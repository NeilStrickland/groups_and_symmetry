\documentclass{amsart}
\usepackage{amssymb}
\usepackage{fullpage}

\newcommand{\R}         {{\mathbb{R}}}
\newcommand{\ip}[1]     {\langle #1\rangle}
\newcommand{\sse}       {\subseteq}
\newcommand{\st}        {\;|\;}
\newcommand{\xra}       {\xrightarrow}
\renewcommand{\:}       {\colon}

\newtheorem{theorem}{Theorem}
\newtheorem{conj}[theorem]{Conjecture}
\newtheorem{construction}[theorem]{Construction}
\newtheorem{lemma}[theorem]{Lemma}
\newtheorem{proposition}[theorem]{Proposition}
\newtheorem{corollary}[theorem]{Corollary}
\theoremstyle{definition}
\newtheorem{remark}[theorem]{Remark}
\newtheorem{definition}[theorem]{Definition}
\newtheorem{example}[theorem]{Example}

\begin{document}
\title{All isometries are affine}
\author{N.~P.~Strickland}
\date{\today}
\bibliographystyle{abbrv}

\maketitle

\begin{definition}
 An \emph{isometry} of $\R^n$ is a bijective map $f\:\R^n\xra{}\R^n$
 that preserves distances, ie $d(f(x),f(y))=d(x,y)$ for all
 $x,y\in\R^n$.  We write $I_n$ for the group of all such isometries.
 We also write $FI_n:=\{f\in I_n\st f(0)=0\}$.
\end{definition}

It is clear that any orthogonal matrix $A\in O_n$ gives rise to an
isometry $f_A\:\R^n\xra{}\R^n$ by $f_A(x)=Ax$.  We will blur the
distinction between $A$ and $f_A$ and thus write $O_n\sse I_n$.  In
fact $f_A(0)=0$ so $O_n\sse FI_n$.  Moreover,
Proposition~\ref{prop-orthogonal} tells us that
$GL_n\cap I_n=O_n$.

For an example of an isometry not in $O_n$, let $a$ be a point in
$\R^n$ and define $T_ax=x+a$.  This is clearly a bijection (with
$T_a^{-1}=T_{-a}$) and
\[ d(T_ax,T_ay)=\|(x+a)-(y+a)\|=\|x-y\|=d(x,y), \]
so it is an isometry.  Isometries of this form are called
\emph{translations}.  As $T_aT_b=T_{a+b}$, we see that the
translations form an Abelian subgroup of $I_n$, which we denote by
$T_n$.  The assignment $a\mapsto T_a$ gives an isomorphism $\R^n\simeq
T_n$.  If $a\neq 0$ then $T_a^n=T_{na}\neq 1$ for all $n\neq 0$, so
$T_a$ has infinite order.  Moreover, we have $T_ax\neq x$ for all $x$,
so $T_a$ has no fixed points.

\begin{theorem}\label{thm-isometry}
 Any distance preserving map $f\:\R^n\xra{}\R^n$ has the form
 $f(x)=Ax+b$ for some $A\in O_n$ and $b\in\R^n$.  It is thus
 automatically bijective.
\end{theorem}
The proof will be given after a number of preparatory lemmas.

\begin{lemma}
 For any $x,y\in\R^n$, the point $z=(x+y)/2$ is the unique point such
 that $d(x,z)=d(y,z)=d(x,y)/2$.
\end{lemma}
\begin{proof}
 It is trivial to check that $d(x,z)=d(y,z)=d(x,y)/2$, and it should
 be geometrically clear (at least in dimensions $\leq 3$) that it is
 the unique point with this property.  For a formal proof, suppose we
 also have $d(x,w)=d(y,w)=d(x,y)/2$.  Put $x'=x-z=(x-y)/2$ and
 $y'=y-z=(y-x)/2=-x'$ and $w'=w-z$ and $r=d(x,y)$.  Then
 $\|x'\|=\|y'\|=r/2$, and also $\|x'-w'\|=\|y'-w'\|=r/2$.  Thus
 $\ip{x'-w',x'-w'}=r^2/4=\ip{x',x'}$, and by expanding this out we
 see that $\|w'\|^2=2\ip{x',w'}$.  Similarly, we have
 $\|w'\|^2=2\ip{y',w'}$.  By adding these equations together and
 using the fact that $y'=-x'$ we see that $\|w'\|^2=0$, so $w'=0$, so
 $w=z$.
\end{proof}

\begin{lemma}
 If $f\:\R^n\xra{}\R^n$ preserves distances then
 $f((x+y)/2)=(f(x)+f(y))/2$.
\end{lemma}
\begin{proof}
 Put $z=(x+y)/2$ and $z'=(f(x)+f(y))/2$ and $r=d(x,y)$.  We then have
 $d(f(x),f(z))=d(x,z)=r/2$ and $d(f(y),f(z))=d(y,z)=r/2$ and
 $d(f(x),f(y))=d(x,y)=r$.  However, $z'$ is the \emph{unique} point
 with $d(f(x),z')=d(f(y),z')=d(f(x),f(y))/2$, so we must have
 $f(z)=z'$.  In other words, $f((x+y)/2)=(f(x)+f(y))/2$, as claimed.
\end{proof}

\begin{lemma}
 If $f\:\R^n\xra{}\R^n$ preserves distances and $f(0)=0$ then
 $\ip{f(x),f(y)}=\ip{x,y}$.
\end{lemma}
\begin{proof}
 First, we have $f(0)=0$ and thus
 \[ \|x\|=d(x,0)=d(f(x),f(0))=d(f(x),0)=\|f(x)\| \]
 for all $x\in\R^n$.

 Next, we can take $y=0$ in the previous lemma to see that
 $f(x/2)=f(x)/2$ for all $x\in\R^n$.  We can then feed this back into
 the previous lemma to see that $f((x+y)/2)=f(x+y)/2$ and thus that
 $f(x+y)=f(x)+f(y)$ for all $x,y\in\R^n$.  If we take the inner
 product of this equation with itself, we find that
 \[ \|f(x+y)\|^2 = \|f(x)\|^2 + \|f(y)\|^2 + 2\ip{f(x),f(y)}. \]
 Thus, we have
 \begin{align*}
  2\ip{f(x),f(y)} &= \|f(x+y)\|^2 - \|f(x)\|^2 - \|f(y)\|^2 \\
                  &= \|x+y\|^2 - \|x\|^2 - \|y\|^2 \\
                  &= \ip{x+y,x+y} - \ip{x,x} - \ip{y,y} \\
                  &= 2\ip{x,y},
 \end{align*}
 as required.
\end{proof}

\begin{corollary}\label{cor-isometry}
 If $f\:\R^n\xra{}\R^n$ preserves distances and $f(0)=0$ then
 $f\in O_n$.
\end{corollary}
\begin{proof}
 Let $e_1,\ldots,e_n$ be the usual basis of $\R^n$, and put
 $u_k=f(e_k)$.  As $f$ preserves inner products, we see that
 $\ip{u_i,u_j}=0$ unless $i=j$, and $\ip{u_i,u_i}=1$.  Define a linear
 map $g\:\R^n\xra{}\R^n$ by
 \[ g(x_1,\ldots,x_n) = x_1u_1 + \ldots + x_nu_n.  \]
 We then have
 \[ \ip{g(x),g(y)} = \sum_{i,j} x_iy_j \ip{u_i,u_j} =
      \sum_i x_iy_i = \ip{x,y},
 \]
 so $g\in O_n$.  I claim that $f=g$.  To see this, consider the map
 $h=g^{-1}f$, which again preserves distances and sends $0$ to $0$,
 and thus preserves inner products.  By construction, we have
 $h(e_k)=e_k$ for all $k$.  As $h$ preserves inner products, we have
 \begin{align*}
  x_k &= \ip{x,e_k} \\
      &= \ip{h(x),h(e_k)} \\
      &= \ip{h(x),e_k} \\
      &= h(x)_k,
 \end{align*}
 so $x=h(x)$ for all $x\in\R^n$.  Thus $h$ is the identity, and $f=g$
 as claimed.
\end{proof}

\begin{proof}[Proof of Theorem~\ref{thm-isometry}]
 Suppose that $f\:\R^n\xra{}\R^n$ preserves distances.  Define
 $b=f(0)$ and $g(x)=f(x)-b$.  Clearly $g$ preserves distances and
 $g(0)=0$.  By the above Corollary, there is an orthogonal matrix $A$
 such that $g(x)=Ax$ for all $x$, so $f(x)=Ax+b$ for all $x$ s
 claimed.  It follows that $f$ is a bijection, with
 $f^{-1}(y)=A^{-1}(y-b)$.
\end{proof}

We conclude this section by giving a simple criterion for when an
isometry is the identity.
\begin{definition}
 A list $u_0,\ldots,u_n$ of $n+1$ points in $\R^n$ is in \emph{general
 position} if the vectors $u_1-u_0,\ldots,u_n-u_0$ form a basis of
 $\R^n$.
\end{definition}

\begin{proposition}\label{prop-gen-pos}
 If $u_0,\ldots,u_n$ are in general position, $f\in I_n$ and
 $f(u_i)=u_i$ for all $i$, then $f=1$.
\end{proposition}
\begin{proof}
 We have $f(x)=Ax+b$ for some $A,b$.  It follows that
 \[ A(u_i-u_0) = (Au_i+b)-(Au_0+b) = f(u_i)-f(u_0) = u_i-u_0 \]
 for all $i$.  As the vectors $u_i-u_0$ form a basis, we deduce that
 $A=I$, so $f(x)=x+b$ for all $x$.  In particular, $u_0=f(u_0)=u_0+b$,
 so $b=0$.  Thus $f(x)=x$ for all $x$ as claimed.
\end{proof}

\end{document}
