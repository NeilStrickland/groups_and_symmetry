\documentclass{amsart}
\usepackage{amssymb}
\usepackage{fullpage}
\usepackage[pdftex]{graphicx}
\usepackage{tikz}
\usepackage[type={CC},modifier={by-nc-sa},version={3.0}]{doclicense}

\usetikzlibrary{patterns}

\DeclareMathOperator{\Tet}      {Tet}
\DeclareMathOperator{\Cube}     {Cube}
\DeclareMathOperator{\Oct}      {Oct}
\DeclareMathOperator{\Dodec}    {Dodec}
\DeclareMathOperator{\Icos}     {Icos}

\DeclareMathOperator{\Aut}      {Aut}
\DeclareMathOperator{\Dir}      {Dir}
\DeclareMathOperator{\Fix}      {Fix}
\DeclareMathOperator{\Trans}    {Trans}
\DeclareMathOperator{\Isom}     {Isom}
\DeclareMathOperator{\Symm}     {Symm}

\DeclareMathOperator{\sgn}      {sgn}
\DeclareMathOperator{\orb}      {orb}
\DeclareMathOperator{\stab}     {stab}

\newcommand{\Z}         {{\mathbb{Z}}}
\newcommand{\Q}         {{\mathbb{Q}}}
\newcommand{\R}         {{\mathbb{R}}}
\newcommand{\Rt}        {{\mathbb{R}^\times}}

\newcommand{\bsm}       {\left(\begin{smallmatrix}}
\newcommand{\esm}       {\end{smallmatrix}\right)}
\newcommand{\blm}       {\left(\begin{array}{cc}}
\newcommand{\elm}       {\end{array}\right)}

\newcommand{\cv}[1]     {\left(\begin{array}{c}#1\end{array}\right)}
\newcommand{\bcf}[2]{\left(\begin{array}{c}{#1}\\{#2}\end{array}\right)}

\newcommand{\half}      {{\textstyle\frac{1}{2}}}
\newcommand{\quarter}   {{\textstyle\frac{1}{4}}}
\newcommand{\thf}       {{\textstyle\frac{\tau}{2}}}
\newcommand{\ip}[1]     {\langle #1\rangle}
\newcommand{\st}        {\;|\;}
\newcommand{\tm}        {\times}
\newcommand{\xra}       {\xrightarrow}
\newcommand{\ov}[1]     {\overline{#1}}
\newcommand{\sm}        {\setminus}
\newcommand{\sse}       {\subseteq}
\newcommand{\tC}        {\widetilde{C}}
\newcommand{\tD}        {\widetilde{D}}
\newcommand{\tR}        {\widetilde{R}}
\newcommand{\tS}        {\widetilde{S}}
\newcommand{\ab}        {\overline{a}}
\newcommand{\vh}        {\hat{v}}
\newcommand{\wh}        {\hat{w}}
\newcommand{\CP}        {{\mathcal{P}}}
\newcommand{\CX}        {{\mathcal{X}}}
\newcommand{\blob}      {circle(0.03cm)}

\newcommand{\al}        {\alpha}
\newcommand{\bt}        {\beta}
\newcommand{\gm}        {\gamma}
\newcommand{\ep}        {\epsilon}
\newcommand{\tht}       {\theta}
\newcommand{\lm}        {\lambda}
\newcommand{\sg}        {\sigma}

\renewcommand{\:}{\colon}

\newtheorem{theorem}{Theorem}[section]
\newtheorem{conj}[theorem]{Conjecture}
\newtheorem{construction}[theorem]{Construction}
\newtheorem{lemma}[theorem]{Lemma}
\newtheorem{proposition}[theorem]{Proposition}
\newtheorem{corollary}[theorem]{Corollary}
\theoremstyle{definition}
\newtheorem{remark}[theorem]{Remark}
\newtheorem{definition}[theorem]{Definition}
\newtheorem{example}[theorem]{Example}

\newtheorem{notation}{Notation}
\renewcommand{\thenotation}{} % make the notation environment unnumbered

%\numberwithin{equation}{subsection}


\begin{document}
\title{Groups and Symmetry}
\author{N.~P.~Strickland}
\date{\today}
\bibliographystyle{abbrv}

\maketitle

\begin{center}
 This work is licensed under a 
 \href{https://creativecommons.org/licenses/by-nc-sa/3.0/deed.en}{
  Creative Commons Attribution-NonCommercial-ShareAlike license}.
 
 \bigskip

 \doclicenseImage 
\end{center}

\section{Symmetry groups in $\R^n$}

\subsection{General linear groups}

We write $M_n$ or $M_n(\R)$ for the set of $n\tm n$ matrices over the
real numbers.  Recall that an $n\tm n$ matrix $A$ is \emph{invertible}
if there is a matrix $B$ such that $AB=I=BA$.  This holds iff
$\det(A)\neq 0$, and in that case the matrix $B$ is unique, and we
call it $A^{-1}$.

We write $GL_n$ or $GL_n(\R)$ for the set of invertible $n\tm n$
matrices over $\R$.

Recall that a \emph{group} is a set $G$ equipped with a binary
operation $*$ and an element $e\in G$ such that
\begin{itemize}
 \item The set $G$ is closed under $*$, in other words $a*b\in G$
  whenever $a,b\in G$.
 \item The operation is associative, in other words
  $a*(b*c)=(a*b)*c$ whenever $a,b,c\in G$.
 \item $e$ is a neutral element, in other words $e*a=a=a*e$ for all
  $a\in G$.
 \item The operation has inverses: for any $a\in G$ there exists an
  element $a^{-1}\in G$ with $a*a^{-1}=e=a^{-1}*a$.
\end{itemize}
For most groups in this course, we will write $ab$ for $a*b$ and $1$
for $e$.

It is easy to check that $GL_n$ is a group under matrix
multiplication; it is called the \emph{general linear group}.

% \subsection{Indexing conventions}

% If $A$ is a matrix, we write $A_{ij}$ for the entry in the $i$'th row
% and $j$'th column, so for example a $2\tm 2$ matrix looks like this:
% \[ A = \blm A_{11} & A_{12} \\ A_{21} & A_{22} \elm. \]
% If $A$ is an $n\tm m$ matrix (ie it has $n$ rows and $m$ columns) and
% $B$ is an $m\tm l$ matrix then $C:=AB$ is defined and is an $n\tm l$
% matrix, with
% \[ C_{ik} = \sum_{j=1}^m A_{ij}B_{jk}. \]
% If $x$ is a vector of length $m$ then $y:=Ax$ is defined and is a
% vector of length $n$, with
% \[ y_i = \sum_{j=1}^m A_{ij} x_j. \]

% For example:
% \[ \blm A_{11} & A_{12} \\ A_{21} & A_{22} \elm
%    \blm B_{11} & B_{12} \\ B_{21} & B_{22} \elm =
%    \blm A_{11} B_{11} + A_{12} B_{21} &
%         A_{11} B_{12} + A_{12} B_{22} \\
%         A_{21} B_{11} + A_{22} B_{21} &
%         A_{21} B_{12} + A_{22} B_{22}
%    \elm
% \]
% and
% \[ \blm A_{11} & A_{12} \\ A_{21} & A_{22} \elm
%    \left( \begin{array}{c} x_1 \\ x_2 \end{array} \right) =
%    \left( \begin{array}{c}
%     A_{11}x_1 + A_{12}x_2 \\
%     A_{21}x_1 + A_{22}x_2
%    \end{array} \right).
% \]

\subsection{Orthogonal groups}

Given vectors $x=(x_1,\ldots,x_n)$ and $y=(y_1,\ldots,y_n)$ in $\R^n$,
we define
\begin{align*}
 \ip{x,y} &= \sum_{i=1}^n x_i y_i \\
 \|x\|    &= \sqrt{\ip{x,x}} = \text{ the length of } x \\
 d(x,y)   &= \|x-y\| = \text{ the distance from $x$ to $y$ }.
\end{align*}

\begin{proposition}[The Cauchy-Schwartz inequality]
 For any $x,y\in\R^n$ we have $|\ip{x,y}|\leq\|x\|\,\|y\|$.
\end{proposition}
\begin{proof}
 (This is included for completeness but is not examinable.)

 For any $t\in\R$ we define
 \begin{align*}
   f(t) &= \|x+ty\|^2 \\
        &= \ip{x+ty,x+ty} \\
        &= \ip{x,x} + 2t\ip{x,y} + t^2\ip{y,y} \\
        &= \|x\|^2 + t^2\|y\|^2 + 2t\ip{x,y}.
 \end{align*}
 From the first part of the definition we see that $f(t)\geq 0$ for
 all $t$.  We now take $t=-\ip{x,y}/\|y\|^2$; the geometric
 interpretation is that in this case $x+ty$ is the projection of $x$
 perpendicular to $y$.  Then
 \[ t^2\|y\|^2=\ip{x,y}^2\|y\|^2/\|y\|^4=
    \ip{x,y}^2/\|y\|^2
 \]
 and
 \[ 2t\ip{x,y} = -2\ip{x,y}^2/\|y\|^2 \]
 so
 \[ f(t) = \|x\|^2 + t^2\|y\|^2 + 2t\ip{x,y} =
           \|x\|^2 - \ip{x,y}^2/\|y\|^2.
 \]
 Thus, the inequality $f(t)\geq 0$ gives
 $\|x\|^2\geq\ip{x,y}^2/\|y\|^2$ or equivalently
 $\|x\|\,\|y\|\geq\ip{x,y}$.  All this assumes that $y\neq 0$ but the
 case $y=0$ is trivial.
\end{proof}
This allows us to define the angle between two nonzero vectors $x$ and
$y$ to be the number $\theta\in[0,\pi]$ such that
$\ip{x,y}=\|x\|\|y\|\cos(\theta)$.

\begin{proposition}[The triangle inequality]
 We have $\|x+y\|\leq \|x\|+\|y\|$, and
 $d(x,z)\leq d(x,y)+d(y,z)$.
\end{proposition}
\begin{proof}
 Using the Cauchy-Schwartz inequality, we have
 \begin{align*}
  (\|x\|+\|y\|)^2 &= \|x\|^2 + \|y\|^2 + 2 \|x\|\,\|y\| \\
     &\geq \|x\|^2 + \|y\|^2 + 2 \ip{x,y} \\
     &= \|x+y\|^2.
 \end{align*}
 By taking square roots we get $\|x\|+\|y\|\geq\|x+y\|$, and thus
 \[ d(x,y) + d(y,z) = \|y-x\| + \|z-y\|
   \geq \|(y-x)+(z-y)\| = \|z-x\| = d(x,z).
 \]
\end{proof}

Next recall that $A^T$ denotes the transpose of $A$, so the rows of
$A^T$ are the columns of $A$, or in other words $(A^T)_{ij}=A_{ji}$.
It is easy to check that
\[ \ip{x,Ay} = \ip{A^Tx,y} = \sum_{i,j} x_i A_{ij} y_j. \]

\begin{proposition}\label{prop-orthogonal}
 If $A\in M_n$, then the following conditions are equivalent:
 \begin{itemize}
 \item[(a)] $A$ is invertible with $A^{-1}=A^T$.
 \item[(b)] $A$ preserves inner products, or in other words
  $\ip{Ax,Ay}=\ip{x,y}$ for all $x,y\in\R^n$.
 \item[(c)] $A$ preserves lengths, or in other words $\|Ax\|=\|x\|$
  for all $x\in\R^n$.
 \item[(d)] $A$ preserves distances, or in other words
  $d(Ax,Ay)=d(x,y)$ for all $x,y\in\R^n$.
 \end{itemize}
\end{proposition}
\begin{proof}
 (a)$\Rightarrow$(b):
  If $A^T=A^{-1}$ then
  \[ \ip{Ax,Ay}=\ip{A^TAx,y}=\ip{A^{-1}Ax,y}=\ip{x,y}. \]

 (b)$\Rightarrow$(c)$\Rightarrow$(d): this is trivial, as lengths are
  defined in terms of inner products, and distances are defined in
  terms of lengths.

 (d)$\Rightarrow$(c): If $A$ preserves distances then
  $\|Ax\|=d(Ax,A0)=d(x,0)=\|x\|$.

 (c)$\Rightarrow$(b): Note that
  \[ \|x+y\|^2=\ip{x+y,x+y}=\|x\|^2 + \|y\|^2 + 2\ip{x,y}, \]
  so
  \[ \ip{x,y} = (\|x+y\|^2 - \|x\|^2 - \|y\|^2)/2. \]
  Thus, if $A$ preserves lengths we have
  \begin{align*}
   \ip{Ax,Ay} &= (\|Ax+Ay\|^2 - \|Ax\|^2 - \|Ay\|^2)/2 \\
              &= (\|A(x+y)\|^2 - \|Ax\|^2 - \|Ay\|^2)/2 \\
              &= (\|x+y\|^2 - \|x\|^2 - \|y\|^2)/2 \\
              &= \ip{x,y}.
  \end{align*}
  
 (b)$\Rightarrow$(a): Suppose that $\ip{Ax,Ay}=\ip{x,y}$ for all
  $x,y$.  We also have $\ip{Ax,Ay}=\ip{x,A^TAy}$, so we deduce that
  $\ip{x,y-A^TAy}=0$.  This means that $y-A^TAy$ is orthogonal to
  every vector in $\R^n$.  In particular, it is orthogonal to itself,
  so it must be zero, so $y=A^TAy$ for all $y$.  This shows that
  $A^TA=I$, so $A^T$ is an inverse for $A$.  (Here we are using the
  fact that if $A$ and $B$ are square matrices of the same size and
  $BA=I$ then $AB=I$ also.  Why is this false for non-square
  matrices?)
\end{proof}

\begin{definition}
 A matrix $A$ is \emph{orthogonal} if it satisfies the equivalent
 conditions in the Proposition.  We write $O_n$ for the set of
 $n\tm n$ orthogonal matrices, and call this the \emph{orthogonal
 group}.
\end{definition}
\begin{proposition}\label{prop-On-subgroup}
 $O_n$ is a subgroup of $GL_n$.
\end{proposition}
\begin{proof}
 We need to check that (1)~the identity matrix is in $O_n$, (2)~if
 $A\in O_n$ then $A^{-1}\in O_n$ and~(3) if $A,B\in O_n$ then
 $AB\in O_n$.  Condition~(1) is clear, because $I^T=I=I^{-1}$.  If
 $A\in O_n$ then $A^TA=I$ and $A^{TT}=A$ so if we put $C=A^T$ we see
 that $CC^T=I$, so $C\in O_n$.  On the other hand, we also have
 $C=A^{-1}$ so $A^{-1}\in O_n$ as required.  Finally, if $A,B\in O_n$
 then $(AB)^T=B^TA^T$ and $BB^T=I$ and $AA^T=I$, so
 $AB(AB)^T=ABB^TA^T=AA^T=I$.  Thus $AB\in O_n$.
\end{proof}

\subsection{Determinants}

Recall that the \emph{determinant} of an $n\tm n$ matrix $A$ is given
by the formula
\[ \det(A) = \sum_{\sg\in S_n} \sgn(\sg) \prod_{i=1}^n A_{i,\sg(i)}.
\]
In other words, for each permutation $\sg$ of $\{1,\ldots,n\}$ we form
the product
\[ A_{1,\sg(1)} A_{2,\sg(2)} \ldots A_{n,\sg(n)}, \]
we multiply by the signature of $\sg$ and then add all these terms up
to get the determinant.  For example, when $n=2$ we just have the
identity permutation $\iota$ and the transposition $\tau=(1\;2)$, and
so we have the familiar formula
\[ \det\blm A_{11} & A_{12} \\ A_{21} & A_{22} \elm =
    \sgn(\iota) A_{1,\iota(1)} A_{2,\iota(2)} +
    \sgn(\tau)  A_{1, \tau(1)} A_{2, \tau(2)} =
    A_{11} A_{22} - A_{12} A_{21}.
\]
Much more important than the definition is the following list of
properties:
\begin{itemize}
 \item[(a)] $\det(I)=1$
 \item[(b)] $\det(AB)=\det(A)\det(B)$
 \item[(c)] $\det(A^T)=\det(A)$
 \item[(d)] If we multiply a single row in $A$ by a number $t$ to get
  a new matrix $A'$, then $\det(A')=t\det(A)$.  The same thing works
  for columns instead of rows.
 \item[(e)] If we add a multiple of one row in $A$ to another row to
  get a new matrix $A'$, then $\det(A')=\det(A)$.
 \item[(f)] $\det(tA)=t^n\det(A)$ for $t\in\R$ (for example 
  $\det\left(\begin{array}{cc}ta & tb\\ tc & td\end{array}\right) = 
   t^2ad-t^2bc=
   t^2\det\left(\begin{array}{cc}a&b\\c&d\end{array}\right)$).
\end{itemize}

If $A$ is invertible then $\det(A)\det(A^{-1})=\det(I)=1$, so
$\det(A)\neq 0$.  Thus $\det$ can be thought of as a function from
$GL_n=\{\text{invertible $n\tm n$ matrices}\}$ to the set
$\Rt:=\R\sm\{0\}$.  Let $D_t$ be the matrix obtained from $I$ by
multiplying the first row by $t$; for example, when $n=4$ we have
\[ D_t = \left(\begin{array}{c|ccc}
    t & 0 & 0 & 0 \\ \hline
    0 & 1 & 0 & 0 \\
    0 & 0 & 1 & 0 \\
    0 & 0 & 0 & 1
  \end{array} \right).
\]
We then have $\det(D_t)=t$; this shows that the function
$\det\:GL_n\xra{}\Rt$ is surjective.  Moreover, $\Rt$ is a group under
multiplication, and properties~(a) and~(b) can be restated as follows:
\begin{proposition}\label{prop-det-GL}
 The determinant gives a surjective homomorphism
 $\det\:GL_n\xra{}\Rt$. \qed
\end{proposition}

We next recall the First Isomorphism Theorem:
\begin{theorem}\label{thm-first-iso}
 If $\phi\:G\xra{}H$ is a surjective homomorphism of groups and
 $N=\{g\in G\st\phi(g)=1\}$ is the kernel of $\phi$, then:
 \begin{itemize}
  \item[(a)] $N$ is a normal subgroup of $G$; in other words, it
   contains $1$, is closed under multiplication and inversion, and
   satisfies $gNg^{-1}=N$ for all $g\in G$.
  \item[(b)] It follows that there is a quotient group $G/N$.  The
   elements of $G/N$ are the cosets of $N$.  For each coset $C$ we
   can choose $g\in G$ such that $C=gN$, but there will usually be
   many choices for $g$. 
  \item[(c)] There is a unique function $\ov{\phi}\:G/N\xra{}H$
   with $\ov{\phi}(gN)=\phi(g)$ for all $g\in G$.
  \item[(d)] The function $\ov{\phi}$ is actually an isomorphism of
   groups. \qed
 \end{itemize}
\end{theorem}

\begin{definition}
 We write
 \[ SL_n = \ker(\det\:GL_n\xra{}\Rt) =
    \{ n\tm n \text{ matrices } A \text{ such that } \det(A)=1\},
 \]
 and call this the \emph{special linear group}.
\end{definition}

The First Isomorphism Theorem implies:
\begin{proposition}
 $SL_n$ is a normal subgroup of $GL_n$, and there is a natural
 isomorphism
 \[ \ov{\det}\:GL_n/SL_n\xra{}\Rt. \qed\]
\end{proposition}

\subsection{Orthogonal determinants}

\begin{lemma}\label{lem-det-O}
 If $A\in O_n$ then $\det(A)\in\{1,-1\}=\{\pm 1\}$.
\end{lemma}
\begin{proof}
 $\det(A)^2=\det(A)\det(A^T)=\det(AA^T)=\det(I)=1$.
\end{proof}

Clearly $\{\pm 1\}$ is a subgroup of $\Rt$, and $\det$ gives a
homomorphism from $O_n$ to $\{\pm 1\}$.  Clearly
$D_{-1}^TD_{-1}=D_{-1}^2=I$, so $D_{-1}\in O_n$, and
$\det(D_{-1})=-1$, so our homomorphism $\det\:O_n\xra{}\{\pm 1\}$ is
surjective.

\begin{definition}\label{defn-SO}
 We write
 \[ SO_n = \ker(\det\:O_n\xra{}\{\pm 1\}) =
    \{ n\tm n \text{ orthogonal matrices } A
       \text{ such that } \det(A)=1\},
 \]
 and call this the \emph{special orthogonal group}.
\end{definition}

The First Isomorphism Theorem gives:
\begin{proposition}
 $SO_n$ is a normal subgroup of $O_n$, and there is a natural
 isomorphism
 \[ \ov{\det}\:O_n/SO_n\xra{}\{\pm 1\}. \]
\end{proposition}

\subsection{One dimension}

A $1\tm 1$ matrix is just a number.  Thus $GL_1=\Rt$, and
$O_1=\{\pm 1\}$.  The determinant map is just the identity, so
$SL_1=SO_1=\{1\}$, the trivial group.

\subsection{Two dimensions}

Given an angle $\theta$, we write $c=\cos(\theta)$ and $s=\sin(\theta)$ (so
$s^2+c^2=1$) and define matrices as follows:
\[
 R_\theta = \blm c & -s\\ s &  c\elm \hspace{6em}
 S_\theta = \blm c &  s\\ s & -c\elm.
\]
It is easy to see that these are orthogonal, and that
$\det(R_\theta)=1$ and $\det(S_\theta)=-1$.  Thus $R_\theta\in SO_2$ and
$S_\theta\in O_2\sm SO_2$.

\begin{theorem}
 Any matrix $A\in SO_2$ has the form $R_\theta$ for some $\theta$.  Any
 matrix $A\in O_2\sm SO_2$ has the form $S_\theta$ for some $\theta$.
\end{theorem}
\begin{proof}
 Suppose $A\in O_2$.  We have $A=\bsm a&b\\c&d\esm$ for some
 $a,b,c,d$.  As $A$ is orthogonal we have $I=A^TA$, so
 \[ \blm 1&0\\0&1\elm = \blm a&c\\ b&d\elm\blm a&b\\c&d\elm =
    \blm a^2 + c^2 & ab + cd \\ ab + cd & b^2 + d^2 \elm,
 \]
 so $a^2+c^2=b^2+d^2=1$ and $ab+cd=0$.  In other words, the vectors
 $u=(a,c)$ and $v=(b,d)$ have length one and are orthogonal to each
 other.  As $u$ is a unit vector, we have $u=(\cos(\theta),\sin(\theta))$
 for some $\theta$, so $a=\cos(\theta)$ and $c=\sin(\theta)$.  It is
 geometrically clear (see the diagram below) that the only unit
 vectors orthogonal to $u$ are $(-c,a)=(-\sin(\theta),\cos(\theta))$ and
 $(c,-a)=(\sin(\theta),\cos(\theta))$.  If $v=(-c,a)$ we find that
 $A=R_\theta$, and if $v=(c,-a)$ we find that $A=S_\theta$.  By equating
 determinants, we see that the first case must occur if $A\in SO_2$,
 and the second case must occur if $A\in O_2\sm SO_2$.
 \begin{center}
  \begin{tikzpicture}[scale=2] % [60]{-2.2}{2.2}{-1.2}{1.2}
   \draw (0,0) circle(1.0);
   \draw[->] (-1.1,0) -- (1.1,0);
   \draw[->] (0,-1.1) -- (0,1.1);
   % the point (0.8,0.6) has radius 1 and argument 37 degrees
   \draw (0,0) -- (0.8,0.6);
   \draw (0.6,-0.8) -- (-0.6,0.8);
   \fill (0,0) \blob;
   \fill (0.8,0.6) \blob;
   \fill (0.6,-0.8) \blob;
   \fill (-0.6,0.8) \blob;
   \draw (0,0) +(0:0.2) arc(0:37:0.2);
   \draw (0,0) +(90:0.2) arc(90:127:0.2);
   \draw (0,0) +(-90:0.2) arc(-90:-53:0.2);
   \draw (-0.65,0.85) node[anchor=south east]
     {$\pm v=(-c,a)=(-\sin(\theta),\cos(\theta))$};
   \draw (0.65,-0.85) node[anchor=north west]
     {$\pm v=(c,-a)=(\sin(\theta),-\cos(\theta))$};
   \draw (0.85,0.65) node[anchor=south west]
     {$u=(a,c)=(\cos(\theta),\sin(\theta))$};
   \draw ( 0.27, 0.09) node {$\theta$};
   \draw ( 0.27, 0.09) node {$\theta$};
   \draw (-0.09, 0.27) node {$\theta$};
   \draw ( 0.09,-0.27) node {$\theta$};
  \end{tikzpicture} 
 \end{center}
\end{proof}

In $\R^2$ it is often convenient to use polar coordinates.  We will
write $[r,\phi]$ for the point at distance $r$ from the origin and
angle $\phi$ to the $x$ axis, so
\[ [r,\phi] = (r\cos(\phi), r\sin(\phi)). \]
Note that $[r,\phi]=[r',\phi']$ iff $r=r'=0$ or
($r=r'\neq 0$ and $\phi-\phi'$ is an integer multiple of $2\pi$).

\begin{proposition}\label{prop-R-rot}
 We have $R_\theta.[r,\phi]=[r,\theta+\phi]$, so $R_\theta$ represents an
 anticlockwise rotation through an angle $\theta$.
\end{proposition}
\begin{proof}
 \begin{align*}
  R_\theta.[r,\phi] &=
   \blm \cos(\theta) & -\sin(\theta) \\ \sin(\theta) & \cos(\theta) \elm
   \cv{ r\cos(\phi) \\ r\sin(\phi) } \\
  &= r\cv{\cos(\theta)\cos(\phi)-\sin(\theta)\sin(\phi) \\
          \sin(\theta)\cos(\phi)+\cos(\theta)\sin(\phi)}
  &= \cv{r\cos(\theta+\phi)\\ r\sin(\theta+\phi)} = [r,\theta+\phi].
 \end{align*}
\end{proof}

We also want to characterise $S_\theta$ geometrically.  Firstly, a very
similar calculation shows that $S_\theta.[r,\phi]=[r,\theta-\phi]$.  Now
define $u_+=[1,\theta/2]$ and $u_-=[1,(\theta+\pi)/2]$, so that $u_+$ and
$u_-$ are unit vectors and are orthogonal to each other.  Note also
that $-[r,\phi]=[r,\phi-\pi]$, so $-u_-=[1,(\theta-\pi)/2]$.  We have
\begin{align*}
 S_\theta.u_+ &= [1,\theta-\theta/2] = [1,\theta/2] = u_+ \\
 S_\theta.u_- &= [1,\theta-\theta/2-\pi/2]=[1,(\theta-\pi)/2]=-u_-.
\end{align*}
Thus $u_+$ and $u_-$ are eigenvectors of $S_\theta$ with eigenvalues
$+1$ and $-1$ respectively.  This means that $S_\theta$ represents
reflection across the line through $0$ and $u_+$.  We summarise our
conclusions as follows:
\begin{proposition}\label{prop-S-ref}
 We have $S_\theta.[r,\phi]=[r,\theta-\phi]$, and $S_\theta$ represents
 reflection across a line $L$ through $0$ at angle $\theta/2$ to the
 $x$-axis. \qed
\end{proposition}

\begin{center}
 \begin{tikzpicture}[scale=2] % [60]{-2.2}{2.2}{-1.2}{1.2}
 \draw (0,0) circle(1.0);
 \draw[->] (-1.1,0) -- (1.1,0);
 \draw[->] (0,-1.1) -- (0,1.1);
 % the point (0.8,0.6) has radius 1 and argument 37 degrees
 \draw (-0.96,-0.72) -- (0.96,0.72);
 \draw (0.6,-0.8) -- (-0.6,0.8);
 \fill (0,0) \blob;
 \fill (0.8,0.6) \blob;
 \fill (0.6,-0.8) \blob;
 \fill (-0.6,0.8) \blob;
 \draw (0,0) +(0:0.2) arc(0:37:0.2);
 \draw (0,0) +(90:0.2) arc(90:127:0.2);
 \draw (0,0) +(-90:0.2) arc(-90:-53:0.2);
 \draw (-0.65,0.85) node[anchor=south east] {$u_-$};
 \draw (0.65,-0.85) node[anchor=north west] {$-u_-=S_\theta u_-$};
 \draw (0.88,0.6) node[anchor=north west] {$u_+=S_\theta u_+$};
 \draw (0.27,0.09) node[anchor=west] {$\theta/2$};
 \draw (1.04,0.78) node[anchor=west] {$L$};
 \end{tikzpicture} 
\end{center}

By working in polar coordinates, it is now easy to check the following
facts:
\begin{align*}
 R_\theta &= R_\phi \hspace{5em} \text{ iff } \theta-\phi\in 2\pi\Z \\
 S_\theta &= S_\phi \hspace{5em} \text{ iff } \theta-\phi\in 2\pi\Z \\
 R_\theta R_\phi &= R_{\theta+\phi} \\
 R_\theta S_\phi &= S_{\theta+\phi} \\
 S_\theta R_\phi &= S_{\theta-\phi} \\
 S_\theta S_\phi &= R_{\theta-\phi} \\
 R_\theta^{-1}   &= R_{-\theta}     \\
 S_\theta^{-1}   &= S_\theta        \\
 R_\theta S_\phi R_\theta^{-1} &= S_{\phi+2\theta}.
\end{align*}

In particular, we have $R_\theta R_\phi=R_\phi R_\theta$, so the group
$SO_2$ is Abelian.

We also have $S_\theta S_\theta=R_{\theta-\theta}=R_0=I$, so all reflections
have order $2$.  A rotation $R_\theta$ has order dividing $m$ iff
$m\theta$ is an integer multiple of $2\pi$, iff
$\theta=2\pi r/m\pmod{2\pi\Z}$ for some $r\in\{0,1,\ldots,m-1\}$.  It
has order exactly $m$ iff $(r,m)=1$.  Most rotations have infinite
order.

% Lecture 2

\subsection{Symmetries of geometric objects}

Let $X$ be a subset of $\R^n$.  For any $A\in O_n$ we put
$AX=\{Ax\st x\in X\}$, the image of $X$ under $A$.  The \emph{symmetry
  group} of $X$ is
\[ \Symm(X) = \{A\in O_n\st AX=X\}. \]
The \emph{direct symmetry group} is
\[ \Dir(X) = \{A\in SO_n\st AX=X\} = \Symm(X)\cap SO_n. \]

\begin{example}
 Let $X$ be a rectangle as shown below.
 \begin{center}
  \begin{tikzpicture}[scale=3]
   \draw (-1,-0.5) -- (1,-0.5) -- (1,0.5) -- (-1,0.5) -- cycle;
   \draw (-1.2,0) -- (1.2,0);
   \draw (0,-0.6) -- (0,0.6);
   \draw[->] (1.1,0) -- (1.1, 0.1);
   \draw[->] (1.1,0) -- (1.1,-0.1);
   \draw[->] (0,0.55) -- ( 0.1,0.55);
   \draw[->] (0,0.55) -- (-0.1,0.55);
   \draw (1.2,0.1) node {$S_0$};
   \draw (0.2,0.6) node {$S_\pi$};
  \end{tikzpicture}
 \end{center}
 It is clearly invariant under the reflections $S_0$ (across the
 $x$-axis) and $S_\pi$ (across the $y$-axis), and also under a
 half-turn (which is $R_\pi$).  We have
  $S_0S_\pi=S_\pi S_0=R_\pi=R_{-\pi}$ and $R_0=I$.  The symmetry groups
 are
 \begin{align*}
  \Symm(X) &= \{I,S_0,S_\pi,R_\pi\} \\
  \Dir(X)  &= \{I,R_\pi\}.
 \end{align*}
\end{example}

Now suppose that $A$ is \emph{not} a symmetry of $X$.  Then $AX$ is
different from $X$, but it has the same shape and thus is ``just as
symmetrical'' as $X$.  However, it it not true (as one might naively
think) that $\Symm(AX)=\Symm(X)$; instead, $\Symm(AX)$ is conjugate to
$\Symm(X)$.  The slogan is that ``conjugacy is doing the same thing
somewhere else''.


\begin{proposition}\label{prop-Symm-AX}
 For any $X\sse\R^n$ and $A\in O_n$ we have
 $\Symm(AX)=A\Symm(X)A^{-1}$ and $\Dir(AX)=A\Dir(X)A^{-1}$.
\end{proposition}
\begin{proof}
 If $B\in\Symm(X)$ then $BX=X$ so $(ABA^{-1})(AX)=ABX=AX$, which shows
 that $ABA^{-1}\in\Symm(AX)$.  Thus $A\Symm(X)A^{-1}\sse\Symm(AX)$.
 Conversely, suppose that $C\in\Symm(AX)$.  If we put $B=A^{-1}CA$,
 then a similar argument shows that $B\in\Symm(X)$.  Thus, the matrix
 $C=ABA^{-1}$ lies in $A\Symm(X)A^{-1}$, proving that
 $\Symm(AX)\sse A\Symm(X)A^{-1}$ as required.

 The argument for $\Dir(X)$ is the same.  It works even if
 $A\not\in SO_n$, because
 \[ \det(ABA^{-1})=\det(A)\det(B)\det(A)^{-1}=\det(B), \]
 so $B$ lies in $SO_n$ iff $ABA^{-1}$ lies in $SO_n$.
\end{proof}

For another example of this sort of phenomenon let $L$ be a line
through the origin, and let $S_L$ be the reflection across $L$.  If
$L$ has angle $\phi$ to the $x$-axis, then $S_L=S_{2\phi}$.  The line
$R_\theta L$ has angle $\theta+\phi$ to the $x$-axis, so
$S_{R_\theta L}=S_{2(\theta+\phi)}$.  On the other hand, from our formulae
for compositions of reflections and rotations, we see that
$R_{\theta}S_{2\phi}R_\theta^{-1}=S_{2\phi+2\theta}$.  In summary, we have:
\begin{proposition}\label{prop-ref-conj}
 For any rotation $R\in SO_2$ and any line $L$ in $\R^2$ we have
 $RS_LR^{-1}=S_{RL}$. \qed
\end{proposition}

\begin{example}
 \begin{center}
  \begin{tikzpicture}[scale=2]
   \draw (-1,-0.5) -- (1,-0.5) -- (1,0.5) -- (-1,0.5) -- cycle;
   \draw[<->] (1.1,-0.2) -- (1.1, 0.2);
   \draw[<->] (-0.2,0.55) -- ( 0.2,0.55);
   \draw ( 0.0,0.0) node {$X$};
   \draw (1.25,0.0) node {$\scriptstyle S_0$};
   \draw ( 0.0,0.65) node {$\scriptstyle S_\pi$};     
   \begin{scope}[xshift=3cm,rotate=30]
    \draw (-1,-0.5) -- (1,-0.5) -- (1,0.5) -- (-1,0.5) -- cycle;
    \draw[<->] (1.1,-0.2) -- (1.1, 0.2);
    \draw[<->] (-0.2,0.55) -- ( 0.2,0.55);
    \draw ( 0.0,0.0) node {$R_{\pi/6}X$};
    \draw (1.25,0.0) node {$\scriptstyle S_{\pi/3}$};
    \draw (-0.2,0.75) node {$\scriptstyle S_{4\pi/3}$};   
   \end{scope}
  \end{tikzpicture}
 \end{center}
 We can see directly that
 \begin{align*}
  \Symm(X)          &= \{I,S_0,S_\pi,R_\pi\}            \\
  \Symm(R_{\pi/6}X) &= \{I,S_{\pi/3},S_{4\pi/3},R_\pi\}
 \end{align*}
 We also have
 \begin{align*}
  R_{\pi/6} I     R_{\pi/6}^{-1} &= I \\
  R_{\pi/6} S_0   R_{\pi/6}^{-1} &= S_{\pi/3} \\
  R_{\pi/6} S_\pi R_{\pi/6}^{-1} &= S_{4\pi/3} \\
  R_{\pi/6} R_\pi R_{\pi/6}^{-1} &= R_\pi.
 \end{align*}
 This shows that $\Symm(R_{\pi/6}X)=R_{\pi/6}\Symm(X)R_{\pi/6}^{-1}$,
 illustrating Proposition~\ref{prop-Symm-AX}.  Also, if we let $L$
 denote the long axis of $X$ we see that $S_0=S_L$ and $S_{\pi/3}$ is
 the reflection in the long axis of $R_{\pi/6}X$, which is
 $R_{\pi/6}L$.  This illustrates Proposition~\ref{prop-ref-conj}.
\end{example}

\section{Polygons}

\subsection{Cyclic and dihedral groups}

Fix an integer $n>0$.  For $k=0,\ldots,n-1$ we put
\[ v_k = [1,2\pi k/n] = (\cos(2\pi k/n),\sin(2\pi k/n)). \]
We then let $X_n$ be the regular $n$-gon with vertices
$v_0,\ldots,v_{n-1}$.  In the case $n=1$ this is to be interpreted as
the line segment from $(0,0)$ to $v_0=(1,0)$.

\begin{center}
 \begin{tikzpicture}[scale=2] % [50]{-1}{6.5}{-1.1}{0.8}
  \draw[->] (-0.5,0) -- (0.5,0);
  \draw[->] ( 1.0,0) -- (2.0,0);
  \draw[->] ( 2.5,0) -- (3.5,0);
  \draw[->] ( 4.0,0) -- (5.0,0);
  \draw[->] ( 5.5,0) -- (6.5,0);
  \draw[->] (0.0,-0.5) -- (0.0,0.5);
  \draw[->] (1.5,-0.5) -- (1.5,0.5);
  \draw[->] (3.0,-0.5) -- (3.0,0.5);
  \draw[->] (4.5,-0.5) -- (4.5,0.5);
  \draw[->] (6.0,-0.5) -- (6.0,0.5);
  \begin{scope}[very thick,blue]
   \draw (0,0) -- (0.35,0);
   \fill (0.35,0) \blob;

   \draw (1.15,0) -- (1.85,0);
   \fill (1.15,0) \blob;
   \fill (1.85,0) \blob;

   \draw (3.35,0) -- (2.83,0.30) -- (2.83,-0.3) -- (3.35,0);
   \fill (3.35,0) \blob;
   \fill (2.83,0.30) \blob;
   \fill (2.83,-0.3) \blob;

   \draw (4.85,0) -- (4.5,0.35) -- (4.15,0) -- (4.5,-0.35) -- (4.85,0);
   \fill (4.85,0) \blob;
   \fill (4.5,0.35) \blob;
   \fill (4.15,0) \blob;
   \fill (4.5,-0.35) \blob;

   \draw (6.35,0) -- (6.10,0.33) -- (5.71,0.20) --
     (5.71,-0.20) -- (6.10,-0.33) -- (6.35, 0);
   \fill (6.35,0) \blob;
   \fill (6.10,0.33) \blob;
   \fill (5.71,0.20) \blob;
   \fill (5.71,-0.20) \blob;
   \fill (6.10,-0.33) \blob;
  \end{scope}
  \draw (0  ,-0.7) node[anchor=north] {$X_1$};
  \draw (1.5,-0.7) node[anchor=north] {$X_2$};
  \draw (3  ,-0.7) node[anchor=north] {$X_3$};
  \draw (4.5,-0.7) node[anchor=north] {$X_4$};
  \draw (6  ,-0.7) node[anchor=north] {$X_5$};
 \end{tikzpicture}
\end{center}

We also define
\begin{align*}
 C_n &= \Dir(X_n) \\
 D_n &= \Symm(X_n) \\
 R   &= R_{2\pi/n} = \text{ $1/n$-turn around the origin } \\
 S   &= S_0 = \text{ reflection across the $x$-axis }.
\end{align*}
We call $C_n$ the \emph{cyclic group}, and $D_n$ the \emph{dihedral
  group}.

\begin{theorem}
 We have
 \begin{align*}
  C_n &= \{ R^i\st 0\leq i<n \} \\
  D_n &= \{ R^i\st 0\leq i<n \} \cup
         \{ R^iS\st 0\leq i<n \}.
 \end{align*}
\end{theorem}
\begin{proof}
 First, it is clear that $R\in C_n$ and $S\in D_n$, so
 $C_n\supseteq\{R^i\st 0\leq i<n\}$ and
 $D_n\supseteq\{R^i\st 0\leq i<n\}\cup\{R^iS\st 0\leq i<n\}$.
 Suppose that $A\in C_n$.  Then $Av_0\in RX_n$ and
 $\|Av_0\|=\|v_0\|=1$.  However, it is easy to see that the only
 vectors in $X_n$ of length $1$ are the vertices, so $Av_0=v_i$ for
 some $i$ with $0\leq i<n$.  This means that the matrix $A':=R^{-i}A$
 satisfies $A'v_0=v_0$.  Also, $A'$ is a rotation, and the only way a
 rotation of the plane can have a nonzero fixed point is if it is the
 identity.  Thus $A'=I$, so $A=R^i$.  Thus $C_n=\{A^i\st 0\leq i<n\}$
 as claimed.

 Now suppose that $B\in D_n$.  If $B\in C_n$ then $B=R^i$ for some $i$
 by the above.  If $B\not\in C_n$ then $\det(B)=-1$, so $BS\in D_n$
 and $\det(BS)=\det(B)\det(S)=1$, so $BS=R^i$ for some $i$.  This
 means that $B=BSS=R^iS$, which proves the claim about $D_n$.
\end{proof}

\begin{remark}\label{rem-Cn-Dn-normal}
 Because $SO_n$ is normal in $O_n$, we see that $C_n$ is normal in
 $D_n$.  It is easy to see that $D_n/C_n\simeq\{\pm 1\}$.
\end{remark}

\subsection{The classification of subgroups}

\begin{proposition}\label{prop-SO-two-subgroups}
 Let $G$ be a finite subgroup of $SO_2$.  Then $G=C_n$ for some $n$.
\end{proposition}
\begin{proof}
 Let $\theta$ be the smallest angle in the range $(0,2\pi]$ such that
 $R_\theta\in G$.  I claim that $\theta=2\pi/n$ for some $n$, and that
 $G=C_n$.  To see this, let $\phi$ be any angle such that $\phi\geq 0$
 and $R_\phi\in G$.  Let $k$ be the largest integer such that
 $k\theta\leq\phi$ and put $\psi=\phi-k\theta$.  We then have
 $0\leq\psi<\theta\leq 2\pi$, and $R_\psi=R_\phi R_\theta^{-k}\in G$.
 If $\psi$ were in the range $(0,2\pi]$, this would contradict our
 definition of $\theta$, so we must have $\psi=0$.  Thus
 $\phi=k\theta$ and $R_\phi=R_\theta^k$.  This shows that the elements
 of $G$ are precisely the powers of $R_\theta$.

 In particular, we have $R_{2\pi}=I\in G$, so we can apply the above
 argument with $\phi=2\pi$ and deduce that $2\pi=n\theta$ for some
 $n>0$, so $\theta=2\pi/n$.  Thus $G$ consists of the powers of
 $R_{2\pi/n}$, in other words $G=C_n$.
\end{proof}

% Lecture 3

\begin{theorem}\label{thm-O-two-subgroups}
 Let $G$ be a finite subgroup of $O_2$.  Then either $G=C_n=\Dir(X_n)$
 for some $n$, or $G=R_\theta D_n R_\theta^{-1}=\Symm(R_\theta X_n)$ for
 some $n$ and $\theta$.
\end{theorem}
\begin{proof}
 Put $H=G\cap SO_2$; the Proposition tells us that $H=C_n$ for some
 $n$.  If $G\leq SO_2$ then $G=H=C_n$.  Otherwise $G$ contains some
 reflection, say $S_{2\theta}\in G$.  If $A\in G$ then either
 \begin{itemize}
  \item[(a)] $\det(A)=1$, so $A\in H$ and $A=R_{2\pi/n}^k$ for some
   $k$; or
  \item[(b)] $\det(A)=-1$ so $AS_{2\theta}\in G$ and
   $\det(AS_{2\theta})=1$ so $AS_{2\theta}=R_{2\pi/n}^k$ for some $k$ so
   $A=R_{2\pi/n}^kS_{2\theta}$.
 \end{itemize}
 Next, note that $R_\theta^{-1}R_{2\pi/n}^kR_\theta=R_{2\pi/n}^k$ and
 $R_\theta^{-1}S_{2\theta}R_\theta=S_0$.  It follows that the group
 $G':=R_\theta^{-1}GR_\theta$ consists of the elements $R_{2\pi/n}^k$ and
 $R_{2\pi/n}^kS_0$, or in other words $G'=D_n$.  Thus
 $G=R_\theta G'R_\theta^{-1}=R_\theta D_n R_\theta^{-1}$, as required.
\end{proof}

\section{Affine isometries}

\begin{definition}\label{defn-isom}
 An \emph{isometry} of $\R^n$ is a function $f\:\R^n\xra{}\R^n$ of the
 form $f(x)=Ax+a$ for some orthogonal matrix $A\in O_n$ and some
 vector $a\in\R^n$.  We write $\Isom_n$ for the set of all such
 functions.
\end{definition}
\begin{remark}\label{rem-isom-dist}
 If we have an isometry $f(x)=Ax+a$ as above, then
 $d(f(x),f(y))=d(x,y)$ for all $x$ and $y$ in $\R^n$, or in other
 words, $f$ preserves distances.  To see this, note that
 \[ d(f(x),f(y)) = \|(Ax+a)-(Ay+a)\| = \|A(x-y)\| = 
     \|x-y\| = d(x,y).
 \]
 (At the third step we used the fact that $A$ is an orthogonal matrix,
 so $\|Az\|=\|z\|$ for any vector $z\in\R^n$.)
\end{remark}
\begin{remark}
 Let $f\:\R^n\xra{}\R^n$ be any function that preserves distances.  It
 can be shown that there is a matrix $A\in O_n$ and a vector
 $a\in\R^n$ such that $f(x)=Ax+a$ for all $x$, so $f\in\Isom_n$.  (The
 proof takes about a page and a half, but we will not give it here.)
 This means that Definition~\ref{defn-isom} is compatible with the
 general definition of isometries for metric spaces.
\end{remark}

\begin{remark}\label{rem-isom-group}
 Suppose we have isometries $f(x)=Ax+a$ and $g(x)=Bx+b$.  Then
 \begin{align*}
  f(g(x))   &= (AB)x + (Ab+a) \\
  f^{-1}(x) &= A^{-1}x + (- A^{-1}a). 
 \end{align*}
 We have $AB\in O_n$ and $Ab+a\in\R^n$ so $f\circ g\in\Isom_n$.
 Similarly $A^{-1}\in O_n$ and $-A^{-1}a\in\R^n$ so
 $f^{-1}\in\Isom_n$.  This shows that $\Isom_n$ is a group under
 composition.  We will usually write $fg$ instead of $f\circ g$, and
 write $1$ for the identity map.
\end{remark}

We will not distinguish between a matrix $A\in O_n$ and the
corresponding isometry $f(x)=Ax$.  We thus think of $O_n$ as a
subgroup of $\Isom_n$.

For any $a\in\R^n$ we have an isometry $T_a$ defined by $T_a(x)=x+a$;
this is called a \emph{translation}.  We clearly have $T_aT_b=T_{a+b}$
and $T_a^{-1}=T_{-a}$.  It follows that the translations form an
abelian subgroup $\Trans_n\leq\Isom_n$.  Using the correspondence
$T_a\leftrightarrow a$ we can identify $\Trans_n$ with $\R^n$.

\subsection{The homomorphism $\psi$}
\label{subsec-psi}

\begin{definition}\label{defn-psi}
 Given an isometry $f(x)=Ax+a$, we define $\psi(f)=A\in O_n$ and
 $\det(f)=\det(A)=\det(\psi(f))\in\{1,-1\}$.  This
 gives functions $\psi\:\Isom_n\xra{}O_n$ and
 $\det\:\Isom_n\xra{}\{1,-1\}$. 
\end{definition}

\begin{proposition}
 The map $\psi$ is a surjective homomorphism with kernel $\Trans_n$, and
 thus it induces an isomorphism $\Isom_n/\Trans_n\simeq O_n$.
 Moreover, $\det\:\Isom_n\xra{}\{\pm 1\}$ is also a homomorphism.
\end{proposition}
\begin{proof}
 First suppose we have $f(x)=Ax+a$ and $g(x)=Bx+b$.  Then
 \[ fg(x) = f(Bx+b) = A(Bx+b)+a = ABx + (Ab+a), \]
 which shows that $\psi(fg)=AB=\psi(f)\psi(g)$.  This shows that
 $\psi$ is a homomorphism, and it follows that
 $\det\:\Isom_n\xra{}\{\pm 1\}$ is also a homomorphism.  For any
  $A\in O_n$ we can define an isometry $f$ by $f(x)=Ax$ and then
 $\psi(f)=A$, which shows that $\psi$ is surjective.  We have
 $\psi(f)=I$ iff $f(x)=x+a$ for all $x$, iff $f$ is a translation, so
 $\ker(\psi)$ is the translation subgroup $\Trans_n$.  It now follows
 from the First Isomorphism Theorem that $\Isom_n/\Trans_n\simeq O_n$.
\end{proof}

\begin{proposition}\label{prop-semidirect}
 For any $f\in \Isom_n$ and $b\in\R^n$ we have
 $f T_b f^{-1}=T_{\psi(f)a}$.

 (This is meaningful because $\psi(f)\in O_n$ is a matrix and $a$ is a
 vector so $\psi(f)a$ is another vector, so we have a translation
 function $T_{\psi(f)a}$.)
\end{proposition}
\begin{proof}
 We can write $f(x)=Ax+a$, where $A=\psi(f)$.  We then have
 \[ f T_b(x) = f(x+b) = Ax + Ab + a = f(x) + Ab = T_{Ab} f(x). \]
 Thus, $fT_b=T_{Ab}f$, and we can multiply by $f^{-1}$ on the right to
 get $fT_bf^{-1}=T_{Ab}=T_{\psi(f)b}$.
\end{proof}

\begin{definition}
 For any subset $X\sse\R^n$, we put 
 \[ \Isom(X)=\{f\in \Isom_n\st f(X)=X\}, \]
 and call this the \emph{isometry group} of $X$.
\end{definition}

\begin{example}\label{eg-p-one}
 Let $X$ be the subset of $\R^2$ illustrated below.  It extends
 infinitely in all directions, and the distance between adjacent faces
 is one unit.  
 \begin{center}
  \begin{tikzpicture}[scale=1.2]
   \draw[->] (-2.5cm,0) -- (4cm,0);
   \draw[->] (0,-2.5cm) -- (0,4cm);
   \foreach \x in {-1,0,1,2} {
    \foreach \y in {-1,0,1,2} {
     \begin{scope}[xshift=1.5*\x cm,yshift=1.5*\y cm,blue]
      \draw (0,0) circle(0.3);
      \draw (-0.1,0.15) circle(0.03);
      \draw ( 0.1,0.15) circle(0.03);
      \draw ( 0,-0.05) arc(-90:90:0.05);
      \draw (0,0) +(225:0.18) arc(225:315:0.18);
     \end{scope}
    }
   }
  \end{tikzpicture}
 \end{center}
 If we shift $X$ by $n$ units to the right and $m$ units up, we just
 get $X$ again (assuming that $n$ and $m$ are integers).  In other
 words, $T_{(n,m)}X=X$, so $T_{(n,m)}\in \Isom(X)$.  In fact, one can
 check that these are the only symmetries, so
 $\Isom(X)=\{T_{(n,m)}\st(n,m)\in\Z^2\}$. 
\end{example}


We conclude this section by giving a simple criterion for when an
isometry is the identity.
\begin{definition}
 A list $u_0,\ldots,u_n$ of $n+1$ points in $\R^n$ is in \emph{general
 position} if the vectors $u_1-u_0,\ldots,u_n-u_0$ form a basis of
 $\R^n$.
\end{definition}

\begin{proposition}\label{prop-gen-pos}
 If $u_0,\ldots,u_n$ are in general position, $f\in \Isom_n$ and
 $f(u_i)=u_i$ for all $i$, then $f=1$.
\end{proposition}
\begin{proof}
 We have $f(x)=Ax+b$ for some $A,b$.  It follows that
 \[ A(u_i-u_0) = (Au_i+b)-(Au_0+b) = f(u_i)-f(u_0) = u_i-u_0 \]
 for all $i$.  As the vectors $u_i-u_0$ form a basis, we deduce that
 $A=I$, so $f(x)=x+b$ for all $x$.  In particular, $u_0=f(u_0)=u_0+b$,
 so $b=0$.  Thus $f(x)=x$ for all $x$ as claimed.
\end{proof}


\section{Plane isometries}

We next define some special types of isometries of $\R^2$.
\begin{enumerate}
 \item For any $a\in\R^2$ and any angle $\theta$, we put
  $R_{\theta,a}=T_aR_\theta T_{-a}$, so that
  \[ R_{\tht,a}(x) = a + R_\tht(x-a) = R_\tht x + (1-R_\tht)a. \]
  Note that $R_{\theta,a}(a+x)=a+R_\theta x$, which means that
  $R_{\theta,a}$ is the rotation through angle $\theta$ around $a$.
  If $\theta$ is not a multiple of $2\pi$ then for all $x\neq 0$ we
  have $R_\theta x\neq 0$; it follows that $a$ is the unique fixed
  point of $R_{\theta,a}$.  Note also that $\psi(R_{\tht,a})=R_\tht$.
 \item For any line $L<\R^2$ (not necessarily passing through the
  origin) we let $S_L$ be the reflection across $L$.  If $L$ has angle
  $\theta/2$ to the $x$-axis and $a\in L$ one checks that
  $S_L=T_aS_\theta T_{-a}$.  We also have $\psi(S_L)=S_\al$, where
  $\al$ is the angle between $L$ and the $x$-axis.
 \item For any line $L<\R^2$ and any vector $b$ that is parallel to
  $L$, we define $G_{L,b}=T_bS_L$.  It is not hard to check
  geometrically that $G_{L,b}=S_LT_b$ also, and it follows that
  \[ G_{L,b}^2 = (T_b S_L)(S_L T_b) = T_b^2 = T_{2b}. \]
  \begin{center}
   \begin{tikzpicture}[scale=2]
    \draw (-1.5,0) -- (1.5,0);
    \fill (-1,0.5) \blob (1,0.5) \blob (-1,-0.5) \blob (1,-0.5) \blob;
    \draw[->,blue] (-0.95, 0.5) -- (0.95, 0.5);
    \draw[->,blue] (-0.95,-0.5) -- (0.95,-0.5);
    \draw[<->,dotted,red] (-1,-0.42) -- (-1,0.42);
    \draw[<->,dotted,red] ( 1,-0.42) -- ( 1,0.42);
    \draw (-1.55,0) node[anchor=east]{$L$};
    \draw (-1.05, 0.5) node[anchor=east]{$S_Lx$};
    \draw (-1.05,-0.5) node[anchor=east]{$x$};
    \draw (0,-0.55) node[anchor=north]{$b$};
    \draw (0, 0.55) node[anchor=south]{$b$};
    \draw (1.05, 0.5) node[anchor=west]{$T_bS_Lx=S_LT_bx$};
    \draw (1.05,-0.5) node[anchor=west]{$T_bx$};
   \end{tikzpicture}
  \end{center}
  Clearly $G_{L,0}=S_L$.  Maps of the form $G_{L,b}$ with $b\neq 0$
  are called \emph{glide-reflections}.  We have $\psi(G_{L,b})=S_\al$,
  where $\al$ is the angle between $L$ and the $x$-axis.
\end{enumerate}

\begin{proposition}
 For any $f\in \Isom_2$, precisely one of the following holds:
 \begin{itemize}
  \item[(a)] $f=1$
  \item[(b)] $f=T_a$ for some $a\in\R^2\sm\{0\}$
  \item[(c)] $f=R_{\theta,a}$ for some $a\in\R^2$ and $\theta\in(0,2\pi)$
  \item[(d)] $f=S_L$ for some line $L<\R^2$
  \item[(e)] $f=G_{L,b}$ for some $L$ and some nonzero vector $b$
   parallel to $L$.
 \end{itemize}
\end{proposition}
\begin{proof}
 We know that there exists a matrix $A\in O_2$ and a vector $b$ such
 that $f(x)=Ax+b$ for all $x$.  If $A=I$ then we are in case~(a) (if
 $b=0$) or case~(b) (if $b\neq 0$).  We may thus assume that
 $A\neq I$.

 If $A$ is a rotation we have $A=R_\theta$ for some $\theta\in(0,2\pi)$.
 As $A$ is a nontrivial rotation, for all $x$ we have $x\neq Ax$ so
 $(I-A)x\neq 0$.  Thus, the kernel of $I-A$ is zero, so $I-A$ is
 invertible.  Put $a=(I-A)^{-1}b$, so that $b=a-Aa$.  Then
 \[ R_{\theta,a}x=T_a A T_{-a}x=A(x-a) +a=Ax +a-Aa=Ax+b = f(x), \]
 so $f=R_{\theta,a}$.

 Now suppose instead that $A$ is a reflection, say $A=S_\theta$.  As
 before we put $u_+=[1,\theta/2]$ and $u_-=[1,(\theta+\pi)/2]$, so $u_+$
 and $u_-$ are unit vectors and are orthogonal to each other.  We can
 write any vector $x$ in the form $x_++x_-$, where $x_{\pm}$ is a
 multiple of $u_{\pm}$, and then $Ax=x_+-x_-$.  It follows that
 \[ f(x) = Ax+b = x_+ - x_- + b_+ + b_- = (b_+ + x_+) + (b_- - x_-).
 \]
 Now let $L$ be the line through $b_-/2$ at angle $\theta/2$ to the
 $x$-axis.  We can write any vector $x$ as
 $(x_+ +\half b_-)+(x_- -\half b_-)$, where $(x_+ +\half b_-)\in L$
 and $(x_- -\half b_-)$ is orthogonal to $L$.  It follows that
 \[ S_L x = (x_+ + \half b_-) - (x_- -\half b_-) =
             x_+ + b_- - x_-,
 \]
 and thus that
 \[ G_{L,b_+} x = b_+ + x_+ + b_- - x_- = f(x). \]
 Thus, $f=G_{L,b_+}$, so $f$ is a reflection (if $b_+=0$) or a
 glide-reflection (if $b_+\neq 0$).
\end{proof}

\begin{remark}\label{rem-isom-classify}
 Suppose we have an isometry $f$, and we want to know where it falls
 in the above clasification.  One can check using the above proof that
 the following method will work.
 \begin{itemize}
  \item[(a)] Find the matrix $A=\psi(f)\in O_2$.
  \item[(b)] If $A$ is the identity, then $f=T_u$ for some $u$.  To
   find $u$, let $x$ be any point for which one can easily find
   $f(x)$, and then $u=f(x)-x$.
  \item[(c)] Now suppose that $\psi(f)=R_\tht$ for some angle
   $\tht\in(0,2\pi)$.  Then there is a unique point $a$ such that
   $f(a)=a$, and it works out that $f=R_{\tht,a}$.
  \item[(d)] Suppose instead that $\psi(f)=S_\tht$ for some $\tht$.
   Then we choose a point $x$ for which we can easily calculate
   $f(f(x))$, and put $u=(f(f(x))-x)/2$.  We then put
   $L=\{x\st f(x)=x+u\}$.  It works out that $L$ is always a line
   parallel to $u$, and that $f=G_{L,u}$ (if $u\neq 0$) or $f=S_L$
   (if $u=0$).  
 \end{itemize}
\end{remark}

\subsection{Subgroups with no translations}

\begin{theorem}\label{thm-no-trans}
 Let $H$ be a subgroup of $\Isom_2$, and suppose that $H$ contains no
 translations (other than the trivial translation $T_0=1$).  Then
 there is a point $a\in\R^2$ such that $f(a)=a$ for all $f\in H$, and
 thus $H\leq T_a O_2 T_a^{-1}$.
\end{theorem}

This theorem implies a classification of finite subgroups of $\Isom_2$, as
will be explained in Corollary~\ref{cor-plane-finite}.  The proof relies
on the following lemma.
\begin{lemma}\label{lem-no-trans}
 \begin{itemize}
  \item[(a)] $(R_{a,\theta}S_L)^2=T_c$, where $c=(1-R_\theta)(a-S_L(a))$.
  \item[(b)] $R_{a,\theta}R_{b,\phi}R_{a,\theta}^{-1}R_{b,\phi}^{-1}=T_d$,
   where $d=(1-R_\phi)(1-R_\tht)(a-b)$.
  \item[(c)] If $K$ and $L$ are parallel then there is a vector $u$
   perpendicular to $K$ and $L$ such that $L=K+u$, and
   $S_LS_K=T_{2u}$.
  \item[(d)] If $K$ and $L$ are not parallel then they meet at a
   unique point $a$, and there is a unique angle $\theta\in[0,\pi)$
   such that $L=R_{a,\theta}K$, and $S_LS_K=R_{2\theta,a}$.
 \end{itemize}
\end{lemma}
We will first prove the theorem using the lemma, then we will prove
the lemma.

\begin{proof}[Proof of Theorem~\ref{thm-no-trans}]
 I first claim that $H$ contains no glide-reflections.  Indeed, if
 $G_{L,b}\in H$ then $G_{L,b}^2\in H$ but $G_{L,b}^2=T_{2b}$ and
 $2b\neq 0$, contrary to our assumption about $H$.  Thus every element
 of $H$ is either the identity, a nontrivial rotation, or a
 reflection.

 Now suppose that $H$ contains a nontrivial rotation $R_{a,\theta}$.
 Because this is nontrivial we have $R_\theta(x)\neq x$ for all $x$, so
 $(1-R_\theta)(x)\neq 0$, so $1-R_\theta$ is invertible.  I claim that
 $f(a)=a$ for all $f\in H$.  This is clear if $f=1$.  If $f$ is a
 nontrivial rotation, say $f=R_{b,\phi}$, then we note that the
 element $g=R_{a,\theta}R_{b,\phi}R_{a,\theta}^{-1}R_{b,\phi}^{-1}$ also
 lies in $H$.  Part~(b) of the lemma tells us that $g=T_d$, where
 $d=(1-R_\phi)(1-R_\tht)(a-b)$.  As $H$ contains no nontrivial
 translations, we have $d=0$.  As $1-R_\theta$ and $1-R_\phi$ are
 invertible, we must have $a-b=0$, and so $a=b$, so $f=R_{a,\phi}$.
 Thus the element $f=R_{a,\phi}$ has $f(a)=a$ as claimed.

 Now suppose instead that $f$ is a reflection, say $f=S_L$.  We then
 note that the element $h=(R_{a,\theta}S_L)^2$ also lies in $H$.
 Part~(a) of the lemma tells us that $h=T_c$, where
 $c=(1-R_\theta)(a-S_L(a))$.  It follows that $c=0$, and $1-R_\theta$ is
 invertible so $a-S_L(a)=0$, so $S_L(a)=a$.  Thus $f(a)=S_L(a)=a$ as
 required. 

 This proves the theorem when $H$ contains a nontrivial rotation.  Now
 suppose instead that $H$ contains only reflections and the identity
 map.  I claim that $H$ contains at most one reflection.  If not, let
 $S_K$ and $S_L$ be two different reflections in $H$, so $S_LS_K$ also
 lies in $H$.  We see from parts~(c) and~(d) of the lemma that
 $S_LS_K$ is either a nontrivial translation or a nontrivial rotation,
 giving a contradiction.  It follows that $H$ is either the trivial
 group $\{1\}$ or a group of the form $\{1,S_L\}$ for some line $L$.
 In the first case we can take $a$ to be any point at all, and in the
 second case $a$ can be any point on $L$.
\end{proof}

\begin{proof}[Proof of Lemma~\ref{lem-no-trans}]
 We first check the general type of the various isometries considered,
 using the method described in Remark~\ref{rem-isom-classify}.
 \begin{itemize}
  \item[(a)] Clearly $\psi(R_{a,\tht}S_L)$ is a rotation times a
   reflection, which is another reflection.  Every reflection in $O_2$
   squares to the identity, so $\psi((R_{a,\tht}S_L)^2)=1$, so
   $(R_{a,\tht}S_L)^2=T_c$ for some $c$.
  \item[(b)] We have
   \[ \psi(R_{a,\theta}R_{b,\phi}R_{a,\theta}^{-1}R_{b,\phi}^{-1}) = 
       R_\tht R_\phi R_{-\tht} R_{-\phi} = 
       R_{\tht+\phi-\tht-\phi} = 1,
   \]
   so 
   \[ R_{a,\theta}R_{b,\phi}R_{a,\theta}^{-1}R_{b,\phi}^{-1}= T_d \]
   for some $d$.
  \item[(c)] As $L$ and $K$ are parallel, they have the same angle
   with the $x$-axis, say $\al$.  We thus have 
   \[ \psi(S_LS_K) = S_\al S_\al = 1, \]
   so $S_LS_K=T_e$ for some $e$.
  \item[(d)] Here $\psi(S_LS_K)$ is a product of two different
   reflections in $O_2$, so it is a rotation, say $R_\phi$.  This
   means that $S_LS_K=R_{a,\phi}$ for some $a$ and $\phi$.
 \end{itemize}

 We next find the details.
 \begin{itemize}
  \item[(a)] To find $c$, we choose any convenient point $x$, and then
   $c$ will be $c=(R_{a,\theta}S_L)^2(x)-x$.  We will take $x=S_L(a)$.
   We then have $S_L(x)=a$ so $R_{a,\theta}S_L(x)=R_{a,\theta}(a)=a$ so
   \begin{align*}
    (R_{a,\theta}S_L)^2(x)
      &= R_{a,\theta}S_L(a) \\
      &= R_\tht S_L(a) + (1-R_\tht)a \\
    c &= (R_{a,\theta}S_L)^2(x) - x \\
      &= R_\tht S_L(a) + (1-R_\tht)a - S_L(a) \\
      &= (1-R_\tht)a + (R_\tht-1)S_L(a) \\
      &= (1-R_\tht)(a-S_L(a)).
   \end{align*}
  \item[(b)] Put
   $f=R_{a,\theta}R_{b,\phi}R_{a,\theta}^{-1}R_{b,\phi}^{-1}$.  We
   have seen that $\psi(f)=1$, so $f=T_d$ for some $d$.  To find $d$,
   we choose any convenient point $x$, and then $d$ will be $f(x)-x$.
   We will take $x=R_{b,\phi}R_{a,\tht}(b)$, so 
   \begin{align*}
    f(x) &= R_{a,\theta}R_{b,\phi}R_{a,\theta}^{-1}R_{b,\phi}^{-1}
            R_{b,\phi}R_{a,\tht}(b) \\
         &= R_{a,\tht} R_{b,\phi}(b) \\
         &= R_{a,\tht}(b) \\
         &= R_\tht b + (1-R_\tht)a.
   \end{align*}
   (At the third step we used the fact that $R_{b,\phi}$ is a rotation
   around $b$, so it sends $b$ to itself.)

   We also have 
   \begin{align*}
    x &= R_{b,\phi}R_{a,\tht}(b) \\
      &= R_{b,\phi}(R_\tht b + (1-R_\tht)a) \\
      &= R_\phi R_\tht b + R_\phi(1-R_\tht)a + (1-R_\phi)b.
   \end{align*}
   By subtracting these, we get
   \begin{eqnarray*}
    d & = & f(x) - x \\
      & = & R_\tht b + (1-R_\tht)a \\
      &   & - R_\phi R_\tht b - R_\phi(1-R_\tht)a - (1-R_\phi)b \\
      & = & (1-R_\phi)R_\tht b + (1-R_\phi)(1-R_\tht)a - (1-R_\phi)b \\
      & = & (1-R_\phi)(1-R_\tht)(a-b).
   \end{eqnarray*}
  \item[(c)]
   We know that $S_LS_K=T_e$ for some $e$. 
   Choose any point $x$ on the line $K$, so $S_K(x)=x$.  We then have
   $e=S_LS_K(x)-x=S_L(x)-x$.
   
   If we move away from $x$ towards $L$ in a direction perpendicular
   to $K$ and $L$, we will eventually reach $L$.  In other words,
   there is a vector $u$ perpendicular to $K$ and $L$ such that the
   point $y:=x+u$ lies in $L$.  As $L$ is parallel to $K$, it is easy
   to see that $L=K+u$.  Moreover, $S_L(x)$ is the reflection of $x$
   across $L$, which is just $x+2u$.  It follows that $e=S_L(x)-x=2u$
   as claimed.
   \begin{center}
    \begin{tikzpicture}[scale=2]
     \draw (-1.5,0) -- (1.5,0);
     \fill (-1,0) \blob (0,0) \blob (1,0) \blob;
     \draw[->,blue] (-0.95,0) -- (-0.05,0);
     \draw[->,blue] (0.05,0) -- (0.95,0);
     \draw (-1,-1) -- (-1,0.1);
     \draw (-1,0.4) -- (-1,1);
     \draw ( 0,-1) -- ( 0,0.1);
     \draw ( 0,0.4) -- ( 0,1);
     \draw (-0.5,-0.1) node {$u$};
     \draw ( 0.5,-0.1) node {$u$};
     \draw (-1,0.25) node {$x$};
     \draw ( 0,0.25) node {$y$};
     \draw ( 1,0.25) node {$S_Lx$};
     \draw (-1,1.1) node {$K$};
     \draw ( 0,1.1) node {$L$};
    \end{tikzpicture}
   \end{center}
  \item[(d)] Let $K$ and $L$ be lines that are not parallel.  It is
   geometrically clear that they meet in a unique point, which we call
   $a$.  Let $\al$ be the angle between the $x$-axis and $K$, measured
   anticlockwise from the axis.  Let $\theta$ be the angle between $K$
   and $L$, measured anticlockwise from $K$, so that $\theta\in[0,\pi)$.
   Clearly $L$ is obtained by rotating $K$ around $a$ through an angle
   of $\theta$, in other words $L=R_{a,\theta}K$.  We also have
   $S_K=T_aS_{2\al}T_{-a}$ and $S_L=T_aS_{2\al+2\theta}T_{-a}$ and
   $S_{2\al+2\theta}S_{2\al}=R_{2\theta}$ so
   $S_LS_K=T_aR_{2\theta}T_{-a}=R_{a,2\theta}$.
 \end{itemize}
\end{proof}

\begin{corollary}\label{cor-plane-finite}
 Let $H$ be a finite subgroup of $\Isom_2$.  Then either
 $H=T_aC_nT_a^{-1}$ for some $a$ and $n$, or
 $H=T_aR_\theta D_n R_\theta^{-1}T_a^{-1}$ for some $a$, $n$ and $\theta$.
\end{corollary}
\begin{proof}
 Every element of $H$ has finite order, and thus cannot be a
 nontrivial translation.  It follows from the theorem that
 $H\leq T_a O_2 T_a^{-1}$ for some $a$, so the group
 $H':=T_a^{-1}HT_a$ is contained in $O_2$.
 Theorem~\ref{thm-O-two-subgroups} tells us that $H'$ has the form
 $C_n$ or $R_\theta D_n R_\theta^{-1}$ and clearly $H=T_aH'T_a^{-1}$.  The
 claim follows.
\end{proof}

\section{Wallpaper}
\label{sec-wallpaper}

In this section we study symmetry groups of ``wallpaper patterns'',
which for our purposes will mean ``reasonable'' subsets of $\R^2$
which are translationally symmetric in two different directions.  (I
say ``reasonable'' to exclude sets like $\Q^2$; we will be more
precise later.)  The real importance of this study (and its
three-dimensional analogue) is in the physical chemistry of crystals:
the symmetry group of a crystal is a useful tool in studying the way
it vibrates, refracts X-rays, and so on.

The simplest wallpaper group was discussed in Example~\ref{eg-p-one}.
It turns out that there are precisely 17 types of wallpaper up to a
suitable notion of equivalence.  Here we will analyse a small
selection of these types, and prove some of the key results in the
general classification.

We start with some general concepts.
\begin{definition}
 For any subgroup $H\leq \Isom_2$, the \emph{point group} of $H$ is the
 subgroup $\psi(H)=\{\psi(h)\st h\in H\}\leq O_2$, where $\psi$ is as
 in Section~\ref{subsec-psi}.  We also write
 $\Trans(H)=\{a\in\R^2\st T_a\in H\}$ and call this the translation
 subgroup of $H$.

 For any point $a\in\R^2$, we also define
 $\sg_a(H)=\{A\in O_2\st T_aAT_a^{-1}\in H\}$, which is a subgroup of
 $O_2$.  This is the part of $H$ that encodes the rotational and
 reflectional symmetry about $a$.
\end{definition}

\begin{proposition}
 For any $a\in A$ we have $\sg_a(H)\sse\psi(H)$.
\end{proposition}
\begin{proof}
 If $A\in\sg_a(H)$ then $T_aAT_a^{-1}\in H$, so
 $\psi(T_a)\psi(A)\psi(T_a)^{-1}\in\psi(H)$.  We have $\psi(T_a)=I$
 and $\psi(A)=A$, so $A\in\psi(H)$, as required.
\end{proof}

\begin{definition}\label{defn-gens}
 Let $G$ be a group, and let $x_1,\ldots,x_r$ be elements of $G$.  We
 say that these elements \emph{generate} $G$ if every element in
 $g\in G$ can be expressed in terms of the elements $x_i$, say
 \[ g = x_{i_1}^{n_1}x_{i_2}^{n_2}\dotsb x_{i_r}^{n_r} \]
 for some indices $i_1,\ldots,i_r$ and integers $n_1,\ldots,n_r$.  

 Equivalently, the $x_i$ generate $G$ iff the only subgroup of $G$
 containing all the $x_i$ is $G$ itself.
\end{definition}

We will be interested in finding small sets of generators for some of
the wallpaper groups. 

% The following result will be helpful.
% \begin{proposition}
%  Let $H$ be a subgroup of $\Isom_n$, and let
%  $t_1,\ldots,t_m,h_1,\ldots h_n$ be elements of $H$ such that the
%  $t_i$ are translations and the $h_j$ are not translations.  Suppose
%  that $t_1,\ldots,t_n$ generate $\Trans(H)$ and
%  $\psi(h_1),\ldots,\psi(h_m)$ generate $\psi(H)$.  
% \end{proposition}

\subsection{The group p4g}

Let $M$ be the figure shown on the left below, and let $M'$ be its
mirror image, as shown on the right.
\begin{center}
 \begin{tikzpicture}
  \draw[<->] (-1.3,-1) -- (-1.3,1);
  \draw (-1.4,0) node{$\frac{1}{3}$};
  \draw (-1,-1) -- (1,1) -- (0,1) -- (0,-1) -- cycle;
  \draw (-1,1) -- (1,-1) -- (1,0) -- (-1,0) -- cycle;
  \begin{scope}[xshift=4cm]
   \draw (-1,-1) -- (1,1) -- (1,0) -- (-1,0) -- cycle;
   \draw (-1,1) -- (1,-1) -- (0,-1) -- (0,1) -- cycle;
  \end{scope}
 \end{tikzpicture}
\end{center}

It is easy to see that $\Symm(M)=\Dir(M)=C_4$.

Now let $X$ be the figure shown below, consisting of a copy of $M$
centred at each point of the form $(n,m)$ with $n,m\in\Z$, together
with a copy of $M'$ centred at each point of the form
$(n+\half,m+\half)$.  We will study the group
$\Isom(X):=\{f\in \Isom_2\st f(X)=X\}$, which is known in chemistry as p4g.

\newsavebox{\windR}
\savebox{\windR}{
 \begin{picture}(10,10)(6,0)
  \setlength{\unitlength}{1.6pt}
  \put( -5, -5){\line( 1, 1){10}}
  \put( -5,  0){\line( 1, 0){10}}
  \put( -5,  5){\line( 1,-1){10}}
  \put(  0, -5){\line( 0, 1){10}}
  \put( -5, -5){\line( 1, 0){ 5}}
  \put(  5, -5){\line( 0, 1){ 5}}
  \put(  5,  5){\line(-1, 0){ 5}}
  \put( -5,  5){\line( 0,-1){ 5}}
 \end{picture}
}

\newsavebox{\windL}
\savebox{\windL}{
 \begin{picture}(10,10)(6,0)
  \setlength{\unitlength}{1.6pt}
  \put( -5, -5){\line( 1, 1){10}}
  \put( -5,  0){\line( 1, 0){10}}
  \put( -5,  5){\line( 1,-1){10}}
  \put(  0, -5){\line( 0, 1){10}}
  \put( -5, -5){\line( 0, 1){ 5}}
  \put(  5, -5){\line(-1, 0){ 5}}
  \put(  5,  5){\line( 0,-1){ 5}}
  \put( -5,  5){\line( 1, 0){ 5}}
 \end{picture}
}

\[ \begin{picture}(250,220)(-10,-20)
 \setlength{\unitlength}{1.6pt}

 \multiput( 5, 5)(32, 0){4}{
  \multiput( 0, 0)( 0,32){4}{
   \usebox{\windR}}}

 \multiput(21,21)(32, 0){4}{
  \multiput( 0, 0)( 0,32){4}{
   \usebox{\windL}}}

 % axes
 \put(-10, 37){\vector( 1, 0){140}}
 \put( 37,-10){\vector( 0, 1){140}}

 % reflection lines
 \put(-10, 45){\line( 1, 0){140}}
 \put(135, 45){\makebox(0,0){$L$}}

 \put(-10,100){\line( 1,-1){110}}
 \put(106, -8){\makebox(0,0){$K$}}
\end{picture} \]

Let $K$ and $L$ be the lines marked in the diagram, so $K$ has
equation $x+y=\half$ and $L$ has equation $y=\quarter$.  We define
isometries as follows:
\[ \begin{array}{rlll}
 T_1 &= T_{(1,0)} &\hspace{2em}& (x,y)\mapsto(x+1,y) \\
 T_2 &= T_{(0,1)}      && (x,y)\mapsto(x,y+1) \\
 G   &= T_{(1/2,0)}S_L && (x,y)\mapsto(\half+x,\half-y) \\
 R   &= R_{\pi/2}      && (x,y)\mapsto(-y,x) \\
 S   &= S_K            && (x,y)\mapsto(\half-y,\half-x).
\end{array} \]
I claim that $X$ is invariant under all these isometries; this is
clear by inspection.  For example, if we reflect the pattern across
$L$ and then shift half a unit to the right, we get the original
pattern back, which shows that $GX=X$.

\begin{proposition}
 $\Isom(X)$ is generated by $T_1$, $T_2$, $G$ and $R$.
\end{proposition}
\begin{proof}
 Let $H$ be the group generated by $T_1$, $T_2$, $G$ and $R$.  As
 these isometries preserve $X$, we have $H\leq \Isom(X)$.  Now let $f_0$
 be an arbitrary element of $\Isom(X)$.  If $\det(f_0)=-1$ we define
 $f_1=G^{-1}f_0$, otherwise we put $f_1=f_0$.  Either way we have
 $f_1\in \Isom(X)$ and $\det(f_1)=1$.  It is geometrically obvious that
 $f_1$ must send the copy of $M$ centred at $(0,0)$ to some other copy
 of $M$, and thus that $f_1(0,0)=(n,m)$ for some $n,m\in\Z$.  Now put
 $f_2=T_1^{-n}T_2^{-m}f_1$, so that $f_2\in \Isom(X)$ and $\det(f_2)=1$
 and $f_2(0,0)=(0,0)$.  This implies that $f_2\in SO_2$, in other
 words $f_2$ is a rotation, and clearly the angle must be a multiple
 of $\pi/2$, so $f_2=R^k$ for some $k$.  Thus $f_0$ has the form
 $T_2^m T_1^n R^k$ or $GT_2^m T_1^n R^k$, which means that $f_0\in H$.
 Thus $\Isom(X)\sse H$ as required.
\end{proof}

\begin{corollary}
 $\Isom(X)$ is generated by $R$ and $S$.
\end{corollary}
\begin{proof}
 It will suffice to write the generators $T_1$, $T_2$ and $G$ in terms
 of $R$ and $S$.  The relevant formulae are as follows:
 \begin{align*}
  G   &= SR^{-1} \\
  T_1 &= G^2 = SR^{-1}SR^{-1} \\
  T_2 &= RT_1R^{-1} = RSR^{-1}SR^{-2}
 \end{align*}
 These facts can be proved from the formulae given above in terms of
 $x$ and $y$, or by geometric arguments.
\end{proof}

\begin{remark}
 We have $\psi(T_1)=\psi(T_2)=I$ and $\psi(R)=R$ and $\psi(G)=S_0$.
 The group $\Isom(X)$ is generated by $T_1$, $T_2$, $R$ and $G$, so
 $\psi(\Isom(X))$ is generated by $R$ and $S_0$, so $\psi(\Isom(X))=D_4$.  On
 the other hand, one checks that for each $a\in\R^2$, the group
 $\sg_a(\Isom(X))$ is either $C_1$, $C_2$ or $C_4$.  In particular, there
 is no point $a$ for which $\sg_a(\Isom(X))=\psi(\Isom(X))$.
\end{remark}

\subsection{The group p4m}
\label{subsec-p-four-m}

Let $C_{n,m}$ denote the circle of radius $1/3$ centred at $(n,m)$,
and let $X$ denote the union of all the circle $C_{n,m}$ for
$(n,m)\in\Z^2$.
\[ \begin{picture}(250,200)(-10,-15)
 \setlength{\unitlength}{1.6pt}

 % axes
 \put(-10, 37){\vector( 1, 0){140}}
 \put( 39,-10){\vector( 0, 1){125}}

 \multiput(  0,  0)( 0,32){4}{
  \multiput(  5,  5)(32, 0){4}{
   {\oval(13,13)}}}
\end{picture} \]
The symmetry type of this pattern is known as p4m.

Let $T_1$, $T_2$ and $R$ be as before, and let $S_0$ denote reflection
in the $x$-axis (as usual).  These are easily seen to preserve $X$,
and I claim that they generate $\Isom(X)$.

To see this, suppose that $f_0\in \Isom(X)$.  Then $f_0$ must send
$C_{0,0}$ to one of the other circles in the pattern, say
$f(C_{0,0})=C_{n,m}$.  Put $f_1=T_1^{-n}T_2^{-m}f_0$, so that
$f_1\in \Isom(X)$ and $f_1(C_{0,0})=C_{0,0}$.  This means that $f_1$ is
either a rotation around $(0,0)$ or a reflection across a line through
$(0,0)$.  If it is a rotation, the angle must clearly be a multiple of
$\pi/2$, and thus $f_1=R^k$ for some $k$.  If it is a reflection, one
sees by inspection that the slope of the line must be $k\pi/4$ for
some $k$, and thus that $f_1=S_{k\pi/2}=R^kS_0$.  Thus
$f_0=T_1^nT_2^mR^k$ or $f_1=T_1^nT_2^mR^kS_0$, as required.

In this case, the point group is generated by $\psi(T_1)$,
$\psi(T_2)$, $\psi(R)$ and $\psi(S_0)$.  We have
$\psi(T_1)=\psi(T_2)=1$ and $\psi(R)=R$ and $\psi(S_0)=S_0$, so the
point group is generated by $R$ and $S_0$ and thus is equal to $D_4$
again.  In contrast to the p4g case, we have
$\sg_0(\Isom(X))=D_4=\psi(\Isom(X))$.

\subsection{The group p6m}
\label{subsec-p-six-m}

Put $u=(1,0)$ and $v=R_{\pi/6}(u)=(1/2,\sqrt{3}/2)$, so that $0$, $u$
and $v$ are the vertices of an equilateral triangle of side $1$.  Let
$C_{n,m}$ be a circle of radius $1/3$ centred at $nu+mv$, and let $X$
be the union of the circles $C_{n,m}$

\[ \begin{picture}(250,200)(-10,-15)
 \setlength{\unitlength}{1.6pt}

 % axes
 \put(-10, 37){\vector( 1, 0){140}}
 \put( 39,-10){\vector( 0, 1){125}}

 \multiput(  0,  0)( 0,56){2}{
  \multiput( 21,  9)(32, 0){4}{
   {\oval(13,13)}}}

 \multiput(  0,  0)( 0,56){2}{
  \multiput(  5, 37)(32, 0){4}{
   {\oval(13,13)}}}
\end{picture} \]

This can be analysed in much the same way as the previous example.  We
find that $\Isom(X)$ is generated by $T_u$, $T_v$, $R_{\pi/3}$ and $S_0$.
The point group is $D_6$, which is the same as $\sg_0\Isom(X)$.

\subsection{Steps towards the classification}

We will adopt the following definition.
\begin{definition}\label{defn-wallpaper}
 A \emph{wallpaper group} or \emph{two-dimensional crystallographic
 group} is a subgroup $H\leq \Isom_2$ such that
 \begin{itemize}
  \item[(a)] $\psi(H)$ is finite.
  \item[(b)] There exist linearly independent vectors $u,v\in \Trans(H)$
   such that every vector in $\Trans(H)$ can be written as $nu+mv$ for some
   $n,m\in\Z$.
 \end{itemize}
\end{definition}
It is usual to use a somewhat different definition, which can be shown
to be equivalent to that given above.

Let $H$ be a wallpaper group.  We say that $H$ is \emph{oriented} if
$\psi(H)\leq SO_2$; if so, we know from
Theorem~\ref{thm-O-two-subgroups} that $\psi(H)=C_n$ for some $n$.  We
call $n$ the \emph{rotational order} of $H$.

Now suppose that $H$ is not oriented, so
$\psi(H)=R_\theta D_n R_\theta^{-1}$ for some $n$ and $\theta$.  We again
call $n$ the rotational order of $H$.

\begin{lemma}\label{lem-psi-T}
 If $A\in\psi(H)$ and $b\in \Trans(H)\subset\R^2$ then $Ab\in \Trans(H)$.
\end{lemma}
\begin{proof}
 As $b\in \Trans(H)$ we have $T_b\in H$.  As $A\in\psi(H)$, there is an
 element $f\in H$ of the form $f(x)=Ax+c$ for some $c$.  It follows
 that $fT_bf^{-1}\in H$, and we see from
 Proposition~\ref{prop-semidirect} that $fT_bf^{-1}=T_{Ab}$, so
 $Ab\in \Trans(H)$.
\end{proof}

To explain what the next lemma is about, consider the group
$V\leq\R^2$ consisting of vectors of the form $n(-1,0)+m(\sqrt{2},0)$
with $n,m\in\Z$.  We can choose a rational number $n/m$ which is a
very good (but not perfect) rational approximation to $\sqrt{2}$, and
we find that $n(-1,0)+m(\sqrt{2},0)$ is very small (but nonzero).  By
making this precise, we find that for any $\ep>0$ there exists
$v\in V\sm\{0\}$ such that $\|v\|<\ep$.  Thus, there is no shortest
vector in $V\sm\{0\}$.  This phenomenon can only happen because
$(-1,0)$ and $(\sqrt{2},0)$ are linearly dependent vectors; in
particular, it does not occur in $\Trans(H)$.  The point of the next lemma
is to prove this.
\begin{lemma}\label{lem-discrete}
 If $H$ is a wallpaper group then there exists $w\in \Trans(H)\sm\{0\}$
 such that $\|b\|\geq\|w\|$ for all $b\in \Trans(H)\sm\{0\}$.
\end{lemma}
\begin{proof}
 Let $u$ and $v$ be as in Definition~\ref{defn-wallpaper}.  We claim
 that there is a positive constant $K>0$ such that 
 \[ \|nu+mv\| \geq \sqrt{n^2+m^2}/K. \]
 To see this, define
 $f\:[0,2\pi]\xra{}\R$ by $f(\theta)=\|\cos(\theta)u+\sin(\theta)v\|$.  As
 $u$ and $v$ are linearly independent we have
 $\cos(\theta)u+\sin(\theta)v\neq 0$ and thus $f(\theta)>0$ for all $\theta$.
 It follows that $1/f$ is a positive continuous function on the closed
 interval $[0,2\pi]$, so $1/f$ is bounded by some number $K>0$, so
 $f(\tht)\geq 1/K$ for all $\tht$.  Now, for any $n$ and $m$ we can
 write $(n,m)=r(\cos(\tht),\sin(\tht))$ for some $\tht$, where
 $r=\sqrt{n^2+m^2}$.  This means that
 \begin{align*}
  \|nu+mv\| &= \|r (\cos(\tht)u + \sin(\tht)v)\| \\
            &= r f(\tht) \geq r/K = \sqrt{n^2+m^2}/K,
 \end{align*}
 as claimed.
 
 Now consider a disc $D$ of radius $R$ centred at the origin, and put
 $S=(\Trans(H)\sm\{0\})\cap D$, the set of nonzero vectors in
 $\Trans(H)$ of length at most $R$.  We choose $R$ large enough that
 $D$ contains at least one of the nonzero points in $\Trans(H)$, so
 $S\neq\emptyset$.  If $nu+mv\in S$ then
 $R\geq\|nu+mv\|\geq\sqrt{n^2+m^2}/K$, so $|n|,|m|\leq RK$.  This
 means that there are only finitely many possibilities for $n$ and
 $m$, so there are only finitely many points in $S$.  Among this
 finite list of points, we choose one that is as close as possible to
 zero, and call it $w$.  This clearly has the required property.
\end{proof}

\begin{theorem}
 The rotational order of $H$ is $1$, $2$, $3$, $4$ or $6$.
\end{theorem}
\begin{proof}
 Let $n$ be the rotational order, so the element $R:=R_{2\pi/n}$ lies
 in $\psi(H)$.  Let $w\in \Trans(H)$ be as in Lemma~\ref{lem-discrete}.
 Lemma~\ref{lem-psi-T} tells us that $R(w)\in \Trans(H)$ and
 $\Trans(H)$ is a subgroup of $\R^2$ so $R(w)-w\in \Trans(H)$, so
 $\|R(w)-w\|\geq\|w\|$ by the definition of $w$.  However, for any $x$
 and $\theta$ we have $\|R_\theta(x)-x\|=2\sin(\theta/2)\|x\|$, as we see
 from the diagram below.
 \begin{center}
  \begin{tikzpicture}[scale=2] % [60]{-0.2}{2.2}{-0.2}{2.2}
   \fill (0,0) \blob;
   \fill (1.73,1) \blob;
   \fill (0,2) \blob;
   \draw (0,0) -- (1.73,1) -- (0,2) -- (0,0);
   \draw (0,0) -- (1,1.73);
   \draw (0,0) +(30:2) arc(30:90:2);
   \draw (0,0) +(30:0.4) arc(30:90:0.4);
   \draw ( 1.78,1   ) node[anchor=west] {$x$};
   \draw ( 0.90,0.45) node[anchor=north west] {$\|x\|$};
   \draw (-0.05,2   ) node[anchor=east] {$R_\theta x$};
   \draw (-0.05,1   ) node[anchor=east] {$\|x\|$};
   \draw ( 0.12,0.6 ) node {$\theta/2$};
   \draw ( 0.50,0.5 ) node {$\theta/2$};
  \end{tikzpicture}
 \end{center}
 It follows that $\|R(w)-w\|=2\sin(\pi/n)\|w\|$, so we must have
 $2\sin(\pi/n)\geq 1$, so $\sin(\pi/n)\leq 1/2=\sin(\pi/6)$, so
 $n\leq 6$.

 All that is left is to show that the case $n=5$ leads to a
 contradiction, which we do by a variation of the preceeding argument.
 Clearly $w+R^{-2}w\in \Trans(H)$, but if $n=5$ then
 $-R^{-2}w=R_\pi R_{-4\pi/5}=R_{\pi/5}w$ so
 $\|w+R^{-2}w\|=\|w-R_{\pi/5}w\|=2\sin(\pi/10)\|w\|<\|w\|$, which
 contradicts our choice of $w$, as required.
\end{proof}

\begin{proposition}\label{prop-lattice}
 Suppose that $H$ has rotational order $n$, where $n\in\{3,4,6\}$.
 Let $w$ be as in Lemma~\ref{lem-discrete}, and put
 $x=R_{2\pi/n}(w)$.  Then $\Trans(H)=\{pw+qx\st p,q\in\Z\}$.
\end{proposition}
\begin{proof}
 Put $L=\{pw+qx\st p,q\in\Z\}\leq \Trans(H)$ and $r=\|w\|$.  I claim that
 for each $a\in\R^2$, there exists $b\in L$ such that $d(a,b)<r$.
 Assuming this, when $a\in \Trans(H)$ we have $a-b\in \Trans(H)$ and $\|a-b\|<r$
 so $a-b=0$ by our choice of $w$, so $a=b$; this proves that $\Trans(H)=L$
 as required.

 To prove the claim, we first consider the case $n=4$, where $w$ and
 $x$ are orthogonal.  After a suitable change of coordinates we have
 $w=(r,0)$ and $x=(0,r)$, and the claim is that every point in $\R^2$
 lies in the open ball of radius $r$ centred at $(pr,qr)$ for some
 $p,q\in\Z$.  This should be geometrically clear from the following
 diagram.
 \begin{center}
  \begin{tikzpicture} % [20]{-1}{4}{-1}{4}
   \draw (-1,-1) +(0:1) arc(0:90:1);
   \draw (0,-1) +(0:1) arc(0:180:1);
   \draw (1,-1) +(0:1) arc(0:180:1);
   \draw (2,-1) +(0:1) arc(0:180:1);
   \draw (3,-1) +(0:1) arc(0:180:1);
   \draw (4,-1) +(90:1) arc(90:180:1);

   \draw (-1,0) +(-90:1) arc(-90:90:1);
   \draw (0,0) circle(1);
   \draw (1,0) circle(1);
   \draw (2,0) circle(1);
   \draw (3,0) circle(1);
   \draw (4,0) +(90:1) arc(90:270:1);

   \draw (-1,1) +(-90:1) arc(-90:90:1);
   \draw (0,1) circle(1);
   \draw (1,1) circle(1);
   \draw (2,1) circle(1);
   \draw (3,1) circle(1);
   \draw (4,1) +(90:1) arc(90:270:1);

   \draw (-1,2) +(-90:1) arc(-90:90:1);
   \draw (0,2) circle(1);
   \draw (1,2) circle(1);
   \draw (2,2) circle(1);
   \draw (3,2) circle(1);
   \draw (4,2) +(90:1) arc(90:270:1);

   \draw (-1,3) +(-90:1) arc(-90:90:1);
   \draw (0,3) circle(1);
   \draw (1,3) circle(1);
   \draw (2,3) circle(1);
   \draw (3,3) circle(1);
   \draw (4,3) +(90:1) arc(90:270:1);

   \draw (-1,4) +(-90:1) arc(-90:0:1);
   \draw (0,4) +(-180:1) arc(-180:0:1);
   \draw (1,4) +(-180:1) arc(-180:0:1);
   \draw (2,4) +(-180:1) arc(-180:0:1);
   \draw (3,4) +(-180:1) arc(-180:0:1);
   \draw (4,4) +(-180:1) arc(-180:-90:1);

   \fill (-1,-1) \blob;
   \fill (0,-1) \blob;
   \fill (1,-1) \blob;
   \fill (2,-1) \blob;
   \fill (3,-1) \blob;
   \fill (4,-1) \blob;
   \fill (-1,0) \blob;
   \fill (0,0) \blob;
   \fill (1,0) \blob;
   \fill (2,0) \blob;
   \fill (3,0) \blob;
   \fill (4,0) \blob;
   \fill (-1,1) \blob;
   \fill (0,1) \blob;
   \fill (1,1) \blob;
   \fill (2,1) \blob;
   \fill (3,1) \blob;
   \fill (4,1) \blob;
   \fill (-1,2) \blob;
   \fill (0,2) \blob;
   \fill (1,2) \blob;
   \fill (2,2) \blob;
   \fill (3,2) \blob;
   \fill (4,2) \blob;
   \fill (-1,3) \blob;
   \fill (0,3) \blob;
   \fill (1,3) \blob;
   \fill (2,3) \blob;
   \fill (3,3) \blob;
   \fill (4,3) \blob;
   \fill (-1,4) \blob;
   \fill (0,4) \blob;
   \fill (1,4) \blob;
   \fill (2,4) \blob;
   \fill (3,4) \blob;
   \fill (4,4) \blob;
  \end{tikzpicture} 
 \end{center}
 For an algebraic proof, note that $\{w,x\}$ is a basis for $\R^2$, so
 any vector $a\in\R^2$ can certainly be written in the form $pw+qx$
 for some $p,q\in\R$.  We can choose $p',q'\in\Z$ with
 $|p-p'|\leq\half$ and $|q-q'|\leq\half$, and then put
 $b=p'w+q'x\in L$.  We then have $a-b=(p-p')w+(q-q')x$ and $w$ and $x$
 are orthogonal so
 \[ \|a-b\|^2=(p-p')^2r^2+(q-q')^2r^2\leq(\quarter+\quarter)r^2<r^2,
 \]
 so $\|a-b\|<r$ as required.

 We next turn to the case $n=6$.  It should be clear from the way the
 previous case worked that the value of $r$ is irrelevant, so we
 assume that $r=1$.  We may also change coordinates and assume that
 $w=(1,0)$, so $x=R_{2\pi/6}w=(1/2,\sqrt{3}/2)$.  The lattice $L$
 consists of the dots in the following diagram:
 \begin{center}
  \begin{tikzpicture} % [20]{-1}{4}{-0.9}{2.7}
   \draw (-1,-0.86) -- (4,-0.86);
   \draw (-1, 0   ) -- (4, 0   );
   \draw (-1, 0.86) -- (4, 0.86);
   \draw (-1, 1.73) -- (4, 1.73);
   \draw (-1, 2.60) -- (4, 2.60);
   \draw (-1,1.73) -- (-0.5,2.6) -- (1.5,-0.86) -- (3.5,2.6) -- (4,1.73) --
         (2.5,-0.86) -- (0.5,2.6) -- (-1,0) -- (-0.5,-0.86) -- (1.5,2.6) --
         (3.5,-0.86) -- (4,0) -- (2.5,2.6) -- (0.5,-0.86) -- (-1,1.73);
   \fill (-0.5,-0.86) \blob;
   \fill (0.5,-0.86) \blob;
   \fill (1.5,-0.86) \blob;
   \fill (2.5,-0.86) \blob;
   \fill (3.5,-0.86) \blob;
   \fill (-0.5,0.86) \blob;
   \fill (0.5,0.86) \blob;
   \fill (1.5,0.86) \blob;
   \fill (2.5,0.86) \blob;
   \fill (3.5,0.86) \blob;
   \fill (-0.5,2.6) \blob;
   \fill (0.5,2.6) \blob;
   \fill (1.5,2.6) \blob;
   \fill (2.5,2.6) \blob;
   \fill (3.5,2.6) \blob;
   \fill (-1,0) \blob;
   \fill (0,0) \blob;
   \fill (1,0) \blob;
   \fill (2,0) \blob;
   \fill (3,0) \blob;
   \fill (4,0) \blob;
   \fill (-1,1.73) \blob;
   \fill (0,1.73) \blob;
   \fill (1,1.73) \blob;
   \fill (2,1.73) \blob;
   \fill (3,1.73) \blob;
   \fill (4,1.73) \blob;
  \end{tikzpicture} 
 \end{center}
 Each of the triangles is equilateral with side $1$, and every point
 in such a triangle lies at distance $<1$ from at least one of the
 vertices.  (In fact, if $T$ is an equilateral triangle of side $1$
 with vertices $A$, $B$ and $C$ and $X\in T$ then the distances
 $d(A,X)$, $d(B,X)$ and $d(C,X)$ are \emph{all} less than one unless
 $X$ is itself a vertex; in the exceptional case, of course $X$ lies
 at distance $0$ from one of the vertices.)  This settles the case
 $n=6$.

 Finally, we treat the case $n=3$.  With assumptions as in the case
 $n=6$, we have $w=(1,0)$ and $x=R_{2\pi/3}(w)=(-1/2,\sqrt{3}/2)$.  Put
 $y=R_{2\pi/6}(w)=(1/2,\sqrt{3}/2)$ and notice that $y=x+w$ and
 $x=y-w$.  This shows that every integer combination of $w$ and $x$ is
 an integer combination of $w$ and $y$, and \emph{vice versa}.  This
 means that the lattice for the $n=3$ case is exactly the same as for
 the $n=6$ case, so again every point in $\R^2$ is at distance $<1$
 from a lattice point.
\end{proof}

We now see that $\psi(H)$ is conjugate to $C_n$ or $D_n$ where
$n\in\{1,2,3,4,6\}$, which gives twelve possibilities for $\psi(H)$.
In the cases $n\geq 3$ we have a strong information about $\Trans(H)$.
Even if we know $\psi(H)$ and $\Trans(H)$ there may be more than one
possibility for $H$, as exemplified by the difference between p4g and
p4m.  Nonetheless, we are well on the way to the complete
classification of wallpaper groups.

% Lecture 9

\section{Polyhedra}

We now turn to the study of symmetries in three dimensions.  In this
context we will not consider translations, so we are really just
looking at subgroups of $O_3$.  It will turn out that this is strongly
related to the theory of regular polyhedra, otherwise known as
Platonic solids.

\subsection{Actions of groups on sets}

In our study of subgroups of $O_3$ (and in later sections of the
course) it will be helpful to think about actions of groups on sets.
\begin{definition}
 Let $G$ be a group and $X$ a set.  An \emph{action} of $G$ on $X$ is
 a rule which assigns to each element $g\in G$ and each element
 $x\in X$ an element $g*x\in X$, such that
 \begin{itemize}
 \item[A1] $1*x=$ for all $x\in X$
 \item[A2] $g*(h*x)=(gh)*x$ for all $g,h\in G$ and $x\in X$.
 \end{itemize}
 We will often write $gx$ for $g*x$.
\end{definition}

\begin{example}\label{eg-O-three-action}
 Put $X=S^2=\{x\in\R^3\st \|x\|=1\}$.  If $x\in S^2$ and $A\in O_3$
 then $Ax\in S^2$.  Clearly $Ix=x$ and $(AB)x=A(Bx)$, so this gives an
 action of $O_3$ on $S^2$.
\end{example}
\begin{example}\label{eg-D-four-action}
 Consider the group $G=D_4=\{1,R,R^2,R^3,S,RS,R^2S,R^3S\}$, where
 $R=R_{\pi/2}$ and $S=S_0$.  Let $L_0$ be the line with equation
 $x=y$, and let $L_1$ be the line with equation $x=-y$.  One checks
 that $S(L_0)=L_1$ and $S(L_1)=L_0$, and similarly $R(L_0)=L_1$ and
 $R(L_1)=L_0$.  It follows that for each $g\in D_4$ we either have
 $g(L_0)=L_0$ or $g(L_0)=L_1$, and similarly we either have
 $g(L_1)=L_1$ or $g(L_1)=L_0$.  Thus, if we put $X=\{L_0,L_1\}$ then
 $D_4$ acts on $X$.
\end{example}
\begin{example}\label{eg-conj-action}
 Let $G$ be any group.  For any $g,x\in G$ we define $g*x=gxg^{-1}$.
 This satisfies $1*x=x$ and
 \[ g*(h*x)=g*(hxh^{-1})=ghxh^{-1}g^{-1}=ghx(gh)^{-1}=(gh)*x. \]
 Thus, we have an action of $G$ on itself, called the
 \emph{conjugation action}.  In this case it would of course be a
 mistake to write $gx$ instead of $g*x$.
\end{example}

We next introduce a different way of thinking about group actions.
\begin{definition}\label{defn-SX}
 A \emph{permutation} of a set $X$ is a bijective function
 $\sg\:X\xra{}X$.  We write $S(X)$ for the group of all permutations
 of $X$, and we write $S_n=S(\{1,\ldots,n\})$.
\end{definition}

Suppose we have an action of $G$ on $X$.  For any $g\in G$, we can
define a function $\phi(g)\:X\xra{}X$ by $\phi(g)(x)=g*x$.  This
satisfies $\phi(1)(x)=1*x=x$, so $\phi(1)$ is the identity map.
Moreover, we have
\[ (\phi(g)\circ\phi(h))(x) = \phi(g)(\phi(h)(x)) =
    g*(h*x)= (gh)*x = \phi(gh)(x),
\]
so $\phi(gh)=\phi(g)\circ\phi(h)$.  In particular, we have
$\phi(g)\phi(g^{-1})=\phi(1)=1$, and similarly
$\phi(g^{-1})\phi(g)=1$.  Thus $\phi(g)$ is a bijection, with inverse
$\phi(g^{-1})$.  We have thus defined a homomorphism
$\phi\:G\xra{}S(X)$.  Conversely, if we start with a homomorphism
$\phi\:G\xra{}S(X)$ we can define an action by $g*x=\phi(g)(x)$.
Thus, actions of $G$ on $X$ are essentially the same as homomorphisms
from $G$ to $S(X)$.

\begin{example}
 Let $V=\{v_0,v_1,v_2,v_3,v_4\}$ be the set of vertices of the
 standard pentagon, so the group $D_5$ acts on $V$, giving a
 homomorphism $\phi\:D_5\xra{}S(V)$.  If we write $R=R_{2\pi/5}$ and
 $S=S_0$ as usual then
 \begin{align*}
  \phi(R)(v_0) &= v_1 \\
  \phi(R)(v_1) &= v_2 \\
  \phi(R)(v_2) &= v_3 \\
  \phi(R)(v_3) &= v_4 \\
  \phi(R)(v_4) &= v_0.
 \end{align*}
 We can write this in cycle notation as
 $\phi(R)=(v_0\;v_1\;v_2\;v_3\;v_4)$.  If we identify $V$ with
 $\{0,1,2,3,4\}$ in the obvious way then $\phi(R)$ becomes the
 permutation $(0\;1\;2\;3\;4)$.  Similarly, we have
 $\phi(S)=(1\;4)(2\;3)$.
 \begin{center}
  \begin{tikzpicture}[scale=2.4] % [80]{-1.5}{1.5}{-1.2}{1.3}
   \fill ( 1.00, 0.00) \blob;
   \fill ( 0.31, 0.95) \blob;
   \fill (-0.81, 0.59) \blob;
   \fill (-0.81,-0.59) \blob;
   \fill ( 0.31,-0.95) \blob;
   \fill (0,0) \blob;
   \draw (1,0) -- (0.31,0.95) -- (-0.81,0.59) --
         (-0.81,-0.59) -- (0.31,-0.95) -- (1,0);
   \draw (0,0) -- (0.31,0.95);
   \draw (-1.1,0) -- (1.3,0);
   \draw[->] (1.3,0) -- (1.3,0.2);
   \draw[->] (1.3,0) -- (1.3,-0.2);
   \draw ( 0.00, 0.00) +(0:0.2) arc(0:72:0.2);
   \draw ( 0.40, 0.30) node {$2\pi/5$};
   \draw ( 1.04, 0.10) node[anchor=west]       {$v_0$};
   \draw ( 0.31, 1.00) node[anchor=south]      {$v_1$};
   \draw (-0.86, 0.59) node[anchor=south east] {$v_2$};
   \draw (-0.86,-0.59) node[anchor=north east] {$v_3$};
   \draw ( 0.31,-1.00) node[anchor=north]      {$v_4$};
   \draw ( 1.35, 0.10) node[anchor=west] {$S_0$};
  \end{tikzpicture} 
 \end{center}
\end{example}

\subsection{Rotations and axes}
\label{subsec-axes}

We have already seen a very simple and concrete description of the
elements of $SO_2$; they are just the rotations $R_\theta$ for
$0\leq\theta\leq 2\pi$.  Our next task is to see how far this
generalises to $SO_3$, or to $SO_n$ for $n>3$.

\begin{proposition}
 If $A\in SO_n$ and $n$ is odd then $1$ is an eigenvalue of $A$.
\end{proposition}
\begin{proof}
 We have $A^T=A^{-1}$, so
 \[ A^T(A-I) = I-A^T = -(A-I)^T. \]
 For any $n\tm n$ matrix $B$ we have $\det(B^T)=\det(B)$ and
 $\det(-B)=(-1)^n\det(B)=-\det(B)$ (as $n$ is odd).  We can thus take
 determinants in the displayed equation to get
 \[ \det(A)\det(A-I)=-\det(A-I). \]
 As $A\in SO_n$ we have $\det(A)=1$ so $\det(A-I)=-\det(A-I)$, so
 $\det(A-I)=0$ as required.
\end{proof}

\begin{corollary}\label{cor-gauss}
 If $A\in SO_3$ then there is an orthonormal basis $\{u,v,w\}$ of
 $\R^3$ and an angle $\theta$ such that
 \begin{align*}
  Au &= u \\
  Av &=  \cos(\theta)v + \sin(\theta)w \\
  Aw &= -\sin(\theta)v + \cos(\theta)w.
 \end{align*}
 Thus, $A$ is conjugate in $O_3$ to a matrix of the form
 \[ U_\theta = \left(\begin{array}{ccc}
     1 & 0 & 0 \\
     0 & \cos(\theta) & -\sin(\theta) \\
     0 & \sin(\theta) & \cos(\theta)
 \end{array} \right). \]
\end{corollary}
\begin{proof}
 As $1$ is an eigenvalue, there is a vector $u'\neq 0$ such that
 $Au'=u'$.  Put $u=u'/\|u'\|$, so $\|u\|=1$ and $Au=u$.  Let $v$ be
 any unit vector perpendicular to $u$, and let $w$ be either of the
 two unit vectors that are perpendicular to the plane spanned by $u$
 and $v$.  As $Au=u$ and $A$ preserves inner products, we have
 $\ip{Av,u}=\ip{Av,Au}=\ip{v,u}=0$, so $Av$ is perpendicular to $u$.
 It is clear that $v$ and $w$ form a basis for the plane perpendicular
 to $u$, so $Av=cv+sw$ for some $c,s\in\R$.  Moreover, we have
 \[ 1 = \|v\|^2 = \|Av\|^2 = \ip{cv+sw,cv+sw}=c^2+s^2, \]
 so we have $(c,s)=(\cos(\theta),\sin(\theta))$ for some $\theta$.
 Similarly, we have $Aw=c'v+s'w$ for some $c',s'$ with
 $(c')^2+(s')^2=1$.  As $\ip{v,w}=0$ we have $\ip{Av,Aw}=0$ and thus
 $cc'+ss'=0$.  Thus $(c',s')$ is a unit vector in $\R^2$ which is
 orthogonal to $(c,s)$; one sees easily that the only possibilities
 are $(c',s')=(-s,c)$ and $(c',s')=(s,-c)$.  For the moment we simply
 assume that $(c',s')=(-s,c)$; we will explain later why the other
 case is impossible.  Define $\bt\:\R^3\xra{}\R^3$ by
 $\bt(x,y,z)=xu+yv+zw$.  As $u$, $v$ and $w$ are orthonormal we see
 that
 \[ \|\bt(x,y,z)\|^2=\ip{xu+yv+zw,xu+yv+zw}=
     x^2+y^2+z^2=\|(x,y,z)\|^2.
 \]
 Thus $\bt$ is a norm-preserving linear map, so the corresponding
 matrix $B$ is orthogonal.  We have
 \begin{align*}
  AB(x,y,z) &= A(xu+yv+zw) \\
            &= xAu+yAv+zAw \\
            &= xu + y(cv+sw) + z (-sv+cw) \\
            &= xu + (cy-sz)v + (sy+cz)w \\
            &= BU_\theta(x,y,z),
 \end{align*}
 so $B^{-1}AB=U_\theta$.  Thus $A$ is conjugate to $U_\theta$ in $O_3$, as
 claimed.

 Now suppose instead that $(c',s')=(s,-c)$.  Then we would have
 $B^{-1}AB=U'_\theta$, where $U'_\theta$ is obtained from $U_\theta$ by
 multiplying the last column by $-1$.  However, we have $A\in SO_3$ by
 assumption, so $\det(B^{-1}AB)=\det(B)^{-1}\det(A)\det(B)=1$.  We see
 by direct calculation that $\det(U'_\theta)=-1$, and this gives a
 contradiction.  Thus we must have $(c',s')=(-s,c)$ after all.
\end{proof}

\begin{proposition}\label{prop-rot-fix}
 Suppose that $A\in SO_3$ and that there are two linearly independent
 vectors $u$ and $v$ such that $Au=u$ and $Av=v$.  Then $A=I$.
\end{proposition}
\begin{proof}
 If $A$ is not the identity, then it must be a nontrivial rotation,
 around an axis $L$ say.  This means that $A$ fixes all the points on
 $L$, and moves all other points.  As $Au=u$ and $Av=v$, we see that
 $u$ and $v$ must both lie on the line $L$.  This is impossible,
 because they are linearly independent. 
\end{proof}

\begin{proposition}\label{prop-product}
 If $G\leq O_3$ and $-1\in G$ and $H=G\cap SO_3$ then
 $G=H\tm\{\pm 1\}$.
\end{proposition}
\begin{proof}
 Define $\mu\:H\tm\{\pm 1\}\xra{}G$ by $\mu(A,t)=tA$.  As
 multiplication by any number commutes with multiplication by any
 matrix, we have
 \[ \mu(A,t)\mu(A',t') = tAt'A'=tt'AA'=\mu(AA',tt'), \]
 so $\mu$ is a homomorphism.  Suppose that $\mu(A,t)=I$; then
 either $A=I$ and $t=1$ or $A=-I$ and $t=-1$, but the second case is
 impossible because $-I\not\in SO_3$.  This shows that
 $\ker(\mu)$ is the trivial group, so $\mu$ is injective.  Next
 consider an element $B\in G$.  If $\det(B)=1$ then $B\in H$ so
 $B=\mu(B,1)$ so $B$ is in the image of $\mu$.  If $\det(B)=-1$ then
 $-B\in G$ (because $B$ and $-1$ both lie in $G$) and $\det(-B)=1$ so
 $-B\in H$.  We also have $B=\mu(-B,-1)$, so we again see that $B$ is
 in the image of $\mu$.  This shows that $\mu$ is surjective as well
 as injective, so it is an isomorphism of groups.
\end{proof}
\begin{corollary}
 In particular, we have $O_3=SO_3\tm\{\pm 1\}$ as groups. \qed
\end{corollary}

\subsection{Symmetries of the tetrahedron}

Let $\Tet$ be a regular tetrahedron centred at the origin, whose edges
have length $1$, and let $v_1,\ldots,v_4$ be the vertices of $\Tet$.
% tetrahedron
\begin{center}
 \begin{tikzpicture} % [30]{-2}{2}{-1.7}{2.3}
  \draw (0,1.71) -- (1.05,-0.351);
  \draw (0,1.71) -- (-1.61,-0.519);
  \draw (0,1.71) -- (0.559,-0.835);
  \draw[dashed] (1.05,-0.351) -- (-1.61,-0.519);
  \draw (1.05,-0.351) -- (0.559,-0.835);
  \draw (-1.61,-0.519) -- (0.559,-0.835);
  \fill (0,1.71) \blob;
  \fill (1.05,-0.351) \blob;
  \fill (-1.61,-0.519) \blob;
  \fill (0.559,-0.835) \blob;
  \draw (0,1.88)       node {$\scriptstyle{1}$};
  \draw (1.15,-0.386)  node {$\scriptstyle{2}$};
  \draw (-1.77,-0.571) node {$\scriptstyle{3}$};
  \draw (0.614,-0.919) node {$\scriptstyle{4}$};
 \end{tikzpicture}
\end{center}
The action of $\Symm(\Tet)$ on the vertices gives rise to a
homomorphism $\phi\:\Symm(\Tet)\xra{}S_4$.  For example, let $g$ be a
$1/3$-twist about the $z$-axis, anticlockwise as seen from above.
Then $g$ fixes $v_1$ and sends $v_2$ to $v_3$, $v_3$ to $v_4$ and
$v_4$ back to $v_2$.  Thus $\phi(g)$ is the $3$-cycle $(2\;3\;4)$.

\begin{theorem}
 The homomorphism $\phi\:\Symm(\Tet)\xra{}S_4$ is an isomorphism, and
 it also gives an isomorphism $\Dir(\Tet)\xra{}A_4$.
\end{theorem}
\begin{proof}
 Given any pair of vertices $v_i,v_j$, let $v_k$ and $v_l$ be the two
 remaining vertices, and let $P$ be the plane through $v_k$,
 $v_l$ and $(v_i+v_j)/2$.  Let $A$ be the reflection across $P$ (if
 $n$ is a unit normal to $P$ then $A$ is given by $Ax=x-2\ip{n,x}n$.)
 We find that $Av_i=v_j$ and $Av_j=v_i$, and that $v_k$ and $v_l$ are
 fixed by $A$.  Thus $\phi(A)$ is the transposition $(i\;j)$.  The
 diagram below illustrates the case $i=3$, $j=4$.
 \begin{center}
  \begin{tikzpicture} % [30]{-2}{2}{-1.7}{2.3}
   \draw (0,1.71) -- (1.05,-0.351);
   \draw (0,1.71) -- (-1.61,-0.519);
   \draw (0,1.71) -- (0.559,-0.835);
   \draw (1.05,-0.351) -- (-1.61,-0.519);
   \draw (1.05,-0.351) -- (0.559,-0.835);
   \draw (-1.61,-0.519) -- (0.559,-0.835);
   \fill (0,1.71) \blob;
   \fill (1.05,-0.351) \blob;
   \fill (-1.61,-0.519) \blob;
   \fill ( 0.56,-0.835) \blob;
   \draw ( 0.00, 1.880) node {$\scriptstyle{1}$};
   \draw ( 1.15,-0.386) node {$\scriptstyle{2}$};
   \draw (-1.77,-0.571) node {$\scriptstyle{3}$};
   \draw ( 0.61,-0.919) node {$\scriptstyle{4}$};
   \draw[pattern=crosshatch] %
    (0,1.71) -- (1.05,-0.351) -- (-0.525,-0.677) -- cycle;
  \end{tikzpicture}
 \end{center}

 The image of $\phi$ is a subgroup of $S_4$ containing all the
 transpositions, and any permutation can be written as a product of
 transpositions, so the image is all of $S_4$, so $\phi$ is surjective.
 If $A\in\ker(\phi)$ then $Av_i=v_i$ for all $i$.  It is easy to see
 that $\{v_1,v_2,v_3\}$ is a basis of $\R^3$ so we can conclude that
 $A=I$.  This proves that $\phi$ is injective as well as surjective, so
 it is an isomorphism.  We have seen that $\phi^{-1}$ sends each
 transposition to a reflection, so it sends any product of $n$
 transpositions to a product of $n$ reflections, and we see that
 $\det(\phi^{-1}(\sg))=\sgn(\sg)$ for all $\sg\in S_4$, so $\phi^{-1}$
 carries $A_4$ to $\Dir(\Tet)$.  By putting $\sg=\phi(g)$ we deduce
 that $\sgn(\phi(g))=\det(g)$, so $\phi$ carries $\Dir(\Tet)$ to $A_4$.
 Thus $\phi$ gives an isomorphism $\Dir(\Tet)\simeq A_4$ as claimed.
\end{proof}

\begin{remark}
 If $g$ is a half turn around the axis shown on the left, then
 $\phi(g)$ is the permutation $(1\;2)(3\;4)$.  If $h$ is a one-third
 turn around the axis shown on the right, turning anticlockwise as
 seen from above, then $\phi(h)=(2\;3\;4)$.  Note that this rotation
 looks clockwise when seen from below.
 \begin{center}
  \begin{tikzpicture} % [30]{-2}{2}{-1.7}{2.3}
   \draw (0,1.71) -- (1.05,-0.351);
   \draw (0,1.71) -- (-1.61,-0.519);
   \draw (0,1.71) -- (0.559,-0.835);
   \draw (1.05,-0.351) -- (-1.61,-0.519);
   \draw (1.05,-0.351) -- (0.559,-0.835);
   \draw (-1.61,-0.519) -- (0.559,-0.835);
   \fill (0,1.71) \blob;
   \fill (1.05,-0.351) \blob;
   \fill (-1.61,-0.519) \blob;
   \fill (0.559,-0.835) \blob;
   \draw (0,1.88) node {$\scriptstyle{1}$};
   \draw (1.15,-0.386) node {$\scriptstyle{2}$};
   \draw (-1.77,-0.571) node {$\scriptstyle{3}$};
   \draw (0.614,-0.919) node {$\scriptstyle{4}$};
   \draw[thick] (0.525,0.677) -- (0.787,1.02);
   \draw[thick] (-0.525,-0.677) -- (-0.787,-1.02);
   \draw[thick,dashed] (0.525,0.677) -- (-0.525,-0.677);
  \end{tikzpicture}
  \hspace{5em}
  \begin{tikzpicture} % [30]{-2}{2}{-1.7}{2.3}
   \draw (0,1.71) -- (1.05,-0.351);
   \draw (0,1.71) -- (-1.61,-0.519);
   \draw (0,1.71) -- (0.559,-0.835);
   \draw (1.05,-0.351) -- (-1.61,-0.519);
   \draw (1.05,-0.351) -- (0.559,-0.835);
   \draw (-1.61,-0.519) -- (0.559,-0.835);
   \fill (0,1.71) \blob;
   \fill (1.05,-0.351) \blob;
   \fill (-1.61,-0.519) \blob;
   \fill (0.559,-0.835) \blob;
   \draw (0.2,1.88) node {$\scriptstyle{1}$};
   \draw (1.15,-0.386) node {$\scriptstyle{2}$};
   \draw (-1.77,-0.571) node {$\scriptstyle{3}$};
   \draw (0.614,-0.919) node {$\scriptstyle{4}$};
   \draw[thick] (0,1.71) -- (0,2.05);
   \draw[thick] (0,-0.569) -- (0,-0.853);
   \draw[thick,dashed] (0,1.71) -- (0,-0.569);
   \fill (0,-0.569) \blob;
  \end{tikzpicture}
 \end{center}
\end{remark}


\subsection{Symmetries of the cube}
\label{subsec-cube}

We now study the symmetries of a cube.  We take our standard cube to
have vertices $(\pm 1,\pm 1,\pm 1)$, so the centre is at $(0,0,0)$ and
the edges have length $2$.
\begin{center}
 \begin{tikzpicture} % [30]{-2}{2}{-1.7}{2.3}
  \draw (0.598,-1.21) -- (1.28,-0.881);
  \draw (0.598,-1.21) -- (-1.28,-1.09);
  \draw (0.598,-1.21) -- (0.598,0.762);
  \draw (1.28,-0.881) -- (-0.598,-0.762);
  \draw (1.28,-0.881) -- (1.28,1.09);
  \draw (-1.28,-1.09) -- (-0.598,-0.762);
  \draw (-1.28,-1.09) -- (-1.28,0.881);
  \draw (-0.598,-0.762) -- (-0.598,1.21);
  \draw (0.598,0.762) -- (1.28,1.09);
  \draw (0.598,0.762) -- (-1.28,0.881);
  \draw (1.28,1.09) -- (-0.598,1.21);
  \draw (-1.28,0.881) -- (-0.598,1.21);
  \fill (0.598,-1.21) \blob;
  \fill (1.28,-0.881) \blob;
  \fill (-1.28,-1.09) \blob;
  \fill (-0.598,-0.762) \blob;
  \fill (0.598,0.762) \blob;
  \fill (1.28,1.09) \blob;
  \fill (-1.28,0.881) \blob;
  \fill (-0.598,1.21) \blob;
  \draw ( 0.657, 0.838) node {$\scriptstyle{1}$};
  \draw ( 1.410, 1.2  ) node {$\scriptstyle{2}$};
  \draw (-0.657, 1.33 ) node {$\scriptstyle{3}$};
  \draw (-1.410, 0.969) node {$\scriptstyle{4}$};
  \draw (-0.657,-0.838) node {$\scriptstyle{5}$};
  \draw (-1.410,-1.2  ) node {$\scriptstyle{6}$};
  \draw ( 0.657,-1.33 ) node {$\scriptstyle{7}$};
  \draw ( 1.410,-0.969) node {$\scriptstyle{8}$};
 \end{tikzpicture}
\end{center}
We have marked the vertices so that the vertex labelled $i$ is
opposite the one labelled $i+4$, which will be convenient later.
 
Note that $(x,y,z)$ lies in the cube if
and only if $(-x,-y,-z)$ does, so $-1\in\Symm(\Cube)$ (the
corresponding thing is \emph{not} true for the tetrahedron).  We
therefore see from Proposition~\ref{prop-product} that
$\Symm(\Cube)=\{\pm 1\}\tm\Dir(\Cube)$, so we will focus attention on
$\Dir(\Cube)$.

The action of $\Dir(\Cube)$ on the eight vertices gives rise to an
injective homomorphism $\Dir(\Cube)\xra{}S_8$, but it turns out that
this is far from being surjective.  In fact, $|\Dir(\Cube)|=4!=24$
whereas $|S_8|=8!=40320$, so the image of our homomorphism is a rather
small subgroup of $S_8$.  We therefore use a different approach to
study $\Dir(\Cube)$.  Let $L_1$, $L_2$, $L_3$ and $L_4$ be the four
long diagonals of the cube, as shown below.
\begin{center}
 \begin{tikzpicture} % [30]{-2}{2}{-1.7}{2.3}
  \draw (0.598,-1.21) -- (1.28,-0.881);
  \draw (0.598,-1.21) -- (-1.28,-1.09);
  \draw (0.598,-1.21) -- (0.598,0.762);
  \draw[dashed] (1.28,-0.881) -- (-0.598,-0.762);
  \draw (1.28,-0.881) -- (1.28,1.09);
  \draw[dashed] (-1.28,-1.09) -- (-0.598,-0.762);
  \draw (-1.28,-1.09) -- (-1.28,0.881);
  \draw[dashed] (-0.598,-0.762) -- (-0.598,1.21);
  \draw (0.598,0.762) -- (1.28,1.09);
  \draw (0.598,0.762) -- (-1.28,0.881);
  \draw (1.28,1.09) -- (-0.598,1.21);
  \draw (-1.28,0.881) -- (-0.598,1.21);
  \fill (0.598,-1.21) \blob;
  \fill (1.28,-0.881) \blob;
  \fill (-1.28,-1.09) \blob;
  \fill (-0.598,-0.762) \blob;
  \fill (0.598,0.762) \blob;
  \fill (1.28,1.09) \blob;
  \fill (-1.28,0.881) \blob;
  \fill (-0.598,1.21) \blob;
  \draw[dotted,blue] (0.897,1.14) -- (-0.897,-1.14);
  \draw[dotted,blue] (1.92,1.63) -- (-1.92,-1.63);
  \draw[dotted,blue] (-0.897,1.81) -- (0.897,-1.81);
  \draw[dotted,blue] (-1.92,1.32) -- (1.92,-1.32);
  \draw ( 0.956, 1.22) node {$\scriptstyle{1}$};
  \draw ( 2.050, 1.74) node {$\scriptstyle{2}$};
  \draw (-0.956, 1.93) node {$\scriptstyle{3}$};
  \draw (-2.050, 1.41) node {$\scriptstyle{4}$};
  \draw (-0.956,-1.22) node {$\scriptstyle{1}$};
  \draw (-2.050,-1.74) node {$\scriptstyle{2}$};
  \draw ( 0.956,-1.93) node {$\scriptstyle{3}$};
  \draw ( 2.050,-1.41) node {$\scriptstyle{4}$};
 \end{tikzpicture}
\end{center}
Note that $L_i$ passes through $v_i$ and $v_{i+4}$.

If $g\in\Dir(\Cube)$ then $g$ must send each long diagonal to another
long diagonal, so $\Dir(\Cube)$ acts on $\{L_1,L_2,L_3,L_4\}$.  We
therefore have a homomorphism $\phi\:\Dir(\Cube)\xra{}S_4$ such that
$g(L_i)=L_{\phi(g)(i)}$.


\begin{lemma}
 The homomorphism $\phi$ is injective.
\end{lemma}
\begin{proof}
 Suppose that $g\in\Dir(\Cube)$ and that $\phi(g)=1$; we must show
 that $g=1$.  Because $\phi(g)=1$ we have $g(L_i)=L_i$ for all $i$.
 In particular, we have $v_1\in L_1$ so $g(v_1)\in g(L_1)=L_1$, so
 either $g(v_1)=v_1$ or $g(v_1)=v_5=-v_1$.  Thus $g(v_1)=\ep_1v_1$ for
 some $\ep_1\in\{1,-1\}$, and similarly we have $g(v_i)=\ep_iv_i$ for
 some $\ep_i\in\{1,-1\}$ for $i=2,3,4$.

 Now suppose that $\ep_1=\ep_2=\ep_3=-1$, so $g(v_1)=-v_1$,
 $g(v_2)=-v_2$ and $g(v_3)=-v_3$.  As $v_1$, $v_2$ and $v_3$ are
 linearly independent (they do not all lie in any plane through the
 origin), they form a basis of $\R^3$.  Given this, it is clear that
 $g=-1$, so $\det(g)=-1$, contradicting the assumption that
 $g\in\Dir(\Cube)\leq SO_3$.  So we cannot have $\ep_1=\ep_2=\ep_3=-1$
 after all.

 More generally, any three of $\{v_1,v_2,v_3,v_4\}$ form a basis, so
 no three of the $\ep$'s can be $-1$.  Thus at most two of the $\ep$'s
 are $-1$, so at least two of them are $+1$, say $\ep_i=\ep_j=1$ with
 $i\neq j$.  This means that $g(v_i)=v_i$ and $g(v_j)=v_j$, so $g$ has
 two linearly independent fixed points.  If $g$ were a nontrivial
 rotation then all the fixed points would lie on the axis and thus
 any two would be linearly dependent.  Thus $g$ must be the identity.
\end{proof}

\begin{theorem}
 The homomorphism $\phi\:\Dir(\Cube)\xra{}S_4$ is an isomorphism.
\end{theorem}
\begin{proof}
 Let $g$ be a half turn around the axis shown on the left below.  It
 is clear that $g$ exchanges $L_3$ and $L_4$.  The line $L_1$ is
 perpendicular to the axis of $g$, so when we perform the half turn we
 send $L_1$ to itself, just reversing the direction.  Thus
 $g(L_1)=L_1$.  Similarly, we have $g(L_2)=L_2$ and so
 $\phi(g)=(3\;4)$.
 \begin{center}
  \begin{tikzpicture} % [30]{-2}{2}{-1.7}{2.3}
   \draw (0.598,-1.21) -- (1.28,-0.881);
   \draw (0.598,-1.21) -- (-1.28,-1.09);
   \draw (0.598,-1.21) -- (0.598,0.762);
   \draw[dashed] (1.28,-0.881) -- (-0.598,-0.762);
   \draw (1.28,-0.881) -- (1.28,1.09);
   \draw[dashed] (-1.28,-1.09) -- (-0.598,-0.762);
   \draw (-1.28,-1.09) -- (-1.28,0.881);
   \draw[dashed] (-0.598,-0.762) -- (-0.598,1.21);
   \draw (0.598,0.762) -- (1.28,1.09);
   \draw (0.598,0.762) -- (-1.28,0.881);
   \draw (1.28,1.09) -- (-0.598,1.21);
   \draw (-1.28,0.881) -- (-0.598,1.21);
   \fill (0.598,-1.21) \blob;
   \fill (1.28,-0.881) \blob;
   \fill (-1.28,-1.09) \blob;
   \fill (-0.598,-0.762) \blob;
   \fill (0.598,0.762) \blob;
   \fill (1.28,1.09) \blob;
   \fill (-1.28,0.881) \blob;
   \fill (-0.598,1.21) \blob;
   \draw[dotted,blue] (0.777,-1.57) -- (-0.777,1.57);
   \draw[dotted,blue] (1.67,-1.15) -- (-1.67,1.15);
   \draw[dotted,blue] (-1.67,-1.42) -- (1.67,1.42);
   \draw[dotted,blue] (-0.777,-0.991) -- (0.777,0.991);
   \draw (0.837,1.07) node {$\scriptstyle{1}$};
   \draw (1.79,1.52) node {$\scriptstyle{2}$};
   \draw (-0.837,1.69) node {$\scriptstyle{3}$};
   \draw (-1.79,1.23) node {$\scriptstyle{4}$};
   \draw[thick,red] (-0.94,1.04) -- (-1.32,1.46);
   \draw[thick,red] (0.94,-1.04) -- (1.32,-1.46);
   \draw[thick,red,dashed] (-0.94,1.04) -- (0.94,-1.04);
   \fill (-0.94,1.04) \blob;
   \fill (0.94,-1.04) \blob;
  \end{tikzpicture}
  \hspace{5em}
  \begin{tikzpicture} % [30]{-2}{2}{-1.7}{2.3}
   \draw (0.598,-1.21) -- (1.28,-0.881);
   \draw (0.598,-1.21) -- (-1.28,-1.09);
   \draw (0.598,-1.21) -- (0.598,0.762);
   \draw[dashed] (1.28,-0.881) -- (-0.598,-0.762);
   \draw (1.28,-0.881) -- (1.28,1.09);
   \draw[dashed] (-1.28,-1.09) -- (-0.598,-0.762);
   \draw (-1.28,-1.09) -- (-1.28,0.881);
   \draw[dashed] (-0.598,-0.762) -- (-0.598,1.21);
   \draw (0.598,0.762) -- (1.28,1.09);
   \draw (0.598,0.762) -- (-1.28,0.881);
   \draw (1.28,1.09) -- (-0.598,1.21);
   \draw (-1.28,0.881) -- (-0.598,1.21);
   \fill (0.598,-1.21) \blob;
   \fill (1.28,-0.881) \blob;
   \fill (-1.28,-1.09) \blob;
   \fill (-0.598,-0.762) \blob;
   \fill (0.598,0.762) \blob;
   \fill (1.28,1.09) \blob;
   \fill (-1.28,0.881) \blob;
   \fill (-0.598,1.21) \blob;
   \draw[dotted,blue] (0.777,-1.57) -- (-0.777,1.57);
   \draw[dotted,blue] (1.67,-1.15) -- (-1.67,1.15);
   \draw[dotted,blue] (-1.67,-1.42) -- (1.67,1.42);
   \draw[dotted,blue] (-0.777,-0.991) -- (0.777,0.991);
   \draw (0.837,1.07) node {$\scriptstyle{1}$};
   \draw (1.79,1.52) node {$\scriptstyle{2}$};
   \draw (-0.837,1.69) node {$\scriptstyle{3}$};
   \draw (-1.79,1.23) node {$\scriptstyle{4}$};
   \draw[thick,red] (0.342,1.15) -- (0.479,1.61);
   \draw[thick,red] (-0.342,-1.15) -- (-0.479,-1.61);
   \draw[thick,red,dashed] (0.342,1.15) -- (-0.342,-1.15);
   \fill (0.342,1.15) \blob;
   \fill (-0.342,-1.15) \blob;
  \end{tikzpicture}
  \hspace{5em}
  \begin{tikzpicture} % [30]{-2}{2}{-1.7}{2.3}
   \draw (0.598,-1.21) -- (1.28,-0.881);
   \draw (0.598,-1.21) -- (-1.28,-1.09);
   \draw (0.598,-1.21) -- (0.598,0.762);
   \draw[dashed] (1.28,-0.881) -- (-0.598,-0.762);
   \draw (1.28,-0.881) -- (1.28,1.09);
   \draw[dashed] (-1.28,-1.09) -- (-0.598,-0.762);
   \draw (-1.28,-1.09) -- (-1.28,0.881);
   \draw[dashed] (-0.598,-0.762) -- (-0.598,1.21);
   \draw (0.598,0.762) -- (1.28,1.09);
   \draw (0.598,0.762) -- (-1.28,0.881);
   \draw (1.28,1.09) -- (-0.598,1.21);
   \draw (-1.28,0.881) -- (-0.598,1.21);
   \fill (0.598,-1.21) \blob;
   \fill (1.28,-0.881) \blob;
   \fill (-1.28,-1.09) \blob;
   \fill (-0.598,-0.762) \blob;
   \fill (0.598,0.762) \blob;
   \fill (1.28,1.09) \blob;
   \fill (-1.28,0.881) \blob;
   \fill (-0.598,1.21) \blob;
   \draw[dotted,blue] (0.777,-1.57) -- (-0.777,1.57);
   \draw[dotted,blue] (1.67,-1.15) -- (-1.67,1.15);
   \draw[dotted,blue] (-1.67,-1.42) -- (1.67,1.42);
   \draw[dotted,blue] (-0.777,-0.991) -- (0.777,0.991);
   \draw (0.837,1.07) node {$\scriptstyle{1}$};
   \draw (1.79,1.52) node {$\scriptstyle{2}$};
   \draw (-0.837,1.69) node {$\scriptstyle{3}$};
   \draw (-1.79,1.23) node {$\scriptstyle{4}$};
   \draw[thick,red] (0.94,0.925) -- (1.32,1.3);
   \draw[thick,red] (-0.94,-0.925) -- (-1.32,-1.3);
   \draw[thick,red,dashed] (0.94,0.925) -- (-0.94,-0.925);
   \fill (0.94,0.925) \blob;
   \fill (-0.94,-0.925) \blob;
  \end{tikzpicture}
 \end{center}
 Similarly, if we do a half turn about the other two axes we get the
 transpositions $(2\;3)$ and $(1\;2)$.  The transpositions $(1\;2)$,
 $(2\;3)$ and $(3\;4)$ lie in the image of $\phi$ and generate $S_4$,
 so $\phi$ is surjective.  We have already seen that it is injective,
 so it must be an isomorphism.
\end{proof}

\begin{remark}
 Let $h$ be a one-third turn about $L_4$, rotating clockwise as seen
 from above.
 \begin{center}
  \begin{tikzpicture} % [30]{-2}{2}{-1.7}{2.3}
   \draw (0.598,-1.21) -- (1.28,-0.881);
   \draw (0.598,-1.21) -- (-1.28,-1.09);
   \draw (0.598,-1.21) -- (0.598,0.762);
   \draw[dashed] (1.28,-0.881) -- (-0.598,-0.762);
   \draw (1.28,-0.881) -- (1.28,1.09);
   \draw[dashed] (-1.28,-1.09) -- (-0.598,-0.762);
   \draw (-1.28,-1.09) -- (-1.28,0.881);
   \draw[dashed] (-0.598,-0.762) -- (-0.598,1.21);
   \draw (0.598,0.762) -- (1.28,1.09);
   \draw (0.598,0.762) -- (-1.28,0.881);
   \draw (1.28,1.09) -- (-0.598,1.21);
   \draw (-1.28,0.881) -- (-0.598,1.21);
   \fill (0.598,-1.21) \blob;
   \fill (1.28,-0.881) \blob;
   \fill (-1.28,-1.09) \blob;
   \fill (-0.598,-0.762) \blob;
   \fill (0.598,0.762) \blob;
   \fill (1.28,1.09) \blob;
   \fill (-1.28,0.881) \blob;
   \fill (-0.598,1.21) \blob;
   \draw[dotted,blue] (0.777,-1.57) -- (-0.777,1.57);
   \draw[dotted,blue] (1.67,-1.15) -- (-1.67,1.15);
   \draw[dotted,blue] (-1.67,-1.42) -- (1.67,1.42);
   \draw[dotted,blue] (-0.777,-0.991) -- (0.777,0.991);
   \draw (0.837,1.07) node {$\scriptstyle{1}$};
   \draw (1.79,1.52) node {$\scriptstyle{2}$};
   \draw (-0.837,1.69) node {$\scriptstyle{3}$};
   \draw (-1.79,1.23) node {$\scriptstyle{4}$};
   \draw[thick,red] (-1.28,0.881) -- (-1.67,1.15);
   \draw[thick,red] (1.28,-0.881) -- (1.67,-1.15);
   \draw[thick,red,dashed] (-1.28,0.881) -- (1.28,-0.881);
  \end{tikzpicture}
 \end{center}
 If we look down $L_4$ at the cube we see the following picture:
 \begin{center}
  \begin{tikzpicture}[scale=2] % [40]{-2}{2}{-2}{2}
   \def\PA{( 0.000, 0.000)}
   \def\PB{( 0.866, 0.500)}
   \def\PC{(-0.866, 0.500)}
   \def\PD{( 0.000, 1.000)}
   \def\PE{( 0.000,-1.000)}
   \def\PF{( 0.866,-0.500)}
   \def\PG{(-0.866,-0.500)}
   \fill \PA \blob;
   \fill \PB \blob;
   \fill \PC \blob;
   \fill \PD \blob;
   \fill \PE \blob;
   \fill \PF \blob;
   \fill \PG \blob;
   \draw \PB -- \PD -- \PC -- \PG -- \PE -- \PF -- \PB;
   \draw \PA -- \PD;
   \draw \PA -- \PG;
   \draw \PA -- \PF;
   \draw[dashed] \PA -- \PB;
   \draw[dashed] \PA -- \PC;
   \draw[dashed] \PA -- \PE;
   \draw ( 1.05,-0.60) node {$v_1$};
   \draw ( 1.05, 0.60) node {$v_2$};
   \draw ( 0.00, 1.20) node {$v_3$};
   \draw (-1.05, 0.60) node {$v_5$};
   \draw (-1.05,-0.60) node {$v_6$};
   \draw ( 0.00,-1.20) node {$v_7$};
  \end{tikzpicture} 
 \end{center}
 Thus $h$ gives the following permutation of vertices:
 \begin{align*}
  v_1 &\mapsto v_6 \mapsto v_3 \mapsto v_1 \\
  v_5 &\mapsto v_2 \mapsto v_7 \mapsto v_5
 \end{align*}
 As $L_1$ joins $v_1$ to $v_5$, $L_2$ joins $v_6$ to $v_2$ and $L_3$
 joins $v_3$ to $v_7$ we see that the permutation of $L$'s is
 $L_1\mapsto L_2\mapsto L_3\mapsto L_1$, so $\phi(h)=(1\;2\;3)$.

 Now let $k$ be a quarter turn around the $z$-axis, anticlockwise as
 seen from above.
 \begin{center}
  \begin{tikzpicture} % [30]{-2}{2}{-1.7}{2.3}
   \draw (0.598,-1.21) -- (1.28,-0.881);
   \draw (0.598,-1.21) -- (-1.28,-1.09);
   \draw (0.598,-1.21) -- (0.598,0.762);
   \draw[dashed] (1.28,-0.881) -- (-0.598,-0.762);
   \draw (1.28,-0.881) -- (1.28,1.09);
   \draw[dashed] (-1.28,-1.09) -- (-0.598,-0.762);
   \draw (-1.28,-1.09) -- (-1.28,0.881);
   \draw[dashed] (-0.598,-0.762) -- (-0.598,1.21);
   \draw (0.598,0.762) -- (1.28,1.09);
   \draw (0.598,0.762) -- (-1.28,0.881);
   \draw (1.28,1.09) -- (-0.598,1.21);
   \draw (-1.28,0.881) -- (-0.598,1.21);
   \fill (0.598,-1.21) \blob;
   \fill (1.28,-0.881) \blob;
   \fill (-1.28,-1.09) \blob;
   \fill (-0.598,-0.762) \blob;
   \fill (0.598,0.762) \blob;
   \fill (1.28,1.09) \blob;
   \fill (-1.28,0.881) \blob;
   \fill (-0.598,1.21) \blob;
   \draw[dotted,blue] (0.777,-1.57) -- (-0.777,1.57);
   \draw[dotted,blue] (1.67,-1.15) -- (-1.67,1.15);
   \draw[dotted,blue] (-1.67,-1.42) -- (1.67,1.42);
   \draw[dotted,blue] (-0.777,-0.991) -- (0.777,0.991);
   \draw (0.837,1.07) node {$\scriptstyle{1}$};
   \draw (1.79,1.52) node {$\scriptstyle{2}$};
   \draw (-0.837,1.69) node {$\scriptstyle{3}$};
   \draw (-1.79,1.23) node {$\scriptstyle{4}$};
   \draw[thick,red] (0,0.985) -- (0,1.28);
   \draw[thick,red] (0,-0.985) -- (0,-1.28);
   \draw[thick,red,dashed] (0,0.985) -- (0,-0.985);
  \end{tikzpicture}
 \end{center}
 We then have $\phi(k)=(1\;2\;3\;4)$ and $\phi(k^2)=(1\;3)(2\;4)$.  We
 have thus found rotations giving representatives of all the cycle
 types in $S_4$.
\end{remark}

\section{Duality and the Octahedron}

We next study the symmetries of the octahedron.  We take our standard
octahedron to have vertices as follows:
\[ \setlength{\arraycolsep}{4em} \begin{array}{ccc}
   w_1=( 1, 0, 0) & w_2=( 0, 1, 0) & w_3=( 0, 0, 1) \\
   w_4=(-1, 0, 0) & w_5=( 0,-1, 0) & w_6=( 0, 0,-1).
  \end{array}
\]

\begin{center}
 \begin{tikzpicture} % [30]{-2}{2}{-1.7}{2.3}
  \draw[dashed] (0.513,0.245) -- (-1.41,0.0891);
  \draw[dashed] (0.513,0.245) -- (0,1.48);
  \draw[dashed] (0.513,0.245) -- (1.41,-0.0891);
  \draw[dashed] (0.513,0.245) -- (0,-1.48);
  \draw (-1.41,0.0891) -- (0,1.48);
  \draw (-1.41,0.0891) -- (-0.513,-0.245);
  \draw (-1.41,0.0891) -- (0,-1.48);
  \draw (0,1.48) -- (-0.513,-0.245);
  \draw (0,1.48) -- (1.41,-0.0891);
  \draw (-0.513,-0.245) -- (1.41,-0.0891);
  \draw (-0.513,-0.245) -- (0,-1.48);
  \draw (1.41,-0.0891) -- (0,-1.48);
  \fill (0.513,0.245) \blob;
  \fill (-1.41,0.0891) \blob;
  \fill (0,1.48) \blob;
  \fill (-0.513,-0.245) \blob;
  \fill (1.41,-0.0891) \blob;
  \fill (0,-1.48) \blob;
  \draw (0.616,0.294) node {$\scriptstyle{4}$};
  \draw (-1.69,0.107) node {$\scriptstyle{5}$};
  \draw (0,1.77) node {$\scriptstyle{3}$};
  \draw (-0.616,-0.294) node {$\scriptstyle{1}$};
  \draw (1.69,-0.107) node {$\scriptstyle{2}$};
  \draw (0,-1.77) node {$\scriptstyle{6}$};
 \end{tikzpicture}
\end{center}

It turns out that there is a close relationship (called ``duality'')
between the cube and the octahedron.  As illustrated in the picture on
the left, the vertices of the octahedron are the centres of the faces
of the cube.
\begin{center}
 \begin{tikzpicture} % [30]{-2}{2}{-1.7}{2.3}
  \begin{scope}[blue]
   \draw[dashed] (0.513,0.245) -- (-1.41,0.0891);
   \draw[dashed] (0.513,0.245) -- (0,1.48);
   \draw[dashed] (0.513,0.245) -- (1.41,-0.0891);
   \draw[dashed] (0.513,0.245) -- (0,-1.48);
   \draw (-1.41,0.0891) -- (0,1.48);
   \draw (-1.41,0.0891) -- (-0.513,-0.245);
   \draw (-1.41,0.0891) -- (0,-1.48);
   \draw (0,1.48) -- (-0.513,-0.245);
   \draw (0,1.48) -- (1.41,-0.0891);
   \draw (-0.513,-0.245) -- (1.41,-0.0891);
   \draw (-0.513,-0.245) -- (0,-1.48);
   \draw (1.41,-0.0891) -- (0,-1.48);
   \fill (0.513,0.245) \blob;
   \fill (-1.41,0.0891) \blob;
   \fill (0,1.48) \blob;
   \fill (-0.513,-0.245) \blob;
   \fill (1.41,-0.0891) \blob;
   \fill (0,-1.48) \blob;
  \end{scope}
  \begin{scope}[thick,red]
   \draw (0.897,-1.81) -- (1.92,-1.32);
   \draw (0.897,-1.81) -- (-1.92,-1.63);
   \draw (0.897,-1.81) -- (0.897,1.14);
   \draw[dashed] (1.92,-1.32) -- (-0.897,-1.14);
   \draw (1.92,-1.32) -- (1.92,1.63);
   \draw[dashed] (-1.92,-1.63) -- (-0.897,-1.14);
   \draw (-1.92,-1.63) -- (-1.92,1.32);
   \draw[dashed] (-0.897,-1.14) -- (-0.897,1.81);
   \draw (0.897,1.14) -- (1.92,1.63);
   \draw (0.897,1.14) -- (-1.92,1.32);
   \draw (1.92,1.63) -- (-0.897,1.81);
   \draw (-1.92,1.32) -- (-0.897,1.81);
  \end{scope}
 \end{tikzpicture}
 \hspace{5em}
 \begin{tikzpicture} % [30]{-2}{2}{-1.7}{2.3}
  \begin{scope}[red]
   \draw (0.299,-0.604) -- (0.641,-0.441);
   \draw (0.299,-0.604) -- (-0.641,-0.544);
   \draw (0.299,-0.604) -- (0.299,0.381);
   \draw[dashed] (0.641,-0.441) -- (-0.299,-0.381);
   \draw (0.641,-0.441) -- (0.641,0.544);
   \draw[dashed] (-0.641,-0.544) -- (-0.299,-0.381);
   \draw (-0.641,-0.544) -- (-0.641,0.441);
   \draw[dashed] (-0.299,-0.381) -- (-0.299,0.604);
   \draw (0.299,0.381) -- (0.641,0.544);
   \draw (0.299,0.381) -- (-0.641,0.441);
   \draw (0.641,0.544) -- (-0.299,0.604);
   \draw (-0.641,0.441) -- (-0.299,0.604);
   \fill (0.299,-0.604) \blob;
   \fill (0.641,-0.441) \blob;
   \fill (-0.641,-0.544) \blob;
   \fill (-0.299,-0.381) \blob;
   \fill (0.299,0.381) \blob;
   \fill (0.641,0.544) \blob;
   \fill (-0.641,0.441) \blob;
   \fill (-0.299,0.604) \blob;
  \end{scope}
  \begin{scope}[thick,blue]
   \draw[dashed] (0.513,0.245) -- (-1.41,0.0891);
   \draw[dashed] (0.513,0.245) -- (0,1.48);
   \draw[dashed] (0.513,0.245) -- (1.41,-0.0891);
   \draw[dashed] (0.513,0.245) -- (0,-1.48);
   \draw (-1.41,0.0891) -- (0,1.48);
   \draw (-1.41,0.0891) -- (-0.513,-0.245);
   \draw (-1.41,0.0891) -- (0,-1.48);
   \draw (0,1.48) -- (-0.513,-0.245);
   \draw (0,1.48) -- (1.41,-0.0891);
   \draw (-0.513,-0.245) -- (1.41,-0.0891);
   \draw (-0.513,-0.245) -- (0,-1.48);
   \draw (1.41,-0.0891) -- (0,-1.48);
  \end{scope}
 \end{tikzpicture}
\end{center}
To see this algebraically, note that the vertices of the top face of
the cube are $(1,1,1)$, $(-1,1,1)$, $(1,-1,1)$ and $(-1,-1,1)$.  Thus,
the centre of the top face is
\[ \frac{1}{4}((1,1,1)+(-1,1,1)+(1,-1,1)+(-1,-1,1))=(0,0,1). \]
This is just the top vertex of the octahedron.  The calculation for
the other faces follows the same pattern.

On the other hand, the centres of the faces of the octahedron are the
vertices of a cube one third as big as the one we started with, as
illustrated in the picture on the right.

\begin{proposition}
 The group $\Symm(\Oct)$ is the same as $\Symm(\Cube)$ (and thus is
 isomorphic to $S_4\tm\{\pm 1\}$).
\end{proposition}
\begin{proof}
 Suppose that $g\in\Symm(\Cube)$.  Let $w$ be a vertex of the
 octahedron.  Then $w$ is the centre of some face $F$ of the cube.  As
 $g$ is a symmetry of the cube, $gF$ is another face, and $gw$ is the
 centre of $gF$, so $gw$ is a vertex of the octahedron.  Thus $g$
 sends vertices of the octahedron to vertices, and it follows that it
 sends the octahedron to itself.  Thus $\Symm(\Cube)\sse\Symm(\Oct)$.

 Now suppose that $h\in\Symm(\Oct)$.  Let $v$ be a vertex of the large
 cube, so $v/3$ is a vertex of the small cube, so $v/3$ is the centre
 of some face $F'$ of the octahedron.  As $h$ is a symmetry of the
 octahedron, $hF'$ is another face, and $h(v/3)$ is the centre of
 $hF'$, so $h(v)/3=h(v/3)$ is a vertex of the small cube, so $h(v)$ is
 a vertex of the large cube.  Thus $h$ sends vertices of the cube to
 vertices, and it follows that it sends the cube to itself.
 Thus $\Symm(\Oct)\sse\Symm(\Cube)$.
\end{proof}

\begin{remark}
 You might hope that a similar picture would give interesting
 information about the tetrahedron.  However, the centres of the faces
 of a tetrahedron are just the vertices of a smaller tetrahedron, as
 illustrated below, so we just conclude that the two different
 tetrahedra have the same symmetry group.
 \begin{center}
  \begin{tikzpicture}[scale=0.5] % [17]{-6}{6}{-2.7}{5.3}
   \begin{scope}[blue]
    \draw (0,-1.71) -- (-1.05,0.351);
    \draw (0,-1.71) -- (1.61,0.519);
    \draw (0,-1.71) -- (-0.559,0.835);
    \draw (-1.05,0.351) -- (1.61,0.519);
    \draw (-1.05,0.351) -- (-0.559,0.835);
    \draw (1.61,0.519) -- (-0.559,0.835);
    \fill (0,-1.71) \blob;
    \fill (-1.05,0.351) \blob;
    \fill (1.61,0.519) \blob;
    \fill (-0.559,0.835) \blob;
   \end{scope}
   \begin{scope}[thick,red]
    \draw (0,5.12) -- (3.15,-1.05);
    \draw (0,5.12) -- (-4.82,-1.56);
    \draw (0,5.12) -- (1.68,-2.51);
    \draw[dashed] (3.15,-1.05) -- (-4.82,-1.56);
    \draw (3.15,-1.05) -- (1.68,-2.51);
    \draw (-4.82,-1.56) -- (1.68,-2.51);
    \fill (0,5.12) \blob;
    \fill (3.15,-1.05) \blob;
    \fill (-4.82,-1.56) \blob;
    \fill (1.68,-2.51) \blob;
   \end{scope}
  \end{tikzpicture}
 \end{center}
 For an algebraic approach to this, note that the centre of the
 tetrahedron is $(v_1+v_2+v_3+v_4)/4$, but by assumption the centre is
 at the origin, so we must have $v_1+v_2+v_3+v_4=0$.  The vertices of
 the face opposite $v_1$ are $v_2$, $v_3$ and $v_4$ so the centre of
 the face is $(v_2+v_3+v_4)/3=-v_1/3$.  More generally, the centre of
 the face opposite $v_k$ is $-v_k/3$, and the points $-v_1/3$,
 $-v_2/3$, $-v_3/3$ and $-v_4/3$ clearly form a tetrahedron one third
 as big as the one we started with.
\end{remark}

\section{The construction of the dodecahedron}

\begin{proposition}
 There is a solid (called the dodecahedron) with $12$ faces, each of
 which is a regular pentagon with edges of length $1$.
\end{proposition}
The rest of this section will constitute the proof; a picture of the
dodecahedron is shown below.
% Dodecahedron
\begin{center}
 \begin{tikzpicture}[scale=1.6] % [50]{-2}{2}{-1.7}{2.3}
  \draw (0.391,-0.983) -- (0.453,-1.33);
  \draw (0.391,-0.983) -- (0.975,-0.175);
  \draw (0.391,-0.983) -- (-0.553,-0.698);
  \draw (1.08,-0.729) -- (0.453,-1.33);
  \draw (1.08,-0.729) -- (1.4,-0.0174);
  \draw[dashed] (1.08,-0.729) -- (0.553,-0.286);
  \draw (-1.08,-0.865) -- (-0.453,-1.25);
  \draw (-1.08,-0.865) -- (-0.553,-0.698);
  \draw (-1.08,-0.865) -- (-1.4,0.0174);
  \draw[dashed] (-0.391,-0.61) -- (-0.453,-1.25);
  \draw[dashed] (-0.391,-0.61) -- (0.553,-0.286);
  \draw[dashed] (-0.391,-0.61) -- (-0.975,0.175);
  \draw (0.391,0.61) -- (0.975,-0.175);
  \draw (0.391,0.61) -- (-0.553,0.286);
  \draw (0.391,0.61) -- (0.453,1.25);
  \draw (1.08,0.865) -- (1.4,-0.0174);
  \draw[dashed] (1.08,0.865) -- (0.553,0.698);
  \draw (1.08,0.865) -- (0.453,1.25);
  \draw (-1.08,0.729) -- (-0.553,0.286);
  \draw (-1.08,0.729) -- (-1.4,0.0174);
  \draw (-1.08,0.729) -- (-0.453,1.33);
  \draw[dashed] (-0.391,0.983) -- (0.553,0.698);
  \draw[dashed] (-0.391,0.983) -- (-0.975,0.175);
  \draw[dashed] (-0.391,0.983) -- (-0.453,1.33);
  \draw (0.453,-1.33) -- (-0.453,-1.25);
  \draw (0.975,-0.175) -- (1.4,-0.0174);
  \draw[dashed] (0.553,-0.286) -- (0.553,0.698);
  \draw (-0.553,-0.698) -- (-0.553,0.286);
  \draw[dashed] (-1.4,0.0174) -- (-0.975,0.175);
  \draw (0.453,1.25) -- (-0.453,1.33);
  \fill (0.391,-0.983) \blob;
  \fill (1.08,-0.729) \blob;
  \fill (-1.08,-0.865) \blob;
  \fill (-0.391,-0.61) \blob;
  \fill (0.391,0.61) \blob;
  \fill (1.08,0.865) \blob;
  \fill (-1.08,0.729) \blob;
  \fill (-0.391,0.983) \blob;
  \fill (0.453,-1.33) \blob;
  \fill (-0.453,-1.25) \blob;
  \fill (0.975,-0.175) \blob;
  \fill (1.4,-0.0174) \blob;
  \fill (0.553,-0.286) \blob;
  \fill (0.553,0.698) \blob;
  \fill (-0.553,-0.698) \blob;
  \fill (-0.553,0.286) \blob;
  \fill (-1.4,0.0174) \blob;
  \fill (-0.975,0.175) \blob;
  \fill (0.453,1.25) \blob;
  \fill (-0.453,1.33) \blob;
 \end{tikzpicture}
\end{center}

We will construct the dodecahedron by attaching ``tents'' to a cube as
shown below.  We have only shown two tents here but eventually we will
use six tents, one for each face.
% A cube with two tents ready to attach
\begin{center}
 \begin{tikzpicture}[scale=1.6] % [50]{-2}{2}{-1.7}{2.3}
  \draw (0.391,-0.983) -- (0.453,-1.33);
  \draw (0.391,-0.983) -- (0.975,-0.175);
  \draw (0.391,-0.983) -- (-0.553,-0.698);
  \draw (1.08,-0.729) -- (0.453,-1.33);
  \draw (1.08,-0.729) -- (1.4,-0.0174);
  \draw[dashed] (1.08,-0.729) -- (0.553,-0.286);
  \draw (-1.08,-0.865) -- (-0.453,-1.25);
  \draw (-1.08,-0.865) -- (-0.553,-0.698);
  \draw (-1.08,-0.865) -- (-1.4,0.0174);
  \draw[dashed] (-0.391,-0.61) -- (-0.453,-1.25);
  \draw[dashed] (-0.391,-0.61) -- (0.553,-0.286);
  \draw[dashed] (-0.391,-0.61) -- (-0.975,0.175);
  \draw (0.391,0.61) -- (0.975,-0.175);
  \draw (0.391,0.61) -- (-0.553,0.286);
  \draw (0.391,0.61) -- (0.453,1.25);
  \draw (1.08,0.865) -- (1.4,-0.0174);
  \draw[dashed] (1.08,0.865) -- (0.553,0.698);
  \draw (1.08,0.865) -- (0.453,1.25);
  \draw (-1.08,0.729) -- (-0.553,0.286);
  \draw (-1.08,0.729) -- (-1.4,0.0174);
  \draw (-1.08,0.729) -- (-0.453,1.33);
  \draw[dashed] (-0.391,0.983) -- (0.553,0.698);
  \draw[dashed] (-0.391,0.983) -- (-0.975,0.175);
  \draw[dashed] (-0.391,0.983) -- (-0.453,1.33);
  \draw (0.453,-1.33) -- (-0.453,-1.25);
  \draw (0.975,-0.175) -- (1.4,-0.0174);
  \draw[dashed] (0.553,-0.286) -- (0.553,0.698);
  \draw (-0.553,-0.698) -- (-0.553,0.286);
  \draw[dashed] (-1.4,0.0174) -- (-0.975,0.175);
  \draw (0.453,1.25) -- (-0.453,1.33);
  \fill (0.391,-0.983) \blob;
  \fill (1.08,-0.729) \blob;
  \fill (-1.08,-0.865) \blob;
  \fill (-0.391,-0.61) \blob;
  \fill (0.391,0.61) \blob;
  \fill (1.08,0.865) \blob;
  \fill (-1.08,0.729) \blob;
  \fill (-0.391,0.983) \blob;
  \fill (0.453,-1.33) \blob;
  \fill (-0.453,-1.25) \blob;
  \fill (0.975,-0.175) \blob;
  \fill (1.4,-0.0174) \blob;
  \fill (0.553,-0.286) \blob;
  \fill (0.553,0.698) \blob;
  \fill (-0.553,-0.698) \blob;
  \fill (-0.553,0.286) \blob;
  \fill (-1.4,0.0174) \blob;
  \fill (-0.975,0.175) \blob;
  \fill (0.453,1.25) \blob;
  \fill (-0.453,1.33) \blob;
 \end{tikzpicture}
\end{center}

The edges of the cube will have length $d$; later on we will work out
exactly what $d$ has to be.  The tents will be as shown below, with
dotted edges of length $d$ and solid edges of length $1$.
% A tent with long edges dotted
\begin{center}
 \begin{tikzpicture}[scale=1.3] % [40]{-2}{2}{-1.7}{2.3}
  % cube
  \draw (0.391,-0.983) -- (0.391,0.61) -- (1.08,0.865) -- (1.08,-0.729) --
        (0.391,-0.983) -- (-1.08,-0.865) -- (-1.08,0.729) -- (0.391,0.61) --
        (-1.08,0.729) -- (-0.391,0.983) -- (1.08,0.865);
  \draw[dashed] (-1.08,-0.865) -- (-0.391,-0.61) -- (-0.391,0.983) --
                (-0.391,-0.61) -- (1.08,-0.729);
  % right tent
  \draw[red] (1.3,-1.06) -- (1.98,-0.802) -- (2.3,-0.0908) -- (1.88,-0.248) --
         (1.3,-1.06) -- (1.3,0.537) -- (1.88,-0.248) -- (2.3,-0.0908) --
         (1.98,0.791) -- (1.3,0.537);
  \draw[red,dashed] (1.98,-0.802) -- (1.98,0.791);
  % top tent
  \draw[blue] (0.391,1.3) -- (1.08,1.55) -- (0.453,1.94) -- (0.391,1.3) --
        (-1.08,1.42) -- (-0.453,2.02) -- (0.453,1.94);
  \draw[blue,dashed] (-1.08,1.42) -- (-0.391,1.67) -- (-0.453,2.02) --
        (-0.391,1.67) -- (1.08,1.55) -- (0.453,1.94) -- (1.08,1.55) -- 
        (0.391,1.3);
  % attachment lines
  \draw[dotted] (0.391,-0.983) -- (1.3,-1.06);
  \draw[dotted] (1.08,-0.729) -- (1.98,-0.802);
  \draw[dotted] (0.391,0.61) -- (1.3,0.537);
  \draw[dotted] (1.08,0.865) -- (1.98,0.791);
  \draw[dotted] (0.391,0.61) -- (0.391,1.3);
  \draw[dotted] (1.08,0.865) -- (1.08,1.55);
  \draw[dotted] (-1.08,0.729) -- (-1.08,1.42);
  \draw[dotted] (-0.391,0.983) -- (-0.391,1.67);
 \end{tikzpicture}
 \hspace{5em}
 \begin{tikzpicture} % [30]{-2}{2}{-0.7}{1.5}
  \draw (1.31,-0.325) -- (-0.174,0.814) -- (0.174,1.16) -- (1.87,0.228);
  \draw (-1.87,-0.228) -- (-0.174,0.814) -- (0.174,1.16) -- (-1.31,0.325);
  \draw[dotted] (-1.31,0.325) -- (-1.87,-0.228) -- (1.31,-0.325) --
               (1.87,0.228) -- (-1.31,0.325);
 \end{tikzpicture}
\end{center}


The next diagram shows the result of attaching tents to cubes of three
different sizes.
\begin{center}
 \def\x{1.25}
 \begin{tikzpicture} % [30]{-2}{2}{-1.7}{2.3}
  \draw[white] (0,1.6) -- (1,1.6);
  % right tent
  \draw[red] (1.11,-0.745) -- (1.38*\x,-0.0385*\x) --
             (0.884*\x,-0.189*\x) -- (0.296,-0.989);
  \draw[red] (1.11, 0.848) -- (1.38*\x,-0.0385*\x) --
             (0.884*\x,-0.189*\x) -- (0.296,0.605);
  % top tent
  \draw[blue] (-1.11,0.745) -- (-0.433*\x,1.33*\x) --
              (0.433*\x,1.25*\x) -- (0.296,0.605); 
  \draw[blue] (0.433*\x,1.25*\x) -- (1.11,0.848); 
  \draw[dashed,blue] (-0.296,0.989) -- (-0.433*\x,1.33*\x);
  % cube
  \draw (1.11,-0.745) -- (0.296,-0.989) -- (-1.11,-0.848) --
        (-1.11,0.745) -- (0.296,0.605) -- (0.296,-0.989);
  \draw[dashed] (1.11,-0.745) -- (1.11,0.848) -- (-0.296,0.989) --
                (-0.296,-0.605) -- (1.11,-0.745);
  \draw[dashed] (-1.11,0.745) -- (-0.296,0.989);
  \draw[dashed] (-1.11,-0.848) -- (-0.296,-0.605);
  \draw[very thick] (0.296,0.605) -- (1.11,0.848);
 \end{tikzpicture}
 \hspace{4em}
 \def\x{1}
 \begin{tikzpicture} % [30]{-2}{2}{-1.7}{2.3}
  \draw[white] (0,1.6) -- (1,1.6);
  % right tent
  \draw[red] (1.11,-0.745) -- (1.38*\x,-0.0385*\x) --
             (0.884*\x,-0.189*\x) -- (0.296,-0.989);
  \draw[red] (1.11, 0.848) -- (1.38*\x,-0.0385*\x) --
             (0.884*\x,-0.189*\x) -- (0.296,0.605);
  % top tent
  \draw[blue] (-1.11,0.745) -- (-0.433*\x,1.33*\x) --
              (0.433*\x,1.25*\x) -- (0.296,0.605); 
  \draw[blue] (0.433*\x,1.25*\x) -- (1.11,0.848); 
  \draw[dashed,blue] (-0.296,0.989) -- (-0.433*\x,1.33*\x);
  % cube
  \draw (1.11,-0.745) -- (0.296,-0.989) -- (-1.11,-0.848) --
        (-1.11,0.745) -- (0.296,0.605) -- (0.296,-0.989);
  \draw[dashed] (1.11,-0.745) -- (1.11,0.848) -- (-0.296,0.989) --
                (-0.296,-0.605) -- (1.11,-0.745);
  \draw[dashed] (-1.11,0.745) -- (-0.296,0.989);
  \draw[dashed] (-1.11,-0.848) -- (-0.296,-0.605);
  \draw[very thick] (0.296,0.605) -- (1.11,0.848);
 \end{tikzpicture}
 \hspace{4em}
 \def\x{0.8}
 \begin{tikzpicture} % [30]{-2}{2}{-1.7}{2.3}
  \draw[white] (0,1.6) -- (1,1.6);
  % right tent
  \draw[red] (1.11,-0.745) -- (1.38*\x,-0.0385*\x) --
             (0.884*\x,-0.189*\x) -- (0.296,-0.989);
  \draw[red] (1.11, 0.848) -- (1.38*\x,-0.0385*\x) --
             (0.884*\x,-0.189*\x) -- (0.296,0.605);
  % top tent
  \draw[blue] (-1.11,0.745) -- (-0.433*\x,1.33*\x) --
              (0.433*\x,1.25*\x) -- (0.296,0.605); 
  \draw[blue] (0.433*\x,1.25*\x) -- (1.11,0.848); 
  \draw[dashed,blue] (-0.296,0.989) -- (-0.433*\x,1.33*\x);
  % cube
  \draw (1.11,-0.745) -- (0.296,-0.989) -- (-1.11,-0.848) --
        (-1.11,0.745) -- (0.296,0.605) -- (0.296,-0.989);
  \draw[dashed] (1.11,-0.745) -- (1.11,0.848) -- (-0.296,0.989) --
                (-0.296,-0.605) -- (1.11,-0.745);
  \draw[dashed] (-1.11,0.745) -- (-0.296,0.989);
  \draw[dashed] (-1.11,-0.848) -- (-0.296,-0.605);
  \draw[very thick] (0.296,0.605) -- (1.11,0.848);
 \end{tikzpicture}
\end{center}

Note that we have a bent pentagon with the thick line cutting across
it.  If $d$ is small as shown on the left, then the pentagon is bent
outwards along the thick line.  If $d$ is too large as shown on the
right, then the pentagon is bent inwards along the thick line.  If we
choose exactly the right value of $d$ as shown in the middle, we get a
flat pentagon.

On the other hand, for any value of $d$ we can flatten out the
pentagon and lay it out in the plane.
\begin{center}
 \begin{tikzpicture}[scale=1.4] % [50]{-1.2}{1.2}{-0.2}{2.0}
  \draw (0.5,0) -- (0.647,0.989) -- (0,1.75) --
    (-0.647,0.989) -- (-0.5,0) -- (0.5,0);
  \draw[very thick] (0.647,0.989) -- (-0.647,0.989);
  \draw (0,0.8) node {$\scriptstyle{d}$};
  \draw (0,-.15) node {$\scriptstyle{1}$};
 \end{tikzpicture}
 \hspace{4em}
 \begin{tikzpicture}[scale=1.4] % [50]{-1.2}{1.2}{-0.2}{2.3}
  \draw (0.5,0) -- (0.809,0.951) -- (0,1.54) --
    (-0.809,0.951) -- (-0.5,0) -- (0.5,0);
  \draw[very thick] (0.809,0.951) -- (-0.809,0.951);
  \draw (0,0.76) node {$\scriptstyle{d}$};
  \draw (0,-.15) node {$\scriptstyle{1}$};
 \end{tikzpicture}
 \hspace{4em}
 \begin{tikzpicture}[scale=1.4] % [50]{-1.2}{1.2}{-0.2}{2.3}
  \draw (0.5,0) -- (0.971,0.882) -- (0,1.12) --
    (-0.971,0.882) -- (-0.5,0) -- (0.5,0);
  \draw[very thick] (0.971,0.882) -- (-0.971,0.882);
  \draw (0,0.69) node {$\scriptstyle{d}$};
  \draw (0,-.15) node {$\scriptstyle{1}$};
 \end{tikzpicture}
\end{center}
If $d$ is too small or too large then the pentagon will not be
regular.  The miraculous thing is that the value of $d$ that makes the
pentagon flat is the same value that makes it regular; our next task
is to prove this.

\begin{lemma}\label{lem-tau}
 In the regular pentagon with sides of length $1$, the distance $\tau$
 in the diagram below satisfies $\tau=2\cos(\pi/5)=(1+\sqrt{5})/2$,
 and moreover we have $1/(\tau-1)=\tau$.
 \begin{center}
  \begin{tikzpicture}[scale=1.4] % [50]{-2}{2}{-0.2}{2.3}
   \draw (0.5,0) -- (0.809,0.951) -- (0,1.54) --
    (-0.809,0.951) -- (-0.5,0) -- (0.5,0);
   \draw (0.77,0.523) node {$\scriptstyle{1}$};
   \draw (0.445,1.37) node {$\scriptstyle{1}$};
   \draw (-0.445,1.37) node {$\scriptstyle{1}$};
   \draw (-0.77,0.523) node {$\scriptstyle{1}$};
   \draw (0.,-0.1) node {$\scriptstyle{1}$};
   \draw (0,0.8) node {$\tau$};
   \draw[very thick] (0.809,0.951) -- (-0.809,0.951);
  \end{tikzpicture}
 \end{center}
\end{lemma}
\begin{proof}
 We can divide the pentagon into right angled triangles as shown on
 the left below.  All the angles in the middle are equal to $\theta$ and
 there are ten of them so $\theta=2\pi/10=\pi/5$.  As the angles of any
 triangle add up to $\pi$, we have $\phi=\pi/2-\theta=3\pi/10$.
 \begin{center}
  \begin{tikzpicture}[scale=3] % [90]{-.7}{.7}{-0.7}{.7}
   \draw (0,0.618) -- (-0.588,0.191) -- (-0.363,-0.5) --
          (0.363,-0.5) -- (0.588,0.191) -- (0,0.618);
   \draw (0,0.618) -- (0.,-0.5);
   \draw (-0.588,0.191) -- (0.476,-0.155);
   \draw (-0.363,-0.5) -- (0.294,0.405);
   \draw (0.363,-0.5) -- (-0.294,0.405);
   \draw (0.588,0.191) -- (-0.476,-0.155);
   \draw (-0.342,0.369) -- (-0.307,0.321) -- (-0.21,0.391) -- (-0.245,0.44);
   \draw (-0.457,-0.212) -- (-0.4,-0.193) -- (-0.437,-0.0789) -- (-0.494,-0.0974);
   \draw (0.06,-0.5) -- (0.06,-0.44) -- (-0.06,-0.44) -- (-0.06,-0.5);
   \draw (0.494,-0.0974) -- (0.437,-0.0789) -- (0.4,-0.193) -- (0.457,-0.212);
   \draw (0.245,0.44) -- (0.21,0.391) -- (0.307,0.321) -- (0.342,0.369);
   \draw (0.05,0.15) node {$\theta$};
   \draw (0.05,0.53) node {$\phi$};
  \end{tikzpicture}
  \hspace{4em}
  \begin{tikzpicture}[scale=3] % [90]{-.7}{.7}{-0.7}{.7}
   \draw (0,0.618) -- (-0.588,0.191) -- (-0.363,-0.5) --
          (0.363,-0.5) -- (0.588,0.191) -- (0,0.618);
   \draw (-0.588,0.191) -- (0.588,0.191);
   \draw (0.,0.191) -- (0,0.618);
   \draw (0.06,0.191) -- (0.06,0.251) -- (-0.06,0.251) -- (-0.06,0.191);
   \draw ( 0.400,0.270) node {$\psi$};
   \draw ( 0.050,0.530) node {$\phi$};
   \draw (-0.400,0.270) node {$\psi$};
   \draw (-0.050,0.530) node {$\phi$};
   \draw ( 0.250,0.120) node {$\cos(\psi)$};
   \draw (-0.250,0.120) node {$\cos(\psi)$};
   \draw (-0.353,0.485) node {$\scriptstyle{1}$};
   \draw ( 0.353,0.485) node {$\scriptstyle{1}$};
  \end{tikzpicture}
 \end{center}
 Now consider the picture on the right, which shows that
 $\tau=2\cos(\psi)$.  We have a triangle with angles $\pi/2$, $\phi$
 and $\psi$ so $\psi=\pi-\pi/2-\phi=\pi/2-\phi=\theta=\pi/5$, so we
 conclude that $\tau=2\cos(\pi/5)$.

 We next claim that $\tau^2-\tau-1=0$.  To see this, put
 $\xi=e^{\pi i/5}=\cos(\pi/5)+i\sin(\pi/5)$, so
 $\xi^{-1}=e^{-\pi i/5}=\cos(\pi/5)-i\sin(\pi/5)$, so
 $\xi+\xi^{-1}=2\cos(\pi/5)=\tau$.  We find that
 \begin{align*}
  \tau^2-\tau-1 &= (\xi^2+2+\xi^{-2}) - (\xi+\xi^{-1}) - 1 \\
                &= \xi^2 - \xi + 1 - \xi^{-1} + \xi^{-2}, \\
  \intertext{ so } 
  (1+\xi)(\tau^2-\tau-1) &=
    (1+\xi)(\xi^2 - \xi + 1 - \xi^{-1} + \xi^{-2}) \\
    &= \xi^3 + \xi^{-2} = \xi^{-2}(\xi^5 + 1).
 \end{align*}
 However, we also have $\xi^5=e^{i\pi}=-1$, so $\xi^5+1=0$, so 
 $(1+\xi)(\tau^2-\tau-1)=0$.  As $\xi\neq -1$ we can divide by $\xi$
 to deduce that $\tau^2-\tau-1=0$ as claimed.  

 Solving this equation gives $\tau=(1\pm\sqrt{5})/2$, but
 $(1-\sqrt{5})/2<0$ and $\tau$ is clearly positive so we must have
 $\tau=(1+\sqrt{5})/2$.  We can also rearrange the equation
 $\tau^2-\tau-1=0$ as $(\tau-1)\tau=1$ and so $1/(\tau-1)=\tau$.
\end{proof}

Now let $T$ be a tent whose base is a square of side $\tau$ and whose
other edges have length $1$.  We place $T$ with its base in the
$xy$-plane parallel to the axes with the centre of the base at the
origin and with the ridge parallel to the $x$-axis.
% Labelled tent
\begin{center}
 \begin{tikzpicture}[scale=1.6] % [60]{-2}{2}{-1.7}{2.3}
  \draw (-2.6,0.26) -- (2.6,-0.26);
  \draw (2.86,-0.287) node {$\scriptstyle{x}$};
  \draw (-1.5,-0.451) -- (2.,0.602);
  \draw (2.15,0.647) node {$\scriptstyle{y}$};
  \draw (0,-0.591) -- (0,1.38);
  \draw (0,1.57) node {$\scriptstyle{z}$};
  \begin{scope}[very thick,red]
   \draw (0.592,-0.384) -- (-2.21,-0.103) -- (-0.866,1.07) -- (0.866,0.898) --
          (0.592,-0.384) -- (2.21,0.103) -- (0.866,0.898);
   \draw[dashed] (2.21,0.103) -- (-0.592,0.384) -- (-0.866,1.07);
   \draw[dashed] (-0.592,0.384) -- (-2.21,-0.103);
  \end{scope}
  \draw ( 2.35, 0.123) node {$\scriptstyle{A}$};
  \draw (-0.80, 0.469) node {$\scriptstyle{B}$};
  \draw ( 0.60,-0.480) node {$\scriptstyle{C}$};
  \draw (-2.35,-0.123) node {$\scriptstyle{D}$};
  \draw (-0.95, 1.190) node {$\scriptstyle{E}$};
  \draw ( 0.95, 1.000) node {$\scriptstyle{F}$};
 \end{tikzpicture}
\end{center}
It should be clear that the coordinates of $A,\ldots,D$ are as
follows:
\[ \begin{array}{lll}
    A = ( \tau/2, \tau/2,0) &\hspace{3em}& B = (-\tau/2, \tau/2,0) \\
    C = ( \tau/2,-\tau/2,0) &\hspace{3em}& D = (-\tau/2,-\tau/2,0)
\end{array} \]
Moreover, the line $EF$ is horizontal and lies in the $xz$ plane and
it crosses the $z$-axis at its midpoint.  This means that the $y$
coordinates of $E$ and $F$ are zero, their $z$-coordinates are the
same, and the $x$ coordinate of $E$ is minus the $x$ coordinate of
$F$.  Thus for some $a,b$ we have $E=(-a,0,b)$ and $F=(a,0,b)$.

Next, recall that the edges $FA$, $FC$, $EB$, $ED$ and $EF$ have
length $1$.  As $\vec{EF}=(2a,0,0)$ and $EF$ has length $1$ we must
have $a=1/2$.  Thus
\begin{align*}
 \vec{FA} &= A-F =(\tau/2-a,\tau/2,-b) \\
          &= ((\tau-1)/2,\tau/2,-b) \\
          &= ((\sqrt{5}-1)/4,(\sqrt{5}+1)/4,-b).
\end{align*}
As $FA$ has length $1$ we conclude that
\begin{align*}
 1 &= (\sqrt{5}-1)^2/16 + (\sqrt{5}+1)^2/16 + (-b)^2 \\
   &= (6-2\sqrt{5})/16 + (6+2\sqrt{5})/16 + b^2 \\
   &= 3/4 + b^2.
\end{align*}
This gives $b^2=1/4$ so $b=1/2$.  In summary, we have
\[ \begin{array}{lll}
    A = ( \tau/2, \tau/2,0) &\hspace{3em}& B = (-\tau/2, \tau/2,0) \\
    C = ( \tau/2,-\tau/2,0) &\hspace{3em}& D = (-\tau/2,-\tau/2,0) \\
    E = ( -1/2, 0, 1/2)     &\hspace{3em}& F = (1/2,0,1/2).
\end{array} \]

\begin{lemma}
 The angles $\al$ and $\bt$ indicated below are the same.
 % Tent with angles
 \begin{center}
  \begin{tikzpicture}[scale=2] % [60]{-2}{2}{-1.7}{2.3}
   \draw (0.592,-0.384) -- (-2.21,-0.103) -- (-0.866,1.07) -- (0.866,0.898) --
          (0.592,-0.384) -- (2.21,0.103) -- (0.866,0.898);
   \draw[dashed] (2.21,0.103) -- (-0.592,0.384) -- (-0.866,1.07);
   \draw[dashed] (-0.592,0.384) -- (-2.21,-0.103);
   \draw[dotted] (0.809,0.243) -- (-0.809,-0.243);
   \draw[dotted] (1.4,-0.14) -- (-1.4,0.14);
   \draw ( 2.35, 0.123) node {$\scriptstyle{A}$};
   \draw (-0.80, 0.469) node {$\scriptstyle{B}$};
   \draw ( 0.60,-0.480) node {$\scriptstyle{C}$};
   \draw (-2.35,-0.123) node {$\scriptstyle{D}$};
   \draw (-0.95, 1.190) node {$\scriptstyle{E}$};
   \draw ( 0.95, 1.000) node {$\scriptstyle{F}$};
   \draw[very thick] (0,0) -- (0,0.985) -- (-0.809,-0.243) -- (0,0);
   \draw (-0.135,0.55) node {$\scriptstyle{\alpha}$};
   \draw[very thick] (1.4,-0.14) -- (0.866,-0.0868) -- (0.866,0.898) -- (1.4,-0.14);
   \draw (1.17,0.0305) node {$\scriptstyle{\beta}$};
  \end{tikzpicture}
 \end{center}
\end{lemma}
\begin{proof}
 The left hand triangle looks like this.
 \begin{center}
  \begin{tikzpicture}[scale=2] % [60]{-2}{1}{-0.5}{1.5}
   \draw[very thick] (0,0) -- (-1.6,0) -- (0,1) -- (0,0);
   \draw (-1.7,0) node[anchor=east]
    {${\textstyle\frac{1}{2}}(C+D)=(0,-{\frac{\textstyle{\tau}}{2}},0)$};
   \draw (0.1,0) node[anchor=west]
    {$(0,0,0)$};
   \draw (0.1,1) node[anchor=west]
    {${\textstyle\frac{1}{2}}(E+F)=(0,0,\frac{\textstyle{1}}{2})$};
   \draw (-0.8,-0.1) node[anchor=north] {$\textstyle\frac{\tau}{2}$};
   \draw (0.1,0.5) node[anchor=west] {$\textstyle\frac{1}{2}$};
   \draw (-0.1,0.8) node {$\alpha$};
  \end{tikzpicture} 
 \end{center}
 The top vertex is the midpoint of $EF$ which is $\half(E+F)$ and
 using our formulae for $E$ and $F$ we see that this is just
 $(0,0,\half)$.  The bottom right vertex is in the $xy$-plane directly
 underneath $(0,0,\half)$, so it must be $(0,0,0)$.  The bottom left
 vertex is the midpoint of $CD$, which is $\half(C+D)=(0,-\thf,0)$.
 It follows easily that the sides have length $\half$ and $\thf$ as
 shown, and thus that $\tan(\al)=\thf/\half=\tau$.

 In a similar way, we see that the right hand triangle is as follows:
 \begin{center}
  \begin{tikzpicture}[scale=2] % [60]{-1}{2}{-0.5}{1.5}
   \draw[very thick] (0,0) -- (0.6,0) -- (0,1) -- (0,0);
   \draw (-0.1,0) node[anchor=east]
    {$({\textstyle{1}\over{2}},0,0)$};
   \draw (-0.1,1) node[anchor=east]
    {$F=({\textstyle{1}\over{2}},0,{\textstyle{1}\over{2}})$};
   \draw (0.7,0) node[anchor=west]
    {${\textstyle{1}\over{2}}(A+C)=({\textstyle{\tau}\over{2}},0,0)$};
   \draw (0.3,-0.1) node[anchor=north] {${\tau-1}\over{2}$};
   \draw (-0.1,0.5) node[anchor=east] {${\textstyle{1}\over{2}}$};
   \draw (0.4,0.1) node {$\beta$};
  \end{tikzpicture} 
 \end{center}
 This shows that $\tan(\bt)=\half/{\tau-1}\over{2}=1/(\tau-1)$.  We
 know from Lemma~\ref{lem-tau} that $1/(\tau-1)=\tau$ so
 $\tan(\bt)=\tan(\al)$ so $\bt=\al$.
\end{proof}

Now suppose we attach two tents to a cube of side $\tau$ as shown on
the left.
\begin{center}
 \begin{tikzpicture}[scale=1.6] % [50]{-1.3}{1.8}{-1.2}{1.5}
  \draw (0.296,-0.989) -- (1.11,-0.745) -- (1.38,-0.0385) -- (0.884,-0.189) --
         (0.296,-0.989) -- (0.296,0.605) -- (0.884,-0.189) -- (1.38,-0.0385) --
         (1.11,0.848) -- (0.433,1.25) -- (0.296,0.605) -- (-1.11,0.745) --
         (-0.433,1.33) -- (0.433,1.25) -- (1.11,0.848);
  \draw (0.296,-0.989) -- (-1.11,-0.848) -- (-1.11,0.745);
  \draw[dashed] (1.11,-0.745) -- (1.11,0.848) -- (-0.296,0.989) --
                (-0.296,-0.605) -- (1.11,-0.745);
  \draw[dashed] (-1.11,0.745) -- (-0.296,0.989) -- (-0.433,1.33);
  \draw[dashed] (-1.11,-0.848) -- (-0.296,-0.605);
  \fill (0.433,1.25) \blob;
  \fill (0.296,0.605) \blob;
  \fill (0.884,-0.189) \blob;
  \draw (0.44,1.35) node[anchor=south west] {$P$};
  \draw (0.28,0.50) node[anchor=north east] {$Q$};
  \draw (0.8,-0.19) node[anchor=east] {$R$};
  \draw[very thick] (0.296,0.605) -- (1.11,0.848);
 \end{tikzpicture}
 \hspace{5em}
 \begin{tikzpicture} % [25]{-3}{3}{-2}{2.9}
  \draw (-1.6,-1.6) -- (1.6,-1.6) -- (1.6,1.6) -- (-1.6,1.6) -- (-1.6,-1.6) --
         (-1.6,1.6) -- (-1,2.6) -- (1,2.6) -- (1.6,1.6) -- (2.6,0) -- (1.6,-1.6);
  \draw[dotted] (-1.8,0) -- (2.8,0);
  \fill (1.0,2.6) \blob;
  \fill (1.6,1.6) \blob;
  \fill (2.6,0.0) \blob;
  \draw (1.1,2.7) node[anchor=south west] {$P$};
  \draw (1.7,1.7) node[anchor=south west] {$Q$};
  \draw (2.7,0.1) node[anchor=south west] {$R$};
  \draw (2.3,0.2) node {$\alpha$};
  \draw (1.3,1.8) node {$\beta$};
 \end{tikzpicture} 
\end{center}
Looking from the side we see the picture on the right.  As $\al=\bt$
we see that $P$, $Q$ and $R$ lie on a straight line, so the pentagon
is flat as required.

We now attach a tent to each face, giving the following picture.
\begin{center}
 \begin{tikzpicture}[scale=2] % [50]{-2}{2}{-1.7}{2.3}
  \begin{scope}[red]
   \draw (0.391,-1.06) -- (0.453,-1.32);
   \draw (0.391,-1.06) -- (0.975,-0.26);
   \draw (0.391,-1.06) -- (-0.553,-0.79);
   \draw (1.08,-0.68) -- (0.453,-1.32);
   \draw (1.08,-0.68) -- (1.4,-0.0259);
   \draw[dashed] (1.08,-0.68) -- (0.553,-0.176);
   \draw (-1.08,-0.883) -- (-0.453,-1.21);
   \draw (-1.08,-0.883) -- (-0.553,-0.79);
   \draw (-1.08,-0.883) -- (-1.4,0.0259);
   \draw[dashed] (-0.391,-0.503) -- (-0.453,-1.21);
   \draw[dashed] (-0.391,-0.503) -- (0.553,-0.176);
   \draw[dashed] (-0.391,-0.503) -- (-0.975,0.26);
   \draw (0.391,0.503) -- (0.975,-0.26);
   \draw (0.391,0.503) -- (-0.553,0.176);
   \draw (0.391,0.503) -- (0.453,1.21);
   \draw (1.08,0.883) -- (1.4,-0.0259);
   \draw[dashed] (1.08,0.883) -- (0.553,0.79);
   \draw (1.08,0.883) -- (0.453,1.21);
   \draw (-1.08,0.68) -- (-0.553,0.176);
   \draw (-1.08,0.68) -- (-1.4,0.0259);
   \draw (-1.08,0.68) -- (-0.453,1.32);
   \draw[dashed] (-0.391,1.06) -- (0.553,0.79);
   \draw[dashed] (-0.391,1.06) -- (-0.975,0.26);
   \draw[dashed] (-0.391,1.06) -- (-0.453,1.32);
   \draw (0.453,-1.32) -- (-0.453,-1.21);
   \draw (0.975,-0.26) -- (1.4,-0.0259);
   \draw[dashed] (0.553,-0.176) -- (0.553,0.79);
   \draw (-0.553,-0.79) -- (-0.553,0.176);
   \draw[dashed] (-1.4,0.0259) -- (-0.975,0.26);
   \draw (0.453,1.21) -- (-0.453,1.32);
  \end{scope}
  \fill ( 0.39,-1.06) \blob;
  \fill ( 1.08,-0.68) \blob;
  \fill (-1.08,-0.88) \blob;
  \fill (-0.39,-0.50) \blob;
  \fill ( 0.39, 0.50) \blob;
  \fill ( 1.08, 0.88) \blob;
  \fill (-1.08, 0.68) \blob;
  \fill (-0.39, 1.06) \blob;
  \fill ( 0.45,-1.32) \blob;
  \fill (-0.45,-1.21) \blob;
  \fill ( 0.98,-0.26) \blob;
  \fill ( 1.40,-0.03) \blob;
  \fill ( 0.55,-0.18) \blob;
  \fill ( 0.55, 0.79) \blob;
  \fill (-0.55,-0.79) \blob;
  \fill (-0.55, 0.18) \blob;
  \fill (-1.40, 0.03) \blob;
  \fill (-0.98, 0.26) \blob;
  \fill ( 0.45, 1.21) \blob;
  \fill (-0.45, 1.32) \blob;
  \draw[thick] (0.391,-1.06) -- (0.391,0.503) -- (1.08,0.883) -- (1.08,-0.68) --
         (0.391,-1.06) -- (-1.08,-0.883) -- (-1.08,0.68) -- (0.391,0.503) --
         (-1.08,0.68) -- (-0.391,1.06) -- (1.08,0.883);
  \draw[thick,dashed] (-1.08,-0.883) -- (-0.391,-0.503) -- (-0.391,1.06);
  \draw[thick,dashed] (-0.391,-0.503) -- (1.08,-0.68);
 \end{tikzpicture}
\end{center}
The same argument as before shows that all the pentagons are flat.
You can just now just look at the picture to see that we have twelve
regular pentagonal faces, as required.

\begin{proposition}
 The dodecahedron has $20$ vertices and $30$ edges.
\end{proposition}
\begin{proof}
 There are $12$ faces each with $5$ edges, apparently giving
 $5\times 12=60$ edges.  However, each edge is an edge of two
 different faces, so we have counted each edge twice; there are really
 only $60/2=30$ edges.  Similarly, there are $12$ faces each with $5$
 vertices, but each vertex occurs on three different faces, so there
 are $12\times 5/3=20$ vertices altogether.
\end{proof}

\section{Symmetries of the dodecahedron}

We now investigate the group $G:=\Dir(\Dodec)$ of direct symmetries of
the dodecahedron.

If $g$ is a symmetry of the cube, it may or may not move the tents
around in such a way as to preserve the dodecahedron.  For example, if
we do a quarter around the $z$-axis then the ridge of the top tent
ends up at right angles to the way it originally was, so the
dodecahedron is not preserved.  However, a half turn about the
$z$-axis (or the $x$-axis or $y$-axis) does send tents to tents and
thus gives a symmetry of the dodecahedron.  These half twists are
given by the matrices
\[ \bsm 1 & 0 & 0 \\ 0 & -1 & 0 \\ 0 & 0 & -1 \esm
   \qquad,\qquad
   \bsm -1 & 0 & 0 \\ 0 & 1 & 0 \\ 0 & 0 & -1 \esm
   \qquad \text{ and } \qquad
   \bsm -1 & 0 & 0 \\ 0 &-1 & 0 \\ 0 & 0 &  1 \esm.
\]
These matrices commute and have order two, and the product of any two
of them is the third one.  It follows that together with the identity
they form a group of order four.

Next, if we do a one-third twist about a long diagonal of the cube we
get another symmetry of the dodecahedron, this time of order $3$.

Finally, if we let $L$ be the line joining the centres of two opposite
faces then a rotation through $2\pi/5$ also preserves the
dodecahedron.  It would take some work to prove this from our
construction, but I hope that it is reasonably clear geometrically.

\begin{proposition}\label{prop-sixty}
 The group $G=\Dir(\Dodec)$ has order $60$, and it is generated by the
 rotations of order $2$, $3$ and $5$ mentioned above.
\end{proposition}
\begin{proof}
 Let $H$ be the subgroup of $G$ generated by the rotations considered
 previously.  We have seen that $H$ has a subgroup of order $4$ and
 that it contains an element of order $3$ and an element of order $5$.
 It follows that $|H|$ is divisible by $4$, $3$ and $5$.  As these
 numbers are coprime, it follows that $|H|$ is divisible by
 $4\times 3\times 5=60$.  As $H$ is a subgroup of $G$, we see that
 $|G|$ is divisible by $|H|$ and thus is divisible by $60$.

 Now let $G$ act on the vertices of the dodecahedron, and choose a
 vertex $v$.  The orbit-stabiliser theorem says that $|G|$ is the
 the order of the orbit $Gv$ times the order of the stabiliser group
 $\stab_G(v)$.  There are $20$ vertices, so $|Gv|\leq 20$.  If
 $g\in\stab_G(v)$ then $g$ must be a rotation around $v$, and by
 looking at the three edges meeting at $v$ we see that the only
 possible angles are $0$ and $\pm 2\pi/3$.  This shows that
 $|\stab_G(v)|=3$ and thus that
 $|G|=|Gv||\stab_G(v)|\leq 20\times 3=60$.  As $|H|\leq |G|\leq 60$
 and $|H|$ is divisible by $60$ we must have $|G|=|H|=60$ and $G=H$.
\end{proof}

\begin{theorem}
 $G$ is isomorphic to $A_5$.
\end{theorem}
\begin{proof}
 Let $D$ denote the dodecahedron.  By construction, $D$ contains an
 inscribed cube $C$, and each face of $D$ is cut across by a single
 edge of $C$.  If $g\in G$ then $gC$ is a cube inscribed in $D$, and
 is typically different from $C$.  Let $X$ be the set of all the cubes
 that arise in this way.  The group $G$ acts on the set $X$, and by
 construction the orbit of of the element $C\in X$ is the whole of
 $X$.  Thus, the orbit-stabiliser theorem says that $|G|=|X||K|$, where
 $K$ is the stabiliser of $C$.  In other words, $K$ is the set of
 rotations that preserve both the cube $C$ and the dodecahedron $D$,
 so $K$ is a subgroup of $\Dir(C)$.  It follows that $24=|\Dir(C)|$ is
 divisible by $|K|$.  However, $K$ is not equal to $\Dir(C)$ (because
 a quarter-twist around the $z$-axis does not preserve $D$) so
 $|K|<24$.  On the other hand, $K$ contains the group of order
 $4$ generated by half-twists around the axes and it also contains
 elements of order $3$ given by rotating around the long diagonals, so
 $|K|$ is divisible by $12$.  It follows that $|K|$ must be equal to
 $12$.  As $|G|=|X||K|$ and $|G|=60$ and $|K|=12$ we have $X=5$.

 Another way to see that $X$ has five elements is as follows.  Choose
 a face $F$ of $D$, and let $E$ be the edge of $C$ that cuts across
 $F$.  Let $g$ be a rotation through $2\pi/5$ around the centre of
 $F$, so $g^kF=F$ for $k=0,\ldots,4$.  Then $g^kE$ is the edge of
 $g^kC$ that cuts across $F$.  As the edges $g^kE$ are all different
 (for $k=0,1,2,3,4$), the cubes $g^kC$ must all be different, so we
 have at least five cubes.
 \begin{center}
  \begin{tikzpicture}[scale=4] % [150]{-.7}{.7}{-0.7}{.7}
   \draw (0,0.618) -- (-0.588,0.191) -- (-0.363,-0.5) --
          (0.363,-0.5) -- (0.588,0.191) -- (0,0.618);
   \begin{scope}[very thick]
    \draw (0,0.618) -- (-0.363,-0.5);
    \draw (-0.588,0.191) -- (0.363,-0.5);
    \draw (-0.363,-0.5) -- (0.588,0.191);
    \draw (0.363,-0.5) -- (0,0.618);
    \draw (0.588,0.191) -- (-0.588,0.191);
   \end{scope}
   \draw ( 0.35 , 0.24 )  node {$\scriptstyle{E}$};
   \draw (-0.149, 0.416)  node {$\scriptstyle{gE}$};
   \draw (-0.442,-0.0127) node {$\scriptstyle{g^2E}$};
   \draw (-0.144,-0.424)  node {$\scriptstyle{g^3E}$};
   \draw ( 0.365,-0.249)  node {$\scriptstyle{g^4E}$};
  \end{tikzpicture}
 \end{center}
 It is a bit more difficult to show that there are \emph{exactly} five
 cubes by this method.

 The action of $G$ on $X$ gives a homomorphism $\phi\:G\xra{}S_5$.  If
 $g$ is as above then $\phi(g)$ is clearly a $5$-cycle, and thus an
 even permutation.  If $h$ is a one-third twist about a vertex of $C$,
 then $\phi(h)^3=\phi(h^3)=1$, so $\phi(h)$ is a permutation of
 $\{1,\ldots,5\}$ of order dividing $3$.  One checks that the only
 possibilities are the identity and the $3$-cycles, and by inspecting
 a model we see that $\phi(h)$ is not the identity so it must be a
 $3$-cycle.  In particular, it is again an even permutation.  Now let
 $k_x$, $k_y$ and $k_z$ be the half-twists about the $x$, $y$ and
 $z$-axes.  By inspecting a model again we see that $\phi(k_x)$,
 $\phi(k_y)$ and $\phi(k_z)$ are distinct elements of $S_5$ of the
 form $(a\;b)(c\;d)$, so they are again even permutations.  It
 follows that $\{1,\phi(k_x),\phi(k_y),\phi(k_z)\}$ is a subgroup of
 $A_5$ of order $4$.  As elements of the three types just considered
 generate $G$, we see that $\phi(x)$ is an even permutation for all
 $x\in G$, so $\phi(G)\leq A_5$.  We have also seen that $\phi(G)$
 contains a group of order $4$ and elements of orders $3$ and $5$, so
 $|\phi(G)|$ is divisible by $3\times 4\times 5=60$.  However, we also
 have $|A_5|=5!/2=60$, so we must have $\phi(G)=A_5$.  Thus
 $\phi\:G\xra{}A_5$ is a surjective map between two sets that both
 have exactly $60$ elements, so $\phi$ must be a bijection.  Thus
 $\phi$ gives an isomorphism $G\simeq A_5$ as claimed.
\end{proof}

\section{The icosahedron}

The icosahedron is the dual of the dodecahedron.  We'll write $D$ for
the dodecahedron and $I$ for the icosahedron.  The centres of the 12
faces of $D$ are the $12$ vertices of $I$.  If $v$ is a vertex of $D$,
then $3$ faces of $D$ (say $F_1$, $F_2$ and $F_3$) meet at $v$.  If we
write $w_i$ for the centre of $F_i$ then $w_1$, $w_2$ and $w_3$ are
vertices of $I$, and they form an equilateral triangle which is a face
of $I$.  This gives $20$ faces, one for each of the $20$ vertices of
$D$.
\begin{center}
 \begin{tikzpicture}[scale=1.5] % [50]{-2}{2}{-1.7}{2.3}
  \begin{scope}[blue]
   \draw (0.858,-0.669) -- (-0.247,-1.05);
   \draw (0.858,-0.669) -- (0.13,-0.286);
   \draw[dotted] (0.858,-0.669) -- (0.247,-0.778);
   \draw (0.858,-0.669) -- (0.931,0.158);
   \draw (0.858,-0.669) -- (0.858,0.462);
   \draw (-0.247,-1.05) -- (0.13,-0.286);
   \draw[dotted] (-0.247,-1.05) -- (0.247,-0.778);
   \draw (-0.247,-1.05) -- (-0.858,-0.462);
   \draw (-0.247,-1.05) -- (-0.931,-0.158);
   \draw (0.13,-0.286) -- (-0.931,-0.158);
   \draw (0.13,-0.286) -- (0.858,0.462);
   \draw (0.13,-0.286) -- (-0.247,0.778);
   \draw[dotted] (0.247,-0.778) -- (0.931,0.158);
   \draw[dotted] (0.247,-0.778) -- (-0.858,-0.462);
   \draw[dotted] (0.247,-0.778) -- (-0.13,0.286);
   \draw[dotted] (0.931,0.158) -- (-0.13,0.286);
   \draw (0.931,0.158) -- (0.858,0.462);
   \draw[dotted] (0.931,0.158) -- (0.247,1.05);
   \draw (-0.858,-0.462) -- (-0.931,-0.158);
   \draw[dotted] (-0.858,-0.462) -- (-0.13,0.286);
   \draw[dotted] (-0.858,-0.462) -- (-0.858,0.669);
   \draw (-0.931,-0.158) -- (-0.247,0.778);
   \draw (-0.931,-0.158) -- (-0.858,0.669);
   \draw[dotted] (-0.13,0.286) -- (0.247,1.05);
   \draw[dotted] (-0.13,0.286) -- (-0.858,0.669);
   \draw (0.858,0.462) -- (-0.247,0.778);
   \draw (0.858,0.462) -- (0.247,1.05);
   \draw (-0.247,0.778) -- (0.247,1.05);
   \draw (-0.247,0.778) -- (-0.858,0.669);
   \draw (0.247,1.05) -- (-0.858,0.669);
  \end{scope}
  \begin{scope}[thick,red]
   \draw (0.391,-1.06) -- (0.453,-1.32);
   \draw (0.391,-1.06) -- (0.975,-0.26);
   \draw (0.391,-1.06) -- (-0.553,-0.79);
   \draw (1.08,-0.68) -- (0.453,-1.32);
   \draw (1.08,-0.68) -- (1.4,-0.0259);
   \draw[dashed] (1.08,-0.68) -- (0.553,-0.176);
   \draw (-1.08,-0.883) -- (-0.453,-1.21);
   \draw (-1.08,-0.883) -- (-0.553,-0.79);
   \draw (-1.08,-0.883) -- (-1.4,0.0259);
   \draw[dashed] (-0.391,-0.503) -- (-0.453,-1.21);
   \draw[dashed] (-0.391,-0.503) -- (0.553,-0.176);
   \draw[dashed] (-0.391,-0.503) -- (-0.975,0.26);
   \draw (0.391,0.503) -- (0.975,-0.26);
   \draw (0.391,0.503) -- (-0.553,0.176);
   \draw (0.391,0.503) -- (0.453,1.21);
   \draw (1.08,0.883) -- (1.4,-0.0259);
   \draw[dashed] (1.08,0.883) -- (0.553,0.79);
   \draw (1.08,0.883) -- (0.453,1.21);
   \draw (-1.08,0.68) -- (-0.553,0.176);
   \draw (-1.08,0.68) -- (-1.4,0.0259);
   \draw (-1.08,0.68) -- (-0.453,1.32);
   \draw[dashed] (-0.391,1.06) -- (0.553,0.79);
   \draw[dashed] (-0.391,1.06) -- (-0.975,0.26);
   \draw[dashed] (-0.391,1.06) -- (-0.453,1.32);
   \draw (0.453,-1.32) -- (-0.453,-1.21);
   \draw (0.975,-0.26) -- (1.4,-0.0259);
   \draw[dashed] (0.553,-0.176) -- (0.553,0.79);
   \draw (-0.553,-0.79) -- (-0.553,0.176);
   \draw[dashed] (-1.4,0.0259) -- (-0.975,0.26);
   \draw (0.453,1.21) -- (-0.453,1.32);
  \end{scope}
 \end{tikzpicture}
 \hspace{5em}
 \begin{tikzpicture}[scale=1.7] % [50]{-2}{2}{-1.7}{2.3}
  \draw (0.947,-0.531) -- (0,-1.11);
  \draw[dashed] (0.947,-0.531) -- (0.585,-0.4);
  \draw (0.947,-0.531) -- (0,-0.611);
  \draw (0.947,-0.531) -- (0.585,0.4);
  \draw (0.947,-0.531) -- (0.947,0.531);
  \draw[dashed] (0,-1.11) -- (0.585,-0.4);
  \draw (0,-1.11) -- (0,-0.611);
  \draw (0,-1.11) -- (-0.947,-0.531);
  \draw[dashed] (0,-1.11) -- (-0.585,-0.4);
  \draw[dashed] (0.585,-0.4) -- (-0.585,-0.4);
  \draw[dashed] (0.585,-0.4) -- (0.947,0.531);
  \draw[dashed] (0.585,-0.4) -- (0,0.611);
  \draw (0,-0.611) -- (0.585,0.4);
  \draw (0,-0.611) -- (-0.947,-0.531);
  \draw (0,-0.611) -- (-0.585,0.4);
  \draw (0.585,0.4) -- (-0.585,0.4);
  \draw (0.585,0.4) -- (0.947,0.531);
  \draw (0.585,0.4) -- (0,1.11);
  \draw[dashed] (-0.947,-0.531) -- (-0.585,-0.4);
  \draw (-0.947,-0.531) -- (-0.585,0.4);
  \draw (-0.947,-0.531) -- (-0.947,0.531);
  \draw[dashed] (-0.585,-0.4) -- (0,0.611);
  \draw[dashed] (-0.585,-0.4) -- (-0.947,0.531);
  \draw (-0.585,0.4) -- (0,1.11);
  \draw (-0.585,0.4) -- (-0.947,0.531);
  \draw[dashed] (0.947,0.531) -- (0,0.611);
  \draw (0.947,0.531) -- (0,1.11);
  \draw[dashed] (0,0.611) -- (0,1.11);
  \draw[dashed] (0,0.611) -- (-0.947,0.531);
  \draw (0,1.11) -- (-0.947,0.531);
  \fill (0.947,-0.531) \blob;
  \fill (0,-1.11) \blob;
  \fill (0.585,-0.4) \blob;
  \fill (0,-0.611) \blob;
  \fill (0.585,0.4) \blob;
  \fill (-0.947,-0.531) \blob;
  \fill (-0.585,-0.4) \blob;
  \fill (-0.585,0.4) \blob;
  \fill (0.947,0.531) \blob;
  \fill (0,0.611) \blob;
  \fill (0,1.11) \blob;
  \fill (-0.947,0.531) \blob;
 \end{tikzpicture}
\end{center}
The picture on the left shows $I$ inside $D$.  The picture on the
right shows $I$ rotated into a more natural position.

\section{Finite subgroups of $SO_3$}

We now consider the classification of finite subgroups of $SO_3$.  We
have already met the groups
\begin{align*}
 G_1 &= \Dir(\Tet)\simeq A_4               & |G_1| &= 12 \\
 G_2 &= \Dir(\Cube)=\Dir(\Oct)\simeq S_4   & |G_2| &= 24 \\
 G_3 &= \Dir(\Dodec)=\Dir(\Icos)\simeq A_5 & |G_3| &= 60.
\end{align*}
The orders of these groups are $12$, $24$ and $60$.

Now let $n$ be any natural number.  Let $\tR$ be a rotation through
$2\pi/n$ around the $z$-axis, anticlockwise as seen from above.  This
clearly has order $n$ so the set $\tC_n=\{1,\tR,\ldots,\tR^{n-1}\}$ is
a subgroup of $SO_3$ which is cyclic of order $n$.  Strictly speaking
this is different from $C_n$ but it is usually harmless to ignore the
distinction.

Now let $\tS$ be a half-turn around the $x$-axis.  Note that the
effect of $\tS$ on the $xy$-plane is the same as a reflection across
the $x$-axis.  One can easily check that $\tS^2=1$ and
$\tS\tR\tS=\tR^{-1}$ and that the set
\[ \tD_n=\{1,\tR,\ldots,\tR^{n-1},\tS,\tR\tS,\ldots,\tR^{n-1}\tS\} \]
is a subgroup of $SO_3$ of order $2n$.  Again, it is usually harmless
to identify this with $D_n$.

Another point of view is as follows.  Given a matrix
$A=\bsm a&b\\c&d\esm\in O_2$, define
\[ \lm(A) = \left(\begin{array}{cc|c}
   a & b & 0 \\ c & d & 0 \\ \hline 0 & 0 & 1/\det(A)
   \end{array}\right).
\]
It is not hard to check that $\lm$ is a homomorphism and that
$\lm(R_{2\pi/n})=\tR$ and $\lm(S_0)=\tS$, so $\tC_n=\lm(C_n)$ and
$\tD_n=\lm(D_n)$.

The classification theorem is as follows.
\begin{theorem}
 Let $G$ be a finite subgroup of $SO_3$.  Then $G$ is conjugate to one
 of the groups $G_1$, $G_2$, $G_3$, $\tC_n$ or $\tD_n$ (for some $n$).
\end{theorem}

The rest of this section will constitute the proof.
\begin{definition}
 If $g\in SO_3\sm\{1\}$ (so $g$ is a nontrivial rotation of $\R^3$)
 then the \emph{poles} of $g$ are the two unit vectors on the axis of
 rotation of $g$, or in other words the two unit vectors $v$ such that
 $g(v)=v$.  If $G$ is a finite subgroup of $SO_3$, the poles of $G$
 are the poles of all the elements $g\in G$ such that $g\neq 1$.
\end{definition}
\begin{remark}
 A unit vector $v$ is a pole of $G$ iff $g(v)=v$ for some $g\in G$
 with $g\neq 1$, iff the group $\stab_G(v)=\{g\in G\st g(v)=v\}$ is
 not the trivial group.
\end{remark}
\begin{remark}
 Often it is natural to describe a rotation $g$ as being a rotation
 around $w$ for some non-unit vector $w$.  We will write
 $\wh=w/\|w\|$, which is a positive multiple of $w$ and is a unit
 vector.  We call this the \emph{normalisation} of $w$.  Clearly $g$
 is also a rotation around $\wh$.
\end{remark}
\begin{remark}
 Let $v$ be any unit vector.  Rotations around $v$ are determined by
 their angles in just the same way that rotations of the plane are
 determined by their angles.  Thus, we can analyse the finite groups
 of rotations around $v$ in the same way that we analysed the finite
 subgroups of $SO_2$; we find that they are all cyclic.  In
 particular, if $v$ is a pole of $G$ we find that $\stab_G(v)$ is a
 nontrivial finite group of rotations about $v$, so it is cyclic of
 order $d$ for some $d>1$.  We call $d$ the \emph{degree} of $v$.
\end{remark}
\begin{definition}
 From now on we fix a finite subgroup $G\leq SO_3$, and we let $P$ be
 the set of poles of $G$.  We write $n=|G|$ and $p=|P|$.
\end{definition}

\begin{lemma}
 The action of $G$ on $\R^3$ preserves the set $P$.
\end{lemma}
\begin{proof}
 Suppose that $v\in P$ and $h\in G$; we must show that $h(v)\in P$.
 As $v\in P$ there is some nontrivial element $g\in G$ with $g(v)=v$.
 Thus $g':=hgh^{-1}$ is another nontrivial element of $g$ and
 $g'(h(v))=hgh^{-1}h(v)=hg(v)=h(v)$.  Thus $h(v)$ is a pole of $g'$
 and thus lies in $P$, as required.
\end{proof}

Our main technique will be to study the orbits of the action of $G$ on
$P$.  As a warm-up we consider the case where there are only two
poles.
\begin{theorem}\label{thm-two-poles}
 If $G$ is a finite subgroup of $SO_3$ and the set $P$ of poles has
 order $2$ then $G$ is cyclic.
\end{theorem}
\begin{proof}
 Choose $v\in P$.  Clearly $-v\in P$ also, and as $|P|=2$ we must have
 $P=\{v,-v\}$.  Given $g\in G\sm\{1\}$, we know that $g$ is a rotation
 about some axis $L$, and we choose a unit vector $w$ on $L$.  Clearly
 $w$ is a pole of $G$, so $w\in P=\{v,-v\}$, so $v=w$ or $v=-w$.
 Either way we see that $gv=v$; thus $v$ is fixed under $G$.  Now let
 $U$ be the plane perpendicular to $v$.  If $u\in U$ then
 $\ip{gu,v}=\ip{gu,gv}=\ip{u,v}=0$, so $gu\in U$ also.  Thus, $G$ is a
 finite subgroup of the group of rotations of the plane $U$, which is
 isomorphic to $SO_2$.  We know from
 Proposition~\ref{prop-SO-two-subgroups} that a finite subgroup of
 $SO_2$ is cyclic, so $G$ is cyclic.
\end{proof}

We next record how the orbits work for the groups we already know
about.

\begin{itemize}
 \item[(a)] The nontrivial elements of the group $G_1=\Dir(\Tet)$ are
  the rotations through $\pm 2\pi/3$ about the vertices of the
  tetrahedron and the rotations through $\pi$ about the midpoints of
  the edges.  Let $v_1,\ldots,v_4$ be the vertices, so the two unit
  vectors on the line through $v_i$ are $\pm\vh_i$.  If $i<j$ then
  there is an edge joining $v_i$ to $v_j$, whose midpoint is
  $v_{ij}:=(v_i+v_j)/2$.  We thus have
  \begin{align*}
   P = \{ & \vh_1,\vh_2,\vh_3,\vh_4, \\
          & -\vh_1,-\vh_2,-\vh_3,-\vh_4,\\
          & \vh_{12},\vh_{13},\vh_{14},\vh_{23},\vh_{24},\vh_{34}\}.
  \end{align*}
  You might think that we should include $-\vh_{12}$ (for example) but
  in fact it is there already.  The centre of the tetrahedron is
  $(v_1+v_2+v_3+v_4)/4$ but this is just the origin, so
  $-(v_1+v_2)=v_3+v_4$.  It follows that $-\vh_{12}=\vh_{34}$, which
  is already in the list.  Similarly, for any $i<j$ we can let $k$ and
  $l$ be the other two numbers in the range $\{1,2,3,4\}$ and we have
  $-\vh_{ij}=\vh_{kl}$.

  It is not hard to see that we can send any vertex of $\Tet$ to any
  other vertex by the action of $G_1$, so the set
  $\{v_1,v_2,v_3,v_4\}$ is an orbit of the action.  This implies that
  $\{-v_1,-v_2,-v_3,-v_4\}$ is also an orbit.  Similarly, as we can
  send any edge to any other edge by the action of $G_1$, we see that
  $\{\vh_{12},\vh_{13},\vh_{14},\vh_{23},\vh_{24},\vh_{34}\}$ is
  another orbit.  Thus there are two orbits of poles of degree $3$ and
  one orbit of poles of degree $2$, making three orbits altogether.
 \item[(b)] The nontrivial elements of the group $G_2=\Dir(\Cube)$
  are:
  \begin{itemize}
  \item rotations through $\pi$ about midpoints of edges of the cube
  \item rotations through $\pm 2\pi/3$ about vertices
  \item rotations through $\pi$ or $\pm\pi/2$ about centres of faces.
  \end{itemize}
  Here the negative of the midpoint of an edge is the midpoint of the
  opposite edge, the negative of a vertex is the opposite vertex, and
  the negative of the centre of a face is the centre of the opposite
  face.  Thus, we do not need to worry about negatives, and the poles
  are the normalisations of midpoints of edges, vertices, and centres
  of faces.  There are $12$ edges, $8$ vertices and $6$ faces so we
  have $12+8+6=26$ poles altogether.  As we can move any edge to any
  other edge by the action of $G_2$, we see that the first $12$ poles
  form an orbit.  As we can move any vertex to any other vertex, we
  see that the next $8$ poles form an orbit.  As we can move any face
  to any other face, we see that the last $6$ poles form an orbit.
  Thus we have one orbit consisting of $12$ poles of degree $2$, one
  orbit consisting of $8$ poles of degree $3$, and one orbit
  consisting of $6$ poles of degree $4$.  Again we have $3$ orbits
  altogether.
 \item[(c)] The nontrivial elements of $G_3=\Dir(\Dodec)$ are:
  \begin{itemize}
  \item rotations through $\pi$ about midpoints of edges of the
   dodecahedron
  \item rotations through $\pm 2\pi/3$ about vertices
  \item rotations through $\pm 2\pi/5$ or $\pm 4\pi/5$ about centres
   of faces.
  \end{itemize}
  Just as in the case of the cube, we do not have to worry about
  negatives.  There are $30$ edges, $20$ vertices and $12$ faces.  We
  thus have one orbit consisting of $30$ poles of degree $2$, one
  orbit consisting of $20$ poles of degree $3$, and one orbit
  consisting of $12$ poles of degree $5$.
 \item[(d)] Now consider the cyclic group $\tC_n$.  All the nontrivial
  elements are rotations around the unit vector $w=(0,0,1)$, so
  $P=\{w,-w\}$.  All elements of the group send $w$ to $w$ and $-w$ to
  $-w$, so $\{w\}$ and $\{-w\}$ are two separate orbits.  There are
  thus $2$ orbits, each consisting of $1$ pole of degree $n$.
 \item[(e)] Finally, consider the dihedral group $\tD_n$.  The poles
  of the elements $R^i$ (for $i=1,\ldots,n-1$) are just $w$ and $-w$
  again.  Now consider the points
  \begin{align*}
   v_k &= (\cos(2k\pi/n),\sin(2k\pi/n),0) \\
   u_k &= (\cos((2k+1)\pi/n),\sin((2k+1)\pi/n),0).
  \end{align*}
  These are defined for all $k\in\Z$ but $v_{k+n}=v_k$ and
  $u_{k+n}=u_k$ so there are really only $n$ $v$'s and $n$ $u$'s.
  We show the case $n=5$ below.
  \begin{center}
   \begin{tikzpicture}[scale=2] % [70]{-1.5}{1.5}{-1.2}{1.3}
     \draw (0,0) circle(1);
     \fill (1,0) \blob;
     \fill (0.81,0.59) \blob;
     \fill (0.31,0.95) \blob;
     \fill (-0.31,0.95) \blob;
     \fill (-0.81,0.59) \blob;
     \fill (-1,0) \blob;
     \fill (-0.81,-0.59) \blob;
     \fill (-0.31,-0.95) \blob;
     \fill (0.31,-0.95) \blob;
     \fill (0.81,-0.59) \blob;
     \draw (1,0) -- (0.31,0.95) -- (-0.81,0.59) --
            (-0.81,-0.59) -- (0.31,-0.95) -- (1,0);
     \draw[dotted] (1,0) -- (-1,0);
     \draw[dotted] (0.31,0.95) -- (-0.31,-0.95);
     \draw[dotted] (-0.81,0.59) -- (0.81,-0.59);
     \draw[dotted] (-0.81,-0.59) -- (0.81,0.59);
     \draw[dotted] (0.31,-0.95) -- (-0.31,0.95);
     \draw ( 1.04, 0.00) node[anchor=west]       {$v_0$};
     \draw ( 0.31, 1.00) node[anchor=south]      {$v_1$};
     \draw (-0.86, 0.59) node[anchor=south east] {$v_2$};
     \draw (-0.86,-0.59) node[anchor=north east] {$v_3$};
     \draw ( 0.31,-1.00) node[anchor=north]      {$v_4$};
     \draw ( 0.86, 0.59) node[anchor=south west] {$u_0$};
     \draw (-0.31, 1.00) node[anchor=south]      {$u_1$};
     \draw (-1.04, 0.00) node[anchor=east]       {$u_2$};
     \draw (-0.31,-1.00) node[anchor=north]      {$u_3$};
     \draw ( 0.86,-0.59) node[anchor=north west] {$u_4$};
    \end{tikzpicture} 
  \end{center}
  It is geometrically clear that a half twist around $v_k$ or $u_k$
  preserves the standard $n$-gon $X_n$ and thus lies in $\tD_n$, so
  the $u$'s and $v$'s are poles of $\tD_n$.  We also need to think
  about the points $-u_i$ and $-v_j$.  When $n$ is odd (as in the
  above picture) we see that $-u_i=v_k$ for some $k$ and $-v_j=u_l$
  for some $l$.  If $n$ is even we see instead that $-u_i$ has the
  form $u_k$ and $-v_j$ has the form $v_l$.  Either way, we get no new
  poles.  Also, any symmetry of $X_n$ sends vertices to vertices, and
  we can move any vertex to any other vertex, so the $v$'s form an
  orbit.  Similarly, the $u$'s form an orbit.  Thus, the $u$'s and
  $v$'s give $2$ orbits, each consisting of $n$ poles of degree $2$.
  Moreover, we have $R^k(w)=w$ and $S(w)=-w$ so $\{w,-w\}$ is an orbit
  consisting of $2$ poles of degree $n$.  Thus we again have three
  orbits altogether.
\end{itemize}

We now let $G$ be an arbitrary finite subgroup of $SO_3$.  Our next
task is to show that the number of poles and orbits for $G$ matches
one of the possibilities discussed above.  Our main tool is the orbit
counting theorem:
\begin{theorem}
 Let $H$ be a finite group that acts on a finite set $X$.  For each
 $h\in H$ put $\Fix(h)=\{x\in X\st hx=x\}$, the set of fixed points of
 $h$.  Then the number of orbits of $H$ in $X$ is
 $|H|^{-1}\sum_{h\in H}|\Fix(h)|$, or in other words the average
 number of fixed points of an element of $H$. \qed
\end{theorem}

\begin{proposition}
 Let $G$ be a nontrivial finite subgroup of $SO_3$, and let $P$ be the
 set of poles of $G$.  Put $n=|G|$ and $p=|P|$, and let $m$ be the
 number of orbits of $G$ in $P$.  Let $d_k$ be the degree of the poles
 in the $k$'th orbit; we can order the orbits in such a way that
 $d_1\leq d_2\leq\ldots\leq d_m$.  Then
 \begin{align*}
  m &= (p+2n-2)/n \\
  p &= \sum_{k=1}^m n/d_k.
 \end{align*}
\end{proposition}
\begin{proof}
 The orbit counting theorem says that
 $m=n^{-1}\sum_{g\in G}|\Fix(g)|$.  If $g\neq 1$ then $\Fix(g)$
 consists of the two unit vectors on the axis of $g$, so
 $|\Fix(g)|=2$.  There are $n-1$ elements $g\in G$ with $g\neq 1$, so
 $\sum_{g\neq 1}|\Fix(g)|=2(n-1)=2n-2$.  In the remaining case $g=1$
 we have $\Fix(g)=P$ and thus $|\Fix(g)|=p$.  This means that
 $\sum_{g\in G}|\Fix(g)|=p+2n-2$ and thus $m=(p+2n-2)/n$.

 Next, choose a point $x_k$ in the $k$'th orbit for each $k$.  Then
 $\stab_G(x_k)$ has order $d_k$.  The orbit-stabiliser theorem says
 that $|G|=|\stab_G(x_k)||\orb_G(x_k)|$, so the size of the $k$'th
 orbit is $|G|/|\stab_G(x_k)|=n/d_k$.  As $P$ is the disjoint union of
 the orbits we see that $p=|P|$ is the sum of the orders of all the
 orbits, so $p=\sum_k n/d_k$, as claimed.
\end{proof}

\begin{proposition}\label{prop-numbers}
 With notation as above we have either
 \begin{itemize}
 \item[(1)] $m=3$, $d_1=2$, $d_2=d_3=3$ and $n=12$; or
 \item[(2)] $m=3$, $d_1=2$, $d_2=3$, $d_3=4$ and $n=24$; or
 \item[(3)] $m=3$, $d_1=2$, $d_2=3$, $d_3=5$ and $n=60$; or
 \item[(4)] $m=p=2$ and $d_1=d_2=n$; or
 \item[(5)] there is an integer $d\geq 2$ such that $m=3$, $n=2d$,
  $d_1=d_2=2$ and $d_3=d$.
 \end{itemize}
\end{proposition}
\begin{proof}
 First note that $d_k$ is the order of the stabiliser group of $x_k$.
 As $x_k$ is a pole, this stabiliser group is nontrivial, so
 $d_k\geq 2$.

 Next, we can rearrange the equation $m=(p+2n-2)/n$ as $m=2+(p-2)/n$.
 By assumption $G$ is a nontrivial group, and any nontrivial element
 has two poles, so $p\geq 2$, which implies that $(p-2)/n\geq 0$ and
 $m\geq 2$.

 Alternatively, we can rearrange to get $p=mn-2n+2$.  We also know
 that $p=\sum_{k=1}^m n/d_k$.  As each $d_k$ is at least $2$, each
 term in the on the right hand side is at most $n/2$, and there are
 $m$ terms, so $p\leq mn/2$.  After feeding this back into the
 equation $p=mn-2n+2$ we find that $mn/2\leq 2n-2<2n$ so $mn<4n$ so
 $m<4$.  As $m\geq 2$ and $m<4$ we must have $m=2$ or $m=3$.

 If $m=2$ then the equation $m=2+(p-2)/2$ implies that $p=2$.  The
 equation $p=\sum_k n/d_k$ now says that $2=n/d_1+n/d_2$.  As $d_k$
 divides $n$ for all $k$ the terms $n/d_1$ and $n/d_2$ are positive
 integers, so the only way their sum can be $2$ is if $n/d_1=n/d_2=1$,
 so $d_1=d_2=n$, so case~(4) holds.

 Now suppose instead that $m=3$.  The equation $m=2+(p-2)/n$ then
 simplifies to give $p=n+2$, and we can feed this into the equation
 $p=\sum n/d_k=n/d_1+n/d_2+n/d_3$ and rearrange to give
 \[ \frac{2}{n} = \frac{1}{d_1} + \frac{1}{d_2} + \frac{1}{d_3} - 1.
 \]
 Recall that $2\leq d_1\leq d_2\leq d_3$.  If the $d$'s are reasonably
 large then $1/d_1$, $1/d_2$ and $1/d_3$ will be small and so
 $1/d_1+1/d_2+1/d_3-1$ will be negative, which is absurd because $2/n$
 is certainly positive.  Thus, the $d$'s must be fairly small.  We can
 complete the proof by making this argument more precise.

 We first claim that $d_1=2$.  Indeed, if not then
 $3\leq d_1\leq d_2\leq d_3$, so $1/d_1$, $1/d_2$ and $1/d_3$ are all
 less than or equal to $1/3$ so $1/d_1+1/d_2+1/d_3-1\leq 3/3-1=0$,
 which contradicts the equation $2/n=(\sum 1/d_k)-1$.

 We thus have $d_1=2$ as claimed.  Suppose we also have $d_2=2$.  Then
 $2/n=1/2+1/2+1/d_3-1=1/d_3$, so $n=2d_3$.  We are thus in case~(5).

 Now suppose instead that $d_2>2$.  We claim that in fact $d_2=3$.
 Indeed, if not then $4\leq d_2\leq d_3$ so $1/d_2$ and $1/d_3$ are at
 most $1/4$, so $1/d_1+1/d_2+1/d_3-1\leq 1/2+1/4+1/4-1=0$, which
 contradicts the equation $2/n=(\sum 1/d_k)-1$.

 We thus have $d_1=2$ and $d_2=3$ and $d_3\geq 3$, so
 $2/n=1/2+1/3+1/d_3-1=1/d_3-1/6$.  If $d_3=3$ this gives
 $2/n=1/3-1/6=1/6$ so $n=12$ and we are in case~(1).  If $d_3=4$ then
 $2/n=1/4-1/6=1/12$ so $n=24$ and we are in case~(2).  If $d_3=5$ then
 $2/n=1/5-1/6=1/30$ so $n=60$ and we are in case~(3).  If $d_3\geq 6$
 then $2/n=1/d_3-1/6\leq 0$, which is absurd.
\end{proof}

\begin{proposition}\label{prop-class-tet}
 If case~(1) holds in Proposition~\ref{prop-numbers} then $G$ is
 conjugate to $G_1=\Dir(\Tet)$.
\end{proposition}
\begin{proof}
 Let $V$ be the third orbit, which has order $n/d_3=12/3=4$, so
 $V=\{v_1,v_2,v_3,v_4\}$ say.  Let $g$ be a one-third turn around
 $v_4$, which lies in $G$ because $v_4$ is a pole of degree $3$.
 Clearly $g$ gives a permutation of $\{v_1,v_2,v_3\}$.  The only way
 that a one-third turn can permute a set of three points is if they
 form an equilateral triangle perpendicular to the axis of rotation,
 with the centre of the triangle on the axis.
% As $g^3=1$ this permutation is either the identity or a $3$-cycle.
% If it were the identity then $v_1,\ldots,v_4$ would all be poles of
% $g$ but any rotation has only $2$ poles so this is impossible.  Thus
% $g$ must give a $3$-cycle on $\{v_1,v_2,v_3\}$.  This means that we
% can get from any of these points to any of these points by rotating
% around $v_4$, and
 It follows that the distances $d(v_1,v_4)$, $d(v_2,v_4)$
 and $d(v_3,v_4)$ are all the same.  Similarly, by rotating around
 $v_3$ we see that $d(v_1,v_3)=d(v_2,v_3)=d(v_4,v_3)$.  We can also
 rotate around $v_1$ or $v_2$ and we find that all the distances
 $d(v_i,v_j)$ (for $i\neq j$) are the same.  This means that $v_1$,
 $v_2$, $v_3$ and $v_4$ are the vertices of a regular tetrahedron $T$.
 As $G$ permutes these vertices, it is a subgroup of $\Dir(T)$, but
 $|G|=12=|\Dir(T)|$ so $G=\Dir(T)$.  Let $r$ be the distance from the
 origin to the vertices of the standard tetrahedron $\Tet$, and put
 $T'=rT$; it is not hard to see that $\Dir(T')=\Dir(T)=G$.  As $T'$ is
 a regular tetrahedron the same size as $\Tet$, we can choose an
 isometry $f\in SO_3$ with $f(\Tet)=T'$.  It follows that
 \[ \Dir(T')=\Dir(f(\Tet))=f\Dir(\Tet)f^{-1}=fG_1f^{-1}, \]
 so $G$ is conjugate to $G_1$.
\end{proof}

\begin{proposition}\label{prop-class-oct}
 If case~(2) holds in Proposition~\ref{prop-numbers} then $G$ is
 conjugate to $G_2=\Dir(\Oct)$.
\end{proposition}
\begin{proof}
 Let $V$ be the third orbit, which has order $n/d_3=24/4=6$.  If
 $v\in V$ then $v$ is a pole of degree $4$ so $-v$ is also a pole of
 degree $4$.  The poles in the other two orbits have degree $2$ or
 $3$, so we must have $-v\in V$.  It follows that $V$ has the form
 $\{v_1,v_2,v_3,-v_1,-v_2,-v_3\}$ for some $v_1$, $v_2$ and $v_3$.
 Let $g$ be a quarter turn around $v_3$, which lies in $G$ because
 $v_3$ is a pole of degree $4$.  Clearly $g(v_3)=v_3$ and
 $g(-v_3)=-v_3$ so $g$ must permute the remaining vertices
 $\{v_1,v_2,-v_1,-v_2\}$.  The only way that a quarter turn can
 permute a set of four points is if they form a square perpendicular
 to the axis of rotation.  It follows that the distances $d(v_3,v_1)$,
 $d(v_3,v_2)$, $d(v_3,-v_1)$ and $d(v_3,-v_2)$ are all the same, equal
 to $r$ say.  We also have
 \[ d(-v_3,v_1)=\|v_1-(-v_3)\|=
    \|v_3+v_1\|=\|v_3-(-v_1)\|=d(v_3,-v_1)=r,
 \]
 and by the same method we find that
 $d(-v_3,-v_1)=d(-v_3,v_2)=d(-v_3,-v_2)=r$.  We can also rotate about
 $v_1$ or $v_2$ instead, and we find that
 \[
  d(\pm v_1,\pm v_2) = d(\pm v_1,\pm v_3) = 
  d(\pm v_2,\pm v_1) = d(\pm v_2,\pm v_3) = r.
 \]
 Using this, we find that the points in $V$ are the vertices of a
 regular octahedron $O$, so $G\leq\Dir(O)$, but $|G|=24=|\Dir(O)|$ so
 $G=\Dir(O)$.  By the same method as in the previous proposition we
 find that $G$ is conjugate to $G_2$.
\end{proof}

\begin{proposition}\label{prop-class-icos}
 If case~(3) holds in Proposition~\ref{prop-numbers} then $G$ is
 conjugate to $G_3=\Dir(\Icos)$.
\end{proposition}
\begin{proof}
 Let $V$ be the third orbit, which has order $n/d_3=60/5=12$.  We will
 show that the points in $V$ are the vertices of an icosahedron.
 
 If $v\in V$ then $v$ is a pole of degree $5$ so $-v$ is also a pole
 of degree $5$.  The poles in the other two orbits have degree $2$ or
 $3$, so we must have $-v\in V$.  Put $V'=V\sm\{v,-v\}$ so $|V'|=10$,
 let $g$ be a one-fifth turn around $v$, and let $H$ be the subgroup
 of order $5$ generated by $g$.  We see geometrically that for any
 $x\in\R^3$ that does not lie on the axis of $g$, the orbit $Hx$ has
 order $5$.  None of the points in $V'$ lie on the axis, so $V'$ must
 split into two orbits of order $5$ under the action of $H$, say
 $V'=W_1\cup W_2$.  All the points in $W_1$ lie at the same distance
 (say $r_1$) from $v$, and all the points in $W_2$ lie at some other
 distance $r_2$ from $v$.  We may assume that $r_1\leq r_2$ (otherwise
 rename $W_1$ as $W_2$ and $W_2$ as $W_1$).  We will actually assume
 that $r_1<r_2$; one can check by going through the following argument
 more carefully that the equation $r_1=r_2$ leads to a contradiction.

 Now let $u$ be any point in $V$.  As $V$ is an orbit there exists an
 element $h\in G$ with $hv=u$.  For any point $w'\in W_1$ we then have
 $d(u,hw')=d(hv,hw')=d(v,w')=r_1$.  Thus all the points in $hW_1$ lie
 at distance $r_1$ from $u$, and similarly the points in $hW_2$ lie at
 distance $r_2$.  This means that $u$ has $5$ nearest neighbours, and
 they all lie at distance $r_1$ from $u$.

 Choose a point $w\in W_1$.  I claim that $-w\in W_2$.  Indeed, $-w$
 is certainly a pole of degree $5$ and $-w\neq\pm v$ so
 $-w\in V'=W_1\cup W_2$.  It will thus be enough to show that
 $-w\not\in W_1$.  If $-w\in W_1$ we have $-w=g^kw$ for some $k$,
 which means that $-w=(-1)^5w=g^{5k}w=w$ (because $g^5=1$).  This
 means that $w=0$, which is impossible as $w$ is a unit vector.  We
 must therefore have $-w\in W_2$ as required.  Note that
 $d(-v,-w)=d(v,w)=r_1$ and $d(-v,w)=d(v,-w)=r_2$, so all the points in
 $W_2$ lie at distance $r_1$ from $-w$ and all the points in $W_1$ lie
 at distance $r_2$ from $-w$.

 We thus have a picture like this.  The points in $W_1$ are the
 vertices of the top pentagon, and the points in $W_2$ are the
 vertices of the bottom pentagon.
 \begin{center}
  \begin{tikzpicture}[scale=2] % [50]{-2}{2}{-1.7}{2.3}
   % rest
   \draw (0.947,-0.531) -- (0,-1.11);
   \draw (0.947,-0.531) -- (0.585,0.4);
   \draw (0.947,-0.531) -- (0.947,0.531);
   \draw[dashed] (0,-1.11) -- (0.585,-0.4);
   \draw (0,-1.11) -- (0,-0.611);
   \draw (0,-1.11) -- (-0.947,-0.531);
   \draw[dashed] (0,-1.11) -- (-0.585,-0.4);
   \draw[dashed] (0.585,-0.4) -- (0.947,0.531);
   \draw[dashed] (0.585,-0.4) -- (0,0.611);
   \draw (0,-0.611) -- (0.585,0.4);
   \draw (0,-0.611) -- (-0.585,0.4);
   \draw (0.585,0.4) -- (0,1.11);
   \draw (-0.947,-0.531) -- (-0.585,0.4);
   \draw (-0.947,-0.531) -- (-0.947,0.531);
   \draw[dashed] (-0.585,-0.4) -- (0,0.611);
   \draw[dashed] (-0.585,-0.4) -- (-0.947,0.531);
   \draw (-0.585,0.4) -- (0,1.11);
   \draw (0.947,0.531) -- (0,1.11);
   \draw[dashed] (0,0.611) -- (0,1.11);
   \draw (0,1.11) -- (-0.947,0.531);
   \fill (0,-1.11) \blob;
   \fill (0,1.11) \blob;

   \begin{scope}[very thick]
    % bottom pentagon
    \draw[dashed] (0.947,-0.531) -- (0.585,-0.4);
    \draw (0.947,-0.531) -- (0,-0.611);
    \draw[dashed] (0.585,-0.4) -- (-0.585,-0.4);
    \draw (0,-0.611) -- (-0.947,-0.531);
    \draw[dashed] (-0.947,-0.531) -- (-0.585,-0.4);
    \fill (0.947,-0.531) \blob;
    \fill (0.585,-0.4) \blob;
    \fill (0,-0.611) \blob;
    \fill (-0.947,-0.531) \blob;
    \fill (-0.585,-0.4) \blob;

    % top pentagon
    \draw (0.585,0.4) -- (-0.585,0.4);
    \draw (0.585,0.4) -- (0.947,0.531);
    \draw (-0.585,0.4) -- (-0.947,0.531);
    \draw[dashed] (0.947,0.531) -- (0,0.611);
    \draw[dashed] (0,0.611) -- (-0.947,0.531);
    \fill (0.585,0.4) \blob;
    \fill (-0.585,0.4) \blob;
    \fill (0.947,0.531) \blob;
    \fill (0,0.611) \blob;
    \fill (-0.947,0.531) \blob;
   \end{scope}

   \draw (0,1.2) node[anchor=south] {$v$};
   \draw (0,-1.2) node[anchor=north] {$-v$};
   \draw ( 1.05, 0.541) node[anchor=west] {$w$};
   \draw (-1.05, 0.541) node[anchor=east] {$W_1$};
   \draw (-1.05,-0.541) node[anchor=east] {$-w$};
   \draw ( 1.05,-0.541) node[anchor=west] {$W_2$};

   \draw (-0.5,-0.9) node[anchor=north east] {$r_1$};
   \draw ( 0.5,-0.9) node[anchor=north west] {$r_1$};
   \draw (-0.5, 0.9) node[anchor=south east] {$r_1$};
   \draw ( 0.5, 0.9) node[anchor=south west] {$r_1$};
  \end{tikzpicture}
 \end{center}
 Because the nearest neighbours of any vertex lie at distance $r_1$
 from that vertex, we see that all the edges have length $r_1$ so we
 have a regular icosahedron, which we call $I$.  Clearly
 $G\leq\Dir(I)$ but $|G|=60=|\Dir(I)|$, so $G=\Dir(I)$.  It follows as
 usual that $G$ is conjugate to $G_3$.
\end{proof}

\begin{proposition}\label{prop-class-cyclic}
 If case~(4) holds in Proposition~\ref{prop-numbers} then $G$ is
 conjugate to $\tC_n$.
\end{proposition}
\begin{proof}
 This is essentially Theorem~\ref{thm-two-poles}.
\end{proof}

\begin{proposition}\label{prop-class-dihedral}
 If case~(5) holds in Proposition~\ref{prop-numbers} and $d>2$ then
 $G$ is conjugate to $\tD_d$.
\end{proposition}
\begin{proof}
 Let $P_1$, $P_2$ and $P_3$ be the three orbits in $P$.  Choose $w$ in
 $P_3$, so $w$ has degree $d$.  Then $-w$ also has degree $d$ and the
 poles in the first two orbits have degree $2$ so $-w\in P_3$.  We
 also have $|P_3|=n/d_3=2d/d=2$, so $P_3=\{w,-w\}$.  Let $g$ be the
 rotation through $2\pi/d$ around $w$, and let $U$ be the plane
 through the origin perpendicular to $w$.

 Now suppose that $v\in P_2$, and let $h$ be the half turn around $v$,
 which lies in $G$ because $v$ is a pole of degree $2$.  As $P_3$ is
 an orbit we have $hP_3=P_3$ so $hw=\pm w$.  The only way that a half
 twist around $v$ can send $w$ to $-w$ is if $v$ is perpendicular to
 $w$.  Thus all the points in $P_2$ lie in the plane $U$, and
 similarly all the points in $P_1$ lie in $U$.

 Note also that the points $v,gv,\ldots,g^{d-1}v$ are all different
 and all lie in $P_2$, and $|P_2|=n/d_2=2d/2=d$, so we must have
 $P_2=\{v,gv,\ldots,g^{d-1}v\}$.  We can choose coordinate so that
 $w=(0,0,1)$ and $v=(1,0,0)$.  Then $U$ is the $xy$-plane and $P_2$
 consists of the vertices of the standard polygon $X_d$, with polar
 coordinates $[1,2k\pi/d]$.

 By a similar argument, the set $P_1$ consists of $d$ equally spaced
 points on the unit circle in the $xy$-plane, and these points are all
 different from the points $g^kv$.  Thus there must be precisely one
 of the the points in $P_1$ lying in the gap between $v$ and $gv$;
 call this point $u$.  Let $\al$ be the angle between $u$ and $v$, and
 let $\bt$ be the angle between $u$ and $gv$.  The half twist around
 $u$ must send $P_2$ to itself, and clearly this can only happen if
 $v$ and $gv$ are exchanged, and this means that $\al=\bt$.  As
 $\al+\bt$ is the angle between $v$ and $gv$, which is $2\pi/d$, we
 have $\al=\bt=\pi/d$.  We also have $P_1=\{u,gu,\ldots,g^{d-1}u\}$,
 which is the set of points with polar coordinates $[1,(2k+1)\pi/d]$.
 The group $G$ consists of the rotations $g^k$ together with the
 half-twists around the points in $P_1$ and $P_2$, so $G=\tD_d$.  This
 refers to $\tD_d$ as defined with respect to our new coordinate
 system: if $\tD_d$ is defined using the original coordinate system,
 then $G$ is merely conjugate to $\tD_d$.
\end{proof}

\begin{proposition}\label{prop-class-dihedral-two}
 If case~(5) holds in Proposition~\ref{prop-numbers} and $d=2$ then
 $G$ is conjugate to $\tD_2$.
\end{proposition}
\begin{proof}
 In this case we have $m=3$, $d_1=d_2=d_3=2$ and $|G|=n=4$.  Let $P_i$
 be the $i$'th orbit, so $|P_i|=n/d_i=2$, so $P_i=\{v_i,w_i\}$, say.
 Let $g_i$ be a half twist around $v_i$, which lies in $G$ because
 $d_i=2$.  Note that every element of $G$ sends $P_i=\{v_i,w_i\}$ to
 itself and $g_i$ sends $v_i$ to $v_i$ so it must send $w_i$ to
 $w_i$.  Thus, $w_i$ is a fixed point of $g_i$ and the only two fixed
 points are $v_i$ and $-v_i$ so we must have $w_i=-v_i$.

 Now consider $g_2v_1$.  As the orbit of $v_1$ is $\{v_1,-v_1\}$ we
 must have $g_2v_1=\pm v_1$ and $v_1$ is not one of the fixed points
 of $g_2$ so we must have $g_2v_1=-v_1$.

 You should be able to see from the following picture that $g_2b=-b$
 if and only if $b$ is perpendicular to $v_2$.
 \begin{center}
  \begin{tikzpicture}[scale=2] % [50]{-1.3}{1.8}{-1.2}{1.5}
   \draw[->] (0,0) -- (0,1);
   \draw (-1,0) -- (1,0);
   \draw (0.6,0.4) -- (0,0);
   \draw (-0.6,0.4) -- (0,0);
   \draw (-0.6,-0.4) -- (0,0);
   \fill (0,0) \blob;
   \fill (1,0) \blob;
   \fill (0.6,0.4) \blob;
   \fill (-0.6,0.4) \blob;
   \fill (-1,0) \blob;
   \fill (-0.6,-0.4) \blob;
   \draw (1.1,0) node[anchor=west] {$b$};
   \draw (0.65,0.45) node[anchor=south west] {$a$};
   \draw (0,1.1) node[anchor=south] {$v_2$};
   \draw (-0.65,0.45) node[anchor=south east] {$g_2a$};
   \draw (-1.1,0) node[anchor=east] {$-b=g_2b$};
   \draw (-0.65,-0.45) node[anchor=north east] {$-a$};
  \end{tikzpicture}
 \end{center}
 As $g_2v_1=-v_1$, the vectors $v_1$ and $v_2$ must be orthogonal to
 each other.  By a similar argument, they are both orthogonal to
 $v_3$.  Thus $G$ consists of the identity together with half twists
 around three orthogonal axes, whereas $\tD_2$ consists of the
 identity together with half-twists about the standard $x$, $y$ and
 $z$-axes.  It follows that $G$ is conjugate to $\tD_2$.
\end{proof}

\section{The Sylow theorems}

Let $G$ be a finite group of order $n$, say.  Lagrange's theorem says
that if $H$ is a subgroup of $G$ and $|H|=d$, then $d$ is a divisor of
$n$.  It is natural to ask whether the converse is true: given a
divisor $d$ of $n$, can we find a subgroup $H\leq G$ such that
$|H|=d$?  The answer is no in general; for example one can check that
the group $A_4$ has order $12$ but there is no subgroup of order $6$.
However, if $d$ is a power of a prime number, then the answer turns
out to be yes, and in fact we can say a great deal more.  This follows
from the Sylow theorems, which we will prove in this section.

Fix a finite group $G$ and a prime $p$.  We can write $|G|$ in the
form $p^vm$, where $p$ does not divide $m$.  A \emph{Sylow
$p$-subgroup} of $G$ is a subgroup $P\leq G$ such that $|P|=p^v$.  We
write $n_p$ for the number of Sylow $p$-subgroups of $G$ (which
\emph{a priori} could be zero).

\begin{theorem}\label{thm-sylow}
 \begin{itemize}
 \item[(a)] There is at least one Sylow $p$-subgroup, so $n_p>0$.
 \item[(b)] Moreover, $n_p$ divides $m$ and is congruent to $1$ mod
  $p$.
 \item[(c)] Any two Sylow $p$-subgroups are conjugate.
 \item[(d)] Any $p$-subgroup of $G$ is contained in a Sylow
  $p$-subgroup.
 \end{itemize}
\end{theorem}

Before giving the proof, we outline some applications.  These will be
discussed in more detail and extended in the next section.

\begin{example}
 Let $G$ be a group of order $35=5\tm 7$.  Then $n_5$ divides $7$ so
 $n_5=1$ or $n_5=7$, but also $n_5=1\pmod{5}$ so we must have $n_5=1$.
 Thus, there is precisely one subgroup $P\leq G$ with $|P|=5$.
 Moreover, we know that $n_7$ divides $5$ and $n_7=1\pmod{7}$ so
 $n_7=1$, so there is a unique subgroup $Q\leq G$ of order $7$.  Using
 the fact that $P$ is unique we see that it is normal in $G$; this
 will be explained in Proposition~\ref{prop-sylow-normal} below.
 Similarly, $Q$ is normal.  The order of $P\cap Q$ divides $|P|=5$ and
 also divides $|Q|=7$, so $|P\cap Q|=1$ so $P\cap Q=\{1\}$.  Using
 this and the fact that $P$ and $Q$ are normal and $|G|=|P||Q|$ one
 can check that $G\simeq P\tm Q\simeq C_5\tm C_7$.  Thus, any group
 of order $35$ is isomorphic to $C_5\tm C_7$.
\end{example}

\begin{example}
 Recall that a group $G$ is \emph{simple} if the only normal subgroups
 of $G$ are $\{1\}$ and $G$.  Let $G$ be a simple group of order
 $60$.  We'll outline a proof that $G$ must be isomorphic to $A_5$.
 Note that $60=2^2\tm 15$ so $n_2$ divides $15$ by part~(b) of the
 Theorem, so $n_2\in\{1,3,5,15\}$.  If $n_2=1$ then the unique Sylow
 $2$-subgroup is normal, contradicting the simplicity of $G$.  A
 slightly more complicated argument shows that $n_2$ cannot be $3$
 either.  Indeed, $G$ acts by conjugation on the set of Sylow
 $2$-subgroups, giving a homomorphism $\phi\:G\xra{}S_{n_2}$.  This is
 nontrivial, because all Sylow $2$-subgroups are conjugate, by
 part~(c) of the Theorem.  This means that $\ker(\phi)\neq G$ and
 $\ker(\phi)$ is a normal subgroup of $G$ so we must have
 $\ker(\phi)=\{1\}$.  This means that $\phi$ is injective, so
 $|G|\leq|S_{n_2}|$, so $n_2!\geq 60$.  As $4!=24$ and $5!=120$ the
 equation $n_2!\geq 60$ is equivalent to $n_2\geq 5$.  We already know
 that $n_2\in\{1,3,5,15\}$ so $n_2=5$ or $n_2=15$.  In fact the case
 $n_2=15$ cannot occur (although we will not prove this here) so
 $n_2=5$.  We thus have an injective homomorphism $\phi\:G\xra{}S_5$.
 As $\phi$ is injective, the image $\phi(G)$ has order $60$, and one
 can check that $A_5$ is the only subgroup of $S_5$ of order $60$, so
 $\phi$ gives an isomorphism $G\simeq A_5$.
\end{example}

\begin{proof}[Proof of~(a)]
 Let $\CX$ be the set of all subsets $X\sse G$ such that $|X|=p^v$.
 In general, any set of order $N$ has $\bcf{N}{M}$ subsets of order
 $M$, so $|\CX|=\bcf{p^vm}{p^v}$.  We claim that this number is not
 divisible by $p$.  To see this, put $q=p^vm-p^v=p^v(m-1)$ and note
 that $\bcf{p^vm}{p^v}=(p^vm)!/(p^v!\,q!)$.  We also have
 \[ (p^vm)!/q!=(q+1)(q+2)\ldots(q+p^v) \]
 so
 \[ \bcf{p^vm}{p^v} =
    \frac{1}{q+1}\frac{2}{q+2}\cdots\frac{p^v}{q+p^v} =
    \prod_{j=1}^{p^v} \frac{j}{q+j}.
 \]
 Now let $w_j$ be the largest number such that $j$ is divisible by
 $p^{w_j}$ (for $j=1,\ldots,p^v$).  As $j\leq p^v$ we must have
 $w_j\leq v$ so the number $q=p^v(m-1)$ is also divisible by
 $p^{w_j}$.  It follows that $q+j$ is divisible by $p^{w_j}$ as well,
 so all the $p$'s on the top in our equation for $\bcf{p^vm}{p^v}$ are
 cancelled out by $p$'s on the bottom.  It follows that
 $|\CX|\neq 0\pmod{p}$, as claimed.

 Now let $G$ act on $\CX$ by $gX=\{gx\st x\in X\}$, and divide $\CX$
 into orbits under this action, say $\CX=\CX_1\cup\ldots\cup\CX_k$.
 Orbits are always disjoint so $|\CX|=\sum_j|\CX_j|$.  If each of the
 numbers $|\CX_j|$ were divisible by $p$, then $|\CX|$ would also be
 divisible by $p$, contrary to what we just proved.  Thus we can
 choose $j$ such that the number $m':=|\CX_j|$ is not divisible by
 $p$.  Choose an element $X\in\CX_j$, so (by the definition of $\CX$)
 $X$ is a subset of $G$ with $p^v$ elements.  Put
 $P=\stab_G(X)=\{g\in G\st gX=X\}$.  The orbit-stabiliser theorem says
 that $|G|=|\stab_G(X)||\orb_G(X)|$, or in other words $p^vm=|P|m'$.
 As $p^v$ divides $|P|m'$ and $p$ does not divide $m'$ we see that
 $p^v$ divides $|P|$.  Next, fix an element $x_0\in X$.  For each
 $g\in P$ we have $gx_0\in gX=X$ so $Px_0\sse X$, so
 $p^v=|X|\geq|Px_0|=|P|$.  As $p^v$ divides $|P|$ and $|P|\leq p^v$ we
 must have $|P|=p^v$, so $P$ is a Sylow $p$-subgroup of $G$.
\end{proof}

For the remaining parts we need the following lemma.  Recall that a
\emph{$p$-group} is a group whose order is a power of the prime $p$.

\begin{lemma}\label{lem-p-fixed}
 Let $P$ be a finite $p$-group, and let $X$ be a set with an action of
 $P$.  Put
 \[ \Fix(P)=\Fix(P,X)=\{x\in X\st gx=x \text{ for all } g\in P\}. \]
 Then $|\Fix(P)|=|X|\pmod{p}$.  In particular, if $|X|\neq 0\pmod{p}$
 then $\Fix(P)\neq\emptyset$.
\end{lemma}
\begin{proof}
 The order of $P$ is $p^v$ for some $v\geq 0$.  Divide $X$ into orbits
 and list them in order of size, say $X=X_1\cup\ldots\cup X_k$ with
 $|X_1|\leq|X_2|\leq\ldots\leq|X_k|$.  As each set $X_j$ is an orbit,
 its order divides $|P|=p^v$, so $|X_j|=p^{w_j}$ for some $w_j$ with
 $0\leq w_1\leq w_2\leq\ldots\leq w_k\leq v$.  For some $r$ (possibly
 $r=0$) we have $w_j=0$ when $1\leq j\leq r$ and $w_j>0$ when $j>r$.
 We have
 \[ |X| = \sum_{j=1}^k |X_j| = \sum_{j=1}^r 1 + \sum_{j=r+1}^k p^{w_j}
        = r + \sum_{j>r} p^{w_j} = r \pmod{p}.
 \]
 On the other hand, an eement $x\in X$ lies in $\Fix(P)$ if and only
 if the orbit $Px$ consists of the single element $x$, so $|\Fix(P)|$
 is the number of orbits of size $1$, which is $r$.  Thus
 $|X|=|\Fix(P)|\pmod{p}$, as claimed.

 Now suppose that $|X|\neq 0\pmod{p}$.  Then
 $|\Fix(P)|\neq 0\pmod{p}$, so $|\Fix(P)|\neq 0$, so
 $\Fix(X)\neq\emptyset$.
\end{proof}

\begin{proof}[Proof of~(c) and~(d)]
 Let $P$ be a Sylow $p$-subgroup of $G$, and let $Q$ be any
 $p$-subgroup of $G$.  Note that $|P|=p^v$ and $|Q|=p^w$ for some
 $w\leq v$.  As usual we write $G/P$ for the set of right cosets of
 $P$, so a typical element of $G/P$ has the form $xP$ for some
 $x\in G$.  Note that $|G/P|=|G|/|P|=m$, which is not divisible by
 $p$.  We let $Q$ act on $G/P$ by $g*(xP)=gxP$ for $g\in Q$.  Note
 that $|G/P|=m\neq 0\pmod{p}$ and $Q$ is a $p$-group so by
 Lemma~\ref{lem-p-fixed} there is a fixed point, in other words a
 coset $xP$ such that $gxP=xP$ for all $g\in Q$.  This means that
 $x^{-1}gxP=P$, so $x^{-1}gx\in P$, so
 $g=x(x^{-1}gx)x^{-1}\in xPx^{-1}$.  This proves that
 $Q\sse xPx^{-1}$.  Now $xPx^{-1}$ is conjugate to $P$ so it is a
 subgroup of $G$ with the same order as $P$, in other words it is
 another Sylow $p$-subgroup.  This shows that $Q$ is contained in a
 Sylow $p$-subgroup, as claimed in~(d).

 Now suppose that $Q$ itself is a Sylow $p$-subgroup.  Then
 $Q\leq xPx^{-1}$ but $|Q|=|xPx^{-1}|=p^v$ so $Q=xPx^{-1}$.  Thus $Q$
 is conjugate to $P$, as claimed in~(c).
\end{proof}

\begin{proof}[Proof of~(b)]
 Let $\CP$ be the set of all Sylow $p$-subgroups of $G$, so
 $n_p=|\CP|$.  Let $G$ act on $\CP$ by conjugation, so $g*P=gPg^{-1}$.
 Choose a Sylow $p$-subgroup $P\in\CP$, and put
 \[ N=\stab_G(P)=\{g\in G \st gPg^{-1}=P\}. \]
 It is clear that $P\leq N\leq G$ so $p^v$ divides $|N|$ and $|N|$
 divides $p^vm$, so $|N|=p^vk$ for some $k$ dividing $m$.

 As all Sylow $p$-subgroups are conjugate to $P$, we have
 $\CP=\orb_G(P)$ and so $|G|=|N||\CP|$, so $p^vm=p^vkn_p$, so
 $m=kn_p$.  Thus $n_p$ divides $m$.

 Note also that $P$ can be thought of as a Sylow $p$-subgroup of $N$.
 Part~(c) of the Theorem works for any finite group, in particular it
 works for the group $N$, so any other Sylow $p$-subgroup $Q$ of $N$
 is conjugate \emph{in $N$} to $P$.  This means that $Q=gPg^{-1}$ for
 some $g\in N$.  By the definition of $N$, this means that $Q=P$.
 Thus, $P$ is the \emph{unique} Sylow $p$-subgroup of $N$.

 Before we considered the action of all of $G$ on $\CP$; now we
 restrict attention to the action of the subgroup $P$.
 Lemma~\ref{lem-p-fixed} tells us that
 $|\Fix(P,\CP)|=|\CP|=n_p\pmod{p}$.  We want to prove that
 $n_p=1\pmod{p}$, so it will be enough to show that
 $\Fix(P,\CP)=\{P\}$.  Clearly if $g\in P$ then $gPg^{-1}=P$, which
 shows that $P\in\Fix(P,\CP)$.  Conversely, suppose that
 $Q\in\Fix(P,\CP)$, so $Q$ is a Sylow $p$-subgroup and $gPg^{-1}=P$
 for all $g\in Q$.  This means that $Q$ is a Sylow $p$-subgroup of
 $N$, which means that $Q=P$ by the previous paragraph.  Thus
 $\Fix(P,\CP)=\{P\}$ and $|\Fix(P,\CP)|=1$ as required.
\end{proof}

\begin{proposition}\label{prop-sylow-normal}
 If $n_p=1$ then the Sylow $p$-subgroup of $G$ is a normal subgroup.
 If $n_p>1$ then none of the Sylow $p$-subgroups is normal.
\end{proposition}
\begin{proof}
 Suppose that $n_p=1$, so there is a unique Sylow $p$-subgroup, which
 we call $P$.  If $g\in G$ then $gPg^{-1}$ is a Sylow $p$-subgroup so
 it must be equal to $P$; this says that $P$ is normal.

 Now suppose that $n_p>1$.  If $P$ is any Sylow $p$-subgroup, we can
 choose a different Sylow $p$-subgroup, say $Q$.  As all such
 subgroups are conjugate, there is some $g\in G$ such that
 $gPg^{-1}=Q\neq P$.  This means that $P$ is not normal.
\end{proof}

\section{Groups of small order}

We now try to classify groups of various small orders, using the Sylow
theorems as one of our main tools.

Many of our results involve the cyclic groups:
\[ C_n = \{1,R,\ldots,R^{n-1}\} \]
with $R^n=1$.  We start with two general facts about these groups.
\begin{lemma}
 If $G$ is a group and $g\in G$ and $g^n=1$, then there is a
 homomorphism $\phi\:C_n\xra{}G$ with $\phi(R)=g$.
\end{lemma}
\begin{proof}
 As $C_n=\{1=R^0,R,\ldots,R^{n-1}\}$, we can define a function
 $\phi\:C_n\xra{}G$ by $\phi(R^i)=g^i$ for $i=0,\ldots,n-1$.  To see
 that this is a homomorphism, consider two elements $R^i,R^j\in C_n$
 with $0\leq i<n$.  If $i+j<n$ then 
 \[ \phi(R^i.R^j)=\phi(R^{i+j})=g^{i+j}=g^i g^j=\phi(R^i)\phi(R^j).
 \]
 Suppose instead that $i+j\geq n$.  By assumption we have
 $0\leq i,j<n$, so $i+j<2n$.  Thus, if we put $k=i+j-n$ then
 $0\leq k<n$, so $\phi(R^k)=g^k$.  We also have 
 \begin{align*}
  R^iR^j &= R^{i+j}=R^{k-n}=R^k(R^n)^{-1} = R^k \\ 
  g^ig^j &= g^{i+j}=g^{k-n}=g^k(g^n)^{-1} = g^k 
 \end{align*}
 so
 \[ \phi(R^i.R^j)=\phi(R^{i+j})=
     \phi(R^k)=g^k=g^ig^j=\phi(R^i)\phi(R^j). 
 \]
 We thus have $\phi(R^i.R^j)=\phi(R^i)\phi(R^j)$ in all cases, showing
 that $\phi$ is a homomorphism.
\end{proof}

\begin{lemma}\label{lem-cyclic-coprime}
 If $n$ and $m$ are coprime, then $C_{nm}\simeq C_n\tm C_m$.
\end{lemma}
\begin{proof}
 We write $r_k$ for the generator of $C_k$, and define
 $\phi\:C_{nm}\xra{}C_n\tm C_m$ by $\phi(r_{nm}^i)=(r_n^i,r_m^i)$.  It
 is easy to see that this is a homomorphism.  Suppose that $r_{nm}^i$
 lies in the kernel of $\phi$.  This means that $(r_n^i,r_m^i)=(1,1)$,
 which means that $r_n^i=1$ in $C_n$ and $r_m^i=1$ in $C_m$.  As
 $r_n^i=1$ we see that $i$ must be divisible by $n$, and as $r_m^i=1$
 we see that $i$ must be divisible by $m$.  As $n$ and $m$ are coprime
 this means that $i$ is divisible by $nm$, so $r_{nm}^i=1$.  This
 shows that the kernel of $\phi$ is the trivial group, so $\phi$ is
 injective.  This means that $|\phi(C_{nm})|=|C_{nm}|=nm=|C_n\tm C_m$,
 so $\phi(C_{nm})$ must be all of $C_n\tm C_m$, so $\phi$ is
 surjective as well as injective, so $\phi$ is an isomorphism.
\end{proof}

We next recall the basic result about groups of prime order.
\begin{proposition}\label{prop-p}
 If $G$ is a group whose order is a prime number $p$, then $G$ is
 isomorphic to $C_p$.
\end{proposition}
\begin{proof}
 Let $g$ be any element of $G$ other than the identity.  Then the
 order of $g$ is not equal to $1$ and it divides $p$ so it must be
 equal to $p$.  The subgroup generated by $g$ is thus equal to the
 whole group, and it follows that $G$ is cyclic of order $p$.  More
 precisely, we can define a homomorphism $\phi\:C_p\xra{}G$ by
 $\phi(R^i)=g^i$, and we find that $\phi$ is an isomorphism.
\end{proof}

We would next like to study groups of order $p^2$, where $p$ is
prime.  We will first need a result about general $p$-groups.

\begin{definition}
 The \emph{centre} of a group $G$ is the set
 $Z(G)=\{z\in G\st zg=gz \text{ for all } g\in G\}$, so an element $z$
 lies in the centre if and only if it commutes with all other
 elements.  One checks that $Z(G)$ is a normal subgroup of $G$ and
 that it is Abelian.
\end{definition}

\begin{example}
 The centre of the symmetric group $S_n$ is the trivial group
 (provided that $n>2$).  To see this, suppose that $\sg$ lies in the
 centre.  For each $i$, let $\rho_i$ be the $(n-1)$-cycle formed by
 the numbers $1,\ldots,n$ with $i$ missing, so $\rho_i(i)=i$ and
 $\rho_i(j)\neq j$ if $j\neq i$.  Now $\rho_i\sg=\sg\rho_i$ so
 $\rho_i(\sg(i))=\sg(\rho_i(i))=\sg(i)$, so $\sg(i)$ is fixed by the
 action of $\rho_i$.  The only fixed point is $i$, so $\sg(i)=i$.
 This holds for all $i$, so $\sg=1$.
\end{example}

\begin{proposition}\label{prop-ZP}
 If $P$ is a nontrivial $p$-group then $Z(P)\neq\{1\}$.
\end{proposition}
\begin{proof}
 Let $P$ act on itself by conjugation, so $g*x=gxg^{-1}$.  Note that
 $g*x=x$ if and only if $gx=xg$, or in other words $g$ commutes with
 $x$.  Thus $x$ is fixed under the action of $P$ iff $g*x=x$ for all
 $g$, iff $x\in Z(P)$.  Thus Lemma~\ref{lem-p-fixed} tells us that
 $|Z(P)|=|P|\pmod{p}$.  As $P$ is a nontrivial $p$-group we have
 $|P|=p^v$ for some $v>0$ so $|P|=0\pmod{p}$, so $|Z(P)|$ is divisible
 by $p$.  Moreover, $1\in Z(P)$ so $|Z(P)|>0$.  It follows that
 $|Z(G)|\geq p$, so $Z(G)\neq\{1\}$.
\end{proof}

\begin{lemma}\label{lem-commuting-product}
 Let $G$ be a finite group, and let $P$ and $Q$ be subgroups of $G$.
 Define a function $\phi\:P\tm Q\xra{}G$ by $\phi(x,y)=xy$.
 \begin{itemize}
  \item[(a)] If every element of $P$ commutes with every element of
   $Q$, then $\phi$ is a homomorphism.
  \item[(b)] If we also have $P\cap Q=\{1\}$, then $\phi$ is
   injective.  
 \end{itemize}
\end{lemma}
\begin{proof}
 \begin{itemize}
  \item[(a)] Suppose that every element of $P$ commutes with every
   element of $Q$.  Consider elements $x_0,x_1\in P$ and
   $y_0,y_1\in Q$, so $(x_0,y_0)$ and $(x_1,y_1)$ are elements of
   $P\tm Q$.  We then have
   \[ \begin{array}{rlr}
    \phi((x_0,y_0)(x_1,y_1)) &\!\!=
        \phi((x_0x_1,y_0y_1)) & 
         \hspace{6em} \text{(definition of $P\tm Q$)} \\
     &\!\!= x_0x_1y_0y_1 & \text{(definition of $\phi$)}\\
     &\!\!= x_0y_0x_1y_1 & \text{(because $x_1$ commutes with $y_0$)} \\
     &\!\!= \phi((x_0,y_0)) \phi((x_1,y_1)) 
         & \text{(definition of $\phi$)}
   \end{array} \]
   This shows that $\phi$ is a homomorphism.
  \item[(b)] Now suppose as well that $P\cap Q=\{1\}$.  Consider an
   element $(x,y)\in\ker(\phi)$.  This means that $(x,y)\in P\tm Q$ and
   and $\phi((x,y))=1$, or in other words, $x\in P$ and $y\in Q$ and
   $xy=1$.  This means that $x=y^{-1}$, and $y\in Q$, so $x\in Q$.  We
   are also given that $x\in P$, so $x\in P\cap Q=\{1\}$, so $x=1$.
   This means that $y^{-1}=1$, so $y=1$, so $(x,y)=(1,1)$.  This
   proves that the kernel of $\phi$ is the trivial group, so $\phi$ is
   injective. 
 \end{itemize}
\end{proof}

\begin{proposition}\label{prop-p-squared}
 Let $G$ be a group of order $p^2$.  Then $G$ is isomorphic either to
 $C_p\tm C_p$ or to $C_{p^2}$ (and so $G$ is always Abelian).
\end{proposition}
\begin{proof}
 If $G$ has an element of order $p^2$ then it is cyclic and thus
 isomorphic to $C_{p^2}$.  Suppose instead that all nontrivial
 elements of $G$ have order $p$.  By Proposition~\ref{prop-ZP}, we can
 choose a nontrivial element $z\in Z(G)$.  This generates a subgroup
 $P\leq Z(G)\leq G$ of order $p$.  Let $g$ be any element of $G$ not
 lying in $P$, and let $Q$ be the subgroup generated by $g$, which
 again has order $p$.  By Lemma~\ref{lem-commuting-product}, we can
 define a homomorphism $\phi\:P\tm Q\xra{}G$ by $\phi(x,y)=xy$.  Let
 $H$ be the image of $\phi$, so $H$ is a subgroup of $G$, so $|H|$
 divides $|G|=p^2$, so $|H|$ is $1$, $p$ or $p^2$.  It is clear that
 $P\leq H$ and $g\in H\setminus P$, so $|H|\geq p+1$, so $|H|$ must be
 $p^2$.  This means that $H=G$, so $\phi$ is surjective.  As
 $P\tm Q$ and $G$ have the same size, any sutrjective function between
 them must be bijective, so $\phi$ is an isomorphism.  We also know
 that $P$ and $Q$ are both isomorphic to $C_p$, so
 $G\simeq P\tm Q\simeq C_p\tm C_p$.
\end{proof}

\begin{proposition}\label{prop-coprime}
 Let $G$ be a finite group, and let $P$ and $Q$ be \emph{normal}
 subgroups of orders $p$ and $q$.  Suppose that $p$ and $q$ are
 coprime, and that $pq=|G|$.  Then $G\simeq P\tm Q$.
\end{proposition}
\begin{proof}
 First, put $r=|P\cap Q|$.  As $P\cap Q$ is a subgroup of $P$, we see
 that $r$ divides $p$.  As $P\cap Q$ is a subgroup of $Q$, we see that
 $r$ also divides $q$.  As $p$ and $q$ are coprime, this means that
 $r=1$, so $P\cap Q$ is the trivial group.

 Now consider elements $x\in P$ and $y\in Q$, and put
 $z=xyx^{-1}y^{-1}$.  We will show that $z\in P\cap Q$, so that
 $z=1$.  Indeed, we have $y\in Q$ and $Q$ is normal, so
 $xyx^{-1}\in Q$.  As $y^{-1}$ also lies in $Q$, we deduce that
 $z=(xyx^{-1})y^{-1}\in Q$.  Similarly, we know that $x^{-1}\in P$ and
 $P$ is normal so $yx^{-1}y^{-1}\in P$.  We also know that $x\in P$,
 so $z=x(yx^{-1}y^{-1})\in P$.  This gives $z\in P\cap Q$, so $z=1$,
 or in other words $1=xyx^{-1}y^{-1}$.  If we multiply this on the
 right by $yx$, we get $yx=xyx^{-1}y^{-1}yx=xy$, so $x$ commutes with
 $y$.  Lemma~\ref{lem-commuting-product} now tells us that we can
 define an injective homomorphism $\phi\:P\tm Q\xra{}G$ by
 $\phi(x,y)=xy$.  As this is injective, we have 
 \[ |\phi(P\tm Q)| = |P\tm Q|=pq=|G|, \]
 so $\phi(P\tm Q)=G$, so $\phi$ is also surjective.  This means that
 $\phi$ is an isomorphism of groups.
\end{proof}

\begin{proposition}\label{prop-pq}
 Let $G$ be a group of order $pq$ where $p$ and $q$ are primes and
 $p<q$.  Suppose also that $q-1$ is not divisible by $p$.  Then
 $G\simeq C_p\tm C_q$.
\end{proposition}
\begin{proof}
 We know that $n_p$ divides $q$ and $q$ is prime so $n_p=1$ or
 $n_p=q$.  However we also know that $n_p=1\pmod{p}$ so $p$ divides
 $n_p-1$.  We are told that $p$ does not divide $q-1$, so $n_p$ cannot
 be equal to $q$, so $n_p=1$.  It follows that there is a unique Sylow
 $p$-subgroup, which we call $P$.  Note that $|P|=p$ and so
 $P\simeq C_p$, and also that $P$ is normal.

 Next, we know that $n_q$ divides $p$, so $n_q=1$ or $n_q=p$.  We also
 know that $n_q=1\pmod{q}$, so $n_q-1$ is divisible by $q$.  Note that
 $0<p-1<q$, so $p-1$ cannot be divisible by $q$, so we must have
 $n_q=1$.  We therefore have a unique Sylow $q$-subgroup, which we
 call $Q$.  We note that $|Q|=q$ and that $Q$ is normal.

 It is now clear that the conditions of Proposition~\ref{prop-coprime}
 are satisfied, so $G\simeq P\tm Q\simeq C_p\tm C_q$.
\end{proof}

\begin{proposition}\label{prop-two-p}
 Let $p$ be a prime number with $p>2$, and let $G$ be a group with
 $|G|=2p$.  Then either $G\simeq C_p\tm C_2\simeq C_{2p}$ or
 $G\simeq D_p$.
\end{proposition}
\begin{proof}
 We know that $n_p$ divides $2$ and that $n_p-1$ is divisible by $p$.
 It follows that $n_p=1$, so there is a unique Sylow $p$-subgroup,
 which we call $P$.  We choose a nontrivial element $g\in P$, and
 define an isomorphism $\phi\:C_p\xra{}P$ by $\phi(R^i)=g^i$.

 Next, let $Q$ be a Sylow $2$-subgroup, so $|Q|=2$, so $Q=\{1,h\}$ for
 some element $h$ with $h^2=1$.  As $P$ is normal, we know that
 $hgh^{-1}\in P$, so $hgh^{-1}=g^a$ for some $a$.  It follows that
 \[ h^2gh^{-2}=hg^ah^{-1} = (hgh^{-1})^a = (g^a)^a = g^{(a^2)}. \]
 On the other hand, we have $h^2=1$, so $h^{-2}=1$, so
 $h^2gh^{-2}=g$.  We thus have $g=g^{a^2}$, so $g^{a^2-1}=1$, so
 $a^2-1$ must be divisible by $p$.  As $p$ is prime and
 $a^2-1=(a+1)(a-1)$, we see that either $a+1$ or $a-1$ must be
 divisible by $p$.

 If $a-1$ is divisible by $p$ then $g^a=g$, so $h^{-1}gh=g$, so $g$
 commutes with $h$.  In this case, the conditions of
 Lemma~\ref{lem-commuting-product} are satisfied and we find that
 $G\simeq P\tm Q\simeq C_p\tm C_2\simeq C_{2p}$.  

 Suppose instead that $a+1$ is divisible by $p$.  This means that
 $g^a=g^{-1}$, so $hgh^{-1}=g^{-1}$.  We then define a function
 $\phi\:D_p\xra{}G$ by $\phi(R^i)=g^i$ and $\phi(R^iS)=g^ih$ for
 $0\leq i<p$.  It is straightforward to check that this is a
 homomorphism, the key point being that $SRS^{-1}=R^{-1}$ in $D_p$,
 which corresponds to the relation $hgh^{-1}=g^{-1}$ in $G$.  The
 image of $\phi$ contains both $P$ and $Q$, so the order of the image
 is divisible by $p$ and $2$ and thus by $2p$.  It follows that $\phi$
 is surjective, but the groups $D_p$ and $G$ have the same order, so
 $\phi$ must actually be an isomorphism.
\end{proof}

\begin{remark}
 The above proposition can be extended to show that when $p$ and $q$
 are distinct primes, any group of order $pq$ is a ``semidirect
 product'' of $C_p$ and $C_q$.
\end{remark}

Now consider groups of order at most $40$.  Using
Propositions~\ref{prop-p}, \ref{prop-p-squared}, \ref{prop-pq}
and~\ref{prop-two-p}, we can classify all groups of the following
orders:
\[ 1,2,3,4,5,6,7,9,10,11,13,14,15,17,19,22,23,
   25,26,29,31,33,35,37,38.
\]
More work is needed to classify groups of the remaining orders:
\[ 8,12,16,18,20,21,24,27,28,32,34,36,39. \]

\section{Automorphisms and semidirect products}

\begin{definition}
 An \emph{automorphism} of a group $G$ is an isomorphism from $G$ to
 itself.  We write $\Aut(G)$ for the set of all automorphisms of $G$.
 This set is itself a group under composition.
\end{definition}

\begin{example}
 Consider the group $G=\Z_2\tm\Z_2$.  Define $\sg\:G\xra{}G$ by
 $\sg(x,y)=(y,x)$.  Clearly $\sg^2(x,y)=\sg(y,x)=(x,y)$ so $\sg^2$ is
 the identity map so $\sg$ is an inverse for itself.  This means that
 $\sg$ is a bijection.  Also 
 \[ \sg((v,w)+(x,y)) = \sg(v+x,w+y)=(w+y,v+x)=
       (w,v)+(y,x)=\sg(v,w)+\sg(x,y),
 \]
 which proves that $\sg$ is a homomorphism and thus is an automorphism
 of $G$.  In othe words, $\sg$ is an element of the group $\Aut(G)$.
 We can define another homomorphism $\rho\:G\xra{}G$ by
 $\rho(x,y)=(y,x+y)$.  We then have
 $\rho^2(x,y)=\rho(y,x+y)=(x+y,y+(x+y))=(x+y,x)$ and
 $\rho^3(x,y)=\rho(x+y,x)=(x,x+(x+y))=(x,y)$, so $\rho^3=1$, so $\rho$
 is an automorphism.  One can check that 
 \[ \Aut(G) = \{1,\rho,\rho^2,\sg,\rho\sg,\rho^2\sg\} \]
 and using this that $\Aut(G)\simeq D_3$.
\end{example}

\begin{proposition}
 If $G$ is cyclic of order $n$ then $\Aut(G)\simeq\Z_n^\tm$.
\end{proposition}
\begin{proof}
 Suppose $a\in\Z_n$.  If $x\in G$ then $x^n=1$ so the powers of $x$
 depend only on the exponent modulo $n$, so it makes sense to talk
 about $x^a$.  We define a function $\al_a\:G\xra{}G$ by
 $\al_a(x)=x^a$.  Because $G$ is cyclic it is Abelian and this implies
 that $(xy)^a=x^ay^a$ or in other words $\al_a(xy)=\al_a(x)\al_a(y)$,
 so $\al_a$ is a homomorphism.  Clearly also
 $\al_a\al_b(x)=\al_a(x^b)=(x^a)^b=x^{ab}=\al_{ab}(x)$, so
 $\al_a\al_b=\al_{ab}$.  Similarly $\al_1(x)=x$, so $\al_1$ is the
 identity map, and $\al_0(x)=1$, so $\al_0$ is the trivial
 homomorphism.

 We next claim that every homomorphism $\bt\:G\xra{}G$ has the form
 $\bt=\al_a$ for a unique element $a\in\Z_n$.  Indeed, we can choose a
 generator $g\in G$ so that $G=\{1,g,\ldots,g^{n-1}\}=\{g^a\st
 a\in\Z_n\}$.  We then have $\bt(g)\in G$ so $\bt(g)=g^a$ for a unique
 element $a\in\Z_n$.  As $\bt$ is a homomorphism we have
 \[ \bt(g^i)=\bt(g)^i=(g^a)^i=g^{ia}=\al_a(g^i) \]
 for all $i$, so $\bt=\al_a$.  

 Now, if $a\in\Z_n^\tm$ then $a$ has an inverse $b\in\Z_n^\tm$ and
 then $\al_a\al_b=\al_{ab}=\al_1=1$ and similarly $\al_b\al_a=1$ so
 $\al_b$ is an inverse for $\al_a$.  This means that $\al_a$ is an
 automorphism of $G$, in other words $\al_a\in\Aut(G)$.

 Conversely, suppose that $\bt$ is an automorphism of $G$.  Then $\bt$
 and $\bt^{-1}$ are homomorphisms from $G$ to itself, say $\bt=\al_a$
 and $\bt^{-1}=\al_b$ for some $a,b\in\Z_n$.  We then have
 $\al_{ab}=\bt\bt^{-1}=1=\al_1$, so $ab=1$, so $a\in\Z_n^\tm$.  This
 shows that the automorphisms of $G$ are precisely the maps $\al_a$
 with $a\in\Z_n^\tm$.  

 We can now define a map $\phi\:\Z_n^\tm\xra{}\Aut(G)$ by
 $\phi(a)=\al_a$, and we find that this is an isomorphism.
\end{proof}

\begin{construction}
 Suppose that $G$ is a group and $N$ is a normal subgroup.  For any
 $g\in G$ we know that $N=gNg^{-1}$.  We can therefore define
 $\gm_g\:N\xra{}N$ by $\gm_g(x)=gxg^{-1}$.  This is a homomorphism
 because
 \[ \gm_g(x)\gm_g(y)= gxg^{-1}gyg^{-1}=gxyg^{-1} = \gm_g(xy). \]
 We also have
 \[ \gm_g(\gm_h(x)) = \gm_g(hxh^{-1})=ghxh^{-1}g^{-1} =
      (gh)x(gh)^{-1} = \gm_{gh}(x),
 \]
 so $\gm_g\gm_h=\gm_{gh}$.  In particular, this means that
 $\gm_{g^{-1}}$ is an inverse for $\gm_g$, so $\gm_g$ is an
 automorphism of $N$, in other words $\gm_g\in\Aut(N)$.
\end{construction}

\begin{construction}
 Now suppose that $G$ is a semidirect product of $N$ and $Q$.  We
 define a function $\phi\:Q\xra{}\Aut(N)$ by $\phi(g)=\gm_g$.  We then
 have $\phi(g)\phi(h)=\gm_g\gm_h=\gm_{gh}=\phi(gh)$, so $\phi$ is a
 homomorphism. 
\end{construction}

\begin{example}
 Consider the group $G=\Z_n\rtimes_a\Z_m$ as the semidirect product of
 $N=\{(v,0)\st v\in\Z_n\}$ and $Q=\{(0,w)\st w\in\Z_m\}$.  We then
 have $\Aut(N)\simeq\Z_n^\tm$ and $Q\simeq\Z_m$.  We also have 
 \[ \gm_{(0,w)}(v,0)=(0,w)(v,0)(0,-w)=(a^wv,w)(0,-w)=(a^wv,0). \]
 This means that the automorphism $\gm_{(0,w)}\in\Aut(N)$ corresonds
 to the element $a^w\in\Z_n^\tm$ and that the homomorphism
 $\phi\:\Z_m\xra{}\Z_n^\tm$ is given by $\phi(w)=a^w$.
\end{example}

\begin{proposition}
 If $G$ is a semidirect product as above and $\phi\:Q\xra{}\Aut(N)$ is
 the trivial homomorphism then every element of $Q$ commutes with
 every element of $N$ and $G\simeq N\tm Q$.
\end{proposition}
\begin{proof}
 If $\phi$ is trivial then for each $g\in Q$, the homomorphism
 $\phi(g)=\gm_g\:N\xra{}N$ is the identity map, in other words
 $\gm_g(x)=x$ for all $x\in N$.  As $\gm_g(x)=gxg^{-1}$ this means
 that $gxg^{-1}=x$ so $gx=xg$ so $g$ commutes with $x$.

 Now define $\mu\:N\tm Q\xra{}G$ by $\mu(x,g)=xg$.  We claim that this
 is a homomorphism.  Recall that the group structure in $N\tm Q$ is
 just given by $(x,g)(y,h)=(xy,gh)$, so we must show that
 $\mu(x,g)\mu(y,h)=\mu(xy,gh)$ or in other words that $xgyh=xygh$.
 This is true because $g$ commutes with $y$.

 As $G$ is a semidirect product of $N$ and $Q$, we have $NQ=G$ and
 $N\cap Q=\{1\}$.  As $NQ=G$ we see that $\mu$ is surjective.  If
 $(x,g)\in\ker(\mu)$ then $xg=1$ so $g=x^{-1}$.  Now $g\in Q$ and
 $x^{-1}\in N$ so the element $g=x^{-1}$ lies in $N\cap Q=\{1\}$ so
 $g=x=1$.  Thus $\ker(\mu)=\{(1,1)\}$, which shows that $\mu$ is
 injective and thus an isomorphism.
\end{proof}

\end{document}

