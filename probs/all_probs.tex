\documentclass{amsart}
\usepackage{fullpage}
\usepackage{amscd,amssymb}
\usepackage{verbatim}
\usepackage{tikz}

\DeclareMathOperator{\Dir}{Dir}
\DeclareMathOperator{\Fix}{Fix}
\DeclareMathOperator{\Symm}{Symm}
\DeclareMathOperator{\Aut}{Aut}
\DeclareMathOperator{\Inn}{Inn}
\DeclareMathOperator{\Out}{Out}
\DeclareMathOperator{\Trans}{Trans}

\DeclareMathOperator{\Tet}{Tet}
\DeclareMathOperator{\Cube}{Cube}
\DeclareMathOperator{\Oct}{Oct}
\DeclareMathOperator{\Dodec}{Dodec}
\DeclareMathOperator{\Icos}{Icos}

\newcommand{\xra}{\xrightarrow}

\renewcommand{\:}{\colon}
\newcommand{\sse}{\subseteq}
\newcommand{\sm}{\setminus}
\newcommand{\ov}{\overline}
\newcommand{\tm}{\times}
\newcommand{\ip}[1]{\langle #1\rangle}
\newcommand{\st}{\;|\;}
\newcommand{\CC}{\mathcal{C}}
\newcommand{\bsm}       {\left(\begin{smallmatrix}}
\newcommand{\esm}       {\end{smallmatrix}\right)}

\newcommand{\N}{\mathbb{N}}
\newcommand{\Z}{\mathbb{Z}}
\newcommand{\Q}{\mathbb{Q}}
\newcommand{\R}{\mathbb{R}}
\newcommand{\C}{\mathbb{C}}

\newcommand{\al}        {\alpha}
\newcommand{\bt}        {\beta} 
\newcommand{\gm}        {\gamma}
\newcommand{\dl}        {\delta}
\newcommand{\ep}        {\epsilon}
\newcommand{\zt}        {\zeta}
\newcommand{\et}        {\eta}
\newcommand{\tht}       {\theta}
\newcommand{\io}        {\iota}
\newcommand{\kp}        {\kappa}
\newcommand{\lm}        {\lambda}
\newcommand{\ph}        {\phi}
\newcommand{\ch}        {\chi}
\newcommand{\ps}        {\psi}
\newcommand{\rh}        {\rho}
\newcommand{\sg}        {\sigma}
\newcommand{\om}        {\omega}

\definecolor{olivegreen}{cmyk}{0.64,0,0.95,0.40}

\theoremstyle{definition}
\newtheorem{exercise}{Exercise}
\newenvironment{solution}{{\noindent\bf Solution:}}{}

\begin{document}
\title{Problems on Groups and Symmetry}
\author{N.~P.~Strickland}

\maketitle 

\begin{exercise}
 Note that $R_\pi(x)=-x$.  Give expressions for $R_{a,\pi}(x)$,
 $R_{a,\pi}^{-1}(x)$ and
 $R_{a,\pi}R_{b,\pi}R_{a,\pi}^{-1}R_{b,\pi}^{-1}(x)$.  Check that your
 last answer is consistent with the formula in the notes for
 $R_{a,\tht}R_{b,\phi}R_{a,\tht}^{-1}R_{b,\phi}^{-1}$. 
\end{exercise}
\begin{solution}
 First, we have $R_{a,\pi}=T_aR_\pi T_{-a}$, so
 $R_{a,\pi}(x)=a+R_\pi(x-a)=a-(x-a)=2a-x$.  If $y=2a-x$ then $x=2a-y$;
 this shows that $R_{a,\pi}^{-1}(y)=2a-y$ and thus
 $R_{a,\pi}^{-1}=R_{a,\pi}$.  Using this we have
 \begin{align*}
  R_{a,\pi}R_{b,\pi}R_{a,\pi}^{-1}R_{b,\pi}^{-1}(x) &=
     2a-(2b-(2a-(2b-x))) \\
  &= 2a-(2b-(2a-2b+x)) \\
  &= 2a-(4b-2a-x)) \\
  &= 4a-4b+x,
 \end{align*}
 so $[R_{a,\pi},R_{b,\pi}]=T_{4(a-b)}$.  On the other hand, we saw in
 lectures that
 $[R_{a,\tht},R_{b,\phi}]=T_{(1-R_\tht)(1-R_\phi)(a-b)}$.  This is
 consistent because $1-R_\pi$ is just twice the identity matrix, so
 $(1-R_\tht)(1-R_\phi)(a-b)=4(a-b)$.
\end{solution}

\begin{exercise}
 Find $\Symm(X)$ and $\Dir(X)$ when $X\subseteq\R^2$ is
 \begin{itemize}
 \item[(a)] The unit disc centred at the origin
 \item[(b)] The isosceles triangle with vertices $(0,2)$, $(-1,-3)$ and
  $(1, -3)$.
 \item[(c)] The four points $(1,1)$, $(1,-1)$, $(-1,1)$ and $(-1,-1)$.
 \item[(d)] The square with vertices $(-1, 2)$, $(-1,-2)$, $(3, 2)$ and
  $(3, -2)$.
 \end{itemize}
\end{exercise}
\begin{solution}
 \begin{itemize}
 \item[(a)] Here $\Symm(X)=O_2$ and $\Dir(X)=SO_2$.  This just means
  that the unit disc is invariant under any rotation about the origin,
  and under any reflection across a line through the origin, which is
  geometrically clear.

  For a more algebraic proof, note that $X=\{x\in\R^2\st\;\|x\|\leq
  1\}$.  For any $A\in O_2$ and $x\in X$ we have $\|Ax\|=\|x\|\leq 1$
  so $Ax\in X$; thus $AX\sse X$.  Conversely, if $y\in X$ then
  $\|A^{-1}y\|=\|y\|\leq 1$ so the point $x:=A^{-1}y$ lies in $X$.  We
  have $y=Ax$ so $y\in AX$.  This shows that $X\sse AX$, so $X=AX$, so
  $A\in\Symm(X)$.  As $A$ was an arbitrary element of $O_2$ we have
  $\Symm(X)=O_2$ and $\Dir(X)=\Symm(X)\cap SO_2=O_2\cap SO_2=SO_2$, as
  claimed.
 \item[(b)] Here it is evident that the only symmetry is under
  reflection across the $y$-axis, which lies at angle $\pi/2$ to the
  horizontal.  Recall that $S_\tht$ is the reflection across the line
  at angle $\tht/2$ to the horizontal, so reflection across the
  $y$-axis is $S_\pi$.  Thus $\Symm(X)=\{1,S_\pi\}$.  This contains no
  rotations other than the identity, so $\Dir(X)=\{1\}$.
  \begin{center}
   \begin{tikzpicture}
    \draw[->] (-1.2,0) -- (1.2,0);
    \draw[->] (0,-3.2) -- (0,2.4);
    \draw[->] (0,2.2) -- (0.2,2.2);
    \draw[->] (0,2.2) -- (-0.2,2.2);
    \draw (0,2) -- (1,-3) -- (-1,-3) -- (0,2);
    \draw (0.9,-1) node {$X$};
    \draw (0.3,2.2) node[anchor=west] {$S_\pi$};
   \end{tikzpicture}
  \end{center}
 \item[(c)] Here $X$ consists of the vertices of a square of side $2$
  with horizontal and vertical sides (note that this is different from
  our usual square $X_4$).  We have $\Symm(X)=D_4$ and $\Dir(X)=C_4$.
  Indeed, it is clear that $X$ is invariant under a rotation $R_\tht$
  if and only if $\tht$ is a multiple of a quarter-turn, or in other
  words a multiple of $\pi/2=2\pi/4$.  Thus if we put $R=R_{\pi/2}$ we
  find that
  \[ \Dir(X)=\{1=R_0,R_{\pi/2},R_\pi,R_{3\pi/2}\}=
      \{1,R,R^2,R^3\}=C_4.
  \]
  It is also clear that $X$ is unchanged if we reflect it across the
  $x$-axis, so $S_0\in\Symm(X)$.  If $A\in\Symm(X)$ is a reflection
  then $AS_0$ is a rotation and lies in $\Symm(X)$ so $AS_0=R^k$ for
  some $k$ so $A=AS_0S_0=R^kS_0$ so $A\in D_4$.  If $A\in\Symm(X)$ is
  a rotation we have seen that $A\in C_4\sse D_4$, so again $A\in D_4$; 
  thus $\Symm(X)\sse D_4$.  As $R$ and $S_0$ preserve $X$ we also have
  $D_4\sse\Symm(X)$, so $\Symm(X)=D_4$.  Alternatively, we can just
  observe geometrically that there are four lines of reflectional
  symmetry at angles $0$, $\pi/4$, $\pi/2$ and $3\pi/4$ to the
  $x$-axis, so the reflections in $\Symm(X)$ are $S_0$, $S_{\pi/2}$,
  $S_\pi$ and $S_{3\pi/2}$. 
 \item[(d)] Here $X$ is an off-centre square.
  \begin{center}
   \begin{tikzpicture}
    \draw[->] (-1.3,0) -- (3.5,0);
    \draw[->] (0,-2.3) -- (0,2.3);
    \draw (-1,-2) -- (-1,2) -- (3,2) -- (3,-2) -- (-1,-2);
    \fill (1,0) circle(0.05);
    \draw[->] (3.2,0) -- (3.2,0.2);
    \draw[->] (3.2,0) -- (3.2,-0.2);
    \draw (3.3,0.3) node[anchor=south west] {$S_0$};
    \draw (1,-0.1) node[anchor=north west] {$(1,0)$};
   \end{tikzpicture}
  \end{center}

  There is a lot of symmetry about the point $(1,0)$ at the centre of
  the square.  However, the question asks about $\Symm(X)$, which is by
  definition the group of symmetries about the point $(0,0)$, and from
  that point of view the picture is much less symmetrical.  In fact,
  the only symmetry is the reflection across the $x$-axis, so
  $\Symm(X)=\{1,S_0\}$ and $\Dir(X)=\{1\}$.
 \end{itemize}
\end{solution}

\begin{exercise}
 \begin{itemize}
  \item[(a)] What can you say about $R_\pi A R_\pi^{-1}$ for
   $A\in O_2$? (Try writing out the matrices explicitly).
  \item[(b)] Show that $R_{\pi/n}X_n\neq X_n$ but that
   $D_n=R_{\pi/n}D_nR_{\pi/n}^{-1}$.
  \item[(c)] Deduce that $R_{\pi k/n}D_nR_{\pi k/n}^{-1}=D_n$ for all
   $k\in\Z$. 
  \item[(d)]
   Let $X$ be the regular heptagon centred at $(0,0)$ with one vertex at
   $(0,1)$.  Find $n$ and $\tht$ such that
   $\Symm(X)=R_\tht D_n R_\tht^{-1}$.  There are three different
   possibilities for $\tht$ in the range $[0,\pi/2)$; find all of
   them.
   \begin{center}
    \begin{tikzpicture}[scale=2]
     \draw[->] (-1.2,0) -- (1.2,0);
     \draw[->] (0,-1.2) -- (0,1.2);
     \def\PA{( 0.00, 1.00)}
     \def\PB{(-0.78, 0.62)}
     \def\PC{(-0.97,-0.22)}
     \def\PD{(-0.43,-0.90)}
     \def\PE{( 0.43,-0.90)}
     \def\PF{( 0.97,-0.22)}
     \def\PG{( 0.78, 0.62)}
     \draw \PA -- \PB -- \PC -- \PD -- \PE -- \PF -- \PG -- \PA -- cycle;     
    \end{tikzpicture}
   \end{center}
 \end{itemize}
\end{exercise}
\begin{solution}
 \begin{itemize}
  \item[(a)] $R_\pi$ is the matrix $\bsm -1&0\\0&-1\esm$, in other
   words $R_\pi=-I$.  This means that $R_\pi^{-1}=-I$ also and thus
   that $R_\pi A R_\pi^{-1}=-(-A)=A$ for all $A$.
  \item[(b)] The polygon $X_n$ has vertices $v_k$ with polar
   coordinates $[1,2\pi k/n]$.  The map $R_{\pi/n}$ sends $v_0$ to
   $[1,\pi/n]$ which is not a vertex, so $R_{\pi/n}X_n\neq X_n$.  The
   elements of $D_n$ have the form $R_{2k\pi/n}$ or $S_{2k\pi/n}$ for
   $k\in\Z$.  Using the equations $R_\al R_\bt R_\al^{-1}=R_\bt$ and
   $R_\al S_\bt R_\al^{-1}=S_{\bt+2\al}$ we see that
   \begin{align*}
    R_{\pi/n} R_{2\pi k/n} R_{\pi/n}^{-1} &= R_{2\pi k/n}\in D_n \\
    R_{\pi/n} S_{2\pi k/n} R_{\pi/n}^{-1} &= S_{2\pi(k+1)/n}\in D_n.
   \end{align*}
   This shows that $R_{\pi/n}D_n R_{\pi/n}^{-1}\sse D_n$ but these two
   sets have the same size so they must actually be equal.
  \item[(c)]
   If $R_{\pi k/n}D_n R_{\pi k/n}^{-1}=D_n$ then we can substitute
   $D_n=R_{\pi/n} D_n R_{\pi/n}^{-1}$ on the left hand side to deduce
   that 
   \[ R_{\pi(k+1)/n}D_n R_{\pi(k+1)/n}^{-1}=
      R_{\pi k/n} R_{\pi/n}D_n R_{\pi/n}^{-1} R_{\pi k/n}^{-1}=
      R_{\pi k/n} D_n R_{\pi k/n}^{-1}=
      D_n.
   \]
   By induction, this proves~(c) for all $k\geq 0$.  If $k<0$ we have
   $k=-m$ for some $m>0$ and we have proved already that 
   $D_n=R_{\pi m/n} D_n R_{\pi m/n}^{-1}$.  If we multiply this
   equation on the left by $R_{\pi m/n}^{-1}$ and on the right by
   $R_{\pi m/n}$ we obtain $R_{\pi -m/n} D_n R_{\pi m/n}=D_n$, or in
   other words $R_{\pi k/n}D_n R_{\pi k/n}^{-1}=D_n$, as required.
  \item[(d)]
   We have $X=R_{\pi/2}X_7$ so 
   \[ \Symm(X) = R_{\pi/2}\Symm(X_7)R_{\pi/2}^{-1}
               = R_{\pi/2} D_7 R_{\pi/2}^{-1} .
   \]
   More generally, by using part~(c) we see that
   $R_{\pi k/7}D_7 R_{\pi k/7}^{-1}=D_7$ and thus that 
   \[ \Symm(X) = R_{\pi/2} D_7 R_{\pi/2}^{-1}
               = R_{\pi/2} R_{\pi k/7}D_7 R_{\pi k/7}^{-1} R_{\pi/2}^{-1}
               = R_{(2k+7)\pi/14} D_7 R_{(2k+7)\pi/14}^{-1}.
   \]
   Thus, if we put $\tht=(2k+7)\pi/14$ we again have
   $\Symm(X)=R_\tht D_7R_\tht^{-1}$. 

   Conversely, suppose that $\tht$ satisfies 
   $\Symm(X)=R_\tht D_7R_\tht^{-1}$.  We then have
   \[ D_7 = R_{\pi/2}^{-1}\Symm(X)R_{\pi/2} =
            R_{\pi/2}^{-1}R_\tht D_7 R_{\pi/2}R_\tht^{-1} = 
            R_{\tht-\pi/2}D_7R_{\tht-\pi/2}^{-1}.
   \]
   As $S_0\in D_7$ this implies that the element
   $S_{2\tht-\pi}=R_{\tht-\pi/2}S_0R_{\tht-\pi/2}^{-1}$ also lies in
   $D_7$.  However, we only have $S_\phi\in D_7$ if $\phi$ is a multiple
   of $2\pi/7$, so $2\tht-\pi=2k\pi/7$ for some $k$, so
   $\tht=\pi/2+k\pi/7=(2k+7)\pi/14$.  Thus we have
   $\Symm(X)=R_\tht D_7R_\tht^{-1}$ if and only if $\tht$ has the form
   $(2k+7)\pi/14$ for some integer $k$.  For $\tht\in[0,\pi/2)$ we
   must have $k=-3$ or $k=-2$ or $k=-1$, so
   $\tht\in\{\pi/14,3\pi/14,5\pi/14\}$. 
 \end{itemize} 
\end{solution}

\begin{exercise}
Prove that $D_n$ is generated by reflections.

(A group $G$ is said to be {\em generated} by a subset $Y$ if every
$g\in G$ can be written in the form
\[ g = y_1^{a_1}y_2^{a_2}\cdots y_n^{a_n} \]
for some $n \geq 1$, $y_1,y_2,\ldots,y_n\in Y$ and
$a_1,a_2,\ldots,a_n\in\{\pm 1\}$.)

\end{exercise}
\begin{solution}
 Write $R=R_{2\pi/n}$ and $S=S_0$ so 
 \[ D_n=\{1,R,\ldots,R^{n-1},S,RS,\ldots,R^{n-1}S\}. \]
 Using the fact that $R_\al S_\bt=S_{\al+\bt}$ we see that
 $R^kS=R_{2k\pi/n}S_0=S_{2k\pi/n}$, which is a reflection.  Also,
 because $S^2=1$ we see that $R^k=(R^kS)S$.  Here $R^kS$ and $S$ are
 reflections lying in $D_n$, so $R^k$ can be written as a product of
 two reflections lying in $D_n$.  Thus every element in $D_n$ is
 either a reflection or a product of reflections, so the reflections
 in $D_n$ generate $D_n$ as claimed.
\end{solution}

\begin{exercise}
 Find the conjugacy classes of the elements of $D_n$ and hence find the
 centre of $D_n$.  [Hint: First deal with the cases of $D_1$ and $D_2$
 which are a bit different.  Then treat $D_3$ and $D_4$. You are then
 probably ready for the general case. See also the case treated in
 lectures!]
\end{exercise}
\begin{solution}
 \begin{itemize}
  \item $D_1=\{1,S\}$; the conjugacy classes are $\{1\}$ and $\{S\}$.
  \item $D_2=\{I,R,S,SR\}$.  We have $R^2=1$ so $R=R^{-1}$.  This
    means that $SRS=R^{-1}=R$, so $SR=RS$, so the group is
    commutative.  This means that each element is in a separate
    conjugacy class, so the classes are $\{1\}$, $\{R\}$, $\{S\}$ and
    $\{SR\}$. 
  \item $D_3=\{1,R,R^2,S,SR,SR^2\}$.  We have $R^3=1$ so
   $S^{-1}RS=R^{-1}=R^2$, showing that $R$ and $R^2$ are conjugate.  
 \end{itemize}
\end{solution}

\begin{exercise}
 Use the First Isomorphism Theorem to prove that $SO_2\simeq\R/2\pi\Z$.
\end{exercise}
\begin{solution}
 Define $\phi\:\R\xra{}SO_2$ by $\phi(\al)=R_\al$.  We have
 \[ \phi(\al)\phi(\bt)=R_\al R_\bt=R_{\al+\bt}=\phi(\al+\bt), \]
 so $\phi$ is a homomorphism.  Any element of $SO_2$ has the form
 $R_\al=\phi(\al)$ for some $\al$, so $\phi$ is surjective.  Thus, the
 First Isomorphism Theorem gives us an isomorphism
 $\ov{\phi}\:\R/\ker(\phi)\xra{}SO_2$.  Moreover, we have
 $\al\in\ker(\phi)$ iff $\phi(\al)=1$ iff $R_\al=R_0$ iff $\al$ is an
 integer multiple of $2\pi$, so $\ker(\phi)=2\pi\Z$.  Thus
 $\R/2\pi\Z\simeq SO_2$ as claimed.
\end{solution}

\begin{exercise}
 Suppose $x_0,x_1,\ldots x_n\in\R^n$ have the property that
 $x_1-x_0,x_2-x_0,\ldots,x_n-x_0 $ is a basis. If $f,g$ are isometries
 of $\R^n$ so that $f(x_i)=g(x_i)$ for $i=0,1,2,\ldots,n$ then show
 that $f=g$. Thus for instance an isometry of $\R^2$ is determined by
 the image of three points.
\end{exercise}
\begin{solution}

\end{solution}

\begin{exercise}
 Let $\R^\infty$ be the set of sequences $x=(x_1,x_2,\ldots)$ such that
 $x_n=0$ for $n\gg 0$.  We define as usual
 \begin{align*}
  \ip{x,y} &= \sum_{k=1}^\infty x_k y_k \\
  \|x\|    &= \sqrt{\ip{x,x}} \\
  d(x,y)   &= \|x-y\|. 
 \end{align*}
 Note that the infinite sum is really only a finite sum because when
 $k$ is large we have $x_k=y_k=0$.  Thus, there is no convergence
 problem to worry about.  Construct a function
 $f\:\R^\infty\xra{}\R^\infty$ that preserves distances but is not a
 bijection. 
\end{exercise}
\begin{solution}
 Define $f(x_1,x_2,x_3,\ldots)=(0,x_1,x_2,x_3,\ldots)$.  This is a
 linear map, and it satisfies
 \[ \|f(x)\|^2 = 0^2 + x_1^2 + x_2^2 + \ldots = \|x\|^2. \]
 We thus have 
 \[ d(f(x),f(y))=\|f(x)-f(y)\|=\|f(x-y)\|=\|x-y\|=d(x,y), \]
 so $f$ preserves distances.  However, the first entry in $f(x)$ is
 always $0$, so $f(x)$ cannot be equal to $(1,0,0,\ldots)$ for any
 $x$, so $f$ is not surjective and thus not bijective.
\end{solution}

\begin{exercise}
 Find the conjugacy classes in $O_2$.  What is the centre of $O_2$?
\end{exercise}
\begin{solution}

\end{solution}

\begin{exercise}
 \begin{itemize}
 \item[(a)] Consider the action of $O_2$ on $\R^2$. Identify the orbits
  as subsets of $\R^2$ and draw a picture. Identify the stabilizers.
 \item[(b)] Consider the action of $I(\R^2)$ on $\R^2$. Identify the orbits
  as subsets of $\R^2$ and draw a picture. Identify the stabilizers.
 \end{itemize}
\end{exercise}
\begin{solution}

\end{solution}

\begin{exercise}
 Let $X$ be the following subset of $\R^2$ (with centre at the
 origin).  The group $D_3$ acts on $X$.  Find the orbits of this
 action, find the fixed points of all the elements of $D_3$, and
 verify the orbit counting theorem.
 \begin{center}
  \begin{tikzpicture}[scale=2.5]
   \foreach \i in {0,60,...,300} {
    \fill (\i:1) circle(0.03);
    \fill ({\i+30}:0.87) circle(0.03);
   }
  \end{tikzpicture}
 \end{center}
\end{exercise}
\begin{solution}
 The points marked $1$ form an orbit of size $3$, the points marked
 $2$ form another orbit of size $3$, and the points marked $3$ form an
 orbit of size $6$.  Thus, there are $3$ orbits altogether.
 \begin{center}
  \begin{tikzpicture}[scale=2.5]
   \foreach \i in {0,120,240} {
    \draw (\i:1) -- (\i:-1);
    \fill[blue] (\i:1) circle(0.03);
    \draw (\i:1.1) node{$1$};
    \fill[red] ({\i+60}:1) circle(0.03);
    \draw ({\i+60}:1.1) node{$2$};
   }
   \foreach \i in {0,60,...,300} {
    \fill[green] ({\i+30}:0.88) circle(0.03);
    \draw ({\i+30}:1.1) node{$3$};
   }
  \end{tikzpicture}
 \end{center}
 Recall that 
 \[ D_3=\{1,R_{2\pi/3},R_{4\pi/3},S_0,S_{2\pi/3},S_{4\pi/3}\}. \]
 The identity element of $D_3$ fixes all $12$ points of $X$.  The
 rotations $R_{2\pi/3}$ and $R_{4\pi/3}$ have no fixed points in $X$.
 The axis of the reflection $S_0$ passes through $2$ of the points of
 $X$.  These two points are fixed under $S_0$, and the remaining
 points are not.  Similarly, the reflections $S_{2\pi/3}$ and
 $S_{4\pi/3}$ have two fixed points each.  Thus
 \[ \frac{1}{|G|}\sum_{g\in G}|\Fix(g)| = 
     \frac{1}{6}(12+0+0+2+2+2) = 3, 
 \]
 which is equal to the number of orbits, as predicted by the
 orbit counting theorem.
\end{solution}

\begin{exercise}
 Find an infinite subgroup $G<SO_2$ such that $G\neq SO_2$.  [Hint:
 think about rational and irrational numbers.]
\end{exercise}
\begin{solution}
 One possibility is to choose an irrational number $\al$ and define
 $R=R_{2\pi\al}$ and 
 \[ G = \ip{R} = \{R^n\st n\in\Z\} = \{R_{2\pi\al n}\st n\in\Z\}. \]
 This is clearly a subgroup of $SO_2$ (because $1=R^0\in G$ and
 $R^nR^m=R^{n+m}\in G$ and $(R^n)^{-1}=R^{-n}\in G$ for all
 $n,m\in\Z$.)  

 I next claim that all the elements $R^n$ are distinct, in other words
 that $R^n\neq R^m$ whenever $n\neq m$.  Indeed, if $R^n=R^m$ then
 $R_{2\pi\al(n-m)}=R^{n-m}=1$ so $2\pi\al(n-m)=2\pi k$ for some
 integer $k$.  If $n\neq m$ then this implies that $\al=k/(n-m)$,
 contradicting our assumption that $\al$ is irrational.  This proves
 that the elements $R^n$ are all distinct, so $G$ is infinite.  

 Finally we must prove that $G\neq SO_2$.  For those of you that know
 about countability, the ``real reason'' is that $G$ is countable and
 $SO_2$ is not.  For a more direct proof, it will suffice to show that
 $R_\pi\not\in G$.  If $R_\pi$ were an element of $G$ we would have
 $R_\pi=R^k$ for some $k\in\Z$, and $k\neq 0$ because certainly
 $R_\pi\neq 1=R^0$.  This would give $R^{2k}=R_\pi^2=1=R^0$,
 contradicting the fact that all the $R^n$'s are distinct.

 Two other possibilities are to take $G=\{R_{a\pi}\st a\in\Q\}$ or to
 take $G=\{R_\tht\st\tht\in\Q\}$.
\end{solution}

\begin{exercise}
 Put $H_n=\{A\in O_2\st A^n=1\}$.  Show that if $n$ is odd then $H_n$ is
 a finite subgroup of $O_2$ (which one?), but if $n$ is even then
 $H_n$ is not a subgroup at all.
\end{exercise}
\begin{solution}
 First suppose that $n$ is odd, say $n=2m+1$.  For any $\tht$ we have
 $S_\tht^2=1$ so $S_\tht^n=(S_\tht^2)^mS_\tht=S_\tht\neq 1$, so
 $S_\tht\not\in H_n$.  On the other hand, we have $R_\tht^n=1$ iff
 $n\tht=2k\pi$ for some $k\in\Z$, iff $\tht=2k\pi/n$ for some $k$, iff
 $R_\tht\in C_n$.  Thus $H_n=C_n$, which is a finite subgroup of
 $O_2$.

 Now suppose instead that $n$ is even, say $n=2m$.  Then for all
 $\tht$ we have $S_\tht^n=(S_\tht^2)^m=1$, so $S_\tht\in H_n$.  For
 most $\tht$ we have $R_\tht^n\neq 1$, so $R_\tht\not\in H_n$.  Thus
 $S_\tht$ and $S_0$ lie in $H_n$ but $S_\tht S_0=R_\tht$ does not;
 this shows that $H_n$ is not a subgroup.
\end{solution}

\begin{exercise}
 Show that if $G/Z(G)$ is a cyclic group then $G$ is abelian (hence
 in fact $Z(G)=G$). Identify $D_4/Z(D_4)$.
\end{exercise}
\begin{solution}

\end{solution}

\begin{exercise}
 The Quaternion group of order 8  is the group
 \[ G = \{\pm 1, \pm i, \pm j, \pm k \} \]
 with $ij=-ji=k$, $jk=-kj=i$, $ki=-ik=j$, $i^2=j^2=k^2=-1$. 
 Do there exist subgroups $N,Q<G$ (with $N\neq G$ and $Q\neq G$) such
 that $G$ is the semidirect product of $N$ and $Q$?  (One approach is
 just to find all the subgroups of $G$).
\end{exercise}
\begin{solution}
 No such subgroups exist.  To prove this, I first claim that any
 nontrivial subgroup $H\leq G$ contains the element $-1$.  Indeed, we
 have $G=\{1,-1,i,-i,j,-j,k,-k\}$ and
 $i^2=(-i)^2=j^2=(-j)^2=k^2=(-k)^2=-1$.  As $H$ is nontrivial it must
 either contain $-1$ (so there is nothing to say) or some element
 $h\in\{\pm i,\pm j,\pm k\}$, in which case it also contains $h^2=-1$,
 as claimed.  

 Now suppose that $G$ is the semidirect product of $N$ and $Q$, where
 $N\neq G$ and $Q\neq G$.  We then have $|N|<|G|$ so $|Q|=|G|/|N|>1$,
 so $Q$ is nontrivial, so $-1\in Q$.  Similarly $-1\in N$, so
 $-1\in N\cap Q$.  However, for a semidirect product we must have
 $N\cap Q=\{1\}$, so this gives a contradiction.
 
 The complete list of subgroups is as follows:
 \begin{align*}
  & \{1\} \\
  & \{1,-1\} \\
  & \{1,-1,i,-i\} \\
  & \{1,-1,j,-j\} \\
  & \{1,-1,k,-k\} \\
  & \{1,-1,i,-i,j,-j,k,-k\}.
 \end{align*}
\end{solution}

\begin{exercise}
 If $X$ is a non-empty subset of a group $G$, we write
 \[ \langle X \rangle = 
    \left\{ x_1^{\varepsilon_1} \cdots x_t^{\varepsilon_t}
     \st t \geq 1, \; x_1,\cdots, x_t\in X, \;
         \varepsilon_1, \cdots \varepsilon_t \in \{\pm 1\} \right\}
 \]
 and call it the subgroup of $G$ generated by $X$.

 Suppose $G$ is a group with elements $a, b \in G$, so that 
 $G=\langle a,b\rangle$. Show that if $a$ and $b$ are of order 2 and
 $ab$ is of order $n$ then $G \cong D_n$.
\end{exercise}
\begin{solution}

\end{solution}

\begin{exercise}
 Use your classification of the conjugacy classes in $D_n$ to find all
 the normal subgroups of $D_n$.
\end{exercise}
\begin{solution}

\end{solution}

\begin{exercise}
 Let $G$ be a subgroup of $SO_2$.  Suppose that there exists a number
 $\ep>0$ such that $R_\tht\not\in G$ for $0<\tht<\ep$.  Prove that
 $G=C_n$ for some $n$ (and thus that $G$ is finite).  [Hint: mimic the
 classification of finite subgroups of $C_n$.]
\end{exercise}
\begin{solution}
 Put $S=\{\phi>0\st R_\phi\in G\}$, so $2\pi\in S$.  If we take $m=1$
 then by assumption we have $S\cap(0,m\ep)=\emptyset$.  If we take $m$
 to be very large then $2\pi\in S\cap(0,m\ep)$ so
 $S\cap(0,m\ep)\neq\emptyset$.  Thus, there must be some intermediate
 value of $m$ such that $S\cap(0,m\ep)=\emptyset$ and
 $S\cap(0,(m+1)\ep)\neq\emptyset$.  Choose
 $\tht\in S\cap(0,(m+1)\ep)$.  I claim that for $\phi\in S$ we have
 $\phi\geq\tht$.  Suppose not, so $\phi<\tht$.  Put $\psi=\tht-\phi$,
 so $\psi>0$ and $R_\psi=R_\tht R_\phi^{-1}\in G$ so $\psi\in S$.  
 We also have $\tht<(m+1)\ep$.  As $S\cap(0,m\ep)=\emptyset$ we
 certainly have $\phi\geq m\ep$, so
 $\psi=\tht-\phi<(m+1)\ep-m\ep=\ep$.  This contradicts the assumption
 that $S\cap(0,\ep)=\emptyset$, so we must have $\phi\geq\tht$ after
 all.  

 Now let $\al$ be any element of $S$.  For some $k\geq 0$ we must have
 $k\tht\leq\al<(k+1)\tht$.  If we put $\bt=\al-k\tht$ then
 $0\leq\bt<\tht$ and $R_\bt=R_\al R_\tht^{-k}\in G$.  As every element
 of $S$ is at least as large as $\tht$ we see that $\bt$ cannot lie in
 $S$, and the only way this can happen is if $\bt=0$.  Thus
 $\al=k\tht$ for some $k>0$.

 By applying this in the case $\al=2\pi$ we see that $2\pi=n\tht$ for
 some $n$ and thus that $\tht=2\pi/n$.  It follows that
 $S=\{2\pi k/n\st k>0\}$ and thus that $G=C_n$.
\end{solution}

\begin{exercise}
 The quaternion group of order 8  is the group
 \[ G = \{\pm 1, \pm i, \pm j, \pm k \} \]
 with $ij=-ji=k$, $jk=-kj=i$, $ki=-ik=j$, $i^2=j^2=k^2=-1$. 
 Show that if $P$ and $Q$ are any two nontrivial subgroups of $G$,
 then $P\cap Q$ is also nontrivial.
\end{exercise}
\begin{solution}
 We first claim that any nontrivial subgroup $P\leq G$ contains the
 element $-1$.  Indeed, we have $G=\{1,-1,i,-i,j,-j,k,-k\}$ and
 $i^2=(-i)^2=j^2=(-j)^2=k^2=(-k)^2=-1$.  As $P$ is nontrivial it must
 either contain $-1$ (so there is nothing to say) or some element
 $x\in\{\pm i,\pm j,\pm k\}$, in which case it also contains $x^2=-1$,
 as claimed.  By the same argument, $Q$ must contain $-1$, so
 $P\cap Q$ contains $-1$, so $P\cap Q$ is nontrivial.
\end{solution}

\begin{exercise}
 Prove or disprove the following statements:
 \begin{itemize}
  \item[(i)] $D_6/Z(D_6)\simeq D_3,$ where $Z(D_6)$ is the centre of $D_6$.
  \item[(ii)] $D_{42}$ has a non-cyclic subgroup of order 21.
  \item[(iii)] $D_8\simeq D_4\times C_2$.
  \item[(iv)] There is an integer $n\geq 1$ such that there is a
   surjective homomorphism $D_n\xra{}Q_8$.
  \item[(v)] $D_{10}\cong D_5\times C_2$.
 \end{itemize}
\end{exercise}
\begin{solution}

\end{solution}

\begin{exercise}
 Let $\delta\:I_n\xra{}\{\pm 1\}$ be the composite of
 $\psi\:I_n\xra{}I_n/T_n\simeq O_n$ and the determinant map (ie
 $\delta(f)=\det(\psi(f))$).
 \begin{itemize}
  \item[(a)] If $f\in O_n$ show that $\delta(f)=\det(f)$.
  \item[(b)] When $n = 2$ find the value of $\delta$ on reflections,
   rotations, translations and glides.
  \item[(c)] If $X \subseteq \R^n$ write
   \[ SI(X) = \{f \in I(X) \st \delta(f) = 1 \}. \]
   Show that $SI(X)$ is a subgroup of $I(X)$ and that either
   $SI(X)=I(X)$ or $[I(X):SI(X)]=2$.
 \end{itemize}
\end{exercise}
\begin{solution}

\end{solution}

\begin{exercise}
 Let $f,g\in I_n$ be given by $f(x)=Ax+a$ and $g(x)=Bx+b$, where
 $A,B\in O_n$ and $a,b\in\R^n$.
 \begin{itemize}
 \item[(a)] Find $C\in O_n$ and $c\in\R^n$ such that $fg(x)=Cx+c$ for
  all $x$.
 \item[(b)] Find $D$ and $d$ such that $fgf^{-1}(x)=Dx+d$ for all $x$.
 \item[(c)] Describe the isometry $fT_bf^{-1}$.
 \item[(d)] Show that $T_{(1,0)}$ is not conjugate to $T_{(0,2)}$ in
  $I_2$.   
 \end{itemize}
\end{exercise}
\begin{solution}
 \begin{itemize}
 \item[(a)] We have $fg(x)=f(Bx+b)=A(Bx+b)+a=ABx+(Ab+a)$, so we can
  take $C=AB$ and $c=Ab+a$.
 \item[(b)] First, if $x=f(y)=Ay+a$ then
  $f^{-1}(x)=y=A^{-1}(x-a)=A^{-1}x-A^{-1}a$.  Thus 
  \begin{align*}
   fgf^{-1}(x) &= fg(A^{-1}x - A^{-1}a) \\
               &= f(BA^{-1}x - BA^{-1}a + b) \\
               &= ABA^{-1}x - ABA^{-1}a + Ab + a.
  \end{align*}
  Thus we can take $D=ABA^{-1}$ and $d=a+Ab-ABA^{-1}a$.
 \item[(c)] This is what we get in the case $B=1$.  We then have
  $ABA^{-1}=1$, so part~(b) gives
  $fT_bf^{-1}(x)=x-a+Ab+a=x+Ab=T_{Ab}(x)$, so $fT_bf^{-1}=T_{Ab}$.
 \item[(d)] If $T_b$ is conjugate to $T_c$ then $T_c=fT_bf^{-1}$ for
  some $f\in I_2$.  If we write $f$ in the form $f(x)=Ax+a$ (with
  $A\in O_2$), we see from~(c) that $fT_bf^{-1}=T_{Ab}$, so $Ab=c$.
  Thus $\|c\|=\|Ab\|=\|b\|$.  As $\|(1,0)\|\neq\|(0,2)\|$, we deduce
  that $T_{(1,0)}$ is not conjugate to $T_{(0,2)}$.  
 \end{itemize}
\end{solution}

\begin{exercise}
 Let $H$ be a wallpaper group, and let $a$ be a point in $\R^2$.
 Recall that $\text{orbit}_H(a)=\{h(a)\st h\in H\}$.  Prove that if
 $\text{orbit}_H(a)=\text{orbit}_{T(H)}(a)$ then $\sg_a(H)=\psi(H)$.
\end{exercise}
\begin{solution}
 For any group $H\leq I_2$ we have seen that $\sg_a(H)\leq\psi(H)$, so
 we need only prove the opposite inequality.  Suppose that
 $B\in\psi(H)$, so there is some element $h\in H$ with $\psi(h)=B$.
 We have $h(a)\in \text{orbit}_H(a)=\text{orbit}_{T(H)}(a)$, so there
 must be some $u\in T(H)$ such that $h(a)=a+u$.  Put $g=T_{-u}h$,
 so $g\in H$ and $g(a)=a$.  This means that the map $f:=T_{-a}gT_a$
 satisfies $f(0,0)=(0,0)$, so $f(x)=Ax$ for some $A\in O_2$.  From the
 definition of $\sg_a(H)$ we see that $A\in\sg_a(H)$.  On the other
 hand, we have 
 \[ A = \psi(f) = \psi(T_{-a}gT_a) = \psi(T_{-u-a}hT_a) =
    \psi(T_{-u-a})\psi(h)\psi(T_a) = 1.B.1 = B,
 \] 
 so $A=B$.  As $A\in\sg_a(H)$, this means that $B\in\sg_a(H)$, as
 required. 
\end{solution}

\begin{exercise}
 Let $H$ be a wallpaper group, and put $L=\{h(0)\st h\in H\}$.  Prove
 that $H\cap O_2\leq\psi(H)$.  Prove also that if $L=\Trans(H)$, then
 $\psi(H)=H\cap O_2$. 
\end{exercise}
\begin{solution}
 First, if $A\in O_2$ then $A=\psi(A)$.  Thus, if $A\in H\cap O_2$
 then $A=\psi(A)\in\psi(H)$, so $H\cap O_2\leq\psi(H)$.
 
 Now suppose that $L=\Trans(H)$; we must show that
 $\psi(H)\leq H\cap O_2$.  If $A\in\psi(H)$ then there is an element
 $h\in H$ with $\psi(h)=A$, which means that $h(x)=Ax+a$ for some
 $a\in\R^2$.  Next, note that $a=h(0)$, so $a\in L$ (by the definition
 of $L$).  We are assuming that $L=\Trans(H)$, so $a\in\Trans(H)$,
 which means that $T_a\in H$.  This means that the function
 $g=T_a^{-1}h$ also lies in $H$.  Clearly $g(x)=h(x)-a=Ax$, so $g$
 corresponds to the element $A\in O_2$.  This means that
 $A\in H\cap O_2$, as claimed.
\end{solution}

\begin{exercise}
 Three infinite wall-paper patterns are represented below
 by a small segment of the pattern.
 \begin{itemize}
  \item[(a)] Find the isometry group of pattern (a).
  \item[(b)] Find the isometry group of pattern (b). 
  \item[(c)] Find the isometry group of pattern (c) and show that it is
   generated by a reflection and a rotation.
 \end{itemize}

 \begin{center}
  \begin{tikzpicture}
   \begin{scope}
    \begin{scope}[scale=0.45]
     \def\a{0.3}
     \def\b{0.1}
     \def\c{0.6}
     \def\d{0.3}
     \foreach \i in {1,2,3,4,5} {
      \foreach \j in {1,2,3,4,5} {
       \begin{scope}[fill=green!40,draw=black,shift={({2.2 * \i},{2.2 * \j})}]
        \filldraw ( \a, \b) -- ({ \a+\c}, \b) -- ({ \a+\c},{ \b+\d}) -- cycle;
        \filldraw ( \a,-\b) -- ({ \a+\c},-\b) -- ({ \a+\c},{-\b-\d}) -- cycle;
        \filldraw (-\a, \b) -- ({-\a-\c}, \b) -- ({-\a-\c},{ \b+\d}) -- cycle;
        \filldraw (-\a,-\b) -- ({-\a-\c},-\b) -- ({-\a-\c},{-\b-\d}) -- cycle;
        \filldraw ( \b, \a) -- ( \b,{ \a+\c}) -- ({ \b+\d},{ \a+\c}) -- cycle;
        \filldraw (-\b, \a) -- (-\b,{ \a+\c}) -- ({-\b-\d},{ \a+\c}) -- cycle;
        \filldraw ( \b,-\a) -- ( \b,{-\a-\c}) -- ({ \b+\d},{-\a-\c}) -- cycle;
        \filldraw (-\b,-\a) -- (-\b,{-\a-\c}) -- ({-\b-\d},{-\a-\c}) -- cycle;
       \end{scope}
      }
     }
    \end{scope}
   \end{scope}
   \begin{scope}[shift={(6.5,1)}]
    \begin{scope}[scale=0.13]
     \def\M{
     \filldraw%
     (-3, 0) -- (-2, 1) -- ( 0, 1) -- (-1, 2) -- ( 0, 3) -- %
     ( 1, 2) -- ( 2, 3) -- ( 2, 1) -- ( 3, 0) -- ( 2,-1) -- %
     ( 2,-3) -- ( 1,-2) -- ( 0,-3) -- (-1,-2) -- ( 0,-1) -- %
     (-2,-1) -- cycle;}
     \foreach \i in {0,1,2} {
      \foreach \j in {0,1,2} {
       \begin{scope}[shift={({12*\i},{12*\j})}]
        \begin{scope}[fill=green!30,xscale=-1] \M \end{scope}
        \begin{scope}[fill=green!30,xscale=-1,shift={(-6,6)}] \M \end{scope}
        \begin{scope}[fill=blue!30 ,draw=black,shift={(6,0)}] \M \end{scope}
        \begin{scope}[fill=blue!30 ,draw=black,shift={(0,6)}] \M \end{scope}     
       \end{scope}
      }
     }
    \end{scope}
   \end{scope}
   \begin{scope}[shift={(12,1)}]
    \begin{scope}[scale=0.46,fill=orange]
     \def\a{0.3}
     \def\b{0.1}
     \def\c{0.6}
     \def\d{0.3}
     \foreach \i in {0,1,2,3,4} {
      \foreach \j in {0,1,2} {
       \foreach \k in {0,1} {
        \begin{scope}[shift={({2*\i+\k},{1.73*(2*\j+\k)})}]
         \foreach \p in {0,120,240} {
          \begin{scope}[rotate=\p]
           \filldraw ( \a, \b) -- ({ \a+\c}, \b) -- ({ \a+\c},{ \b+\d}) -- cycle;
           \filldraw ( \a,-\b) -- ({ \a+\c},-\b) -- ({ \a+\c},{-\b-\d}) -- cycle;
          \end{scope}
         }
        \end{scope}
       }
      }
     }
    \end{scope}
   \end{scope}
   \draw( 3.0,-0.2) node{Pattern (a)};
   \draw( 8.5,-0.2) node{Pattern (b)};
   \draw(14.0,-0.2) node{Pattern (c)};
  \end{tikzpicture}
 \end{center}
\end{exercise}
\begin{solution}
 {\bf Pattern~(a):} Let $R$ be rotation throught $\pi/2$ about $O$,
 and let $S$ be reflection across $L$.  It is clear that $T_u$, $T_v$,
 $R$ and $S$ preserve $X$, so $\ip{T_u,T_v,R,S}\leq I(X)$.  Now
 suppose we have $f_0\in I(X)$.  If $\det(f_0)=-1$ we put $f_1=Sf_0$,
 otherwise we put $f_1=f_0$; either way we have $\det(f_1)=1$ and
 $f_1\in I(X)$.  Clearly $f_1$ must send $O$ to the centre of one of
 the motifs, so $f_1(O)=nu+mv+O$ for some $n,m\in\Z$.  We put
 $f_2=T_u^{-n}T_v^{-m}f_1$, so $f_2\in I(X)$, $\det(f_2)=1$ and
 $f_2(O)=O$.  Thus $f_2$ is a rotation about $O$ that preserves $X$;
 clearly the angle must be a multiple of $\pi/2$, so $f_2=R^k$ for
 some $k$.  We thus have $f_1=T_v^mT_u^nR^k$ and either
 $f_0=T_v^mT_u^nR^k$ or $f_0=ST_v^mT_u^nR^k$.  Thus
 $f_0\in\ip{T_u,T_v,R,S}$, which proves that $I(X)=\ip{T_u,T_v,R,S}$.
 \begin{center}
  \begin{tikzpicture}[scale=1.2]
   \def\a{0.3}
   \def\b{0.1}
   \def\c{0.6}
   \def\d{0.3}
   \def\e{2.2}
   \foreach \i in {-1,0,1} {
    \foreach \j in {-1,0,1} {
     \begin{scope}[fill=green!40,draw=black,shift={({\e * \i},{\e * \j})}]
      \filldraw ( \a, \b) -- ({ \a+\c}, \b) -- ({ \a+\c},{ \b+\d}) -- cycle;
      \filldraw ( \a,-\b) -- ({ \a+\c},-\b) -- ({ \a+\c},{-\b-\d}) -- cycle;
      \filldraw (-\a, \b) -- ({-\a-\c}, \b) -- ({-\a-\c},{ \b+\d}) -- cycle;
      \filldraw (-\a,-\b) -- ({-\a-\c},-\b) -- ({-\a-\c},{-\b-\d}) -- cycle;
      \filldraw ( \b, \a) -- ( \b,{ \a+\c}) -- ({ \b+\d},{ \a+\c}) -- cycle;
      \filldraw (-\b, \a) -- (-\b,{ \a+\c}) -- ({-\b-\d},{ \a+\c}) -- cycle;
      \filldraw ( \b,-\a) -- ( \b,{-\a-\c}) -- ({ \b+\d},{-\a-\c}) -- cycle;
      \filldraw (-\b,-\a) -- (-\b,{-\a-\c}) -- ({-\b-\d},{-\a-\c}) -- cycle;
     \end{scope}
    }
   }
   \draw[blue] ({-1.5 * \e},0) -- ({1.8 * \e},0);
   \draw[blue] ({-0.5 * \e},{-1.5 * \e}) -- ({ 1.8 * \e},{ 0.8 * \e});
   \draw[thick,red,<->] ({-\e},{0.5 * \e}) -- ({-\e},{-0.5*\e}) -- (0,{-0.5*\e});
   \fill (    {-\e},{-0.5*\e}) circle(0.05);
   \fill (        0,        0) circle(0.05);
   \fill (        0,{ 0.5*\e}) circle(0.05);
   \fill ({ 0.5*\e},{ 0.5*\e}) circle(0.05);
   \draw ({ 1.7*\e},        0) node[anchor=north     ] {$L$};
   \draw ({ 1.7*\e},{ 0.7*\e}) node[anchor=north west] {$M$};
   \draw (        0,        0) node[anchor=north     ] {$O$};
   \draw (        0,{ 0.5*\e}) node[anchor=east      ] {$P$};
   \draw ({ 0.5*\e},{ 0.5*\e}) node[anchor=east      ] {$Q$};
   \draw ({-0.5*\e},{-0.5*\e}) node[anchor=north     ] {$u$};
   \draw ({-    \e},        0) node[anchor=north east] {$v$};
  \end{tikzpicture}
 \end{center} 
 In particular, we see that $S_M$, $R_{P,\pi}$ and $R_{Q,\pi/2}$ all
 lie in $\ip{T_u,T_v,R,S}$.  By following the above recipe we find
 that $S_M=ST_vT_uR$, $R_{P,\pi}=T_vR^2$ and $R_{Q,\pi/2}=T_uR$.

 {\bf Pattern~(b):} Clearly $\ip{T_u,T_v,S_L}\leq I(X)$.  Suppose that
 $f_0\in I(X)$, and put $f_1=S_Lf_0$ if $\det(f_0)=-1$ and $f_1=f_0$
 otherwise.  Note that $O$ is the point where the blunt ends of two
 white motifs meet, so $f_1(O)$ must also be the point of intersection
 of the blunt ends of two white motifs, so $f_1(O)=O+nu+mv$ for some
 $n,m\in\Z$.  Put $f_2=T_u^{-n}T_v^{-m}f_1$, so $f_2$ is a rotation
 around $O$ that preserves the pattern $X$.  There is only one dark
 grey motif adjacent to $O$, so $f_2$ must send that motif to itself,
 and this forces $f_2$ to be the identity.  Thus either
 $f_0=T_v^mT_u^n$ or $f_0=S_LT_v^mT_u^n$, and in either case we have
 $f_0\in\ip{S_L,T_u,T_v}$.  This shows that $I(X)=\ip{S_L,T_u,T_v}$.
 \begin{center}
  \begin{tikzpicture}[scale=0.3]
   \def\M{
   \filldraw%
   (-3, 0) -- (-2, 1) -- ( 0, 1) -- (-1, 2) -- ( 0, 3) -- %
   ( 1, 2) -- ( 2, 3) -- ( 2, 1) -- ( 3, 0) -- ( 2,-1) -- %
   ( 2,-3) -- ( 1,-2) -- ( 0,-3) -- (-1,-2) -- ( 0,-1) -- %
   (-2,-1) -- cycle;}
   \foreach \i in {0,1} {
    \foreach \j in {0,1} {
     \begin{scope}[shift={({12*\i},{12*\j})}]
      \begin{scope}[fill=green!30,xscale=-1] \M \end{scope}
      \begin{scope}[fill=green!30,xscale=-1,shift={(-6,6)}] \M \end{scope}
      \begin{scope}[fill=blue!30 ,draw=black,shift={(6,0)}] \M \end{scope}
      \begin{scope}[fill=blue!30 ,draw=black,shift={(0,6)}] \M \end{scope}     
     \end{scope}
    }
   }
   \draw[blue] (-4,6) -- (22,6);
   \draw[<->,thick,red] (3,12) -- (9,6) -- (15,12);
   \fill (9,6) circle(0.2);
   \draw (9,6) node[anchor=north] {$O$};
   \draw ( 4,10) node {$u$};
   \draw (14,10) node {$v$};
   \draw (22,6) node[anchor=north west] {$L$};
  \end{tikzpicture}
 \end{center}

 {\bf Pattern~(c):} Put $S=S_L$ and $R=R_{O,2\pi/3}$, so
 $\ip{R,S}\leq I(X)$. 
 \begin{center}
  \begin{tikzpicture}[scale=1.0]
   \def\a{0.3}
   \def\b{0.1}
   \def\c{0.6}
   \def\d{0.3}
   \foreach \i in {0,1,2,3} {
    \foreach \j in {0,1} {
     \foreach \k in {0,1} {
      \begin{scope}[shift={({2*\i+\k},{1.73*(2*\j+\k)})}]
       \foreach \p in {0,120,240} {
        \begin{scope}[rotate=\p]
         \filldraw[fill=orange] ( \a, \b) -- ({ \a+\c}, \b) -- ({ \a+\c},{ \b+\d}) -- cycle;
         \filldraw[fill=orange] ( \a,-\b) -- ({ \a+\c},-\b) -- ({ \a+\c},{-\b-\d}) -- cycle;
        \end{scope}
       }
      \end{scope}
     }
    }
   }
   \draw[dotted] (-2,0) +( 60: 2.0) -- +( 60:7.5);
   \draw[dotted] ( 0,0) +( 60:-1.5) -- +( 60:7.5);
   \draw[blue  ] ( 2,0) +( 60:-1.5) -- +( 60:7.5);
   \draw[dotted] ( 4,0) +( 60:-1.5) -- +( 60:7.5);
   \draw[dotted] ( 6,0) +( 60:-1.5) -- +( 60:3.5);
   \draw[dotted] ( 0,0) +(120:-1.5) -- +(120:2.0);
   \draw[dotted] ( 2,0) +(120:-1.5) -- +(120:6.0);
   \draw[dotted] ( 4,0) +(120:-1.5) -- +(120:7.5);
   \draw[dotted] ( 6,0) +(120:-1.5) -- +(120:7.5);
   \draw[dotted] ( 8,0) +(120: 0.5) -- +(120:7.5);
   \draw[dotted] ( 0,   0) +(-1,0) -- +(8,0);
   \draw[dotted] ( 0,1.73) +(-1,0) -- +(8,0);
   \draw[dotted] ( 0,3.46) +(-1,0) -- +(8,0);
   \draw[dotted] ( 0,5.19) +(-1,0) -- +(8,0);
   \draw[->,thick,red] (60:4) +(300:2) -- +(300:0.1);
   \draw[->,thick,red] (60:2) +(2,0) -- +(3.9,0);
   \fill (60:4)        circle(0.05);
   \fill (60:2) +(2,0) circle(0.05);
   \fill (60:4) +(2,0) circle(0.05);
   \fill (60:2) +(4,0) circle(0.05);
   \fill (60:4) +(330:1.15) circle(0.05);
   \draw (3.1,2.7) node {$O$};
   \draw (3.1,1.5) node {$P$};
   \draw (1.7,3.4) node {$P_2$};
   \draw (3.7,3.4) node {$P_1$};
   \draw (4.3,1.5) node {$v$};
   \draw (2.2,2.8) node {$u$};
  \end{tikzpicture}
 \end{center}
 As $\psi(S)$ is a reflection and $\psi(R)$ is a
 rotation we see that $\psi(RS)=\psi(R)\psi(S)$ is a reflection and
 thus that $\psi(RSRS)=\psi(RS)^2=1$, so $RSRS$ is a translation.  As
 $P$ lies on $L$ we have $S(P)=P$ and $RS(P)=R(P)=P_1$.  This lies on
 $L$ again, so $SRS(P)=S(P_1)=P_1$, and it follows that
 $RSRS(P)=R(P_1)=P_2$.  Thus $RSRS$ is a translation sending $O$ to
 $P_2$, so we must have $RSRS=T_u$.  Similarly, we have $SRSR=T_v$.
 Next, put $R'=T_u^{-1}T_v^{-1}R$.  This has $\psi(R')=R_{2\pi/3}$, so
 $R'$ must be a rotation through $2\pi/3$ about some point.  We have
 seen that $R(P)=P_1=P+u+v$ so $R'(P)=P$ so $P$ must be the centre of
 the rotation $R'$, so $R'=R_{P,2\pi/3}$.  Now suppose that
 $f\in I(X)$.  Then $f$ must send $P$ to the centre of some motif, say
 $f(P)=nu+mv$ for some $n,m\in\Z$, so $T_v^{-m}T_u^{-n}f$ fixes $P$
 and preserves $X$.  After multiplying by $S$ if necessary we get a
 rotation that fixes $P$ and preserves $X$, which must be a power of
 $R'$.  As $T_u$, $T_v$, $S$ and $R'$ lie in $\ip{R,S}$ we deduce that
 $f\in\ip{R,S}$.  Thus $I(X)=\ip{R,S}$ as required.

\end{solution}

\begin{exercise}
 Let $X$ be the wallpaper pattern shown below.  We have seen that
 $I(X)=\ip{T_u,T_v,R_{\pi/3},S_0}$, where $u=(1,0)$ and
 $v=(1/2,\sqrt{3}/2)$.  We also see geometrically that the rotation
 $R_{P,\pi}$ and the glide $G_{L,u/2}$ preserve $X$, so it must be
 possible to write $R_{P,\pi}$ and $G_{L,u/2}$ in terms of $T_u$,
 $T_v$, $R_{\pi/3}$ and $S_0$.  Do this explicitly.  [Hint: just
 follow through the steps in the proof that
 $I(X)=\ip{T_u,T_v,R_{\pi/3},S_0}$.]
 \begin{center}
  \begin{tikzpicture}[scale=0.9]
   \foreach \i in {0,1,2,3} {
    \foreach \j in {0,1} {
     \foreach \k in {0,1} {
      \begin{scope}[shift={({2*\i-\k},{1.73*(2*\j+\k)})}]
       \filldraw[fill=cyan!70] (0,0) circle(0.6); 
      \end{scope}
     }
    }
   }
   \draw[blue] (-2,{1.73 * 1.5}) -- (7,{1.73 * 1.5});
   \fill (1.0,1.73) circle(0.08);
   \fill (2.0,1.73) circle(0.08);
   \draw (1.0,1.45) node {$O$};
   \draw (2.0,1.45) node {$P$};
   \draw (6.5,2.20) node {$L$};
  \end{tikzpicture}
 \end{center}
\end{exercise}
\begin{solution}
 We have 
 \[ G_{L,u/2}(0,0)=u/2+S_L(0,0)=(1/2,0)+(0,\sqrt{3}/2)=v, \]
 so the map $f=T_{-v}G_{L,u/2}=T_{u/2-v}S_L$ satisfies $f(0,0)=(0,0)$.
 As $L$ is parallel to the $x$-axis we have $\psi(S_L)=S_0$ and
 $\psi(T_a)=1$ for all $a$ so $\psi(f)=S_0$.  Thus $f(x)=S_0(x)+b$ for
 some $b$, but $f(0,0)=(0,0)$ so $b=(0,0)$ so $f=S_0$.  Thus
 $G_{L,u/2}=T_vf=T_vS_0$, which writes $G_{L,u/2}$ in terms of
 $\{T_u,T_v,R_{\pi/3},S_0\}$ as required.

 Similarly, we have $R_{P,\pi}(0,0)=(1,0)=u$ so the map
 $h=T_{-u}R_{P,\pi}$ satisfies $h(0,0)=(0,0)$.  It also has
 $\psi(h)=\psi(T_{-u})\psi(R_{P,\pi})=R_\pi$ so we must have
 $h=R_\pi$, so $R_{P,\pi}=T_uR_\pi=T_uR_{\pi/3}^3$.
\end{solution}

\begin{exercise}
 Let $T$ be a triangle with angles $\pi/2$, $\pi/6$ and $\pi/3$.  Let
 $S_l$, $S_m$ and $S_n$ be the reflections in the three sides of $T$.
 Find a pattern $X$ with isometry group $\text{Isom}(X)$ generated by
 $S_l,S_m$ and $S_n$.
\end{exercise}
\begin{solution}
 Let $X$ be the pattern of hexagons shown below.
 \begin{center}
  \begin{tikzpicture}[scale=1.8]
   \def\r{0.4}
   \def\M{\draw[very thick,magenta] (30:\r) -- (90:\r) -- (150:\r) -- (210:\r) -- (270:\r) -- (330:\r) -- cycle;}
   \begin{scope}[shift={(-1,-1.73)}] \M \end{scope}
   \begin{scope}[shift={( 1,-1.73)}] \M \end{scope}
   \begin{scope}[shift={( 3,-1.73)}] \M \end{scope}
   \begin{scope}[shift={(-2, 0.00)}] \M \end{scope}
   \begin{scope}[shift={( 0, 0.00)}] \M \end{scope}
   \begin{scope}[shift={( 2, 0.00)}] \M \end{scope}
   \begin{scope}[shift={(-1, 1.73)}] \M \end{scope}
   \begin{scope}[shift={( 1, 1.73)}] \M \end{scope}
   \begin{scope}[shift={( 3, 1.73)}] \M \end{scope}
   \fill[yellow] (0,0) -- (-1,0) -- (-1,0.57);
   \draw[red ]  (-2.8, 0) -- ( 3.5,0);
   \draw[blue]  (-1,-2.6) -- (-1,2.6);
   \draw[blue]  ( 0,-2.6) -- ( 0,2.6);
   \draw[olivegreen] (150:2.8) -- (150:-4.0);
   \fill (-2, 0.00) circle(0.04);
   \fill ( 0, 0.00) circle(0.04);
   \fill (-1,-1.73) circle(0.04);
   \fill ( 1,-1.73) circle(0.04);
   \draw[thick,red,->] (-1,-1.73) -- +(120:1.95);
   \draw[thick,red,->] (-1,-1.73) -- +(  0:1.95);
   \draw ( 0.2, 0.2) node {$O$};
   \draw (-1.1, 0.3) node{$l$};
   \draw ( 0.1,-0.8) node{$l'$};
   \draw (-0.6,-0.1) node{$m$};
   \draw (-0.6, 0.5) node{$n$};
   \draw ( 0.3,-1.8) node{$u$};
   \draw (-1.7,-0.8) node{$v$};
  \end{tikzpicture}
 \end{center}
 I claim that $I(X)=\ip{S_l,S_m,S_n}$.  One checks directly that
 $S_l$, $S_m$ and $S_n$ send $X$ to itself, so
 $\ip{S_l,S_m,S_n}\leq I(X)$.  Clearly $S_m=S_0$ and $S_n=S_{-2\pi/6}$
 and it follows that $\ip{S_m,S_n}=D_6$.  Moreover, the $S_{l'}=S_\pi$
 lies in $D_6$ so it can be written in terms of $S_n$ and $S_m$ (an
 explicit expression is $S_{l'}=S_nS_mS_nS_mS_n$).  It follows that
 the map $T_u=S_{l'}S_l$ lies in $\ip{S_l,S_m,S_n}$.  We also have
 $S_mS_n=R_{\pi/3}$ so the map
 $T_v=T_{R_{\pi/3}u}=R_{\pi/3}T_uR_{\pi/3}^{-1}$ also lies in
 $\ip{T_l,T_m,T_n}$.  Given an arbitary element $g\in I(X)$ we see in
 the usual way that the map $h=T_u^{-n}T_v^{-m}g$ fixes $O$ for some
 $n,m\in\Z$, and thus $h$ lies in the symmetry group of the hexagon
 around $O$, which is the group $D_6=\ip{S_m,S_n}$.  It follows that
 the map $g=T_v^mT_u^nh$ lies in $\ip{S_l,S_m,S_n}$, which proves that
 $I(X)=\ip{S_l,S_m,S_n}$ as claimed.
\end{solution}

\begin{exercise}
\begin{itemize}
\item[(a)] Let the group $G$ act on the non empty set $X$.  We say
 that $G$ acts \emph{transitively} on $X$ if for all $x, y \in X$
 there is an element $g\in G$ so that $g*x=y$.

 Show that the following are equivalent
 \begin{itemize}
 \item[(1)] $G$ acts transitively on $X$
 \item[(2)] For any $z\in X$ we have $G * z =X$
 \item[(3)] For some $z\in X$ we have  $G*z=X$.
 \end{itemize}
\item[(b)] Decide which of the following actions are transitive. 
 \begin{itemize}
 \item[(1)] $S_n$ acting naturally on $\{1,2,\ldots,n\}$.
% \item[(2)] $GL_n$ acting naturally on the set of $d$-dimensional
%  subspaces of $\R^n$ (for some fixed $d$ with $0 \leq d \leq n$).
 \item[(2)] $D_4$ acting on the square $X_4$.
 \item[(3)] $S_6$ acting by conjugation on the set of elements of
  $S_6$ having order 3 (so $\theta * \phi = \theta \phi \theta^{-1}$).
 \end{itemize}
\item[(c)] Let $G$ be a group and $H$ be a subgroup.  Then $G$ acts on
 $G/H=\{x H\st x \in G\}$, the set of left cosets of $H$ in $G$, by
 left multiplication: $g*xH=gxH$, for $g\in G$ and $xH\in G/H$.  Show
 $G$ acts transitively on $G/H$.
\item[(d)] If $G$ acts on sets $X_1$ and $X_2$ by
 $\bullet\:G\tm X_1\xra{}X_1$ and $*\:G\tm X_2\xra{}X_2$ we say these
 actions are \emph{equivalent} if there is a bijection
 $\phi\:X_1\xra{}X_2$ such that $g*\phi(x)=\phi(g \bullet x)$ for 
 all $g \in G$ and $x\in X_1$.

 Show that if $G$ acts transitively on a set $X$ then this action is
 equivalent to one of $G$ by left multiplication on $G/H$, for some
 subgroup $H$ of $G$.
\end{itemize}
\end{exercise}
\begin{solution}
\begin{itemize}
 \item[(a)] (1)$\Rightarrow$(2): Suppose $G$ acts transitively and
  that $z\in X$.  Then for any $y\in X$ there exists $g\in G$ such
  that $g*z=y$, by the definition of transitivity.  This means that
  $y\in\{g*z\st g\in G\}=G*z$.  As every $y$ lies in $G*z$, we have
  $G*z=X$, as required.

  (2)$\Rightarrow$(3): if the condition $G*z=X$ holds for every
  element $z$ of the nonempty set $X$, then it certainly holds for
  some element.

  (3)$\Rightarrow$(1): Suppose that $G*z=X$ for some element
  $z\in X$.  Let $x$ and $y$ be points of $X$.  Then $x\in X=G*z$, so
  $x=a*z$ for some $a\in G$.  Similarly $y=b*z$ for some $b\in G$.
  Thus the element $g=ba^{-1}$ satisfies $g*x=(ba^{-1})*a*z=b*z=y$.
  Thus $G$ acts transitively, as claimed.
 \item[(b)] 
  \begin{itemize}
  \item[(1)] Suppose $x,y\in\{1,\ldots,n\}$.  If $x=y$ let
   $\sg\in S_n$ be the identity permutation, otherwise let $\sg$ be
   the transposition $(x\; y)$.  Either way we have $\sg*x=y$, so the
   action is transitive.
%   \item[(2)] Let $U$ and $V$ be $d$-dimensional subspaces of $\R^n$.
%    Choose a basis $\{u_1,\ldots,u_d\}$ for $U$, then choose further
%    vectors $u_{d+1},\ldots,u_n$ so that $\{u_1,\ldots,u_n\}$ is a
%    basis for $\R^n$.  Choose a basis $\{v_1,\ldots,v_d\}$ for $V$,
%    then choose further vectors $v_{d+1},\ldots,v_n$ so that
%    $\{v_1,\ldots,v_n\}$ is a basis for $\R^n$.  Define a linear map
%    $f\:\R^n\xra{}\R^n$ by $f(u_k)=v_k$.  More precisely, we note that
%    every element $x\in\R^n$ can be written uniquely in the form
%    $x=a_1u_1+\ldots+a_nu_n$, and we define
%    $f(x)=a_1v_1+\ldots+a_nv_n$.  We can define $g\:\R^n\xra{}\R^n$ in
%    a similar way with $g(v_k)=u_k$ and we find that $g$ is inverse to
%    $f$, so $f\in GL_n$.  We also have 
%    \begin{align*}
%     f(U) &= f(\text{span}(u_1,\ldots,u_d)) \\
%          &= \text{span}(f(u_1),\ldots,f(u_d)) \\
%          &= \span(v_1,\ldots,v_d) \\
%          &= V,
%    \end{align*}
%    and this shows that $GL_n$ acts transitively.
  \item[(2)] This action is not transitive.  The square $X_4$ contains
   the point $P=(1,0)$, and also the point $Q=(1/2,1/2)$ on the edge
   between $(1,0)$ and $(0,1)$.  There is no element $g\in D_4$ such
   that $g(P)=Q$.
  \item[(3)] This action is not transitive.  To see this, put
   $x=(1\;2\;3)$ and $y=(1\;2\;3)(4\;5\;6)$.  It is easy to see that
   $x^3=y^3=1$, so $x$ and $y$ lie in the set under consideration.  As
   $x$ and $y$ have different cycle types, they are not conjugate.
   More explicitly, for any $g\in S_6$ the permutation $g*x=gxg^{-1}$
   satisfies $(gxg^{-1})(g(4))=g(4)$, but there is no number $i$ with
   $y(i)=i$, so $g*x\neq y$.   
 \end{itemize}
 \item[(c)] Given any two elements $xH,yH\in G/H$, the element
  $g=yx^{-1}$ satisfies $g*(xH)=yH$; this shows that $G$ acts
  transitively. 
 \item[(d)] Suppose that $G$ acts transitively on $X$.  Choose a point
  $a\in X$ and put $H=\{g\in G\st g*a=a\}$.  Define $\phi\:G/H\xra{}X$
  by $\phi(xH)=x*a$.  To see that this is well-defined, note that if
  $xH=yH$ then $x^{-1}y\in H$ so $(x^{-1}y)*a=a$ so
  $x*a=x*(x^{-1}y)*a=y*a$.  Conversely, if $\phi(xH)=\phi(yH)$ then
  $x*a=y*a$ so $a=(x^{-1}y)*a$ so $x^{-1}y\in H$ so $xH=yH$; this
  shows that $\phi$ is injective.  Moreover, for any $b\in X$, there
  is an element $x\in G$ with $x*a=b$, because the action is
  transitive, and so $\phi(xH)=b$.  This shows that $\phi$ is
  surjective and thus bijective.  We also have 
  \[ \phi(g*(xH))=\phi(gxH)=(gx)*a=g*(x*a)=g*\phi(xH), \]
  so $\phi$ gives an equivalence between the two actions.
\end{itemize}
\end{solution}

\begin{exercise}
 Using the formulae in the notes about the geometry of the tent, find
 the coordinates of the vertices $P$, $Q$, $R$, $S$ and $T$ of the
 dodecahedron, as indicated in the diagram below.  
 \begin{center}
  \begin{tikzpicture}[scale=1.5]
   \draw (0.296,-0.989) -- (1.11,-0.745) -- (1.38,-0.0385) -- (0.884,-0.189) --
          (0.296,-0.989) -- (0.296,0.605) -- (0.884,-0.189) -- (1.38,-0.0385) --
          (1.11,0.848) -- (0.433,1.25) -- (0.296,0.605) -- (-1.11,0.745) --
          (-0.433,1.33) -- (0.433,1.25) -- (1.11,0.848);
   \draw (0.296,-0.989) -- (-1.11,-0.848) -- (-1.11,0.745);
   \draw[dashed] (1.11,-0.745) -- (1.11,0.848) -- (-0.296,0.989) --
                 (-0.296,-0.605) -- (1.11,-0.745);
   \draw[dashed] (-1.11,0.745) -- (-0.296,0.989) -- (-0.433,1.33);
   \draw[dashed] (-1.11,-0.848) -- (-0.296,-0.605);
   \fill ( 0.43, 1.25) circle(0.05);
   \fill ( 0.30, 0.61) circle(0.05);
   \fill ( 0.88,-0.19) circle(0.05);
   \fill ( 1.38,-0.04) circle(0.05);
   \fill ( 1.11, 0.85) circle(0.05);
   \draw ( 0.44, 1.35) node[anchor=south west] {$P$};
   \draw ( 0.28, 0.50) node[anchor=north east] {$Q$};
   \draw ( 0.80,-0.19) node[anchor=east] {$R$};
   \draw ( 1.44,-0.04) node[anchor=west] {$S$};
   \draw ( 1.20, 0.94) node[anchor=south west] {$T$};
   \draw[thick] (0.296,0.605) -- (1.11,0.848);   
  \end{tikzpicture}
 \end{center}
 Then find the centre $C$ of the pentagon $PQRST$ and check that
 $d(P,C)^2=d(Q,C)^2=d(R,C)^2=(2+\tau)/5$.

 [{\bf Hint:} You may find it convenient to write all numbers in terms
 of $\tau$.  If a $\sqrt{5}$ turns up you can write it as $2\tau-1$
 (because $\tau=(\sqrt{5}+1)/2$).  If a $\tau^2$ turns up you can
 write it as $\tau+1$ (because $\tau^2-\tau-1=0$).]
\end{exercise}
\begin{solution}
 The cube in the middle has centre at $(0,0,0)$ and the sides have
 length $\tau$ so the coordinates of the vertices are
 $(\pm\tau/2,\pm\tau/2,\pm\tau/2)$.  We use the usual axes so the
 $z$-axis is vertical, the $x$-axis passes through the middle of $RS$
 and the $y$-axis is parallel to $RS$.  With these conventions we have
 $Q=\tau/2.(1,-1,1)$ and $T=\tau/2.(1,1,1)$.  Next, the right hand
 face of the cube has $x=\tau/2$ and we see from the notes that the
 distance from the base of a tent to its ridge is $1/2$ so on the line
 $RS$ we have $x=\tau/2+1/2=(\tau+1)/2$.  The midpoint of $RS$ lies on
 the $x$-axis and thus has coordinates $((\tau+1)/2,0,0)$.  To get
 from this point to $S$ we move half the length of the ridge in the
 positive $y$-direction.  The length of the ridge is $1$ so we end
 up with $S=((\tau+1)/2,0,0)+(0,1/2,0)=(\tau+1,1,0)/2$.  Similarly we
 have $R=(\tau+1,-1,0)/2$.  

 Again, the top face is at height $\tau/2$ so the top ridge is at
 height $\tau/2+1/2=(\tau+1)/2$ so the centre of the top ridge is
 $(0,0,\tau+1)/2$.  To get to $P$ we move a distance of $1/2$ in the
 positive $x$-direction, so $P=(1,0,\tau+1)/2$.  

 We now find that $Q+T=(\tau,0,\tau)$ and $R+S=(\tau+1,0,0)$ so
 $P+Q+R+S+T=(2\tau+3/2,0,\tau+1/2)$, so 
 \[ C = (P+Q+R+S+T)/5 = (3+4\tau,0,1+3\tau)/10. \]
 This implies that
 \begin{align*}
  P-C &= (2-4\tau,0,4+2\tau)/10 \\
  Q-C &= (-3+\tau,-5\tau,-1+2\tau)/10 \\
  R-C &= (2+\tau,-5,-1-3\tau)/10.
 \end{align*}
 It follows that 
 \begin{align*}
  d(P,C)^2 &= \|P-C\|^2 = ((2-4\tau)^2+0^2+(4+2\tau)^2)/10^2 \\
   &= (4-16\tau+16\tau^2+16+16\tau+4\tau^2)/100 \\
   &= (1+\tau^2)/5 \\
   &= (\tau+2)/5,
 \end{align*}
 where we have used the relation $\tau^2=\tau+1$.  Similarly, we have
 \begin{align*}
  d(Q,C)^2 &= ((-3+\tau)^2 + 25\tau^2 +(-1+2\tau)^2)/100 \\
   &= (9-6\tau+\tau^2+25\tau^2+1-4\tau+4\tau^2)/100 \\
   &= (1-\tau+3\tau^2)/10 \\
   &= (2\tau+4)/10 = (\tau+2)/5,
 \end{align*}
 and
 \begin{align*}
  d(R,C)^2 &= ((2+\tau)^2+25+(1+3\tau)^2)/100 \\
   &= (4+4\tau+\tau^2+25+1+6\tau+9\tau^2)/100 \\
   &= (3+\tau+\tau^2)/10 \\
   &= (4+2\tau)/10 = (2+\tau)/5.
 \end{align*}
\end{solution}

\begin{exercise}
 This problem is a more elaborate application of the method we used to
 prove that the number $\tau=2\cos(\pi/5)$ is equal to
 $(1+\sqrt{5})/2$.  

 Define $\zt=e^{2\pi i/15}$ and $\sg=\zt+\zt^{-1}$.  Express the number
 $\rho=\sg^4-\sg^3-4\sg^2+4\sg+1$ in terms of $\zt$.  Then work out
 $(\sg+1)\rho$ and $(\zt^5-1)\zt^5(\sg+1)\rho$.  Show that
 $\zt^5\neq 1$  and that $\sg$ is a positive real number, and deduce
 that $\rho=0$.

 Now recall that the number $\tau=(1+\sqrt{5})/2$ satisfies
 $\tau^2-\tau-1=0$ so $\tau^{-1}=\tau-1$ and $\tau-\tau^{-1}=1$.
 Check that for any $t$ we have 
 $t^4-t^3-4t^2+4t+1=q_0(t)q_1(t)$, where
 \begin{align*}
  q_0(t) &= t^2-\tau t+\tau^{-1}-1 \\
  q_1(t) &= t^2+\tau^{-1}t-\tau-1.
 \end{align*}
 Deduce that either $q_0(\sg)=0$ or $q_1(\sg)=0$.  Given that in fact
 $q_1(\sg)\neq 0$, prove that $\sg=(\tau+\sqrt{9-3\tau})/2$.
\end{exercise}
\begin{solution}
 First, we have
 \begin{align*}
  \sg   &= \zt + \zt^{-1} \\
  \sg^2 &= \zt^2 + 2 + \zt^{-2} \\
  \zt^3 &= \zt^3 + 3\zt + 3\zt^{-1} + \zt^{-3} \\
  \zt^4 &= \zt^4 + 4 \zt^2 + 6 + 4 \zt^{-2} + \zt^{-4},
 \end{align*}
 so
 \begin{align*}
  \rho &= \sg^4-\sg^3-4\sg^2+4\sg+1 \\
       &= (\zt^4 + 4 \zt^2 + 6 + 4 \zt^{-2} + \zt^{-4})
          - (\zt^3 + 3\zt + 3\zt^{-1} + \zt^{-3}) \\
       &\qquad - (4\zt^2 + 8 + 4\zt^{-2}) 
               + (4\zt + 4\zt^{-1}) + 1 \\
       &= \zt^4 - \zt^3 + \zt - 1 + \zt^{-1} - \zt^{-3} + \zt^{-4}.  
 \end{align*}
 It follows that
 \begin{align*}
  (\sg+1)\rho &= \zt\rho + \rho + \zt^{-1}\rho \\
   &= \zt^5 - \zt^4 + \zt^2 - \zt + 1 - \zt^{-2} + \zt^{-3} + \\
   &\qquad
      \zt^4 - \zt^3 + \zt - 1 + \zt^{-1} - \zt^{-3} + \zt^{-4} + \\
   &\qquad
      \zt^3 - \zt^2 + 1 - \zt^{-1} + 1 - \zt^{-4} + \zt^{-5} \\
   &= \zt^5 + 1 + \zt^{-5},
 \end{align*}
 and thus that
 \begin{align*}
  (\zt^5-1)\zt^5(\sg+1)\rho &= (\zt^5-1)(\zt^{10}+\zt^5+1) \\
    &= \zt^{15} + \zt^{10} + \zt^5 - \zt^{10} - \zt^5 - 1 \\
    &= \zt^{15} - 1 = e^{2\pi i} - 1 = 0.
 \end{align*}
 We have $\zt=\cos(2\pi/15)+i\sin(2\pi/15)$ so $\sg=2\cos(2\pi/15)$.
 It is well-known that $\cos(\tht)>0$ for $|\tht|<\pi/2$ and
 $2\pi/15<\pi/2$ so $\sg>0$.  This implies that $\sg+1\neq 0$.  We
 also have $\zt^5=e^{2\pi i/3}=(-1+\sqrt{3}i)/2\neq 1$ so
 $\zt^5-1\neq 0$ and clearly also $\zt^5\neq 0$.  As
 $(\zt^5-1)\zt^5(\sg+1)\rho=0$ and the numbers $\zt^5-1$, $\zt^5$ and
 $\sg+1$ are all nonzero we must have $\rho=0$.  


 Next, we have
 \begin{align*}
  q_0(t)q_1(t) &= (t^2-\tau t+\tau^{-1}-1)(t^2+\tau^{-1}t-\tau-1) \\
   &= t^4 + (\tau^{-1}-\tau)t^3 +
      (-\tau-1 -\tau\tau^{-1} + \tau^{-1}-1) t^2 +
      (\tau(\tau+1)+(\tau^{-1}-1)\tau^{-1}) t -
      (\tau^{-1}-1)(\tau+1) \\
   &= t^4 - (\tau-\tau^{-1}) t^3 -
      (\tau - \tau^{-1} + 3) t^2 + 
      (\tau^2+\tau^{-2}+\tau-\tau^{-1}) t +
      (\tau-\tau^{-1}). 
 \end{align*}
 We can simplify most of the terms using the fact that
 $\tau-\tau^{-1}=1$.  To simplify the coefficient of $t$ we square
 this equation to get $\tau^2-2+\tau^{-2}=1$ and thus
 $\tau^2+\tau^{-2}=3$.  Putting this all in we get
 \begin{align*}
  q_0(t)q_1(t) &= t^4 - t^3 - (1+3) t^2 + (3+1) t + 1 \\
   &= t^4 - t^3 - 4 t^2 + 4t + 1,
 \end{align*}
 as claimed.  This implies that
 $0=\rho=\sg^4-\sg^3-4\sg^2+4\sg+1=q_0(\sg)q_1(\sg)$, so either
 $q_0(\sg)$ or $q_1(\sg)$ is zero.  We are given that $q_1(\sg)\neq 0$
 so $q_0(\sg)=0$.  The roots of the equation $q_0(t)=0$ are 
 $t=(\tau\pm\sqrt{\tau^2-4\tau^{-1}+4})/2$.  We have $\tau^2=\tau+1$
 and $\tau^{-1}=\tau-1$ so
 $\tau^2-4\tau^{-1}+4=\tau+1-4\tau+4+4=9-3\tau$.  Thus the roots are
 $t=(\tau\pm\sqrt{9-3\tau})/2$.  We have
 $\tau=(1+\sqrt{5})/2\simeq 1.618$ and so
 $\sqrt{9-3\tau}\simeq 2.036$, so $(\tau-\sqrt{9-3\tau})/2<0$.  As
 $\sg$ is a positive root of $q_0$ we must have 
 $\sg=(\tau+\sqrt{9-3\tau})/2$.
\end{solution}

\begin{exercise}
 Show that the centre of $S_n$ is  trivial, for $n\geq 3$.
\end{exercise}
\begin{solution}

\end{solution}

\begin{exercise}
 The group $\Dir(\Cube)$ of rotational symmetries of the cube acts on the
 surface of the cube.  Find the sizes of the orbits of points on the surface
 and describe geometrically which points have which orbit sizes. 
\end{exercise}
\begin{solution}
 Let $S$ be the surface of the cube.  For any point $x\in S$ we have
 $|Gx|=|G|/|\text{stab}_G(x)|=24/|\text{stab}_G(x)|$.  Moreover,
 $\text{stab}_G(x)$ is the group of all rotations around $x$ that
 preserve the cube.  For most points $x$ there are no such rotations
 (except for the identity) and so $|\text{stab}_G(x)|=1$ and the orbit
 $Gx$ has order $24$.  If $x$ is the centre of a face then
 $\text{stab}_G(x)$ is cyclic of order $4$ and so $|Gx|=24/4=6$.  In
 fact, $Gx$ consists of the centres of the $6$ faces.  If $x$ is a
 vertex of the cube then $\text{stab}_G(x)$ is cyclic of order $3$ and
 so $|Gx|=24/3=8$.  In fact, in this case $Gx$ consists of the $8$
 vertices of the cube.  If $x$ is the midpoint of an edge then
 $\text{stab}_G(x)$ has order $2$ and so $|Gx|=24/2=12$.  In fact, in
 this case $Gx$ consists of the midpoints of the $12$ edges of the
 cube.  In all other cases we have $|Gx|=24$.
\end{solution}

\begin{exercise}
 Show that if $S\in\Symm(\Cube)$ is a reflection then $S =-R$ for some
 rotation $R\in\Dir(\Cube)$ of order 2.  Conversely, show that if
 $R\in\Dir(\Cube)$ is a rotation of order 2 then $-R\in\Symm(\Cube)$ is
 a reflection.  Hence find how many reflections there in
 $\Symm(\Cube)$.
\end{exercise}
\begin{solution}

\end{solution}

\begin{exercise}
 \begin{itemize}
 \item[(a)] Show that if $G$ is a group of order $p^n$ for some $n$
  then the centre of $G$ is non-trivial.
 \item[(b)] Show that any group of order $p^2$ is abelian. [Hint:
  consider $G/Z(G)$]
 \item[(c)] Conclude that any group of order $p^2$ is either cyclic or
  isomorphic to $C_p \times C_p$.
 \end{itemize}
\end{exercise}
\begin{solution}

\end{solution}

\begin{exercise}
 Show that if $G$ is a semidirect product of $H$ and $K$ with $H$ and
 $K$ {\em both} normal then in fact $G\simeq H \tm K$.
\end{exercise}
\begin{solution}

\end{solution}

\begin{exercise}
 The following diagram shows a cuboctahedron $X$ in $\R^3$, centred at
 the origin.  Its faces are squares and equilateral triangles.
 \begin{center}
  \begin{tikzpicture}
   \draw (1.500, 5.825) -- (2.250, 6.125) -- (3.750, 5.825) -- (3.000, 5.562) -- (1.500, 5.825);
   \draw (3.000, 5.562) -- (1.875, 4.437) -- (3.000, 3.312) -- (4.125, 4.437) -- (3.000, 5.562);
   \draw (1.125, 5.000) -- (1.500, 5.825) -- (1.875, 4.437) -- (1.500, 3.575) -- (1.125, 5.000);
   \draw (3.750, 5.825) -- (4.125, 4.437);
   \draw (1.500, 3.575) -- (3.000, 3.312);
   \draw (3.000, 3.312) -- (3.750, 3.575) -- (4.125, 4.437);
   \draw[dotted]
         (2.250, 3.875) -- (1.125, 5.000) -- (2.250, 6.125) -- (3.375, 5.000) -- (2.250, 3.875);
   \draw[dotted]
         (1.500, 3.575) -- (2.250, 3.875) -- (3.750, 3.575) -- (3.375, 5.000) -- (3.750, 5.825);   
  \end{tikzpicture}
 \end{center}
 Which of the standard finite subgroups of $SO_3$ is isomorphic to
 $\Dir(X)$?  What can you deduce about $\Symm(X)$?

 You may wish to make a model from the net below.
 \begin{center}
  \begin{tikzpicture}[rotate=90,yscale=-1]
   \def\rt{1.732}
   \draw 
   (1,1) -- (1,-1) -- (1+\rt,-2) -- (\rt,-2-\rt) -- (0,-1-\rt) --
   (1,-1) -- (-1,-1) -- (0,-1-\rt) -- (-\rt,-2-\rt) -- (-1-\rt,-2) --
   (-1,-1) -- (-1,1) -- (-1-\rt,2) -- (-\rt,2+\rt) -- (0,1+\rt) --
   (-1,1) -- (1,1) -- (0,1+\rt) -- (\rt,2+\rt) -- (0,3+\rt) --
   (\rt,4+\rt) -- (1+\rt,4+2*\rt) -- (2+\rt,4+\rt) -- (2+2*\rt,3+\rt) --
   (2+\rt,2+\rt) -- (1+\rt,2) -- (\rt,2+\rt) -- (\rt,4+\rt) --
   (2+\rt,4+\rt) -- (2+\rt,2+\rt) -- (\rt,2+\rt) -- (1+\rt,2) -- (1,1);
  \end{tikzpicture}
 \end{center}
\end{exercise}
\begin{solution}
 Put $G=\Dir(X)$, which is a finite subgroup of $SO_3$.  The
 normalisations of the midpoints of the edges are poles of degree $2$,
 the normalisations of the centres of the triangular faces are poles
 of degree $3$, and the normalisations of the centres of the square
 faces are poles of degree $4$.  The only one of the standard groups
 that has poles of degrees $2$, $3$ and $4$ is $G_2$, so $G$ must be
 conjugate to $G_2$ and thus isomorphic to $S_4$.  

 One can see from the picture that multiplication by $-1$ preserves
 $X$ and so Proposition~6.9 in the notes tells us that
 $\Symm(X)\simeq\{\pm 1\}\tm\Dir(X)\simeq\{\pm 1\}\tm S_4$.
\end{solution}

\begin{exercise}
 Let $g\:\R^3\xra{}\R^3$ be the map $g(x,y,z)=(y,z,x)$.
 \begin{itemize}
 \item[(a)] Prove that $g\in O_3$.
 \item[(b)] Find a unit vector $u$ with $g(u)=u$.
 \item[(c)] Find the order of $g$ and deduce that $g\in SO_3$.
 \item[(d)] Show that $g$ preserves the standard cube (with centre at the
  origin and edges of length $2$ parallel to the $x$, $y$ and $z$
  axes). 
 \item[(e)] Describe the effect of $g$ geometrically.
 \end{itemize}
\end{exercise}
\begin{solution}
 \begin{itemize}
 \item[(a)] It is clear that $g$ is linear, and we have 
  \[ \|g(x,y,z)\|^2=y^2+z^2+x^2=x^2+y^2+z^2=\|(x,y,z)\|^2, \]
  so $g$ preserves lengths.  Thus $g\in O_3$.
 \item[(b)] Visibly $g(a,a,a)=(a,a,a)$ for any $a$.  To get a unit
  vector, put $a=1/\sqrt{3}$.
 \item[(c)] We have $g^2(x,y,z)=g(y,z,x)=(z,x,y)$ and
  $g^3(x,y,z)=g(z,x,y)=(x,y,z)$, so $g^3=1$ and $g$ has order $3$.
  Because $g\in O_3$ we have $\det(g)=\pm 1$ and
  $\det(g)^3=\det(g^3)=\det(1)=1$ which would give a contradiction if
  $\det(g)$ were $-1$, so $\det(g)=1$.  Thus $g\in SO_3$.  
 \item[(d)] The vertices of the standard cube are the points $(x,y,z)$ for
  which $x,y,z\in\{1,-1\}$.  Clearly, if $(x,y,z)$ satisfies this
  condition then so does $(y,z,x)$, so $g$ carries vertices of the
  cube to vertices of the cube, so it sends the cube to itself.
 \item[(e)] As $g\in SO_3$, it must be a rotation.  Part~(b) shows
  that the axis of rotation is the line $x=y=z$.  Part~(c) implies
  that the angle of rotation is $2\pi/3$.  As $g(0,0,1)=(0,1,0)$ we
  see that $g$ carries the $z$-axis to the $y$-axis and thus that the
  direction of rotation (as seen while looking from $(1,1,1)$ towards
  the origin) is clockwise.
 \end{itemize}
\end{solution}

\begin{exercise}
 Let $M_1$, $M_2$ and $M_3$ be the $x$, $y$ and $z$-axes.  
 \begin{itemize}
 \item[(a)] Use these to define a homomorphism
  $\psi\:\Dir(\Cube)\xra{}S_3$.
 \item[(b)] Describe some elements $g\in\Dir(\Cube)$ and the
  corresponding permutations $\psi(g)$.
 \item[(c)] Show that $\psi$ is surjective.
 \item[(d)] Show that the kernel of $\psi$ is isomorphic to
  $C_2\tm C_2$. 
 \end{itemize}
\end{exercise}
\begin{solution}
 \begin{itemize}
 \item[(a)] For any $g\in\Dir(\Cube)$ we let $\psi(g)$ be the
  permutation such that $g(M_i)=M_{\sg(i)}$ for $i=1,2,3$.
 \item[(b)] Let $g$ be a half turn around the vector $(0,1,1)$, let
  $h$ be a one-third turn anticlockwise about the vector $(1,1,1)$,
  and let $k$ be a quarter turn anticlockwise about the vector
  $(0,0,1)$.  Then $\psi(g)=(2\; 3)$, $\psi(h)=(1\;2\;3)$ and
  $\psi(k)=(1\; 2)$.
 \item[(c)] Part~(b) shows that the image of $\psi$ contains $(1\;2)$
  and $(2\; 3)$, and these two transpositions generate $S_3$ so $\psi$
  is surjective.  More explicitly, we have
  \[ \begin{array}{lcr}
   \psi(1)      = 1         & \hspace{5em} & \psi(k)  = (1\;2) \\
   \psi(h)      = (1\;2\;3) &              & \psi(g)  = (2\;3) \\
   \psi(h^{-1}) = (1\;3\;2) &              & \psi(gh) = (1\;3).
  \end{array} \]
 \item[(d)] Let $G$ be the group of matrices of the form 
  \[ g = \bsm \ep_1 & 0 & 0 \\ 0 & \ep_2 & 0 \\ 0 & 0 & \ep_3 \esm \]
  such that $\ep_1\ep_2\ep_3=1$.  It is easy to see that $G$ is a
  subgroup of $SO_3$ and that it preserves the cube so it is a
  subgroup of $\Dir(\Cube)$.  If $g\in G$ and $x\in\R$ then
  $g(x,0,0)=(\pm x,0,0)$, so $g$ preserves the $x$-axis, or in other
  words $g(M_1)=M_1$.  Similarly, we have $g(M_2)=M_2$ and
  $g(M_3)=M_3$, so $\psi(g)=1$, so $G\leq\ker(\psi)$.
 
  Conversely, if $g\in\ker(\psi)$ then $g(M_1)=M_1$.  The axis $M_1$
  meets the surface of the cube at $(1,0,0)$ and $(-1,0,0)$, so
  $g(1,0,0)=\ep_1(1,0,0)$ for some $\ep_1\in\{1,-1\}$.  Similarly we
  have $g(0,1,0)=\ep_2(0,1,0)$ and $g(0,0,1)=\ep_3(0,0,1)$ for some
  $\ep_2,\ep_3\in\{1,-1\}$.  Thus, the matrix of $g$ has the form
  described above, and as $g\in\Dir(\Cube)\leq SO_3$ we have
  $\det(g)=1$ so $\ep_1\ep_2\ep_3=1$ so $g\in G$.  This shows that
  $\ker(\psi)=G$.  

  We can define an isomorphism
  $\chi\:C_2\tm C_2=\{\pm 1\}\tm\{\pm 1\}\xra{}G$ by
  $\chi(\ep_1,\ep_2)=(\ep_1,\ep_2,\ep_1\ep_2)$. 
 \end{itemize}
\end{solution}

\begin{exercise}
 Consider the action of $G$ on $\CC$ as in Question 1.
 Observe that the stabilizer of $H$ is
 \[ N_G(H)=\{ g \in G\; |\; gHg^{-1}=H\}. \]
 This is called the {\em normalizer} of $H$ in $G$.

 \begin{itemize}
 \item[(a)] 
  \begin{itemize}
  \item[(i)] Show that $H$ is a normal subgroup of $N_G(H)$.
  \item[(ii)] Show that if $H$ is normal in $K$ then $K\sse N_G(H)$.
  \end{itemize}
 \item[(b)] Consider the action of $H$ on the set $G/H$ of left cosets
  of $H$ by left translation: $h*(gH)=hgH$.
  \begin{itemize}
  \item[(i)] Show that $\{kH\} $ is a complete orbit if and only if
   $k^{-1}Hk=H$.
  \item[(ii)] Let $w$ be the number of singleton orbits.  Show that if
   $H$ is a $p$-group then
   \[ [G:H] \equiv w \pmod{p} . \]
  \item[(iii)] Conclude that if $G$ is a $p$-group and $H \neq G$ then
   $H\neq N_G(H)$.
  \item[(iv)] Deduce from (iii) that if $G$ is a $p$-group and $H$ is a
   subgroup of index $p$ then $H$ is normal in $G$.
  \end{itemize}
 \end{itemize}
\end{exercise}
\begin{solution}

\end{solution}

\begin{exercise}
 Let $G$ be a group of order $77$, and let $X$ be a set of order $96$
 on which $G$ acts.  Suppose there are precisely four orbits.  By
 investigating the possible sizes of the orbits, show that there is
 exactly one point $x\in X$ such that $gx=x$ for all $g\in G$.
\end{exercise}
\begin{solution}
 Let the sizes of the $4$ orbits be $d_1$, $d_2$, $d_3$ and $d_4$ with
 $d_1\leq d_2\leq d_3\leq d_4$.  These sizes must divide $77$, and
 thus must be $1$, $7$, $11$ or $77$.  We must also have
 $d_1+d_2+d_3+d_4=96$.  If $d_4\neq 77$ then $d_i\leq 11$ for all $i$
 so $96=d_1+d_2+d_3+d_4\leq 4\tm 11=44$, which is false.  Thus
 $d_4=77$ and $d_1+d_2+d_3=96-77=19$.  By similar arguments or by
 inspection we must have $d_1=1$, $d_2=7$ and $d_3=11$.  Thus, there
 is precisely one orbit of size $1$.  If this orbit has the form
 $\{x\}$ then $x$ is fixed under the action of $G$.  If $y$ is in any
 of the other orbits then $|Gy|>1$ so $y$ is not fixed.  Thus, there
 is precisely one fixed point.
\end{solution}

\begin{exercise}
 Let $G$ be a finite group, and let $X$ be a set with an action of
 $G$.  Suppose that there is precisely one orbit, and that $|X|>1$.
 Use the orbit counting theorem to show that there is an element $g\in
 G$ such that $\Fix(g)=\emptyset$.
\end{exercise}
\begin{solution}
 The identity element fixes all of $X$, so it has more than one fixed
 point.  The orbit counting theorem says that the average number of
 fixed points is the number of orbits, which is $1$.  As the identity
 has more than one fixed point, some other element $g\in G$ must have
 less than one, so as to bring the average back down to $1$.  Thus
 $|\Fix(g)|<1$ but of course $|\Fix(g)|$ is a nonnegative integer so
 $|\Fix(g)|=0$ so $\Fix(g)=\emptyset$.  

 For a more algebraic presentation, note that $\sum_{g\in G}1=|G|$, so
 $|G|^{-1}\sum_{g\in G}1=1$.  The orbit counting theorem tells us that
 $|G|^{-1}\sum_{g\in G}|\Fix(g)|=\text{ number of orbits }=1$.  By
 subtracting these equations we find that
 $\sum_{g\in G}(|\Fix(g)|-1)=0$.  If we move the $g=1$ term to the
 other side we get $\sum_{g\neq 1}(|\Fix(g)|-1)=1-|X|$.  The right
 hand side is less than $0$, so at least one of the terms on the left
 must be less than $0$, so $|\Fix(g)|-1<0$ for some $g$.  As before,
 this means that $\Fix(g)=\emptyset$.
\end{solution}

\begin{exercise}
 Let $G$ be a finite simple group (so there are no normal subgroups of
 $G$ except for $\{1\}$ and $G$ itself).  Let $p$ be a prime dividing
 the order of $G$, and let $X$ be a set of order $n$ on which the
 group acts.  Suppose that the action is nontrivial, so there is and
 element $g\in G$ and an element $x\in X$ such that $gx\neq x$.  Prove
 that $n\geq p$.

 [{\bf Hint:} Use $X$ to define a homomorphism and consider its
 kernel.  For which integers $m$ does $p$ divide $m!$?]
\end{exercise}
\begin{solution}
 We can use $X$ to define a homomorphism $G\xra{}S_n$ in the usual
 way.  (In other words, we list the elements of $X$ as
 $\{x_1,\ldots,x_n\}$ say, and then let $\phi(g)$ be the permutation
 $\sg$ such that $gx_i=x_{\sg(i)}$ for all $i$.)  The kernel of any
 homomorphism is a normal subgroup, and there are only two normal
 subgroups of $G$ so either $\ker(\phi)=\{1\}$ or $\ker(\phi)=G$.  As
 the action is nontrivial, we have $gx_i\neq x_i$ for some $g\in G$
 and $i\in\{1,\ldots,n\}$, so $\phi(g)(i)\neq i$, so $\phi(g)\neq 1$.
 Thus $g\not\in\ker(\phi)$, so $\ker(\phi)\neq G$, so we must have
 $\ker(\phi)=\{1\}$.  This means that $\phi$ is injective and thus
 that $|G|=|\phi(G)|$.  Moreover, $\phi(G)$ is a subgroup of $S_n$, 
 so $n!=|S_n|$ is divisible by $|\phi(G)|=|G|$, and $|G|$ is divisible
 by $p$, so $n!$ is divisible by $p$.  Now, if $m<p$ then none of the
 numbers $1,2,\ldots,m$ are divisible by $p$ so $m!$ is not divisible
 by $p$.  As $p$ divides $n!$ we must have $n\geq p$ as claimed.
\end{solution}

\begin{exercise}
 We know that $\Dir(\Tet)\simeq A_4$ and that $\Dir(\Cube)\simeq S_4$.
 Find a geometric reason that $\Dir(\Tet)$ is isomorphic to a subgroup
 of $\Dir(\Cube)$.  [Hint: can you see a tetrahedron inside the cube?]
\end{exercise}
\begin{solution}
 The following picture shows a tetrahedron embedded inside a cube.
 We'll assume as usual that the edges of the cube have length $2$.
 \begin{center}
  \begin{tikzpicture}[scale=2]
   \draw (0.598,-1.21) -- (1.28,-0.881);
   \draw (0.598,-1.21) -- (-1.28,-1.09);
   \draw (0.598,-1.21) -- (0.598,0.762);
   \draw[dashed] (1.28,-0.881) -- (-0.598,-0.762);
   \draw (1.28,-0.881) -- (1.28,1.09);
   \draw[dashed] (-1.28,-1.09) -- (-0.598,-0.762);
   \draw (-1.28,-1.09) -- (-1.28,0.881);
   \draw[dashed] (-0.598,-0.762) -- (-0.598,1.21);
   \draw (0.598,0.762) -- (1.28,1.09);
   \draw (0.598,0.762) -- (-1.28,0.881);
   \draw (1.28,1.09) -- (-0.598,1.21);
   \draw (-1.28,0.881) -- (-0.598,1.21);
   \begin{scope}[red,thick]
   \draw (1.28,1.09) -- (-1.28,0.881) -- (0.598,-1.21) -- (1.28,1.09);
    \draw[dashed] (-0.598,-0.762) -- (1.28,1.09);
    \draw[dashed] (-0.598,-0.762) -- (-1.28,0.881);
    \draw[dashed] (-0.598,-0.762) -- (0.598,-1.21);
   \end{scope}
  \end{tikzpicture}
 \end{center}
 Alternatively, we can say that the cube is built from the tetrahedron
 by attaching a pyramid to each face.  All the pyramids have the same
 shape: the base edges have length $2\sqrt{2}$, and the remaining
 edges have length $2$.  Thus, any isometry of the tetrahedron carries
 pyramids to pyramids and thus gives an isometry of the cube.  This
 shows that $\Dir(\Tet)\leq\Dir(\Cube)$.  
\end{solution}

\begin{exercise}
 In this problem we will find the coordinates of the vertices of a
 tetrahedron.  We will place the tetrahedron with its centre at the
 origin and the first vertex $v_1$ on the $z$-axis.  After choosing
 suitable units of length we may assume that $v_1=(0,0,\sqrt{3})$.  We
 then rotate our coordinates if necessary so that the next vertex
 $v_2$ lies in the $xz$-plane, say $v_2=(a,0,-b)$.  There are two more
 vertices; we call the one with positive $y$-coordinate $v_3$, and the
 one with negative $y$-coordinate $v_4$.
 \begin{itemize}
 \item[(a)] Explain why the points $v_3$ and $v_4$ lie in the plane
  $z=-b$.
 \item[(b)] Explain why $v_3=(-a/2,a\sqrt{3}/2,-b)$, and give the
  corresponding formula for $v_4$.
 \item[(c)] By considering the distances $d(O,v_i)$ and $d(v_i,v_j)$
  show that $a^2+b^2=3$ and $a^2+(b+\sqrt{3})^2=3a^2$.
 \item[(d)] Solve these equations for $a$ and $b$, and obtain explicit
  expressions for the coordinates of $v_2$, $v_3$ and $v_4$.
 \item[(e)] Consider the matrix
  \[ g = \bsm 1/3        &  0 & \sqrt{8}/3 \\
              0          & -1 & 0          \\
              \sqrt{8}/3 & 0  & -1/3 \esm
  \]
  Calculate $g(v_i)$ for $i=1,2,3,4$ and thus determine the vertex
  permutation induced by $g$.
 \end{itemize}
\end{exercise}
\begin{solution}
 \begin{itemize}
 \item[(a)] A twist of one third around the $z$-axis sends $v_2$ to
  $v_3$, $v_3$ to $v_4$ and $v_4$ to $v_2$.  Such a twist preserves
  horizontal planes, and $v_2$ lies in the plane $z=-b$ so the same is
  true of $v_3$ and $v_4$.
 \item[(b)] Let $w_2$, $w_3$ and $w_4$ be the points in the $xy$ plane
  lying above $v_2$, $v_3$ and $v_4$.  Then $w_2=(a,0)$ and $w_3$ and
  $w_4$ are obtained from $w_2$ by rotating around the origin through
  $2\pi/3$ in either direction, so the coordinates are
  $(\cos(\pm 2\pi/3)a,\sin(\pm 2\pi/3)a)$, which is equal to
  $(-a/2,\pm a\sqrt{3}/2)$.  In combination with~(a) this means that
  $v_3=(-a/2,a\sqrt{3}/2,-b)$ and $v_3=(-a/2,-a\sqrt{3}/2,-b)$.
 \item[(c)] All the vertices of a tetrahedron have the same distance
  from the centre, so $\|v_2\|^2=\|v_1\|^2$, or in other words
  $a^2+b^2=3$.  The distance between any two vertices is the same, so
  $d(v_1,v_2)=d(v_3,v_4)$.  We have $v_1-v_2=(-a,0,b+\sqrt{3})$ so
  $d(v_1,v_2)^2=a^2+(b+\sqrt{3})^2$.  We also have
  $v_2-v_3=(0,a\sqrt{3},0)$, so $d(v_2,v_3)^2=3a^2$.  It follows that
  $a^2+(b+\sqrt{3})^2=3a^2$ as claimed.
 \item[(d)] Our second equation expands out to give
  $b^2+2b\sqrt{3}+3-2a^2=0$.  Our first equation gives $a^2=3-b^2$ and
  after substituting this in we get $3b^2+2b\sqrt{3}-3=0$, so
  $b^2+2b/\sqrt{3}-1=0$, so $(b+1/\sqrt{3})^2=4/3$.  As $b$ must
  clearly be positive this gives $b=1/\sqrt{3}$.  This implies that
  $a^2=3-b^2=3-1/3=8/3$ so $a=\sqrt{8/3}=2\sqrt{2/3}$.  Putting this
  back into our equations for the $v_i$ gives
  \begin{align*}
   v_1 &= (0,0,\sqrt{3}) \\
   v_2 &= (2\sqrt{2/3},0,-1/\sqrt{3}) \\
   v_3 &= (-\sqrt{2/3},\sqrt{2},-1/\sqrt{3}) \\
   v_4 &= (-\sqrt{2/3},-\sqrt{2},-1/\sqrt{3}).
  \end{align*}
 \item[(e)] Let $V$ be the matrix whose columns are $v_1$, $v_2$,
  $v_3$ and $v_4$, so $gV$ is the matrix whose columns are $gv_1$,
  $gv_2$, $gv_3$ and $gv_4$.  We have
  \begin{align*}
   gV &= \bsm 1/3&0&2\sqrt{2}/3\\0&-1&0\\2\sqrt{2}/3&0&-1/3 \esm 
         \bsm 0 & 2\sqrt{2/3} & -\sqrt{2/3} & -\sqrt{2/3} \\
              0 & 0 & \sqrt{2} & -\sqrt{2} \\
              \sqrt{3} & -1/\sqrt{3} & -1/\sqrt{3} & -1/\sqrt{3}
         \esm \\
      &= \bsm 2\sqrt{2/3} & 0 & -\sqrt{2/3} & -\sqrt{2/3} \\
              0 & 0 & -\sqrt{2} & \sqrt{2} \\
              -1/\sqrt{3} & \sqrt{3} & -1/\sqrt{3} & -1/\sqrt{3}
         \esm, 
  \end{align*}
  so $g(v_1)=v_2$, $g(v_2)=v_1$, $g(v_3)=v_4$ and $g(v_4)=v_3$.  Thus,
  the permutation associated to $g$ is $(1\;2)(3\;4)$.
 \end{itemize}
\end{solution}

\begin{exercise}
Let $(e_1,e_2)$ be the usual basis of $\R^2$, let $L=\Re_1$ be the
$x$-axis and $S=S_e$.
% what's e?

Show that if $G$ is the subgroup $\ip{S T_1,T_2}$ of $I_2$ then the
point group $\psi(G)$ is cyclic of order $2$, whereas $\sg_a(G)$ is
trivial for all $a\in\R^2$.

\noindent
[Hint: For the second part show that every $g\in G$ can be written in the
form: $S^{i_1}T_1^{i_1}T_2^{i_2}.$]
\end{exercise}
\begin{solution}

\end{solution}

\begin{exercise}
 Let $H$ be a wallpaper group such that
 $\Trans(H)=\{(2n,m)\st n,m\in\Z\}$.  Prove that $|\psi(H)|\leq 4$.
\end{exercise}
\begin{solution}
 The group $\Trans(H)\leq\R^2$ looks like this:
 \begin{center}
  \begin{tikzpicture}[scale=1]
   \foreach \i in {-4,-2,...,4} {
    \foreach \j in {-2,-1,...,2} {
     \fill (\i,\j) circle(0.06);
    }
   }
   \draw (-2, 0) node[anchor=north] {$(-2, 0)$};
   \draw ( 0, 0) node[anchor=north] {$( 0, 0)$};
   \draw ( 2, 0) node[anchor=north] {$( 2, 0)$};
   \draw ( 0,-1) node[anchor=north] {$( 0,-1)$};
   \draw ( 0, 1) node[anchor=north] {$( 0, 1)$};
  \end{tikzpicture}
 \end{center}
 We know that if $A\in\psi(H)$ then $A.\Trans(H)=\Trans(H)$ (Lemma 5.8
 in the notes).  If $A$ is a rotation about the origin that sends
 $\Trans(H)$ to itself, then the angle must be $0$ or $\pi$, so $A=I$
 or $R_{\pi}$.  If $A$ is the reflection across a line $L$ through the
 origin, then $L$ must be either the $x$-axis or the $y$-axis, so
 $A=S_0$ or $S_{\pi}$.  Thus $\psi(H)\sse\{1,S_0,S_\pi,S_0S_\pi\}$ and
 so $|\psi(H)|\leq 4$.
 
 Here is a slightly more formal argument.  We know that
 $A.\Trans(H)=\Trans(H)$, so $A(0,1)$ lies in $\Trans(H)$ and has
 length $1$, so $A(0,1)=(0,1)$ or $A(0,1)=(0,-1)$.  In the first case
 we define $A_1=A$, and in the second we define $A_1=S_0A$; either way
 we have $A_1.\Trans(H)=\Trans(H)$ and $A_1(0,1)=(0,1)$.  As $A_1$
 preserves lengths and angles we see that $A_1(2,0)$ is perpendicular
 to $A_1(0,1)=(0,1)$ and $\|A_1(2,0)\|=2$; the only possibilities are
 $A_1(2,0)=(2,0)$ or $A_1(2,0)=(-2,0)$.  In the first case we put
 $A_2=A_1$,and in the second we define $A_2=S_\pi A_1$; either way we
 have $A_2(0,1)=(0,1)$ and $A_2(2,0)=(2,0)$.  As $(0,1)$ and $(2,0)$
 are a basis of $\R^2$, this means that $A_2=1$, and it follows that
 $A$ is either $1$, $S_0$, $S_\pi$ or $S_0S_\pi=R_{\pi}$.
\end{solution}

\begin{exercise}
 Let $\Oct$ be a regular octahedron, centred at the origin in $\R^3$.
 Since $\Oct$ is dual to the cube we know that $\Dir(\Oct)\simeq S_4$.
 Prove this directly as follows:
 \begin{itemize}
  \item[(i)] Describe 24 rotation in $\Dir(\Oct)$.
  \item[(ii)] Describe a set of 4 objects in which $\Dir(\Oct)$ acts
   and such that the induced homorphism $\Dir(\Oct)\xra{}S_4$ is
   injective.  Prove your assertions.
 \end{itemize}
\end{exercise}
\begin{solution}

\end{solution}

\begin{exercise}
 By considering cycle types show that $A_5$ has no elements of order
 $15$.  What does the classification of finite subgroups of $SO_3$
 tell us about subgroups of $A_5$ of order $30$?  Show that there are
 no such subgroups. 
\end{exercise}
\begin{solution}
 $A_5$ consists of elements of the following types:
 \begin{itemize}
 \item the identity, with order $1$
 \item transposition pairs (such as $(1\;2)(3\;4)$), with order $2$
 \item $3$-cycles (such as $(1\;2\;3)$), with order $3$
 \item $5$-cycles (such as $(1\;2\;3\;4\;5)$) with order $5$.
 \end{itemize}
 There are thus no elements of order $15$.  

 Now let $G$ be a subgroup of $A_5$ with $|G|=30$.  The group $A_5$ is
 isomorphic to the subgroup $G_3$ of $SO_3$, so $G$ is isomorphic to
 some subgroup of $G_3$ and thus to a finite subgroup of $SO_3$.  It
 follows by the classification that $G$ is isomorphic to $G_1$, $G_2$
 or $G_3$, or to $\widetilde{C}_n$ or $\widetilde{D}_n$ for some $n$.
 As $|G|=30$ which is different from the orders of $G_1$, $G_2$ and
 $G_3$ we must have $G\simeq\widetilde{C}_{30}$ or
 $G\simeq\widetilde{D}_{15}$.  However $\widetilde{C}_{30}$ and
 $\widetilde{D}_{15}$ both contain elements of order $15$ and $G$ does
 not, which gives a contradiction.  Thus there can be no subgroups of
 $A_5$ of order $30$.
\end{solution}

\begin{exercise}
 Suppose $H$ is a subgroup of a finite group $G$, and consider the
 action of $G$ on the set
 \[ \CC =\{ g H g^{-1}\; | \; g \in G\} \]
 of conjugates of $H$.
 \begin{itemize}
  \item[(i)] Show that $h*K =hKh^{-1}$ defines an action of $G$ on $\CC$.
  \item[(ii)] Conclude that $|\CC|$ divides the order of $G$.
  \item[(iii)] Suppose $G$ is of order $p^ns$ with $(p,s)=1$ and $n_p$
   is the number of conjugates of $P$.  Use Part (ii) to show that if
   $n_p\equiv 1$ mod $p$ then $n_p$ divides $s$.
 \end{itemize}
\end{exercise}
\begin{solution}

\end{solution}

\begin{exercise}
 Let $G$ be a group of order $605$.  Show that $G$ has a normal
 subgroup of order $121$ and hence show that $G$ is a semidirect
 product of proper subgroups.

 Find a group automorphism
 $\tht\:C_{11}\times C_{11}\xra{} C_{11}\times C_{11}$
 of order $5$.  Construct a non-abelian group of order $605$ which has
 a subgroup isomorphic to $C_{11} \times C_{11}$ (You may use general
 facts about automorphism groups of elementary abelian groups without
 proof.)
\end{exercise}
\begin{solution}

\end{solution}

\begin{exercise}
 Write out the multiplication table of the group $\Z_3\rtimes_{-1}\Z_2$.
\end{exercise}
\begin{solution}
 \[ \renewcommand{\arraystretch}{1.5}
  \begin{array}{|c||c|c|c|c|c|c|}\hline
         & ( 0,0) & (+1,0) & (-1,0) & ( 0,1) & (+1,1) & (-1,1) \\
  \hline\hline
  ( 0,0) & ( 0,0) & (+1,0) & (-1,0) & ( 0,1) & (+1,1) & (-1,1) \\ \hline
  (+1,0) & (+1,0) & (-1,0) & ( 0,0) & (+1,1) & (-1,1) & ( 0,1) \\ \hline
  (-1,0) & (-1,0) & ( 0,0) & (+1,0) & (-1,1) & ( 0,1) & (+1,1) \\ \hline
  ( 0,1) & ( 0,1) & (-1,1) & (+1,1) & ( 0,0) & (+1,0) & (-1,0) \\ \hline
  (+1,1) & (+1,1) & ( 0,1) & (-1,1) & (+1,0) & ( 0,0) & (+1,0) \\ \hline
  (-1,1) & (-1,1) & (+1,1) & ( 0,1) & (-1,0) & (-1,0) & ( 0,0) \\ \hline
 \end{array} \]
\end{solution}

\begin{exercise}
 Prove by induction that in $\Z_n\rtimes_a\Z_m$ we have
 \[ (v,w)^k = (v + a^w v + \ldots + a^{(k-1)w} v,kw). \]
 List the elements of the group $G=\Z_5\rtimes_2\Z_4$.  Calculate the
 element $\ov{1}+\ov{2}+\ov{2}^2+\ov{2}^3\in\Z_5$.  Can you explain
 your result (\emph{carefully}) with less calculation?  Show that
 every element of $G$ has order $1$, $2$, $4$ or $5$.
\end{exercise}
\begin{solution}
 The base case says that $(v,w)^1=(v,w)$, which is clear.  Given the
 statement for $(v,w)^k$ we have
 \begin{align*}
  (v,w)^{k+1} &= (v,w)(v,w)^k \\
   &= (v,w)(v + a^w v + \ldots + a^{(k-1)w} v,kw) \\
   &= (v + a^w(v + a^w v + \ldots + a^{(k-1)w} v),w+kw) \\
   &= (v + a^wv + a^{2w}v + \ldots + a^{kw}v,(k+1)w),
 \end{align*}
 which is the required statement for $(v,w)^{k+1}$.

 The elements of $G$ are as follows:
 \[ \renewcommand{\arraystretch}{1.5}
  \begin{array}{ccccc}
  (\ov{0},\ov{0}) & (\ov{1},\ov{0}) & (\ov{2},\ov{0}) &
                    (\ov{3},\ov{0}) & (\ov{4},\ov{0}) \\
  (\ov{0},\ov{1}) & (\ov{1},\ov{1}) & (\ov{2},\ov{1}) & 
                    (\ov{3},\ov{1}) & (\ov{4},\ov{1}) \\
  (\ov{0},\ov{2}) & (\ov{1},\ov{2}) & (\ov{2},\ov{2}) & 
                    (\ov{3},\ov{2}) & (\ov{4},\ov{2}) \\
  (\ov{0},\ov{3}) & (\ov{1},\ov{3}) & (\ov{2},\ov{3}) & 
                    (\ov{3},\ov{3}) & (\ov{4},\ov{3}).
 \end{array} \]
 
 We have $\ov{2}^2=\ov{4}$ and $\ov{2}^3=\ov{8}=\ov{3}$ so
 $\ov{1}+\ov{2}+\ov{2}^2+\ov{2}^3=\ov{1}+\ov{2}+\ov{4}+\ov{3}=\ov{10}=\ov{0}$.
 Alternatively, the group $\Z_5^\tm$ has order $4$, so for any
 $x\in\Z_5^\tm$ we have $x^4=1$ and thus
 $(x-\ov{1})(\ov{1}+x+x^2+x^3)=x^4-\ov{1}=\ov{0}$.  If $x\neq\ov{1}$
 then the element $x-\ov{1}$ is nonzero and thus has a multiplicative
 inverse in $\Z_5$, so we can multiply by this inverse to see that
 $\ov{1}+x+x^2+x^3=\ov{0}$.  By putting $x=\ov{2}$ we see again that
 $\ov{1}+\ov{2}+\ov{2}^2+\ov{2}^3=\ov{0}$.  

 Now, for any element $(v,w)\in G$ we have 
 \[ (v,w)^4=(v+\ov{2}^w v+\ov{2}^{2w} v+\ov{2}^{3w} v,4w)
           = ((\ov{1} + \ov{2}^w + \ov{2}^{2w} + \ov{2}^{3w})v,0).
 \]
 (Here we have used the fact that $4w=0$ for all $w\in\Z_4$.)  If
 $w\neq 0$ we can put $x=\ov{2}^w\neq 1$ in the previous paragraph and
 deduce that $\ov{1} + \ov{2}^w + \ov{2}^{2w} + \ov{2}^{3w}=\ov{0}$,
 so $(v,w)^4=(0,0)$, which implies that the order of $(v,w)$ is $1$,
 $2$ or $4$.  On the other hand, if $w=0$ then it is easy to see that
 $(v,w)^k=(v,0)^k=(kv,0)$ and thus that $(v,w)^5=(0,0)$, so the order
 of $(v,w)$ is $1$ or $5$.
\end{solution}

\begin{exercise}
 Recall that if $H$ and $K$ are groups the Cartesian product $H\tm K$
 is a group under the operation $(h,k)(h',k')=(hh',kk')$.

 Show that if $G$ is a group and $a,b\in G$ are distinct with the
 property that $a,b$ and $ab$ all have order 2 then $L=\{e,a,b,ab\}$ is
 a subgroup of $G$ and $L$ is isomorphic to $C_2\tm C_2$.

 Show that if $K$ is any group of order 4 then either $K\simeq C_4$ or
 $K\simeq C_2\tm C_2$, but not both.  Which of these two alternatives
 hold for $K=D_2$?
\end{exercise}
\begin{solution}
 We first claim that the elements $e$, $a$, $b$ and $ab$ are all
 distinct.  Indeed, if $a=e$ or $b=e$ or $ab=e$ then $a$, $b$ or $ab$
 would have order $1$ rather than $2$, contradicting our assumption.
 We also have $b\neq a$ by assumption.  We cannot have $ab=a$, because
 if we did we could multiply on the left by $a^{-1}$ to get $b=e$.
 Similarly, we cannot have $ab=b$, so all four elements are distinct,
 so $L$ is a set of order $4$.

 The set $L$ clearly contains $e$.  As $a^2=b^2=(ab)^2=e$ we have
 $a^{-1}=a$, $b^{-1}=b$, $(ab)^{-1}=ab$ and of course $e^{-1}=e$; thus
 $L$ is closed under taking inverses.  We also have
 $ab=(ab)^{-1}=b^{-1}a^{-1}=ba$ so $a$ and $b$ commute.  Using this,
 it is easy to fill in the multiplication table as follows:
 \[ \begin{array}{|c|cccc|}
  \hline
      & e  & a  & b  & ab \\ \hline
   e  & e  & a  & b  & ab \\ 
   a  & a  & e  & ab & b  \\ 
   b  & b  & ab & e  & a  \\ 
   ab & ab & b  & a  & e  \\ \hline
 \end{array} \]
 (For example, $(ab)a=aba=aab=b$, which explains the entry in the row
 marked $ab$ and the column marked $a$.)  The table shows that $L$ is
 closed under multiplication, so it is a subgroup.  Recall that
 $C_2=\{1,R\}$ where $R^2=1$.  We can define $\phi\:C_2\tm C_2\xra{}L$
 by $\phi((R^i,R^j))=a^ib^j$, so
 \begin{align*}
  \phi((1,1)) &= e \\
  \phi((R,1)) &= a \\
  \phi((1,R)) &= b \\
  \phi((R,R)) &= ab 
 \end{align*}
 As $C_2\tm C_2=\{(1,1),(R,1),(1,R),(R,R)\}$ we see that $\phi$ is a
 bijection. 

 As $a$ and $b$ commute we have
 \begin{align*}
  \phi((R^i,R^j))\phi((R^k,R^l)) &=
   a^ib^ja^kb^l \\
   &= a^ia^kb^jb^l \\
   &= a^{i+k} b^{j+l} \\
   &= \phi((R^{i+k},R^{j+l})) \\
   &= \phi((R^i,R^j)(R^k,R^l)),
 \end{align*}
 which shows that $\phi$ is a homomorphism and thus an isomorphism.

 Now let $K$ be a group of order $4$.  By Lagrange's theorem, every
 element $a\in K$ has order dividing $4$ and thus equal to $1$, $2$ or
 $4$.  Only the identity element can have order $1$ so the other $3$
 elements must have order $2$ or $4$.  If there are no elements of
 order $4$ then let $a$ and $b$ be any two distinct elements of order
 $2$.  Then $ab$ is another element of $K$, which is not the identity
 because $a\neq b^{-1}=b$, so it must also have order $2$.  Using the
 first part of the question we deduce that $L=\{1,a,b,ab\}$ is a
 subgroup of $K$ and is isomorphic to $C_2\tm C_2$.  As $|L|=|K|=4$ we
 must have $L=K$, so $K\simeq C_2\tm C_2$.

 On the other hand, suppose that not all elements of $K\sm\{1\}$ have
 order $2$.  Let $a$ be an element of order $4$, and put
 $L=\{1,a,a^2,a^3\}$.  It is easy to see that $L$ is a subgroup
 isomorphic to $C_4$, and $|L|=|K|=4$ so $L=K$, so $K\simeq C_4$ as
 claimed.  

 We cannot have both $K\simeq C_2\tm C_2$ and $K\simeq C_4$, for in
 the first case all elements of $K$ have order $2$ whereas in the
 second case some elements have order $4$.
\end{solution}

\begin{exercise}
 Let $G$ be a group of order $18$.
 \begin{itemize}
 \item[(a)] Show that there is a normal subgroup $P\leq G$ with
  $|P|=9$, and another subgroup $Q\leq G$ of the form $Q=\{1,h\}$ with
  $h^2=1$ (and so $h^{-1}=h$).
 \item[(b)] Show that if $P\simeq C_9$ then $G\simeq C_2\tm C_9$ or
  $G\simeq D_9$.
 \item[(c)] Now suppose that $P\simeq C_3\tm C_3$ (so in particular,
  $P$ is abelian, and $x^3=1$ for all $x\in P$).  For any $x\in P$, we
  put  
  \begin{align*}
   x^h &= hxh = h^{-1}xh \\
   x_+ &= x^2\,(x^h)^2 = xxhxhhxh = xxhxxh \\ 
   x_- &= x^2\,x^h = xxhxh \\
   P_+ &= \{x\in P \st x^h=x\} \\
   P_- &= \{x\in P \st x^h=x^{-1}\}.
  \end{align*}
  \begin{itemize}
   \item[(i)] Show that $x_+\in P_+$ and $x_-\in P_-$ and $x=x_+x_-$.
   \item[(ii)] Show that $P_+$ and $P_-$ are subgroups of $P$, and that
    $P_+\cap P_-=\{1\}$.
   \item[(iii)] Deduce that $P\simeq P_+\tm P_-$.
   \item[(iv)] Show that if $|P_+|=9$ then $G\simeq C_3\tm C_3\tm C_2$.
   \item[(v)] Show that if $|P_+|=3$ then $G\simeq C_3\tm D_3$.
  \end{itemize}
  (The case $|P_+|=1$ gives a new group, for which we do not yet have
  a name.)
 \end{itemize}
\end{exercise}
\begin{solution}
 \begin{itemize}
  \item[(a)] We know that $n_3$ divides $2$ and is congruent to $1$
   modulo $3$; the only possibility is $n_3=1$.  This means that there
   is a unique, normal Sylow $3$-subgroup, which we call $P$.  We then
   let $Q$ be any Sylow $2$-subgroup.  As $|Q|=2$, it is clear that
   $Q=\{1,h\}$ for some $h$ with $h^2=1$.
  \item[(b)] Suppose that $P\simeq C_9$.  We can then choose an
   element $g\in P$ that generates $P$, with $g^9=1$.  As $P$ is
   normal, we see that $hgh=g^a$ for some integer $a$.  As $h^2=1$,
   this means that
   \[ g = h^2gh^2 = hg^ah=g^{a^2}, \]
   so $a^2=1\pmod{9}$, so the number $a^2-1=(a+1)(a-1)$ is divisible
   by $9$.  The following table shows all numbers $a$ mod $9$ and
   their squares:
   \[ \begin{array}{|c||c|c|c|c|c|c|c|c|c|} 
       \hline
        a   & -4 & -3 & -2 & -1 & 0 & 1 & 2 & 3 & 4 \\
       \hline
        a^2 & -2 &  0 &  4 &  1 & 0 & 1 & 4 & 0 & -2 \\
       \hline
      \end{array}
   \]
   As $a^2=1\pmod{9}$, we must have $a=1\pmod{9}$ or $a=-1\pmod{9}$,
   so $hgh=g$ or $hgh=g^{-1}$.  If $hgh=g$ then $h$ commutes with $g$
   and we find that $G\simeq P\tm Q\simeq C_2\tm C_9$.  If
   $hgh=g^{-1}$ we find that $G\simeq D_9$.
  \item[(c)] Now suppose instead that $P\simeq C_3\tm C_3$.
   \begin{itemize}
    \item[(i)] We have
     \[ (x_+)^h = hx_+h = h(xxhxxh)h = hxxhxx. \]
     Now, $xx$ lies in $P$ and $P$ is normal so $hxxh$ lies in $P$.
     Moreover, $P$ is abelian, so $xx$ commutes with $hxxh$.  We 
     thus have $hxxhxx=xxhxxh$, or in other words, $(x_+)^h=x_+$.
     This shows that $x_+\in P_+$.

     Next, we have $xxx=x^3=1$ and so 
     \[ (x_-)^h x_- = h(xxhxh)h\,xxhxh = 
         hxxhxxxhxh = hxxhhxh = hxxxh = hh = 1,
     \]
     so $(x_-)^h=(x_-)^{-1}$, so $x_-\in P_-$.

     Finally, we have
     \[ x_+ x_- = (xx)(hxxh)\,(xx)(hxh) = (xx)(xx)(hxxh)(hxh) = 
         xhxxxh = xhh = x.
     \]
     (In the second equality, we used the fact that each of the four
     bracketed terms lies in $P$, so we can commute them past each
     other.) 
    \item[(ii)] Clearly $1\in P_+$ and $1\in P_-$.  If $x,y\in P_+$ we
     have $x=hxh$ and $y=hyh$, so $xy=hxh\,hyh=hxyh$, so $xy\in P_+$.
     In particular, we can take $y=x$ to see that the element
     $x^{-1}=x^2$ lies in $P_+$.  This shows that $P_+$ is a subgroup
     of $P$.  Now suppose instead that $x,y\in P_-$, so that
     $hxh=x^{-1}$ and $hyh=y^{-1}$.  As $x$ and $y$ commute we have
     $x^{-1}y^{-1}=(xy)^{-1}$, so
     $hxyh=hxh\,hyh=x^{-1}y^{-1}=(xy)^{-1}$, so $xy\in P_-$.  It
     follows that $P_-$ is also a subgroup.

     If $x\in P_+\cap P_-$ then $x^h=x$ and also $x^h=x^{-1}$, so
     $x=x^{-1}$, so $x^2=1$.  As $P\simeq C_3\tm C_3$ we also know
     that $x^3=1$, and it follows that $x=x^3(x^2)^{-1}=1$.  This
     shows that $P_+\cap P_-=\{1\}$.
    \item[(iii)] As $P_+$ and $P_-$ are subgroups of the abelian group
     $P$, we see that they commute with each other, so we can define a
     homomorphism $\phi\:P_+\tm P_-\xra{}P$ by $\phi(y,z)=yz$.  For
     any $x\in P$ we have $(x_+,x_-)\in P_+\tm P_-$ and
     $\phi(x_+,x_-)=x_+x_-=x$.  This shows that $\phi$ is surjective.
     We also have $P_+\cap P_-=\{1\}$, which implies that $\phi$ is
     injective.  This means we have an isomorphism
     $P_+\tm P_-\xra{}P$, and so $|P_+||P_-|=9$.
    \item[(iv)] Suppose that $|P_+|=9$, so $P_+=P$ and $P_-=\{1\}$.
     This means that $hxh=x$ for all $x\in P$, so $P$ commutes with
     $Q$, so $G\simeq Q\tm P\simeq C_2\tm C_3\tm C_3$.
    \item[(v)] Now suppose instead that $|P_+|=3$, so $|P_-|=3$ also.
     This means that $P_+\simeq P_-\simeq C_3$, so we can choose
     $a\in P_+$ and $b\in P_-$ such that $P_+=\{1,a,a^2\}$ and
     $P_-=\{1,b,b^2\}$ and $a^3=b^3=1$.  We define
     $\phi\:C_3\tm D_6\xra{}G$ by 
     \begin{align*}
      \phi((R^i,R^j))  &= a^i b^j \\
      \phi((R^i,R^jS)) &= a^i b^j h.
     \end{align*}
     It is easy to check that this is an isomorphism of groups.
   \end{itemize}
 \end{itemize}
\end{solution}

\begin{exercise}
\begin{itemize}
 \item[(a)] Let $G$ be a group of order $21$.  Show that $G$ has a
  normal subgroup of order $7$ and hence show that $G$ is a semidirect
  product of proper subgroups.
 \item[(b)] Construct a non-abelian group of order $21$.
 \item[(c)] Using the fact that $\Z_7^\tm$ is cyclic, show that every
  non-Abelian group of order $21$ is isomorphic to the one in (b).
 \end{itemize}
\end{exercise}
\begin{solution}
 \begin{itemize}
 \item[(a)] If we let $n_7$ be the number of Sylow $7$-subgroups then
  $n_7$ divides $3$ and is congruent to $1$ modulo $7$ so we must have
  $n_7=1$.  If we let $N$ be the unique Sylow $7$-subgroup then it
  follows that $N$ is normal in $G$.  [Explicitly, if $g\in G$ then
  $gNg^{-1}$ is clearly a subgroup of order $7$ but we have seen that
  $N$ is the only such subgroup so $gNg^{-1}=N$, which means that $N$
  is normal.]  Now let $Q$ be a Sylow $3$-subgroup of $G$.  The order
  of $N\cap Q$ divides both $7=|N|$ and $3=|Q|$ so we must have
  $|N\cap Q|=1$ so $N\cap Q=\{1\}$.  This means that
  $|NQ|=|N||Q|/|N\cap Q|=21=|G|$ so $NQ=G$, so $G$ is the semidirect
  product of $N$ and $Q$.
 \item[(b)] Recall that $\Z_n\rtimes_a\Z_m$ is defined whenever
  $a\in\Z_n^\tm$ and $a^m=\ov{1}$.  As $\ov{2}\in\Z_7^\tm$ and
  $\ov{2}^3=\ov{8}=\ov{1}$, we can define the group
  $H=\Z_7\rtimes_2\Z_3$, which is a non-Abelian group of order $21$.
 \item[(c)] Let $G$ be a non-Abelian group of order $21$.  As in
  part~(a) we see that there are subgroups $N$ and $Q$ of $G$ such
  that $N$ is normal, $|N|=7$, $|Q|=3$, and $G$ is the semidirect
  product of $N$ and $Q$.  As the orders of $N$ and $Q$ are prime,
  they must be cyclic groups.  We saw in lectures that all semidirect
  products of cyclic groups have the form $\Z_n\rtimes_a\Z_m$, so
  $G\simeq \Z_7\rtimes_a\Z_3$ for some $a\in\Z_7^\tm$ with $a^3=1$.  

  We know that $\Z_7^\tm$ is cyclic of order $6$, generated by some
  element $b$ say.  From this it is not hard to see that there are
  precisely three elements satisfying $a^3=\ov{1}$, viz. $a=\ov{1}$,
  $a=b^2$ and $a=b^{-2}$.  Explicitly, we have
  \[ \renewcommand{\arraystretch}{1.5}
   \begin{array}{|c|cccccc|} \hline
   a   & \ov{ 1} & \ov{ 2} & \ov{ 3} & \ov{-1} & \ov{-2} & \ov{-3}
    \\ \hline
   a^3 & \ov{ 1} & \ov{ 1} & \ov{-1} & \ov{-1} & \ov{-1} & \ov{ 1}
    \\ \hline
  \end{array} \]
  so the three possibilities are $a=\ov{1}$, $a=\ov{2}$ and
  $a=\ov{-3}=\ov{2}^{-1}$.  [If we take $b=\ov{3}$ then $b$ generates
  $\Z_7^\tm$ and $b^2=\ov{2}$ and $b^{-2}=\ov{-3}$.]  The case $a=1$
  would give the Abelian group $\Z_7\tm\Z_3$, which is impossible as
  $G$ is assumed to be non-Abelian.

  We now see that $G$ is isomorphic either to the group
  $H=\Z_7\rtimes_2\Z_3$ considered in part~(b), or to the group
  $H':=\Z_7\rtimes_{-3}\Z_3$.  It will thus be enough to show that $H'$
  is isomorphic to $H$.

  Define $\phi\:H\xra{}H'$ by $\phi(v,w)=(v,-w)$; this is clearly a
  bijection.  We next show that $\phi$ is a homomorphism.  We will
  write $*$ for the group operation in $H$ and $*'$ for the group
  operation in $H'$, so as to make clear which is which.  We have
  \begin{align*}
   \phi((v,w)*(x,y)) &= \phi(v+\ov{2}^wx,w+y) \\
    &= (v+\ov{2}^wx,-w-y) \\
    &= (v+\ov{-3}^{-w}x,-w-y) \\
    &= (v,-w)*'(x,-y) \\
    &= \phi(v,w)*'\phi(x,y),
  \end{align*}
  which shows that $\phi$ is a homomorphism as claimed, and thus an
  isomorphism.  
 \end{itemize}
\end{solution}

\begin{exercise}
 Let $G$ be a group of order $60$, and suppose that $G$ has a cyclic
 normal subgroup $N$ of order $12$.  Let $Q$ be a Sylow $5$-subgroup of
 $G$.  What standard groups are $N$, $Q$ and $\Aut(N)$ isomorphic to?
 Show that every homomorphism $\phi\:Q\xra{}\Aut(N)$ is trivial.
 Deduce that every element of $Q$ commutes with every element of $N$,
 and thus that $G$ is Abelian.
\end{exercise}
\begin{solution}
 As $N$ is cyclic of order $12$, it is isomorphic to $\Z_{12}$.  As
 $|G|=5\tm 12$ and $5$ does not divide $12$, the Sylow $5$-subgroup
 $Q$ must have order $5$.  Groups of prime order are cyclic so $Q$ is
 isomorphic to $\Z_5$.  Finally,
 $\Aut(N)\simeq\Aut(\Z_{12})\simeq\Z_{12}^\tm$, and we showed in
 lectures that $\Z_{12}^\tm=\{\ov{\pm 1},\ov{\pm 5}\}\simeq C_2\tm
 C_2$, so $\Aut(N)\simeq C_2\tm C_2$.  As the order of $Q$ is coprime
 to the order of $\Aut(N)$, we deduce that any homomorphism
 $\phi\:Q\xra{}\Aut(N)$ is trivial.  [More explicitly, if $g\in Q$
 then $g^5=1$ so $\phi(g)^5=\phi(g^5)=1$.  On the other hand, for any
 $h\in\Aut(N)\simeq C_2\tm C_2$ we have $h^2=1$ so $\phi(g)^2=1$.
 This means that $\phi(g)=\phi(g)^5(\phi(g)^2)^{-2}=1$, so $\phi$ is
 trivial.]

 For any $g\in Q$ we define as usual $\gm_g\:N\xra{}N$ by
 $\gm_g(x)=gxg^{-1}$, so $\gm_g\in\Aut(N)$.  We then have a
 homomorphism $\phi\:Q\xra{}\Aut(N)$ given by $\phi(g)=\gm_g$.  This
 must be trivial, by our first paragraph.  Thus $gxg^{-1}=x$ for all
 $g\in Q$ and $x\in N$.  After multiplying on the right by $g$ we
 deduce that $gx=xg$ for all $g$ and $x$, so every element of $Q$
 commutes with every element of $N$.  

 Now define $\mu\:N\tm Q\xra{}G$ by $\mu(x,g)=xg=gx$.  As $N$ and $Q$
 commute, this function is a homomorphism.  The image is the subgroup
 $NQ$, and because $|N|=12$ and $|Q|=5$ are coprime we have
 $|NQ|=|N||Q|=60=|G|$, so $NQ=G$.  It follows that $\mu$ is surjective
 and $|N\tm Q|=|G|$ so it must also be injective and thus an
 isomorphism.  Thus $G\simeq \Z_{12}\tm\Z_5$, so $G$ is Abelian.
\end{solution}

\begin{exercise}
 Show that every group of order 1225 is abelian.
\end{exercise}
\begin{solution}
 First note that $1225=25\tm 49=5^2 7^2$, so we will study the Sylow
 $5$-subgroups and $7$-subgroups of $G$.  We know that $n_5$ divides
 $49$ (so $n_5\in\{1,7,49\}$) and $n_5=1\pmod{5}$.  As $7=2\pmod{5}$
 and $49=4\pmod{5}$ we see that the only possibility is $n_5=1$.  We
 therefore have a unique Sylow $5$-subgroup, which we call $P$.  Note
 that $P$ is normal, and also that $|P|=5^2$, so Proposition~2.8 tells
 us that $P$ is abelian.

 Next, we know that $n_7$ divides $25$ (so $n_7\in\{1,5,25\}$) and
 $n_7=1\pmod{7}$.  As $5=5\pmod{7}$ and $25=4\pmod{7}$, we see that
 $n_7$ must be $1$.  There is thus a unique Sylow $7$-subgroup, which
 we call $Q$.  We again see that $Q$ is normal and abelian.

 Using Proposition~2.9 in the notes, we see that $G\simeq P\tm Q$.
 This is abelian, because $P$ and $Q$ are.
\end{solution}

\begin{exercise}
 If $G$ is a group let $\tht_g\:G\xra{}G$ be defined by
 $\tht_g(x)=gxg^{-1}$. The group of inner automorphisms is
 \[ \Inn (G)=\{ \theta_g \; |\; g \in G\}. \]
 \begin{itemize}
  \item[(i)] Show that $\Inn(G)$ is a normal subgroup of $\Aut(G)$.
  \item[(ii)] Calculate $\Aut(G)$, $\Inn(G)$ and $\Out(G)=\Aut(G)/\Inn(G)$
  in the following cases
  \begin{itemize}
  \item[(a)] $G=C_7$
  \item[(b)] $G=C_3 \times C_3$
  \item[(c)] $G=S_3$
  \item[(d)] [A bit harder] $G=Q_8$.
  \end{itemize}
 \end{itemize}
\end{exercise}
\begin{solution}
 
\end{solution}

\begin{exercise}
 Given $m \geq 1$ consider the $2\times 2$ matrices
 \[ x = \left(\begin{array}{cc}
     e^{\pi i/m} & 0             \\
     0           & e^{-\pi i/m}  \\
    \end{array} \right) \mbox{ and }
   y = \left( \begin{array}{cc}
     0           & -1            \\
     1           &  0            \\
    \end{array}\right)
 \]
 Let $Q_{4m}=\ip{x,y}$ be the subgroup of $GL_2(\C)$ that they
 generate. This is called the generalized quaternion group.
 \begin{itemize}
  \item[(i)] Show that $x$ is of order $2m$ and $y$ is of order $4$.
  \item[(ii)] Show that $yxy^{-1}=x^{-1}$.
  \item[(iii)] Deduce from Part (ii) that any element of $Q_{4m}$ can
   be written in the form $x^iy^j$. Since $x^m=y^2$, we may assume
   $j=0$ or $1$. List the $4m$ elements of $Q_{4m}$.
  \item[(iv)] Deduce from Part (ii) that for any $l$ the subgroup
   $\ip{x^l}$ is normal in $Q_{4m}$.
  \item[(v)] Show that if $m$ is odd, $Q_{4m}$ is a semidirect product
   of the normal subgroup $N=\ip{x^2}$ by the subgroup $H=\ip{y}$,
   associated to the homomorphism $\phi\: H\xra{}\Aut(N)\cong U(\Z_m)$
   given by $\phi(y)=\psi_{-1}$.
  \item[(vi)] Conclude that $Q_{12}$ as defined here is isomorphic to
   the group with that name in the lectures.
  \item[(vii)] Show that $Q_8$ is not a semidirect product of proper
   subgroups.
 \end{itemize}
\end{exercise}
\begin{solution}

\end{solution}



\end{document}
